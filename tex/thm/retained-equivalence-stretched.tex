\begin{lemma}[Stretched Algorithm Retained Ingest Times]
\label{thm:retained-equivalence-stretched}
If the first $n$ data items $\colorH(\colorTbar) = \colorh$ for each h.v. $\colorh$ are retained, then we are guaranteed to have retained
\begin{align*}
\colorTbar
&\in
\{
  j'2^{\colorh'} - 1
  :
  j' \in 1,2,\,\ldots,2n
  \text{ and }
  \colorh' \in \mathbb{N}
\}.
\end{align*}
Note that, although this formulation nominally includes $\colorTbar > \colorT$, an extension filtering $\colorTbar \in \{0,1,\,\ldots,\colorT\}$ follows trivially.
\end{lemma}
\begin{proof}

Recall that the $j$th instance of hanoi value $\colorh$ appears at ingest time
\begin{align*}
\colorTbar
&= j2^{\colorh + 1} + 2^{\colorh} - 1,
\end{align*}
indexed from $j=0$.

The set of retained data items can be denoted
\begin{align*}
\mathsf{have\_retained} \coloneq
\{
  j2^{\colorh + 1} + 2^{\colorh} - 1
  :
  j \in 0,1,\,\ldots,n-1
  \text{ and }
  \colorh \in \mathbb{N}
\}.
\end{align*}

We will show $\mathsf{have\_retained}$ equivalent to,
\begin{align*}
\mathsf{want\_retained} \coloneq
\{
  j'2^{\colorh'} - 1
  :
  j' \in 1,2,\,\ldots,2n
  \text{ and }
  \colorh' \in \mathbb{N}
\}.
\end{align*}

\begin{proofpart}[$\mathsf{have\_retained} \subseteq \mathsf{want\_retained}$]
Suppose $\colorTbar \in \mathsf{have\_retained}$.
Then $\exists j \in \{0,1,\,\ldots,n-1\}$ and $\colorh \in \mathbb{N}$ such that
\begin{align*}
\colorTbar
&= j2^{\colorh + 1} + 2^{\colorh} - 1\\
&= (2j + 1)2^{\colorh} - 1.
\end{align*}
Noting $2j + 1 \in 1,2,\,\ldots,2n$ gives $\mathsf{have\_retained} \stackrel{\checkmark}{\subseteq} \mathsf{want\_retained}$.
\end{proofpart}

\begin{proofpart}[$\mathsf{want\_retained} \subseteq \mathsf{have\_retained}$]
Suppose $\colorTbar \in \mathsf{want\_retained}$.
Then $\exists j'\in\{1,2,\,\ldots,2n\}$ and $\colorh' \in \mathbb{N}$ such that
\begin{align*}
\colorTbar
&= j'2^{\colorh'} - 1.
\end{align*}

First, where $j' \in \{1,3,5,\,\ldots,2n-1\}$,
\begin{align*}
\colorTbar
&= \frac{j'-1}{2} 2^{\colorh' + 1} + 2^{\colorh'} - 1.
\end{align*}
Because $\frac{j'-1}{2} \in 0, 1, \, \ldots, n - 1$ here, $\mathsf{want\_retained} \stackrel{\checkmark}{\subseteq} \mathsf{have\_retained}$ in this case.

In the case that $j' \in \{0,2,4,\,\ldots,2n\}$,
\begin{align*}
\colorTbar
&= j'2^{\colorh'} - 1\\
&=
\eqnmarkbox[WildStrawberry]{inpositiven}{
  \frac{j'}{2}
}2^{\colorh' + 1} - 1.
\end{align*}
\annotate[yshift=0em]{below,right}{inpositiven}{$\in \{1,2,\,\ldots,n\}$}

Recalling that $\colorH(j'/2 - 1) = \log_2 \max\{ i \in 2^{\mathbb{N}} : j'/2 \bmod i = 0 \}$,
\begin{align*}
\\
\colorTbar
&=
\eqnmarkbox[WildStrawberry]{inoddintegers}{
  \frac{j'/2}{2^{\colorH(j'/2 - 1)}}
}
2^{\colorh'} - 1\\
&=
\eqnmarkbox[WildStrawberry]{inevenintegers}{
  \Big(\frac{j'/2}{2^{\colorH(j'/2 - 1)}} - 1\Big)
}
2^{\colorh'} + 2^{\colorh'} - 1.\\
\end{align*}
\annotate[yshift=1em]{above,right}{inoddintegers}{$\in \{x \in 1,3,5,\,\ldots : x \leq n\}$}
\annotate[yshift=0em]{below,right}{inevenintegers}{$\in \{x \in 0, 2, 4, \,\ldots : x \leq n - 1\}$}
Pulling out a factor of 2 from the first coefficient,
\begin{align*}
  \colorTbar
  &=
  \eqnmarkbox[Orchid]{}{
  \frac{
    \frac{j'/2}{2^{\colorH(j'/2 - 1)}} - 1
  }{2}
  }
2^{\colorh' + 1} + 2^{\colorh'} - 1.
\end{align*}
With
\begin{align*}
  \eqnmarkbox[Orchid]{}{
  \frac{
    \frac{j'/2}{2^{\colorH(j'/2 - 1)}} - 1
  }{2}
  }
  &\in \{x \in 0,1,2,\,\ldots: x \leq (n-1)/2\}\\
  &\stackrel{\checkmark}{\in} \{0,1,2,\,\ldots,n-1\},
\end{align*}
we have $\mathsf{want\_retained} \stackrel{\checkmark}{\subseteq} \mathsf{have\_retained}$ in this case, too.
\end{proofpart}
\end{proof}
