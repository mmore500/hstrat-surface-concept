\begin{lemma}[Placements Overwrite Hanoi Value $\colorh = \colort - 1$]
Placing data items $\colorT$ within segments at position $\colorH(\colorT)$ modulo segment length ensures elimination of \hv{} $\colorh = \colort - 1$ from each segment.
\end{lemma} \label{thm:steady-hv-elimination}
\begin{proof}
Recall that \hv{} $\colorh = \colort + \colors - n - 1$ is placed in the $n$th bunch during epoch $\colort$.
By construction, segments in the $n$th bunch have $\colors - n$ sites.
We must verify,
\begin{align*}
\colort - 1
&\stackrel{?}{=}
\mathsf{invading\_h.v.} - \mathsf{segment\_length}\\
&\stackrel{?}{=}
(\colort + \colors - n - 1) - (\colors - n)
 \\
&\stackrel{\checkmark}{=} \colort - 1.
\end{align*}
\end{proof}

% So, at epoch $\colort$ in reservation $n$ the hanoi value $h(j, r)$ will overwrite hanoi value
% \begin{align*}
% h(j, r) - (s - j) - 1 \\
% &= (r - j) - (s - j) - 1 \\
% &\stackrel{\checkmark}{=} r - s - 1.
% \end{align*}
