\begin{lemma}[Minimum recent items retained per \hv{}]
\label{thm:tilted-most-recent-retained}
At least the most recent $\left\lceil2^{\colors - 1 - \colortau}\right\rceil$ encountered instances of every \hv{} $\colorh$ are retained.
Concretely,
\begin{align*}
\textsf{goal\_tilted} \subset \colorB_{\colorT}.
\end{align*}
\end{lemma}
\begin{proof}
From Lemma \ref{thm:stretched-discarded-incidence-count} we have reservations available to store at least the first $\left\lceil 2^{\colors - 1 - \colortau} \right\rceil$ instances of each hanoi value.
So, we can restrict our consideration to where \hv{} instance count exceeds $\left\lceil 2^{\colors - 1 - \colortau} \right\rceil$.

In the absence of invasion, data item placement (by design) cycles around the sites reserved to a hanoi value.
With at least $\left\lceil 2^{\colors - 1 - \colortau} \right\rceil$ sites reserved, at least the last $\left\lceil 2^{\colors - 1 - \colortau} \right\rceil$ instances of each \hv{} are retained.

What about in the case of invasion?
In this case, again from Lemma \ref{thm:stretched-discarded-incidence-count}, we have the number of reserved sites as dropping from $\left\lceil 2^{\colors - \colortau} \right\rceil$ to $\left\lceil 2^{\colors - 1 - \colortau} \right\rceil$.
Recall from Lemma \ref{thm:tilted-last-touched} each epoch, the final encountered instance of each \hv{} is placed into the rightmost reservation segment.
By design of placement ordering across segments, we therefore know that the final $\left\lceil 2^{\colors - 1 - \colortau} \right\rceil$ instances of a \hv{} encountered during an epoch are laid out left to right in each of the smallest-size remaining segments, $r = \colortau$ ( with the last instance occupying the rightmost reservation segment).

So, at the outset of epoch $\colort$, reassigned sites $\{\colork \in [0\twodots\colorS) : \colorHcal_{\colort - 1}(\colork) \neq \colorHcal_{\colort}(\colork)\}$ always contain the most recent $\left\lceil 2^{\colors - 1 - \colortau} \right\rceil$ instances of \hv{} $\colorh = \colorHcal_{\colort - 1}(\colork)$, arranged left to right.
If data items in these reassigned sites were lost instantaneously at time $\min(\colorT \in \colortsetofT)$, we would not meet our proof objectives.
At that point, we would have none of the most-recent $\left\lceil 2^{\colors - 1 - \colortau} \right\rceil$ \hv{} $\colorh$ data items retained.
However, data items are not lost instantaneously when a site is reassigned.
Instead, data items in reassigned sites $\colork$ linger until they are \textit{actually} overwritten by incoming data items $\colorT \in \colortsetofT$ with $\colorK(\colorT) = \colork$.

From Lemma \ref{thm:tilted-invading-overwrite-order}, we have that, over the course of an epoch, invaded data items are overwritten left to right --- except the leftmost reservation, which is overwritten last.
Ensuring that the most recent $\left\lceil 2^{\colors - 1 - \colortau} \right\rceil$ data items are stored therefore requires two desiderata:
\begin{enumerate}
\item at least two instances of invaded \hv{} $\colorh$ occur before the first invading overwrite, and
\item the cadence of overwrites proceeds slower than that of new instances of invaded hanoi value $\colorh$.
\end{enumerate}


\begin{mybox}
\textbf{Intuition.}
Imagine the sequence of the $\left\lceil 2^{\colors - 1 - \colortau} \right\rceil$ most instances of \hv{} $\colorh$ as the protagonist of the classic video game ``snake'' \citep{de2016complexity}.
In that game, the titular snake slithers by growing at its head and shrinking at its tail.
Analogously, our sequence of most recent \hv{} instances adds new items at the front and has tail items overwritten.
When an invasion occurs and half of ring buffer reservations are reassigned, the snake's body of $\left\lceil 2^{\colors - 1 - \colortau} \right\rceil$ sites will always be stretched out across the reassigned half of the ring buffer.
In other words, our snake is laid out entirely within the \textit{danger zone}!

At that point when an invasion epoch $\colort$ begins, our snake containing $\left\lceil 2^{\colors - 1 - \colortau} \right\rceil$ items will be chased into the preserved half of the ring buffer as overwrites enchroach at its rear.
The two desiderata described above ensure that the snake (1) pulls ahead and (2) stays far enough ahead of invading overwrites to keep all $\left\lceil 2^{\colors - 1 - \colortau} \right\rceil$ body segments intact.
Mixing metaphors, the snake slithers head then tail to safety as the rickety bridge of reassigned but not-yet-overwritten sites it had been occupying collapses behind it.
After escaping the reassigned $\left\lceil 2^{\colors - 1 - \colortau} \right\rceil$ ring buffer sites, the snake of recent \hv{} instances then happily crawls in circles around its $\left\lceil 2^{\colors - 1 - \colortau} \right\rceil$ reserved sites --- at least, until invaded again.
\end{mybox}

\begin{proofpart}[Two Instances of Invaded Hanoi Value before First Invading Overwrite]
Let $\colorT' = \min(\colorT \in \colortsetofTone)$.
Because $\colorT' \in \{2^{\mathbb{N}}\} \forall \colort$ we have $\colorH(\colorT') = 0$.
By the fractal properties of the hanoi sequence, we may then rewrite the sequence of hanoi values encountered during epoch $\colort + 1$ as,
\begin{align*}
\colorH(\colorT'), \colorH(\colorT'+1), \,\ldots, \colorH(2\colorT' - 1) = \colorH(0), \colorH(1), \,\ldots, \colorH(\colorT' - 1).
\end{align*}
Note that $2\colorT' - 1 = \max(\colorT \in \colortsetofTone)$.

By Lemma \ref{thm:tilted-invader-minus-invaded} for \hv{} $\colorHcal_{\colort}(\colork)$ invaded by \hv{} $\colorHcal_{\colort + 1}(\colork)$ (i.e., $\colorHcal_{\colort}(\colork) \neq \colorHcal_{\colort + 1}(\colork)$), we have $\colorHcal_{\colort + 1}(\colork) \geq \colorHcal_{\colort}(\colork) + 2$.
Recalling that the first instance of \hv{} $\colorh$ occurs at ingest time $\colorT = 2^{\colorh} - 1$ and \hv{} $\colorh$ recurs at $\colorT = 2^{\colorh + 1} + 2^{\colorh} - 1$, note that $\forall\colorh$,
\begin{align*}
|\{\colorT \in [0 \twodots 3 \times 2^{\colorh}] : \colorH(\colorT) = \colorh\}| = 2.
\end{align*}
\end{proofpart}
Because $\colorT = 3 \times 2^{\colorh} < 2^{\colorh + 2} - 1$, we have our result.

\begin{proofpart}[Overwrite Cadence Slower than Invaded Hanoi Value Cadence]
The cadence of \hv{} $\colorh$, after its first incidence at ingest time $\colorT=2^{\colorh} - 1$ is to occur every $2^{\colorh + 1}$ data items.
Specifically, where $\colorT \bmod 2^{\colorh + 1} = 2^{\colorh} - 1$.
Observe also that the cadence at which a \hv{} $\geq\colorh$ occurs is every $2^{\colorh}$ items, where $\colorT \bmod 2^{\colorh} = 2^{\colorh} - 1$.

Again, by Lemma \ref{thm:tilted-invader-minus-invaded} for \hv{} $\colorHcal_{\colort}(\colork)$ invaded by \hv{} $\colorHcal_{\colort + 1}(\colork)$ (i.e., $\colorHcal_{\colort}(\colork) \neq \colorHcal_{\colort + 1}(\colork)$), we have $\colorHcal_{\colort + 1}(\colork) \geq \colorHcal_{\colort}(\colork) + 2$.
New incidences of invaded \hv{} $\colorHcal_{\colort}(\colork)$ accrue faster than the final $\left\lceil 2^{\colors - 1 - \colortau} \right\rceil$ instances of \hv{} $\colorh$ from epoch $\colort$ are overwritten because
\begin{align*}
2^{\colorh + 1} \stackrel{\checkmark}{<} 2^{\colorh + 2}.
\end{align*}
\end{proofpart}

\end{proof}
