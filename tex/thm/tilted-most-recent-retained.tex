\begin{lemma}[Minimum Recent Items Retained per Hanoi Value]
\label{thm:tilted-most-recent-retained}
At least the most recent $\left\lceil2^{\colors - 1 - \colortau}\right\rceil$ encountered instances of every h.v. $\colorh$ are retained.
\end{lemma}
\begin{proof}
From Lemma \ref{thm:stretched-discarded-incidence-count} we have that reservations are available to store at least the first $\left\lceil 2^{\colors - 1 - \colortau} \right\rceil$ instances of each hanoi value.
So, we can restrict our consideration to instances where h.v. instance count exceeds $\left\lceil 2^{\colors - 1 - \colortau} \right\rceil$.

In the absence of invasion, data item placement (by design) cycles around the sites reserved to a hanoi value.
With at least $\left\lceil 2^{\colors - 1 - \colortau} \right\rceil$ sites reserved, at least the last $\left\lceil 2^{\colors - 1 - \colortau} \right\rceil$ instances of each h.v. are retained.

What about in the case of invasion?
In this case, again from Lemma \ref{thm:stretched-discarded-incidence-count}, we have the number of reserved sites as dropping from $\left\lceil 2^{\colors - \colortau} \right\rceil$ to $\left\lceil 2^{\colors - 1 - \colortau} \right\rceil$.
Recall from Lemma \ref{thm:tilted-last-touched} each epoch, the final encounttered instance of each h.v. is placed into the rightmost reservation segment.
By design, we therefore know that the final encountered $\left\lceil 2^{\colors - 1 - \colortau} \right\rceil$ instances (ending at the rightmost reservation segment) are laid out left-to-right across the smallest remaining segments, $r = \colortau$.

So, at the outset of epoch $\colort$, to-be-invaded sites $\{\colork \in \colorS : \colorHcal_{\colort - 1}(\colork) \neq \colorHcal_{\colort}(\colork)\}$ always contain the most-recent $\left\lceil 2^{\colors - 1 - \colortau} \right\rceil$ ingested h.v. $\colorh = \colorHcal_{\colort - 1}(\colork)$ data items, sequenced left-to-right.
If data items in these sites were lost instantaneously at time $\min\{\colorT \in \colort\}$, we would not meet our proof objectives, having at that point none of the most-recent $\left\lceil 2^{\colors - 1 - \colortau} \right\rceil$ h.v. $\colorh$ data items retained.
However, this is not the case --- data items in these sites $\colork$ linger until they are \textit{actually} overwritten by incoming data items $\colorT \in \colort$ with $\colorK(\colorT) = \colork$.

From Lemma \ref{thm:tilted-invading-overwrite-order}, we have that --- over the course of an epoch --- data items are overwritten in the left-to-right order they were deposited --- except the leftmost reservation, which is overwritten last.
To ensure that the most recent $\left\lceil 2^{\colors - 1 - \colortau} \right\rceil$ data items are stored, we therefore must show two conditions:
\begin{enumerate}
\item that at least two depositions of our overwritten h.v. $\colorh$ occur before the first invasion overwrite, and
\item that the cadence of overwrites proceeds slower than the cadence of new instances of the hanoi value $\colorh$ being overwritten.
\end{enumerate}

Imagine the sequence of $\left\lceil 2^{\colors - 1 - \colortau} \right\rceil$ most recent data items $\colorT$ for h.v. $\colorH(\colorT) = \colorh$ like the classic video game ``snake'' \citep{de2016complexity}.
In that game, the eponymous snake grows at its head and shrinks at its tail.
Analogously, our sequence of most recent data items adds new items at the front and has tail items overwritten.
When an invasion occurs and half of ring buffer reservations are lost, the snake's head has just arrived to the last site of the \textit{lost half of the ring buffer} and its leading $\left\lceil 2^{\colors - 1 - \colortau} \right\rceil$ sites are \textit{stretched across the lost half of the ring buffer}.
From that point, our $\left\lceil 2^{\colors - 1 - \colortau} \right\rceil$ item snake will be chased into the remaining half of the ring buffer by overwrites at its rear.
The two conditions described above ensure that writes of fresh instances of h.v. $\colorh$ (1) pull far enough ahead and (2) stay far enough ahead of invading overwrites to keep at least the snake's leading $\left\lceil 2^{\colors - 1 - \colortau} \right\rceil$ body segments intact.
Mixing metaphors, the snake drags its head and then its tail to safety in the remaining $\left\lceil 2^{\colors - 1 - \colortau} \right\rceil$ reserved ring buffer sites as the rickety bridge of re-reserved but not-yet-overwritten sites it had been occupying collapses behind it.
After escaping, the snake happily crawls in circles around its $\left\lceil 2^{\colors - 1 - \colortau} \right\rceil$ reserved sites --- at least, until invaded again.

\begin{proofpart}[Before First Overwrite]
Let $\colorT' = \min\{\colorT \in \colort + 1\}$.
Because $\colorT' \in 2^{\mathbb{N}} \forall \colort$, we have $\colorH(\colorT') = 0$ and, by the fractal property of the hanoi sequence,
\begin{align*}
\colorH(\colorT'), \colorH(\colorT'+1), \,\ldots, \colorH(2\colorT' - 1) = \colorH(0), \colorH(1), \,\ldots, \colorH(\colorT' - 1).
\end{align*}
Note that $2\colorT' - 1 = \max\{\colorT \in \colort + 1\}$.

By Lemma \ref{thm:tilted-invader-minus-invaded} for h.v. $\colorHcal_{\colort}(\colork)$ invaded by h.v. $\colorHcal_{\colort + 1}(\colork)$ ($\colorHcal_{\colort}(\colork) \neq \colorHcal_{\colort + 1}(\colork)$), we have $\colorHcal_{\colort + 1}(\colork) \geq \colorHcal_{\colort}(\colork) + 2$.
Recalling that the first instance of h.v. $\colorh$ occurs at ingest time $\colorT = 2^{\colorh} - 1$, note that $\forall\colorh$,
\begin{align*}
|\{\colorT \in 0,\,\ldots, 3 \times 2^{\colorh} : \colorH(\colorT) = \colorh\}| = 2.
\end{align*}
\end{proofpart}
Because $\colorT = 3 \times 2^{\colorh} < 2^{\colorh + 2} - 1$, we have our result.

\begin{proofpart}[Overwrite Cadence]
The cadence of h.v. $\colorh$, after its first incidence at ingest time $\colorT=2^{\colorh} - 1$ is to occur every $2^{\colorh + 1}$ data items, where $\colorT \bmod 2^{\colorh + 1} = 2^{\colorh} - 1$.
Observe also that the cadence at which a h.v. $\geq\colorh$ occurs is every $2^{\colorh}$ items, where $\colorT \bmod 2^{\colorh} = 2^{\colorh} - 1$.

Again, by Lemma \ref{thm:tilted-invader-minus-invaded} for h.v. $\colorHcal_{\colort}(\colork)$ invaded by h.v. $\colorHcal_{\colort + 1}(\colork)$ ($\colorHcal_{\colort}(\colork) \neq \colorHcal_{\colort + 1}(\colork)$), we have $\colorHcal_{\colort + 1}(\colork) \geq \colorHcal_{\colort}(\colork) + 2$.
New incidences of invaded h.v. $\colorHcal_{\colort}(\colork)$ accrue faster than the final $\left\lceil 2^{\colors - 1 - \colortau} \right\rceil$ instances of h.v. $\colorh$ from epoch $\colort$ are overwritten because
\begin{align*}
2^{\colorh + 1} \stackrel{\checkmark}{<} 2^{\colorh + 2}.
\end{align*}
\end{proofpart}

\end{proof}
