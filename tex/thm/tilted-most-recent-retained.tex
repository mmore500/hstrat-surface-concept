\begin{lemma}[Minimum recent items retained per \hv{}]
\label{thm:tilted-most-recent-retained}
At least the most recent $\left\lceil2^{\colors - 1 - \colortau}\right\rceil$ encountered instances of every \hv{} $\colorh$ are retained.
Concretely,
\begin{align*}
\textsf{goal\_tilted} \subseteq \colorB_{\colorT}.
\end{align*}
\end{lemma}
\begin{proof}
From Lemma \ref{thm:stretched-discarded-incidence-count} we have reservations available to store at least the first $2^{\colors - 1 - \colortau}$ instances of each hanoi value.
After this point, data item placement cycles around sites reserved to a \hv as a ring buffer --- keeping most recent $2^{\colors - 1 - \colortau}$ instances.
However, we must validate behavior at the transition points where this ring buffer shrinks due to invasion.

In the case of invasion, the number of reserved sites drops from $\left\lceil 2^{\colors - \colortau} \right\rceil$ to $2^{\colors - 1 - \colortau}$.
Recall from Lemma \ref{thm:tilted-last-touched} that the final instance of each \hv{} each epoch is placed into the rightmost reservation segment.
We therefore know that the final $2^{\colors - 1 - \colortau}$ instances of a \hv{} encountered during an epoch were laid out left to right in each of the smallest-size remaining segments, $r = \colortau$ (with the last instance occupying the rightmost reservation segment).

So, at the outset of epoch $\colort$, reassigned sites $\{\colork \in [0\twodots\colorS) : \colorHcal_{\colort - 1}(\colork) \neq \colorHcal_{\colort}(\colork)\}$ always contain the most recent $2^{\colors - 1 - \colortau}$ instances of \hv{} $\colorh = \colorHcal_{\colort - 1}(\colork)$, arranged left to right.
If data items in these reassigned sites were lost instantaneously at time $\min(\colorT \in \colortsetofT)$, we would not meet our proof objectives.
At that point, we would have none of the most-recent $2^{\colors - 1 - \colortau}$ \hv{} $\colorh$ data items retained.
However, data items are not lost instantaneously when a site is reassigned.
Instead, data items in reassigned sites $\colork$ linger until they are \textit{actually} overwritten by incoming data items $\colorT \in \colortsetofT$ with $\colorK(\colorT) = \colork$.

From Lemma \ref{thm:tilted-invading-overwrite-order}, we have that, over the course of an epoch, invaded data items are overwritten left to right --- except the leftmost reservation, which is overwritten last.
Ensuring retention of the most recent $2^{\colors - 1 - \colortau}$ data items for a \hv{} during invasion therefore requires two desiderata:
\begin{enumerate}
\item at least two instances of invaded \hv{} $\colorh$ occur before the first invading overwrite, and
\item the cadence of overwrites proceeds no faster than fresh instances of invaded \hv{} $\colorh$ accrue.
\end{enumerate}


\begin{mybox}
\textbf{Intuition.}
Imagine the chain of $2^{\colors - 1 - \colortau}$ recent instances of \hv{} $\colorh$ as the protagonist of the classic video game ``snake'' \citep{de2016complexity}.
In that game, the titular snake slithers by growing at its head and shrinking at its tail.
Analogously, our sequence of most recent \hv{} instances adds new items at the front and has tail items overwritten.
When an invasion occurs and half of ring buffer reservations are reassigned, the snake's body of $2^{\colors - 1 - \colortau}$ sites is stretched out across the reassigned half of the ring buffer.
In other words, our snake is laid out entirely within the \textit{danger zone}!

At that point when an invasion epoch $\colort$ begins, our snake containing $2^{\colors - 1 - \colortau}$ items will be chased into the preserved half of the ring buffer as overwrites enchroach at its rear.
The two desiderata described above ensure that the snake (1) pulls ahead and (2) stays ahead of invading overwrites to keep $2^{\colors - 1 - \colortau}$ body segments intact.
Mixing metaphors, the snake slithers head then tail to safety as the rickety bridge of reassigned but not-yet-overwritten sites it had been occupying collapses behind it.
After escaping the reassigned $2^{\colors - 1 - \colortau}$ ring buffer sites, the snake of recent \hv{} instances happily crawls in circles around its $2^{\colors - 1 - \colortau}$ reserved sites --- at least, until invaded again.
\end{mybox}

\begin{proofpart}[Two instances of invaded \hv{} before first invading overwrite]
Let $\colorT' = \min(\colorT \in \colortsetofTone)$.
The fractal properties of the hanoi sequence provides the following equivalence for hanoi values encountered during epoch $\colort + 1$:
\begin{align*}
\colorH(\colorT'), \colorH(\colorT'+1), \,\ldots, \colorH(2\colorT' - 1) = \colorH(0), \colorH(1), \,\ldots, \colorH(\colorT' - 1).
\end{align*}
Recall that $2\colorT' - 1 = \max(\colorT \in \colortsetofTone)$.

By Lemma \ref{thm:tilted-invader-minus-invaded}, for \hv{} $\colorHcal_{\colort}(\colork)$ invaded by \hv{} $\colorHcal_{\colort + 1}(\colork)$ (i.e., $\colorHcal_{\colort}(\colork) \neq \colorHcal_{\colort + 1}(\colork)$), we have $\colorHcal_{\colort + 1}(\colork) \geq \colorHcal_{\colort}(\colork) + 2$.
Hanoi value $\colorh$ occurs first at ingest time $\colorT = 2^{\colorh} - 1$ and then recurs at $\colorT = 2^{\colorh + 1} + 2^{\colorh} - 1 = 3 \times 2^{\colorh} - 1$.
Hence,
\begin{align*}
|\{\colorT \in [0 \twodots 3 \times 2^{\colorh} - 1] : \colorH(\colorT) = \colorh\}| = 2.
\end{align*}
\end{proofpart}
With $3 \times 2^{\colorh} - 1 < 2^{\colorh + 2} - 1 = \min\{\colorT : \colorH(\colorT) = \colorh + 2\}$, we have our result.

\begin{proofpart}[Overwrite cadence slower than invaded \hv{} cadence]
The cadence of a \hv{} $\colorh$, after first occuring at time $\colorT=2^{\colorh} - 1$ is to recur every $2^{\colorh + 1}$ ingests, where $\colorT \bmod 2^{\colorh + 1} = 2^{\colorh} - 1$.
Ingests with \hv{} $\colorH(\colorT) \geq \colorh$ occur twice as frequently, where $\colorT \bmod 2^{\colorh} = 2^{\colorh} - 1$.

Again, by Lemma \ref{thm:tilted-invader-minus-invaded}, for \hv{} $\colorHcal_{\colort}(\colork)$ invaded by \hv{} $\colorHcal_{\colort + 1}(\colork)$ (i.e., $\colorHcal_{\colort}(\colork) \neq \colorHcal_{\colort + 1}(\colork)$), we have $\colorHcal_{\colort + 1}(\colork) \geq \colorHcal_{\colort}(\colork) + 2$.
New incidences of invaded \hv{} $\colorh = \colorHcal_{\colort}(\colork)$ accrue faster than they are overwritten by ingests with $\colorH(\colorT) \geq \colorh + 2$ because
\begin{align*}
2^{\colorh + 1} \stackrel{\checkmark}{<} 2^{\colorh + 2}.
\end{align*}
\end{proofpart}

\end{proof}
