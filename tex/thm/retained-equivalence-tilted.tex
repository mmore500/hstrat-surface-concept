\begin{lemma}[Tilted algorithm retained data items]
\label{thm:retained-equivalence-tilted}
If the most recent $n$ data items $\colorH(\colorTbar) = \colorh$ for each \hv{} $\colorh$ are guaranteed retained, then we are guaranteed to have all
\begin{align*}
\colorTbar
&\in
\{
  2^{\colorh'}\Big(\left\lfloor \frac{\colorT + 1}{2^{\colorh'}} \right\rfloor - j'\Big) - 1
  :
  j' \in [0 \twodots 2n - 1]
  \text{ and }
  \colorh' \in \mathbb{N}
\}.
\end{align*}
Note that, although this formulation nominally includes $\colorTbar < 0$, an extension filtering $\colorTbar \in [0 \twodots \colorT]$ follows trivially.
\end{lemma}
\begin{proof}

Recall that the $j$th instance of hanoi value $\colorh$ appears at ingest time
\begin{align*}
\colorTbar
&= j2^{\colorh + 1} + 2^{\colorh} - 1,
\end{align*}
indexed from $j=0$.

The set of retained data items can be denoted
\begin{align*}
\mathsf{have\_retained} \coloneq
\{
  2^{\colorh + 1}\Big( \left\lfloor\frac{\colorT - 2^{\colorh} + 1}{2^{\colorh + 1}} \right\rfloor - j\Big) + 2^{\colorh} - 1
  :
  j \in [0 \twodots n-1]
  \text{ and }
  \colorh \in \mathbb{N}
\}.
\end{align*}

We will show $\mathsf{have\_retained}$ equivalent to,
\begin{align*}
\mathsf{want\_retained} \coloneq
\{
  2^{\colorh'}\Big(\left\lfloor \frac{\colorT + 1}{2^{\colorh'}} \right\rfloor - j'\Big) - 1
  :
  j' \in [0 \twodots 2n-1]
  \text{ and }
  \colorh' \in \mathbb{N}
\}.
\end{align*}

Sublemma \ref{thm:tilted-rsubset} shows $\mathsf{have\_retained} \subseteq \mathsf{want\_retained}$.
From Sublemma \ref{thm:tilted-subsetr}, $\mathsf{want\_retained} \subseteq \mathsf{have\_retained}$.
Hence, $\mathsf{want\_retained} = \mathsf{have\_retained}$.

\end{proof}

\begin{sublemma}[$\mathsf{have\_retained} \subseteq \mathsf{want\_retained}$]
\label{thm:tilted-rsubset}
Set $\mathsf{have\_retained}$ subsets set $\mathsf{want\_retained}$,
\begin{align*}
&\{
  2^{\colorh + 1}\Big( \left\lfloor\frac{\colorT - 2^{\colorh} + 1}{2^{\colorh + 1}} \right\rfloor - j\Big) + 2^{\colorh} - 1
  :
  j \in [0 \twodots n-1]
  \text{ and }
  \colorh \in \mathbb{N}
\}\\
&\subseteq
\{
  2^{\colorh'}\Big(\left\lfloor \frac{\colorT + 1}{2^{\colorh'}} \right\rfloor - j\Big) - 1
  :
  j' \in [0 \twodots 2n-1]
  \text{ and }
  \colorh' \in \mathbb{N}
\}.
\end{align*}
\end{sublemma}
\begin{proof}
Suppose $\colorTbar \in \mathsf{have\_retained}$.
Then $\exists j \in [0 \twodots n-1]$ and $\colorh \in \mathbb{N}$ such that
\begin{align*}
\colorTbar
&= 2^{\colorh \eqnmarkbox[yellow]{}{+ 1}}\Big(\left\lfloor \frac{\colorT - 2^{\colorh} + 1}{2^{\colorh + 1}} \right\rfloor - j \Big) \eqnmarkbox[orange]{}{+ 2^{\colorh}} - 1\\
% &= 2^{\colorh + 1}\left\lfloor \frac{\colorT - 2^{\colorh}}{2^{\colorh + 1}}\right\rfloor - 2^{\colorh}(2j) + 2^{\colorh} - 1\\
&= 2^{\colorh} \Big(\eqnmarkbox[yellow]{}{2}\left\lfloor \frac{\colorT - 2^{\colorh} + 1}{2^{\colorh + 1}} \right\rfloor  - \eqnmarkbox[yellow]{}{2}j \eqnmarkbox[orange]{}{+ 1} \Big) - 1.
\end{align*}
Denoting $\epsilon \in \{0, 1\}$ as a continuity correction for the integer floor,
\begin{align*}
\colorTbar
&= 2^{\colorh} \Big( \left\lfloor \eqnmarkbox[YellowGreen]{}{2} \frac{\colorT \eqnmarkbox[Orchid]{}{- 2^{\colorh}} + 1}{2^{\colorh+1}} \right\rfloor \eqnmarkbox[YellowGreen]{}{- \epsilon}  - 2j + 1 \Big) - 1\\
&= 2^{\colorh} \Big( \left\lfloor \frac{\colorT + 1}{2^{\colorh}} \right\rfloor \eqnmarkbox[Orchid]{}{- 1} - \epsilon - 2j + 1 \Big) - 1\\
&= 2^{\colorh} \Big(\left\lfloor \frac{\colorT + 1}{2^{\colorh}}\right\rfloor - (2j + \epsilon) \Big) - 1.
\end{align*}
Note that $(2j + \epsilon) \in [0 \twodots 2n-1]$ for $j \in [0 \twodots n-1]$, giving $\mathsf{have\_retained} \stackrel{\checkmark}{\subseteq} \mathsf{want\_retained}$.
\end{proof}

\begin{sublemma}[$\mathsf{want\_retained} \subseteq \mathsf{have\_retained}$]
\label{thm:tilted-subsetr}
Set $\mathsf{want\_retained}$ subsets $\mathsf{have\_retained}$,
\begin{align*}
&\{
  2^{\colorh'}\Big(\left\lfloor \frac{\colorT + 1}{2^{\colorh'}} \right\rfloor - j'\Big) - 1
  :
  j' \in [0 \twodots 2n-1]
  \text{ and }
  \colorh' \in \mathbb{N}
\}\\
&\subseteq
\{
  2^{\colorh + 1}\Big( \left\lfloor\frac{\colorT - 2^{\colorh} + 1}{2^{\colorh + 1}} \right\rfloor - j\Big) + 2^{\colorh} - 1
  :
  j \in [0 \twodots n-1]
  \text{ and }
  \colorh \in \mathbb{N}
\}.
\end{align*}

\end{sublemma}
\begin{proof}
Suppose $\colorTbar \in \mathsf{want\_retained}$.
Then $\exists j' \in [0 \twodots 2n - 1]$ and $\colorh' \in \mathbb{N}$ such that
\begin{align*}
\colorTbar
&= 2^{\colorh'}\Big(\left\lfloor \frac{\colorT + 1}{2^{\colorh'}} \right\rfloor - j'\Big) - 1.
\end{align*}

Begin by calculating how many factors of two divide $\colorTbar + 1$, $\colorH(\colorTbar)$.
Note that we have $\colorH(\colorTbar) \geq \colorh'$ because $2^{\colorh'}$ divides $\colorTbar + 1 = 2^{\colorh'}(\left\lfloor \frac{\colorT + 1}{2^{\colorh'}} \right\rfloor - j')$.
With this fact in hand, we may rearrange our formula for $\colorTbar$,
\begin{align}
\colorTbar
&= 2^{\colorh'}\Big(\left\lfloor \frac{\colorT + 1}{2^{\colorh'}} \right\rfloor - j'\Big) - 1 \nonumber\\
&= 2^{\eqnmarkbox[yellow]{}{\colorH(\colorTbar)}} \Big(
\frac{\left\lfloor \frac{\colorT + 1}{2^{\colorh'}} \right\rfloor - j'}{2^{\eqnmarkbox[yellow]{}{\colorH(\colorTbar) - \colorh'}}}
\Big)
- 1 \nonumber\\
&= 2^{\colorH(\colorTbar)} \Big(
\frac{\left\lfloor \frac{\colorT + 1}{2^{\colorh'}} \right\rfloor \eqnmarkbox[orange]{}{- 2^{\colorH(\colorTbar) - \colorh'}}}{2^{\colorH(\colorTbar) - \colorh'}}
\eqnmarkbox[orange]{}{+ 1}
- \frac{j'}{2^{\colorH(\colorTbar) - \colorh'}}
\Big)
- 1 \nonumber\\
&= 2^{\colorH(\colorTbar)} \Big(
\frac{\left\lfloor \frac{\colorT + 1}{2^{\colorh'}} \eqnmarkbox[orange]{}{- 2^{\colorH(\colorTbar) - \colorh'}} \right\rfloor}{2^{\colorH(\colorTbar) - \colorh'}}
- \frac{j'}{2^{\colorH(\colorTbar) - \colorh'}}
\Big)
\eqnmarkbox[orange]{}{+ 2^{\colorH(\colorTbar)}}
- 1 \nonumber\\
&= 2^{\colorH(\colorTbar) \eqnmarkbox[Orchid]{}{+ 1}} \Big(
\frac{\left\lfloor \frac{\colorT + 1}{2^{\colorh'}} \eqnmarkbox[YellowGreen]{}{- 2^{\colorH(\colorTbar) - \colorh'}} \right\rfloor}{2^{\colorH(\colorTbar) - \colorh' \eqnmarkbox[Orchid]{}{+ 1}}}
-
\frac{j'}{2^{\colorH(\colorTbar) - \colorh' \eqnmarkbox[Orchid]{}{+ 1}}}
\Big)
+ 2^{\colorH(\colorTbar)}
- 1 \nonumber\\
&= 2^{\colorH(\colorTbar) + 1} \Big(
\frac{\left\lfloor \frac{\colorT \eqnmarkbox[YellowGreen]{}{- 2^{\colorH(\colorTbar)}} + 1}{2^{\eqnmarkbox[SkyBlue]{}{\colorh'}}} \right\rfloor}{2^{\colorH(\colorTbar) \eqnmarkbox[SkyBlue]{}{-\colorh'} + 1}}
- \frac{j'}{2^{\colorH(\colorTbar) - \colorh' + 1}}
\Big)
+ 2^{\colorH(\colorTbar)}
- 1.
\nonumber
\end{align}

Letting $\epsilon \in [0, 1/2)$ denote a continuity correction factor for the integer floor,
\begin{align}
\colorTbar
&= 2^{\colorH(\colorTbar) + 1} \Big(
\left\lfloor
\frac{\colorT - 2^{\colorH(\colorTbar)} + 1}{2^{\colorH(\colorTbar) + 1}}
\right\rfloor
\eqnmarkbox[SkyBlue]{}{+ \epsilon}
-
\frac{j'}{2^{\colorH(\colorTbar) - \colorh' + 1}}
\Big)
+ 2^{\colorH(\colorTbar)}
- 1 \nonumber \\
&= 2^{\colorH(\colorTbar) + 1} \Big(
\left\lfloor
\frac{\colorT - 2^{\colorH(\colorTbar)} + 1}{2^{\colorH(\colorTbar) + 1}}
\right\rfloor
-
\eqnmarkbox[WildStrawberry]{proofgoal}{
  \Big(
  \frac{j'}{2^{\colorH(\colorTbar) - \colorh' + 1}}
  - \epsilon
  \Big)
}
\Big)
+ 2^{\colorH(\colorTbar)}
- 1.
\label{eqn:needtoshowj}\\
\nonumber
\end{align}
\annotate[yshift=0em]{below,right}{proofgoal}{need to show $\in [0 \twodots n-1]$}

By definition, $2^{\colorH(\colorTbar)}$ divides $\colorTbar + 1$ and the quotient $(\colorTbar + 1) / 2^{\colorH(\colorTbar)}$ is an odd, positive integer.
So, $(\eqnmarkbox[yellow]{}{\colorTbar} + 1) / 2^{\colorH(\colorTbar)} - 1$ is an even, non-negative integer.
Applying this observation to our expression for $\colorTbar$ from Equation \ref{eqn:needtoshowj},
\begin{align*}
  \frac{
  \eqnmarkbox[yellow]{}{
  2^{\colorboxed{orange}{\colorH(\colorTbar)} + 1} \Big(
    \left\lfloor
    \frac{\colorT - 2^{\colorH(\colorTbar)} + 1}{2^{\colorH(\colorTbar) + 1}}
    \right\rfloor
    -
    \Big(
    \frac{j'}{2^{\colorH(\colorTbar) - \colorh' + 1}}
    - \epsilon
    \Big)
    \Big)
    \colorboxed{YellowGreen}{+ 2^{\colorH(\colorTbar)}}
    \colorboxed{Orchid}{- 1}
  }
  \colorboxed{Orchid}{+ 1}
  }{2^{\colorboxed{orange}{\colorH(\colorTbar)}}}
  \colorboxed{YellowGreen}{- 1}
  &= 2 \Big(
    \left\lfloor
    \frac{\colorT - 2^{\colorH(\colorTbar)} + 1}{2^{\colorH(\colorTbar) + 1}}
    \right\rfloor
    -
    \Big(
    \frac{j'}{2^{\colorH(\colorTbar) - \colorh' + 1}}
    - \epsilon
    \Big)
    \Big)\\
    &\in [0, 2, 4,\;\; \ldots].
\end{align*}

% \frac{\left\lfloor \frac{\colorT + 1}{2^{\colorh'}} \right\rfloor - j'}{2^{\colorH(\colorTbar) - \colorh'}} \in [1, 3, 5, \;\; \ldots],
% \end{align*}
Dividing by 2,
\begin{align*}
  \left\lfloor
  \frac{\colorT - 2^{\colorH(\colorTbar)} + 1}{2^{\colorH(\colorTbar) + 1}}
  \right\rfloor
  -
  \Big(
  \frac{j'}{2^{\colorH(\colorTbar) - \colorh' + 1}}
  - \epsilon
  \Big)
  &\in \mathbb{N}.
\end{align*}

Because
\begin{align*}
\left\lfloor
\frac{
  \colorT - 2^{\colorH(\colorTbar)} + 1
}{
  2^{\colorH(\colorTbar) + 1}
}
\right\rfloor
- \Big(
\frac{j'}{2^{\colorH(\colorTbar) - \colorh' + 1}}
- \epsilon
\Big)
\in \mathbb{Z},
\end{align*}
we necessarily have
\begin{align*}
\frac{j'}{2^{\colorH(\colorTbar) - \colorh' + 1}}
- \epsilon
\in \mathbb{Z}.
\end{align*}
Further, because $\epsilon \in [0, 1/2)$, $j' \in [0 \twodots 2n-1]$, and $\colorH(\colorTbar) - \colorh'  + 1 \geq 1$,
\begin{align*}
\frac{j'}{2^{\colorH(\colorTbar) - \colorh' + 1}}
- \epsilon
\stackrel{\checkmark}{\in}
[0 \twodots n - 1].
\end{align*}
With $\colorh = \colorH(\colorTbar) \stackrel{\checkmark}{\in} \mathbb{N}$, we have the result: $\mathsf{want\_retained} \stackrel{\checkmark}{\subseteq} \mathsf{have\_retained}$.
\end{proof}
