\begin{lemma}[Retained h.v. Equivalence, Tilted]
\label{thm:retained-equivalence-tilted}
If the first $n$ data items $\colorTbar$ of each h.v. $\colorh$ are guaranteed retained, then we are guaranteed
\begin{align*}
\colorTbar
&\in
\{
  2^{\colorh'}\Big(\left\lfloor \colorT / 2^{\colorh'} \right\rfloor - j\Big) - 1
  :
  j' \in 0,1,\,\ldots,2n-1
  \text{ and }
  \colorh' \in \mathbb{N}
\}.
\end{align*}
\end{lemma}
\begin{proof}

Recall that the $j$th instance of hanoi value $\colorh$ appears at ingest time
\begin{align*}
\colorTbar
&= j2^{\colorh + 1} + 2^{\colorh} - 1,
\end{align*}
indexed from $j=0$.

The set of retained data items can be denoted
\begin{align*}
\mathfrak{R} =
\{
  2^{\colorh + 1}\Big( \left\lfloor\frac{\colorT - 2^{\colorh}}{2^{\colorh + 1}} \right\rfloor - j\Big) + 2^{\colorh} - 1
  :
  j \in 0,1,\,\ldots,n-1
  \text{ and }
  \colorh \in \mathbb{N}
\}.
\end{align*}

We will show $\mathfrak{R}$ equivalent to,
\begin{align*}
\mathfrak{R}' =
\{
  2^{\colorh'}\Big(\left\lfloor \colorT / 2^{\colorh'} \right\rfloor - j\Big) - 1
  :
  j' \in 0,1,\,\ldots,2n-1
  \text{ and }
  \colorh' \in \mathbb{N}
\}.
\end{align*}

Sublemma \ref{thm:tilted-rsubset} shows $\mathfrak{R} \subseteq \mathfrak{R}`$.
From Sublemma \ref{thm:tilted-subsetr}, $\mathfrak{R}' \subseteq \mathfrak{R}$.
Hence, $\mathfrak{R}' = \mathfrak{R}$.

\end{proof}

\begin{sublemma}[$\mathfrak{R} \subseteq \mathfrak{R}`$]
\label{thm:tilted-rsubset}
Sets $\mathfrak{R} \subseteq \mathfrak{R}`$,
\begin{align*}
&\{
  2^{\colorh + 1}\Big( \left\lfloor\frac{\colorT - 2^{\colorh}}{2^{\colorh + 1}} \right\rfloor - j\Big) + 2^{\colorh} - 1
  :
  j \in 0,1,\,\ldots,n-1
  \text{ and }
  \colorh \in \mathbb{N}
\}\\
&\subseteq
\{
  2^{\colorh'}\Big(\left\lfloor \colorT / 2^{\colorh'} \right\rfloor - j\Big) - 1
  :
  j' \in 0,1,\,\ldots,2n-1
  \text{ and }
  \colorh' \in \mathbb{N}
\}.
\end{align*}
\end{sublemma}
\begin{proof}
Suppose $\colorTbar \in \mathfrak{R}$.
Then $\exists j,  0 \leq j < n$ and $\colorh \in \mathbb{N}$ such that
\begin{align*}
\colorTbar
&= 2^{\colorh + 1}\Big(\left\lfloor \frac{\colorT - 2^{\colorh}}{2^{\colorh + 1}} \right\rfloor - j \Big) + 2^{\colorh} - 1\\
% &= 2^{\colorh + 1}\left\lfloor \frac{\colorT - 2^{\colorh}}{2^{\colorh + 1}}\right\rfloor - 2^{\colorh}(2j) + 2^{\colorh} - 1\\
&= 2^{\colorh} \Big(2\left\lfloor \frac{\colorT - 2^{\colorh}}{2^{\colorh + 1}} \right\rfloor  - 2j + 1 \Big) - 1.
\end{align*}
Denoting $\epsilon \in \{0, 1\}$ as a continuity correction for the integer floor,
\begin{align*}
\colorTbar
&= 2^{\colorh} \Big( \left\lfloor \frac{\colorT - 2^{\colorh}}{2^{\colorh}} \right\rfloor - \epsilon  - 2j + 1 \Big) - 1\\
&= 2^{\colorh} \Big( \left\lfloor \frac{\colorT}{2^{\colorh}} \right\rfloor - 1 - \epsilon - 2j + 1 \Big) - 1\\
&= 2^{\colorh} \Big(\left\lfloor \frac{\colorT}{2^{\colorh}}\right\rfloor - (2j + \epsilon) \Big) - 1.
\end{align*}
Note that $(2j + \epsilon) \in 0,1,\,\ldots,2n-1$ for $j \in 0,1,\,\ldots,n-1$ giving $\mathfrak{R} \subseteq \mathfrak{R}`$.
\end{proof}

\begin{sublemma}[$\mathfrak{R}' \subseteq \mathfrak{R}$]
\label{thm:tilted-subsetr}
Sets $\mathfrak{R}' \subseteq \mathfrak{R}$,
\begin{align*}
&\{
  2^{\colorh'}\Big(\left\lfloor \colorT / 2^{\colorh'} \right\rfloor - j\Big) - 1
  :
  j' \in 0,1,\,\ldots,2n-1
  \text{ and }
  \colorh' \in \mathbb{N}
\}\\
&\subseteq
\{
  2^{\colorh + 1}\Big( \left\lfloor\frac{\colorT - 2^{\colorh}}{2^{\colorh + 1}} \right\rfloor - j\Big) + 2^{\colorh} - 1
  :
  j \in 0,1,\,\ldots,n-1
  \text{ and }
  \colorh \in \mathbb{N}
\}.
\end{align*}

\end{sublemma}
\begin{proof}
Suppose $\colorTbar \in \mathfrak{R}'$.
Then $\exists j',  0 \leq j' \leq 2n - 1$ and $\colorh \in \mathbb{N}$ such that
\begin{align*}
\colorTbar
&= 2^{\colorh}\Big(\left\lfloor \colorT / 2^{\colorh} \right\rfloor - j'\Big) - 1.
\end{align*}

Begin by calculating the number of factors of two dividing $\colorTbar + 1$, $\colorH(\colorTbar)$.
Take $\colorH(\colorTbar) \geq \colorh$ (TODO prove this),
\begin{align*}
\colorTbar
&= 2^{\colorh}\Big(\left\lfloor \colorT / 2^{\colorh} \right\rfloor - j'\Big) - 1\\
&= 2^{\colorH(\colorTbar)} \Big(
\frac{\left\lfloor \colorT / 2^{\colorh} \right\rfloor - j'}{2^{\colorH(\colorTbar) - \colorh}}
\Big)
- 1\\
&= 2^{\colorH(\colorTbar)} \Big(
\frac{\left\lfloor \colorT / 2^{\colorh} \right\rfloor - 2^{\colorH(\colorTbar) - \colorh}}{2^{\colorH(\colorTbar) - \colorh}}
+ 1
- \frac{j'}{2^{\colorH(\colorTbar) - \colorh}}
\Big)
- 1\\
&= 2^{\colorH(\colorTbar)} \Big(
\frac{\left\lfloor \colorT / 2^{\colorh} - 2^{\colorH(\colorTbar) - \colorh} \right\rfloor}{2^{\colorH(\colorTbar) - \colorh}}
- \frac{j'}{2^{\colorH(\colorTbar) - \colorh}}
\Big)
+ 2^{\colorH(\colorTbar)}
- 1\\
&= 2^{\colorH(\colorTbar) + 1} \Big(
\frac{\left\lfloor \colorT / 2^{\colorh} - 2^{\colorH(\colorTbar) - \colorh} \right\rfloor}{2^{\colorH(\colorTbar) - \colorh + 1}}
- \frac{j'}{2^{\colorH(\colorTbar) - \colorh + 1}}
\Big)
+ 2^{\colorH(\colorTbar)}
- 1.
\end{align*}

By definition $2^{\colorH(\colorTbar)}$ divides $\colorTbar + 1$ and the quoient $\colorTbar / 2^{\colorH(\colorTbar)}$ is an odd, positive integer.
So, $\colorTbar / 2^{\colorH(\colorTbar)} - 1$ is an even, non-negative integer.
Because
\begin{align*}
\frac{\left\lfloor \colorT / 2^{\colorh} \right\rfloor - j'}{2^{\colorH(\colorTbar) - \colorh}} \in 1, 3, 5, \, \ldots
\end{align*}
we have
\begin{align*}
\frac{\left\lfloor \colorT / 2^{\colorh} - 2^{\colorH(\colorTbar) - \colorh} \right\rfloor}{2^{\colorH(\colorTbar) - \colorh}}
- \frac{j'}{2^{\colorH(\colorTbar) - \colorh}} \in 0, 2, 4, \, \ldots \, .
\end{align*}
Thus,
\begin{align*}
\frac{
  \left\lfloor
  \frac{\colorT - 2^{\colorH(\colorTbar)}}{2^{\colorh}}
  \right\rfloor
}{
  2^{\colorH(\colorTbar) - \colorh + 1}
}
- \frac{j'}{
  2^{\colorH(\colorTbar) - \colorh + 1}
}
\in \mathbb{Z}_{\geq0}.
\end{align*}

Letting $0 \leq \epsilon < 0$ denote a continuity correction factor,
\begin{align*}
\colorTbar
&= 2^{\colorH(\colorTbar) + 1}
\Big(
\left\lfloor
\frac{
  \colorT - 2^{\colorH(\colorTbar)}
}{
  2^{\colorH(\colorTbar) + 1}
}
\right\rfloor
+ \epsilon
- \frac{j'}{2^{\colorH(\colorTbar) - \colorh + 1}}
\Big)
+ 2^{\colorH(\colorTbar)}
- 1\\
&= 2^{\colorH(\colorTbar) + 1}
\Big(
\left\lfloor
\frac{
  \colorT - 2^{\colorH(\colorTbar)}
}{
  2^{\colorH(\colorTbar) + 1}
}
\right\rfloor
- (
  \frac{j'}{2^{\colorH(\colorTbar) - \colorh + 1}}
  + \epsilon
)
\Big)
+ 2^{\colorH(\colorTbar)}
- 1\\
\end{align*}

Because
\begin{align*}
\left\lfloor
\frac{
  \colorT - 2^{\colorH(\colorTbar)}
}{
  2^{\colorH(\colorTbar) + 1}
}
\right\rfloor
- (
\frac{j'}{2^{\colorH(\colorTbar) - \colorh + 1}}
+ \epsilon
)
\in \mathbb{Z}
\end{align*}
we have
\begin{align*}
\frac{j'}{2^{\colorH(\colorTbar) - \colorh + 1}}
- \epsilon
\in \mathbb{Z}.
\end{align*}
Further, because $0 \leq \epsilon < 1$ and $\colorH(\colorTbar) - \colorh  + 1 \geq 1$,
\begin{align*}
\frac{j'}{2^{\colorH(\colorTbar) - \colorh + 1}}
- \epsilon
\in
0, 1, \,\ldots, n - 1,
\end{align*}
giving us the result.
\end{proof}
