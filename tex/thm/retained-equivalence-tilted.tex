\begin{lemma}[Tilted Algorithm Retained Ingest Times]
\label{thm:retained-equivalence-tilted}
If the most recent $n$ data items $\colorH(\colorTbar) = \colorh$ for each h.v. $\colorh$ are guaranteed retained, then we are guaranteed to have all
\begin{align*}
\colorTbar
&\in
\{
  2^{\colorh'}\Big(\left\lfloor (\colorT + 1) / 2^{\colorh'} \right\rfloor - j'\Big) - 1
  :
  j' \in 0,1,\,\ldots,2n-1
  \text{ and }
  \colorh' \in \mathbb{N}
\}.
\end{align*}
Note that, although this formulation nominally includes $\colorTbar < 0$, an extension filtering $\colorTbar \in \{0,1,\,\ldots,\colorT\}$ follows trivially.
\end{lemma}
\begin{proof}

Recall that the $j$th instance of hanoi value $\colorh$ appears at ingest time
\begin{align*}
\colorTbar
&= j2^{\colorh + 1} + 2^{\colorh} - 1,
\end{align*}
indexed from $j=0$.

The set of retained data items can be denoted
\begin{align*}
\mathsf{have\_retained} =
\{
  2^{\colorh + 1}\Big( \left\lfloor\frac{\colorT - 2^{\colorh} + 1}{2^{\colorh + 1}} \right\rfloor - j\Big) + 2^{\colorh} - 1
  :
  j \in 0,1,\,\ldots,n-1
  \text{ and }
  \colorh \in \mathbb{N}
\}.
\end{align*}

We will show $\mathsf{have\_retained}$ equivalent to,
\begin{align*}
\mathsf{want\_retained} =
\{
  2^{\colorh'}\Big(\left\lfloor (\colorT + 1)/ 2^{\colorh'} \right\rfloor - j'\Big) - 1
  :
  j' \in 0,1,\,\ldots,2n-1
  \text{ and }
  \colorh' \in \mathbb{N}
\}.
\end{align*}

Sublemma \ref{thm:tilted-rsubset} shows $\mathsf{have\_retained} \subseteq \mathsf{want\_retained}$.
From Sublemma \ref{thm:tilted-subsetr}, $\mathsf{want\_retained} \subseteq \mathsf{have\_retained}$.
Hence, $\mathsf{want\_retained} = \mathsf{have\_retained}$.

\end{proof}

\begin{sublemma}[$\mathsf{have\_retained} \subseteq \mathsf{want\_retained}$]
\label{thm:tilted-rsubset}
Set $\mathsf{have\_retained}$ subsets set $\mathsf{want\_retained}$,
\begin{align*}
&\{
  2^{\colorh + 1}\Big( \left\lfloor\frac{\colorT - 2^{\colorh} + 1}{2^{\colorh + 1}} \right\rfloor - j\Big) + 2^{\colorh} - 1
  :
  j \in 0,1,\,\ldots,n-1
  \text{ and }
  \colorh \in \mathbb{N}
\}\\
&\subseteq
\{
  2^{\colorh'}\Big(\left\lfloor (\colorT + 1) / 2^{\colorh'} \right\rfloor - j\Big) - 1
  :
  j' \in 0,1,\,\ldots,2n-1
  \text{ and }
  \colorh' \in \mathbb{N}
\}.
\end{align*}
\end{sublemma}
\begin{proof}
Suppose $\colorTbar \in \mathsf{have\_retained}$.
Then $\exists j \in \{0,1,\,\ldots,n-1\}$ and $\colorh \in \mathbb{N}$ such that
\begin{align*}
\colorTbar
&= 2^{\colorh + 1}\Big(\left\lfloor \frac{\colorT - 2^{\colorh} + 1}{2^{\colorh + 1}} \right\rfloor - j \Big) + 2^{\colorh} - 1\\
% &= 2^{\colorh + 1}\left\lfloor \frac{\colorT - 2^{\colorh}}{2^{\colorh + 1}}\right\rfloor - 2^{\colorh}(2j) + 2^{\colorh} - 1\\
&= 2^{\colorh} \Big(2\left\lfloor \frac{\colorT - 2^{\colorh} + 1}{2^{\colorh + 1}} \right\rfloor  - 2j + 1 \Big) - 1.
\end{align*}
Denoting $\epsilon \in \{0, 1\}$ as a continuity correction for the integer floor,
\begin{align*}
\colorTbar
&= 2^{\colorh} \Big( \left\lfloor \frac{\colorT - 2^{\colorh} + 1}{2^{\colorh}} \right\rfloor - \epsilon  - 2j + 1 \Big) - 1\\
&= 2^{\colorh} \Big( \left\lfloor \frac{\colorT + 1}{2^{\colorh}} \right\rfloor - 1 - \epsilon - 2j + 1 \Big) - 1\\
&= 2^{\colorh} \Big(\left\lfloor \frac{\colorT + 1}{2^{\colorh}}\right\rfloor - (2j + \epsilon) \Big) - 1.
\end{align*}
Note that $(2j + \epsilon) \in \{0,1,\,\ldots,2n-1\}$ for $j \in \{0,1,\,\ldots,n-1\}$ giving $\mathsf{have\_retained} \stackrel{\checkmark}{\subseteq} \mathsf{want\_retained}$.
\end{proof}

\begin{sublemma}[$\mathsf{want\_retained} \subseteq \mathsf{have\_retained}$]
\label{thm:tilted-subsetr}
Set $\mathsf{want\_retained}$ subsets $\mathsf{have\_retained}$,
\begin{align*}
&\{
  2^{\colorh'}\Big(\left\lfloor (\colorT + 1) / 2^{\colorh'} \right\rfloor - j'\Big) - 1
  :
  j' \in 0,1,\,\ldots,2n-1
  \text{ and }
  \colorh' \in \mathbb{N}
\}\\
&\subseteq
\{
  2^{\colorh + 1}\Big( \left\lfloor\frac{\colorT - 2^{\colorh} + 1}{2^{\colorh + 1}} \right\rfloor - j\Big) + 2^{\colorh} - 1
  :
  j \in 0,1,\,\ldots,n-1
  \text{ and }
  \colorh \in \mathbb{N}
\}.
\end{align*}

\end{sublemma}
\begin{proof}
Suppose $\colorTbar \in \mathsf{want\_retained}$.
Then $\exists j' \in \{0,1,\,\ldots,2n - 1\}$ and $\colorh \in \mathbb{N}$ such that
\begin{align*}
\colorTbar
&= 2^{\colorh}\Big(\left\lfloor (\colorT + 1) / 2^{\colorh} \right\rfloor - j'\Big) - 1.
\end{align*}

Begin by calculating how many factors of two divide $\colorTbar + 1$, $\colorH(\colorTbar)$.
Note that we have $\colorH(\colorTbar) \geq \colorh'$ because $2^{\colorh'}$ divides $\colorTbar + 1 = 2^{\colorh'}(\left\lfloor (\colorT + 1) / 2^{\colorh'} \right\rfloor - j')$.
This fact in hand, we may rearrange our formula for $\colorTbar$,
\begin{align*}
\colorTbar
&= 2^{\colorh'}\Big(\left\lfloor (\colorT + 1) / 2^{\colorh'} \right\rfloor - j'\Big) - 1\\
&= 2^{\colorH(\colorTbar)} \Big(
\frac{\left\lfloor (\colorT + 1) / 2^{\colorh'} \right\rfloor - j'}{2^{\colorH(\colorTbar) - \colorh'}}
\Big)
- 1\\
&= 2^{\colorH(\colorTbar)} \Big(
\frac{\left\lfloor (\colorT + 1) / 2^{\colorh'} \right\rfloor - 2^{\colorH(\colorTbar) - \colorh'}}{2^{\colorH(\colorTbar) - \colorh'}}
+ 1
- \frac{j'}{2^{\colorH(\colorTbar) - \colorh'}}
\Big)
- 1\\
&= 2^{\colorH(\colorTbar)} \Big(
\frac{\left\lfloor (\colorT + 1) / 2^{\colorh'} - 2^{\colorH(\colorTbar) - \colorh'} \right\rfloor}{2^{\colorH(\colorTbar) - \colorh'}}
- \frac{j'}{2^{\colorH(\colorTbar) - \colorh'}}
\Big)
+ 2^{\colorH(\colorTbar)}
- 1\\
&= 2^{\colorH(\colorTbar) + 1} \Big(
\frac{\left\lfloor (\colorT + 1)/ 2^{\colorh'} - 2^{\colorH(\colorTbar) - \colorh'} \right\rfloor}{2^{\colorH(\colorTbar) - \colorh' + 1}}
-
\eqnmarkbox[WildStrawberry]{proofgoal}{
\frac{j'}{2^{\colorH(\colorTbar) - \colorh' + 1}}
}
\Big)
+ 2^{\colorH(\colorTbar)}
- 1.
\\
\end{align*}
\annotate[yshift=0em]{below,right}{proofgoal}{need to show $\in \{0,1,\,\ldots,n-1\}$}

By definition, $2^{\colorH(\colorTbar)}$ divides $\colorTbar + 1$ and the quotient $(\colorTbar + 1) / 2^{\colorH(\colorTbar)}$ is an odd, positive integer.
So, $(\colorTbar + 1) / 2^{\colorH(\colorTbar)} - 1$ is an even, non-negative integer.
Because
\begin{align*}
\frac{\left\lfloor (\colorT + 1) / 2^{\colorh'} \right\rfloor - j'}{2^{\colorH(\colorTbar) - \colorh'}} \in \{1, 3, 5, \, \ldots\},
\end{align*}
we have
\begin{align*}
\frac{\left\lfloor (\colorT + 1) / 2^{\colorh'} - 2^{\colorH(\colorTbar) - \colorh'} \right\rfloor}{2^{\colorH(\colorTbar) - \colorh'}}
- \frac{j'}{2^{\colorH(\colorTbar) - \colorh'}} \in \{0, 2, 4, \, \ldots\}.
\end{align*}
Thus, dividing by 2,
\begin{align*}
\frac{
  \left\lfloor
  \frac{\colorT + 1 - 2^{\colorH(\colorTbar)}}{2^{\colorh'}}
  \right\rfloor
}{
  2^{\colorH(\colorTbar) - \colorh' + 1}
}
- \frac{j'}{
  2^{\colorH(\colorTbar) - \colorh' + 1}
}
\in \{0,1,2,\,\ldots\}.
\end{align*}

Letting $\epsilon \in \{0, 1\}$ denote a continuity correction factor,
\begin{align*}
\colorTbar
&= 2^{\colorH(\colorTbar) + 1}
\Big(
\left\lfloor
\frac{
  \colorT - 2^{\colorH(\colorTbar)} + 1
}{
  2^{\colorH(\colorTbar) + 1}
}
\right\rfloor
+ \epsilon
- \frac{j'}{2^{\colorH(\colorTbar) - \colorh' + 1}}
\Big)
+ 2^{\colorH(\colorTbar)}
- 1\\
&= 2^{\colorH(\colorTbar) + 1}
\Big(
\left\lfloor
\frac{
  \colorT - 2^{\colorH(\colorTbar)} + 1
}{
  2^{\colorH(\colorTbar) + 1}
}
\right\rfloor
- (
  \frac{j'}{2^{\colorH(\colorTbar) - \colorh' + 1}}
  - \epsilon
)
\Big)
+ 2^{\colorH(\colorTbar)}
- 1\\
\end{align*}

Because
\begin{align*}
\left\lfloor
\frac{
  \colorT - 2^{\colorH(\colorTbar)} + 1
}{
  2^{\colorH(\colorTbar) + 1}
}
\right\rfloor
- \Big(
\frac{j'}{2^{\colorH(\colorTbar) - \colorh' + 1}}
- \epsilon
\Big)
\in \mathbb{Z},
\end{align*}
we necessarily have
\begin{align*}
\frac{j'}{2^{\colorH(\colorTbar) - \colorh' + 1}}
- \epsilon
\in \mathbb{Z}.
\end{align*}
Further, because $0 \leq \epsilon < 1$ and $\colorH(\colorTbar) - \colorh'  + 1 \geq 1$,
\begin{align*}
\frac{j'}{2^{\colorH(\colorTbar) - \colorh' + 1}}
- \epsilon
\in
\{0, 1, \,\ldots, n - 1\},
\end{align*}
giving us the result, $\mathsf{want\_retained} \stackrel{\checkmark}{\subseteq} \mathsf{have\_retained}$.
\end{proof}
