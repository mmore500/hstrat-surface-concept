\begin{lemma}[Meta-epochs $\colortau$ correspond to segment subsumption cycles]
\label{thm:stretched-meta-epoch}

The timing of meta-epoch $\colortau$, defined in Section \ref{sec:notation-metaepoch} as lasting $2^{\colortau} - 1$ epochs for $\colortau\geq1$, corresponds to the time window during which the reservation segments initialized with size $r=\colortau$ are removed through ``invasion.''
\end{lemma}

\begin{proof}

Recall that under the stretched algorithm's proposed layout strategy, buffer space is filled without any overwrites during epoch 0.
Then, during subsequent epochs, half of segments (designated ``invading'' segments) grow by addition of new high-\hv{} sites.
The other half of reservation segments are subsumed one site at a time, successively losing low-\hv{} sites to their invading neighbors.
Note that ``invaded'' segments are not allowed to add high-\hv{} sites --- during the invasion process, they are frozen while being eliminated.

By specification, ``invaded'' segments are always those of smallest remaining size.
Owing to the recursively nested structure of segment layout, smallest-remaining segments are always interspersed every second and always constitute half of active segments.

Because invading segments grow by exactly one buffer site per epoch, the number of epochs $\colort$ it takes for a reservation segment to be invaded to elimination corresponds exactly to the invaded segment's reservation size at invasion outset.
Our proof objective can thus be recast as determining the maximal ``mature''' size $R(r)$ reached by segments initialized size $r$ at epoch $\colort=0$ before frozen for elimination.

Recall from Section \ref{sec:notation-metaepoch} that the duration of meta-epoch $\tau$, $|\colort \in \colortausetoft|$, is $2^{\colortau} - 1$.
For reservation segments with $r=1$ (which are invaded in epoch $\colort=1$ and meta-epoch $\colortau = 1$), our goal is therefore to show $|\colort \in \colortausetoft| = 2^{\colortau} - 1$ matches $R(r)$ by showing $R(r) = 2^{r} - 1$.
As already mentioned, initialized-singleton $r=1$ segments are always invaded first, in epoch $\colort=1$.
Trivially, these segments also have $R(1) = 1$. on account of never having the opportunity to act as an invader.
Segments initialized at size $r=2$ are invaded next.
These segments acted as invader during epoch $\colort=1$, and so grew to size $R(2) = 3$.
Note that $R(1) \stackrel{\checkmark}{=} 2^1 - 1$ and $R(2) \stackrel{\checkmark}{=} 2^2 - 1$.

Subsequent segments $r>2$ grow exponentially --- having invaded segments that themselves already grew by invasion.
For instance, segments $r=3$ begin by invading their singleton neighbors $r=1$ during epoch $\colort=1$.
Then $r=3$ segments invade segments that began as $r=2$.
Thus, for $r=3$,
\begin{align*}
R(3)
&= 3 + R(1) + R(2)\\
&= 3 + 1 + 2 + 1\\
&\stackrel{\checkmark}{=} 2^3 - 1.
\end{align*}

This pattern generalizes across initialized segment sizes $r$ as
\begin{align*}
r + \sum_{j=1}^{r-1} j \times 2^{r-1-j}
&\stackrel{\checkmark}{=} 2^{r} - 1.
\end{align*}

\end{proof}

% \begin{corollary}[Number \hv{} 0 Reservations]
% \label{thm:stretched-meta-epoch-hv0}
% The number of buffer sites $n$ reserved to \hv{} $\colorh=0$ is $2^{\colors-1-\colortau}$.
% \end{corollary}

% \begin{proof}
% By construction, the number of buffer sites reserved to \hv{} $\colorh=0$ at epoch $\colort=0$ is $n_{\colorh0}=2^{\colors - 1}$.
% Lemma \ref{thm:stretched-meta-epoch} corresponds meta-epoch $\colortau$ to underlying reservation segment invasion cycles.
% By construction, the number of buffer sites reserved to a \hv{} $\colorh$ halve when that hanoi value is invaded.
% With \hv{} $\colorh=0$ always first to be invaded, this occurs at the opening epoch of each meta-epoch $\min(\colort \in \colortausetoft)$ and so remains consistent within each meta-epoch.
% Thus,

% \begin{align*}
% n_{\colorh0}
% &= 2^{\colors - 1} \times \frac{1}{2}^{\colortau}\\
% &\stackrel{\checkmark}{=} 2^{\colors - \colortau - 1}.
% \end{align*}
% \end{proof}
