\begin{lemma}[Meta-epoch Semantics]
\label{thm:stretched-meta-epoch}

The duration of meta-epochs $\colortau$, defined in Section \ref{sec:notation-metaepoch} as lasting $2^{\colortau} - 1$ epochs for $\colortau\geq1$, corresponds to the succesive epoch durations of ``invasion'' reservation-removal events.
\end{lemma}

\begin{proof}

Recall that under the stretched algorithm's proposed layout strategy, buffer space is filled without any overwrites during epoch 0.
Then, during subsequent epochs, reservations are repeatedly eliminated.
Half of reservations, designated ``invading'' reservations, grow by addition of new high-h.v. sites.
The other half of reservations are subsumed, successively losing low-h.v. sites to their invading neighbors.
Note that ``invaded'' reservations do not add high-h.v. sites --- during the invasion process, they are frozen while being eliminated.
By construction, ``invaded'' reservations are always those of smallest remaining size.
Note that, owing to the recursively nested structure of reservation layout, smallest-remaining-size reservations are always interspersed every-other position and always constitutes half of active reservations.

Because invading reservations grow by exactly one site per epoch, the number of epochs comprising a ``meta-epoch'' during which a reservation set is invaded to eliminations corresponds exactly to their reservation size at invasion outset.
So, our proof objective can be recast as determining the final, frozen-for-elimination size $R$ of reservations that took initial size $r$ at epoch $\colort=0$.
Supposing $\colortau = r$, we would have $R = |\{\colort \in \colortau\}|$.
Our goal is therefore to show $R = 2^{r} - 1$.

As already mentioned, initially-singleton $r=1$ reservations are always invaded first, at epoch $\colort=1$.
Trivially, these reservations also have $R = 1 \stackrel{\checkmark}{=} 2^1 - 1$ on account of never having the opportunity to act as invader.
Initially $r=2$ reservations are invaded next.
These reservations acted as invader during epoch $\colort=1$ so grew to $R = 3 \stackrel{\checkmark}{=} 2^2 - 1$.
Subsequent reservations $r>2$, however, grow exponentially --- owing to having invaded reservations that themselves already grew by invasion.
For instance, reservations $r=3$ begin by invading their singleton neighbors $r=1$ during .
Thus, for $r=3$,
\begin{align*}
R
&= 3 + 1 + 2 + 1\\
&\stackrel{\checkmark}{=} 2^3 - 1.
\end{align*}

This pattern generalizes as
\begin{align*}
r + \sum_{i=1}^{r-1} (r - i - 1) \times 2^{i}
&\stackrel{\checkmark}{=} 2^{r} - 1. TODO cleanup
\end{align*}

\end{proof}

\begin{corollary}[Number h.v. 0 Reservations]
\label{thm:num-hv0-reservations}
The number of buffer sites $n$ reserved to h.v. 0 is $2^{\colors-1-\colortau}$.
\end{corollary}

\begin{proof}
By construction, the number of buffer sites reserved to h.v. 0 at epoch 0 is $2^{\colors - 1}$.
Lemma \ref{thm:stretched-meta-epoch} demonstrates that meta-epoch $\colortau$ corresponds to the total number of active/completed invasion cycles.
By construction, the number of available buffer sites for a hanoi value halve when that hanoi value is invaded.
With h.v. 0 always first to be invaded, occuring at the opening epoch of each meta-epoch,
\begin{align*}
n
&= 2^{\colors - 1} \times \frac{1}{2}^{\colortau}\\
&\stackrel{\checkmark}{=} 2^{\colors - \colortau - 1}.
\end{align*}
\end{proof}

\begin{corollary}[Discarded item h.v. incidence count]
\label{thm:discarded-incidence-count}
No data item is discarded unless its h.v. incidence is greater than number of buffer sites reserved to h.v. 0.
\end{corollary}

\begin{proof}
By construction, this proposition is trivially true for hanoi values with reservation site counts greater than or equal to that for h.v. 0.
Corollary \ref{thm:num-hv0-reservations} gives this as $2^{\colors-1-\colortau}$.
However, we must consider h.v.'s fewer than $2^{\colors-1-\colortau}$ sites reserved TODO and show that the number of incidences of that h.v. encountered is less than or equal to that h.v.'s number of sites reserved.

For our next step, begin by noting that at the outset of each epoch $\colortau$, eall reservations provide a site for all h.v. $\colorh < R$, where $R$ is number of sites in the to-be-invaded reservation when frozen for elimination.
From Lemma \ref{thm:stretched-meta-epoch}, we have $R = 2^{\colortau} - 1$.
So, we can restrict consideration to $\colorh > 2^{\colortau} - 1$ or, equivalently, $\colorh \geq 2^{\colortau}$.

Recall that as a propoerty of the hanoi sequence during epoch $\colort$, we have encountered one of the highest-value reserved h.v., one of the second highest-value reserved h.v., two of the third-highest reserved h.v., etc.
Also recall that highest-value encountered h.v. increases by one per epoch.

Initial reservation sizes are laid out with sizes corresponding to hanoi values.
By construction, retained reservations grow exactly one site per epoch.
Because reservations are eliminated in increasing order of initial size $r$, we will always have the largest (position 0) reservation with a site for the single encountered instance of both the highest-value hanoi values.
We can store the two instances of the the next-smallest h.v. in the largest and second-largest reservations.
Proceeding into deeper uninvaded reservation nesting layers, reservation count doubles in sync with the number of encountered h.v. instances.
So, we can safely store all encountered h.v. instances for the top $\colors - \colortau$ encountered hanoi values.

During epoch $\colort$, the highest-encountered h.v. is $\colors + \colort$.
So, we can safely store all encountered h.v. instances for
\begin{align*}
\colorh
&\geq
\colors + \colort - (\colors - \colortau)\\
&\geq
\colort - \colortau.
\end{align*}

Combining the above, our question boils down to
\begin{align*}
2^{\colortau}
\stackrel{?}{\leq} \colort - \colortau.
\end{align*}

With $\colort \geq 2^{\colortau} - \colortau$, TODO THIS INEQUALITY IS WRONG
\begin{align*}
2^{\colortau}
\stackrel{?}{\leq} \colort - \colortau \\
\stackrel{?}{\leq} 2^{\colortau} - \colortau - \colortau \\
\stackrel{?}{\leq} 2^{\colortau} - 2\colortau.
\end{align*}

\end{proof}
