\begin{lemma}[Meta-epoch Semantics]
\label{thm:stretched-meta-epoch}

The timing of meta-epoch $\colortau$, defined in Section \ref{sec:notation-metaepoch} as lasting $2^{\colortau} - 1$ epochs for $\colortau\geq1$, corresponds to the time window during which reservation segments initialized size $r=\colortau$ at $\colort=\colortau=0$ are removed through ``invasion.''
\end{lemma}

\begin{proof}

Recall that under the stretched algorithm's proposed layout strategy, buffer space is filled without any overwrites during epoch 0.
Then, during subsequent epochs, reservations are repeatedly eliminated.
Half of reservations, designated ``invading'' reservations, grow by addition of new high-h.v. sites.
The other half of reservations are subsumed, successively losing low-h.v. sites to their invading neighbors.

Note that ``invaded'' reservations do not add high-h.v. sites --- during the invasion process, they are frozen while being eliminated.
By construction, ``invaded'' reservations are always those of smallest remaining size.
Owing to the recursively nested structure of reservation layout, smallest-remaining-size reservations are always interspersed every-other position and always constitutes half of active reservations.

Because invading reservations grow by exactly one site per epoch, the number of epochs comprising a ``meta-epoch'' during which a reservation segment is invaded to elimination corresponds exactly to their reservation size at invasion outset.

Our proof objective can thus be recast as determining the mature, frozen-for-elimination size $R(r)$ of reservations initialized size $r$ at epoch $\colort=0$.
With reservation segments $r=1$, by construction, invaded at $\colortau = 1$, our goal is therefore to show $|\{\colort \in \colortau\}| = R(r) = 2^{r} - 1$.

As already mentioned, initially-singleton $r=1$ reservations are always invaded first, in epoch $\colort=1$.
Trivially, these reservations also have $R(1) = 1 \stackrel{\checkmark}{=} 2^1 - 1$ on account of never having the opportunity to act as invader.
Segments initialized size $r=2$ are invaded next.
These segments acted as invader during epoch $\colort=1$, so grew to size $R(2) = 3 \stackrel{\checkmark}{=} 2^2 - 1$.

Subsequent reservations $r>2$ grow exponentially --- having invaded reservations that themselves already grew by invasion.
For instance, reservations $r=3$ begin by invading their singleton neighbors $r=1$ during epoch $\colort=1$.
Thus, for $r=3$,
\begin{align*}
R(3)
&= 3 + 1 + 2 + 1\\
&\stackrel{\checkmark}{=} 2^3 - 1.
\end{align*}

This pattern generalizes across initialized segment sizes $r$ as
\begin{align*}
r + \sum_{j=1}^{r-1} j \times 2^{r-1-j}
&\stackrel{\checkmark}{=} 2^{r} - 1.
\end{align*}

\end{proof}

\begin{corollary}[Number h.v. 0 Reservations]
\label{thm:num-hv0-reservations}
The number of buffer sites $n$ reserved to h.v. $\colorh=0$ is $2^{\colors-1-\colortau}$.
\end{corollary}

\begin{proof}
By construction, the number of buffer sites reserved to h.v. $\colorh=0$ at epoch $\colort=0$ is $n_{\colorh0}=2^{\colors - 1}$.
Lemma \ref{thm:stretched-meta-epoch} corresponds meta-epoch $\colortau$ to underlying reservation segment invasion cycles.
By construction, the number of buffer sites reserved to a h.v. $\colorh$ halve when that hanoi value is invaded.
With h.v. $\colorh=0$ always first to be invaded, this occurs at the opening epoch of each meta-epoch $\min\{\colort \in \colortau\}$ and so remains consistent within each meta-epoch.
Thus,

\begin{align*}
n_{\colorh0}
&= 2^{\colors - 1} \times \frac{1}{2}^{\colortau}\\
&\stackrel{\checkmark}{=} 2^{\colors - \colortau - 1}.
\end{align*}
\end{proof}

\begin{lemma}[Minimum retained h.v. incidences]
\label{thm:discarded-incidence-count}
No data item $\colorTbar'$ is discarded unless its h.v. incidence $|\{\colorTbar \in 0,1,\,\ldots,\colorT : \colorH(\colorTbar) = \colorH(\colorTbar')|$ exceeds $2^{\colors - 1 - \colortau}$.
\end{lemma}

\begin{proof}
By construction, this proposition is trivially true for hanoi values with reservation site counts greater than or equal to $2^{\colors-1-\colortau}$.
However, we must consider h.v.'s with fewer than $2^{\colors-1-\colortau}$ reserved sites.
For these h.v.'s, we must show that they encounter no more items than their number of sites reserved.

\begin{proofpart}[How many hanoi values $\colorh$ have $2^{\colors - 1 - \colortau}$ reserved sites?]

At the outset of each meta-epoch $\colortau$, there are $2^{\colors - 1 - \colortau}$ uninvaded segments.
The smallest invading segment is slated to be invaded next, after $R$ epochs.
Thus, its current size is
\begin{align*}
R(\colortau + 1) - R(\colortau)
&= (2^{\colortau + 1} - 1) - (2^{\colortau} - 1) \tag{by Lemma \ref{thm:stretched-meta-epoch}}\\
&= 2^{\colortau + 1} - 2^{\colortau}\\
&= 2^{\colortau}.
\end{align*}
All $2^{\colors - 1 - \colortau}$ reservations thus provide a site for all h.v. $\colorh < 2^{\tau}$.
We thus can restrict consideration to $\colorh \geq 2^{\colortau}$.
\end{proofpart}

\begin{proofpart}[H.v.'s without $2^{\colors-1-\colortau}$ reserved sites]
Recall that as a propoerty of the hanoi sequence during epoch $\colort$, we have encountered one of the highest-value h.v. $\colorh$, one of the second highest-value h.v. $\colorh-1$, two of the third-highest h.v. $\colorh-2$, etc.
Also be reminded that highest-value encountered h.v. increases by one per epoch.

Initial reservation sizes are laid out with sizes drawn from the hanoi sequence (Formula \ref{eqn:stretched-segment-sizes}).
By construction, retained reservations grow exactly one site per epoch.
Because reservations are eliminated in increasing order of initialized size $r$, we will always have the largest reservation segment $r=\colors$ to provide a site for the single lone instances of the two highest hanoi values $\colorh=\colort+\colors$ and $\colorh=\colort+\colors-1$.
Next, we can store the two instances of the the next-smallest h.v. $\colorh=\colort+\colors-2$ in the largest and second-largest reservations $r=\colors$and $r=\colors-2$.

Proceeding into deeper uninvaded reservation nesting layers, reservation count doubles in step with h.v. instance counts.
Because we have $\colors - \colortau$ active segment layers, we can safely store all encountered h.v. $\colorH(\colorTbar) = \colorh$ instances for the top $\colors - \colortau$ encountered hanoi values $\colorh$.
During epoch $\colort$, the highest-encountered h.v. is $\colorh=\colors + \colort$.
So, we can safely store all encountered instances for h.v.'s
\begin{align*}
\colorh
&\geq
\colors + \colort - (\colors - \colortau)\\
&\geq
\colort - \colortau.
\end{align*}
\end{proofpart}

\begin{proofpart}[Have we accounted for all h.v.'s $\colorh$?]
Combining the above, our question boils down to
\begin{align*}
2^{\colortau}
\stackrel{?}{\geq} \colort - \colortau.
\end{align*}

With $\colort \geq 2^{\colortau} - \colortau$, TODO THIS INEQUALITY IS WRONG
\begin{align*}
2^{\colortau}
&\stackrel{?}{\geq} \colort - \colortau \\
&\stackrel{?}{\geq} 2^{\colortau} - \colortau - \colortau \\
&\stackrel{?}{\geq} 2^{\colortau} - 2\colortau.
\end{align*}
\end{proofpart}

\end{proof}
