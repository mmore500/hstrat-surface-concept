\begin{lemma}[Meta-epochs $\colortau$ Correspond to Invasion Cycles]
\label{thm:stretched-meta-epoch}

The timing of meta-epoch $\colortau$, defined in Section \ref{sec:notation-metaepoch} as lasting $2^{\colortau} - 1$ epochs for $\colortau\geq1$, corresponds to the time window during which reservation segments initialized size $r=\colortau$ at $\colort=\colortau=0$ are removed through ``invasion.''
\end{lemma}

\begin{proof}

Recall that under the stretched algorithm's proposed layout strategy, buffer space is filled without any overwrites during epoch 0.
Then, during subsequent epochs, reservations are repeatedly eliminated.
Half of reservations, designated ``invading'' reservations, grow by addition of new high-h.v. sites.
The other half of reservations are subsumed, successively losing low-h.v. sites to their invading neighbors.

Note that ``invaded'' reservations do not add high-h.v. sites --- during the invasion process, they are frozen while being eliminated.
By construction, ``invaded'' reservations are always those of smallest remaining size.
Owing to the recursively nested structure of reservation layout, smallest-remaining-size reservations are always interspersed every-other position and always constitutes half of active reservations.

Because invading reservations grow by exactly one site per epoch, the number of epochs comprising a ``meta-epoch'' during which a reservation segment is invaded to elimination corresponds exactly to their reservation size at invasion outset.

Our proof objective can thus be recast as determining the mature, frozen-for-elimination size $R(r)$ of reservations initialized size $r$ at epoch $\colort=0$.
With reservation segments $r=1$, by construction, invaded at $\colortau = 1$, our goal is therefore to show $|\{\colort \in \colortau\}| = 2^{\colortau} - 1$ matches $R(r)$ by showing $R(r) = 2^{r} - 1$.

As already mentioned, initially-singleton $r=1$ reservations are always invaded first, in epoch $\colort=1$.
Trivially, these reservations also have $R(1) = 1 \stackrel{\checkmark}{=} 2^1 - 1$ on account of never having the opportunity to act as invader.
Segments initialized size $r=2$ are invaded next.
These segments acted as invader during epoch $\colort=1$, so grew to size $R(2) = 3 \stackrel{\checkmark}{=} 2^2 - 1$.

Subsequent reservations $r>2$ grow exponentially --- having invaded reservations that themselves already grew by invasion.
For instance, reservations $r=3$ begin by invading their singleton neighbors $r=1$ during epoch $\colort=1$.
Thus, for $r=3$,
\begin{align*}
R(3)
&= 3 + 1 + 2 + 1\\
&\stackrel{\checkmark}{=} 2^3 - 1.
\end{align*}

This pattern generalizes across initialized segment sizes $r$ as
\begin{align*}
r + \sum_{j=1}^{r-1} j \times 2^{r-1-j}
&\stackrel{\checkmark}{=} 2^{r} - 1.
\end{align*}

\end{proof}

% \begin{corollary}[Number h.v. 0 Reservations]
% \label{thm:stretched-meta-epoch-hv0}
% The number of buffer sites $n$ reserved to h.v. $\colorh=0$ is $2^{\colors-1-\colortau}$.
% \end{corollary}

% \begin{proof}
% By construction, the number of buffer sites reserved to h.v. $\colorh=0$ at epoch $\colort=0$ is $n_{\colorh0}=2^{\colors - 1}$.
% Lemma \ref{thm:stretched-meta-epoch} corresponds meta-epoch $\colortau$ to underlying reservation segment invasion cycles.
% By construction, the number of buffer sites reserved to a h.v. $\colorh$ halve when that hanoi value is invaded.
% With h.v. $\colorh=0$ always first to be invaded, this occurs at the opening epoch of each meta-epoch $\min\{\colort \in \colortau\}$ and so remains consistent within each meta-epoch.
% Thus,

% \begin{align*}
% n_{\colorh0}
% &= 2^{\colors - 1} \times \frac{1}{2}^{\colortau}\\
% &\stackrel{\checkmark}{=} 2^{\colors - \colortau - 1}.
% \end{align*}
% \end{proof}
