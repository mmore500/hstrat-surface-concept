\begin{lemma}[Minimum Retained Items per Hanoi Value]
\label{thm:stretched-discarded-incidence-count}
No data item $\colorTbar'$ is discarded unless its h.v. incidence $|\{\colorTbar \in 0,1,\,\ldots,\colorT : \colorH(\colorTbar) = \colorH(\colorTbar')|$ exceeds $\left\lceil2^{\colors - 1 - \colortau}\right\rceil$.
\end{lemma}

\begin{proof}
By layout construction, this proposition is trivially true for hanoi values with at least $\left\lceil 2^{\colors-1-\colortau} \right\rceil$ sites.
During final meta-epoch $\colortau = \colors$, every encountered h.v. has exactly $\left\lfloor 2^{-1} \right\rfloor = 1$ reserved site.
However, we must consider h.v.'s with fewer than $2^{\colors-1-\colortau}$ reserved sites, with $\colortau < \colors$.
For these h.v.'s $\colorh$, we must show that no more items $\colorH(\colorTbar) = \colorh$ are encountered than sites reserved to h.v. $\colorh$.

\begin{proofpart}[How many hanoi values $\colorh$ have $2^{\colors - 1 - \colortau}$ reserved sites?]

At the outset of each meta-epoch $\colortau$, there are $2^{\colors - 1 - \colortau}$ uninvaded segments.
At any epoch $\colort>0$, the smallest invading segment will be slated next for invasion, after the current invasion's $R$ epochs.
Thus, size at outset of meta-epoch $\colortau$ is,
\begin{align*}
R(\colortau + 1) - R(\colortau)
&= (2^{\colortau + 1} - 1) - (2^{\colortau} - 1) \tag{by Lemma \ref{thm:stretched-meta-epoch}}\\
&= 2^{\colortau + 1} - 2^{\colortau}\\
&= 2^{\colortau}.
\end{align*}
The $2^{\colors - 1 - \colortau}$ active reservation segments thus provide at least $2^{\colors - 1 - \colortau}$ sites for all h.v. $\colorh < 2^{\colortau}$.
We thus can restrict consideration to $\colorh \geq 2^{\colortau}$.
\end{proofpart}

\begin{proofpart}[Hanoi Values Without $2^{\colors-1-\colortau}$ Reserved Sites]
Recall that as a propoerty of the hanoi sequence during epoch $\colort$, we have encountered one of highest-value h.v. $\colorh$, one of second highest-value h.v. $\colorh-1$, two of the third-highest h.v. $\colorh-2$, etc.
Also be reminded that highest-value encountered h.v. $\colorh$ increases by one per epoch $\colort$.

Initial reservation segments are laid out with sizes drawn from the hanoi sequence (Formula \ref{eqn:stretched-segment-sizes}).
By construction, retained reservations grow exactly one site per epoch.
Because reservations are eliminated in increasing order of their initialized size $r$, we will always ($\colorTbar < 2^{\colors}$) have the largest reservation segment $r=\colors$ to provide a site for the lone instances of our two highest hanoi values $\colorh=\colort+\colors$ and $\colorh=\colort+\colors-1$.
Along these lines, we can store the two instances of the the next-smallest h.v. $\colorh=\colort+\colors-2$ in the largest and second-largest reservations $r=\colors$ and $r=\colors-2$.
Proceeding into deeper uninvaded reservation nesting layers, reservation count doubles in step with h.v. instance counts.

With segments $r \leq \colors - \colortau$ active, we can safely store all encountered h.v. $\colorH(\colorTbar) = \colorh$ instances for the top $\colors - \colortau$ encountered hanoi values $\colorh$.
During epoch $\colort$, the highest-encountered h.v. is $\colorh=\colors + \colort$.
So, we can safely store all encountered instances for h.v.'s
\begin{align*}
\colorh
&\geq
\colors + \colort - (\colors - \colortau)\\
&\geq
\colort + \colortau
\end{align*}
over the entirety of meta-epoch $\colortau$.
\end{proofpart}

\begin{proofpart}[Have we accounted for all hanoi values?]
Combining the above, the question of covering all h.v.'s $0\leq\colorh\leq\colors+\colort$ becomes,
\begin{align*}
2^{\colortau}
\stackrel{?}{\geq} \colort - \colortau.
\end{align*}

With $(\colort = 2^{\colortau} - \colortau) \in \colortau$,
\begin{align*}
2^{\colortau}
&\stackrel{?}{\geq} 2^{\colortau} - \colortau + \colortau \\
&\stackrel{\checkmark}{\geq} 2^{\colortau}.
\end{align*}
\end{proofpart}

\end{proof}
