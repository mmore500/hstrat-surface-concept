\begin{lemma}[Lower bound on stretched criterion]
\label{thm:stretched-ideal-strict}
The ``worst'' stretched criterion (gap size ratio) for a buffer of size $\colorS$ at time $\colorT$ is best-case
\begin{align*}
\frac{\colorG_{\colorT}(\colorTbar)}{\colorTbar}
&\geq
\frac{
  1
}{
  1 + \colorS
  - \colorS \log_{\colorT}\Big(
    (\colorT - \colorS)(\colorT^{1/\colorS} - 1) + 1
  \Big)
}.
  \end{align*}
\end{lemma}

\begin{proof}
Assume $\colorT > \colorS$.
Then, we have $\colorT - \colorS$ discarded data items in our record.
As a consequence of discretization, the smallest possible gap size is 1 data item.

In an optimal, successive gap sizes will grow by a factor $\colorT^{1/\colorS}$.
Equating the total gap space,
\begin{align*}
\colorT - \colorS
&=
\sum_{i = 0}^{j} \colorT^{i/\colorS} % +1?
\\
&=
\frac{
  \colorT^{(j + 1)/\colorS} - 1
}{
  \colorT^{1/\colorS} - 1
}.
\end{align*}

Solving for the number of discrete gaps instantiated $j$,
\begin{align*}
j
&=
\colorS \log_{\colorT}\Big(
  (\colorT - \colorS)(\colorT^{1/\colorS} - 1) + 1
\Big) - 1.
\end{align*}

The smallest gap will be located $\mathrm{num\_gaps} + \mathrm{gap\_space}$ time steps back from the most recent observed time $\colorT$.
The $\mathrm{num\_gaps}$ term arises from the time steps occupied by retained data between gaps.
So, the first gap will occur at time $\colorTbar = \colorT - \mathrm{num\_gaps} - \mathrm{gap\_space}$ and the gap size ratio will be
\begin{align*}
\frac{1}{
\colorT
- \Big[
\colorS \log_{\colorT}\Big(
  (\colorT - \colorS)(\colorT^{1/\colorS} - 1) + 1
\Big) - 1
\Big] - (\colorT - \colorS)
}.
\end{align*}

Simplifying terms gives the result.
\end{proof}
