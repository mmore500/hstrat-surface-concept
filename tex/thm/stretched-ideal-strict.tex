\begin{lemma}[Best-possible Stretched Criterion Satisfaction]
\label{thm:stretched-ideal-strict}
The stretched criterion (largest gap size ratio) for a buffer of size $\colorS$ at time $\colorT$ can be minimized no lower than,
\begin{align*}
\frac{\colorG_{\colorT}(\colorTbar)}{\colorTbar}
&\geq
\frac{
  1
}{
  1 + \colorS
  - \left\lfloor \colorS \log_{\colorT}\Big(
    (\colorT - \colorS)(\colorT^{1/\colorS} - 1) + 1
  \Big)\right\rfloor
}.
\end{align*}
\end{lemma}

\begin{proof}
At time $\colorT > \colorS$, we have discarded at least $\colorT - \colorS$ data items.
That is, total gap size $\sum \colorg \geq \colorT - \colorS$.
As a consequence of discretization, the smallest possible gap size is 1 data item.

Optimal retention grows successive gap sizes by a factor $\colorT^{1/\colorS}$.
Equating total gap space to sum gap sizes
\begin{align*}
\colorT - \colorS
&=
\sum_{i = 0}^{j} \colorT^{i/\colorS} % +1?
\\
&=
\frac{
  \colorT^{(j + 1)/\colorS} - 1
}{
  \colorT^{1/\colorS} - 1
}.
\end{align*}

The number of discrete gaps instantiated $j$ follows as at least,
\begin{align*}
j
&\geq
\left\lfloor
\colorS \log_{\colorT}\Big(
  (\colorT - \colorS)(\colorT^{1/\colorS} - 1) + 1
\Big) - 1
\right\rfloor.
\end{align*}

The smallest gap will be located $\mathsf{num\_gaps} + \mathsf{gap\_space}$ time steps back from the most recent observed time $\colorT$.
The $\mathsf{num\_gaps}$ term arises from the time steps occupied by retained data between gaps.
So, the first gap will occur at time $\colorTbar = \colorT - \mathsf{num\_gaps} - \mathsf{gap\_space}$ and the gap size ratio will be at least.
\begin{align*}
\frac{\colorG_{\colorT}(\colorTbar)}{\colorTbar}
&\geq
\frac{1}{
\colorT
- \left\lfloor
\colorS \log_{\colorT}\Big(
  (\colorT - \colorS)(\colorT^{1/\colorS} - 1) + 1
\Big) - 1
\right\rfloor - (\colorT - \colorS)
}.
\end{align*}

Simplifying terms gives the result.
\end{proof}
