\begin{lemma}[Epochs' Last Instance of a Hanoi Value]
\label{lemma:tilted-last-touched}
The final instance of each h.v. encountered during an epoch is placed in the rightmost site reserved for that hanoi value.
That is, during any epoch $\colort$,
\begin{align*}
\colorK\Big(
  \max\{\colorT \in \colort : \colorH(\colorT) = \colorh\}
\Big)
=
\max\{\colork \in \colorS : \colorHcal(\colork) = \colorh \}
\end{align*}
for all $\colorh \in \{\colorH(0), \colorH(1), \,\ldots, \colorH(\max\{\colorT \in \colort\})\}$.
\end{lemma}

\begin{proof}
Assume $\colorh \leq \colors + \colort$.
Otherwise, no h.v. instances $\colorH(\colorTbar) = \colorh$ are encountered and our task is moot.

Because we cycle through the rightmost reservation segment last (it is rightmost within the innermost segment layer), our proof objective is to show that the number of sites $\colork$ reserved to h.v. $\colorh$ evenly divides the number of h.v. $\colorh$ instances encountered during epoch $\colort$.

How many instances of a hanoi value $\colorh$ are encountered during epoch $\colort$?
This is
\begin{align*}
|\{\colorT \in \colort : \colorH(\colorT) = \colorh\}|
&=
2^{\colort + \colors - \colorh} - \left\lfloor2^{\colort + \colors - \colorh - 1} \right\rfloor.
\end{align*}
Adopting $\mathfrak{C}_{\colorH(\colorTbar)}$ as shorthand for this cardinality, observe that $\mathfrak{C}_{\colorH(\colorTbar)} \in 2^{\mathbb{N}}$ for $\colorh \leq \colors + \colort$.

How many sites are reserved to a hanoi value $\colorh$ during epoch $\colort$?
Recall from Lemma \ref{thm:stretched-discarded-incidence-count} that this is
\begin{align*}
|\{\colork \in \colorS : \colorHcal(\colork) = \colorh\}|
&=
\min\Big(
\left\lceil 2^{\colors - \colortau - 1} \right\rceil,
2^{\colors + \colort - \colorh} - \left\lfloor 2^{\colors + \colort - \colorh - 1} \right\rfloor
\Big).
\end{align*}
Denoting this cardinality $\mathfrak{C}_{\colorHcal(\colork)}$, observe also that
$\mathfrak{C}_{\colorHcal(\colork)} \in 2^{\mathbb{N}}$ for $\colorh \leq \colors + \colort$.

Because both $\mathfrak{C}_{\colorH(\colorTbar)} \in 2^{\mathbb{N}}$ and $\mathfrak{C}_{\colorHcal(\colork)} \in 2^{\mathbb{N}}$, all that remains to be shown is $\mathfrak{C}_{\colorH(\colorTbar)} \geq \mathfrak{C}_{\colorHcal(\colork)}$.
Consider,
\begin{align*}
2^{\colort + \colors - \colorh} - \left\lfloor2^{\colort + \colors - \colorh - 1} \right\rfloor
&\stackrel{?}{\geq}
\min\Big(
\left\lceil 2^{\colors - \colortau - 1} \right\rceil,
2^{\colors + \colort - \colorh} - \left\lfloor 2^{\colors - \colorh - 1} \right\rfloor
\Big)\\
&\stackrel{\checkmark}{\geq}
2^{\colors + \colort - \colorh} - \left\lfloor 2^{\colors - \colorh - 1} \right\rfloor.
\end{align*}
\end{proof}
