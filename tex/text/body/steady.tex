\section{Steady Algorithm} \label{sec:steady}

The steady criterion seeks to retain data items from time points evenly spread across observed history.
The criterion can be formulated as minimization of the largest gap size between retained data items.
For a buffer size $\colorS$ and time elapsed $\colorT$, largest gap size can be minimized to at best $\left\lceil \colorT / \colorS \right\rceil$.
This section presents a stream curation algorithm designed to support the steady criterion, guaranteeing maximum gap size no worse than $2\left\lceil \colorT / \colorS \right\rceil$.
(Disparity from ideal arises, in part, because maintaining uniform gap spacing on an ongoing basis is impossible on account of data item discards merging neighboring gaps.)

\subsection{Steady Algorithm Strategy}
\label{sec:steady-strategy}

Figure \ref{fig:hanoi-intuition-steady} overviews the proposed algorithm's core strategy, which revolves around prioritizing data item retention according to the \hv{} of the sequence indices, $\colorH(\colorTbar)$.
Specifically, we aim to keep data items with the largest hanoi values.

It turns out that with all data items $\colorH(\colorTbar) > m$ retained, gap size is at most $\colorg \leq 2^m - 1$.
To understand, imagine discarding items with $\colorH(\colorTbar) = 0$.
This action would drop every other item, and increase gap size from 0 to $\colorg \leq 1$.
Then, removing items with $\colorH(\colorTbar) = 1$ would again drop every other item, and increase gap size to $\colorg \leq 3$.
Continuing this pattern to prune successive hanoi values provides well-behaved transitions gradually increasing gap size while maintaining even spacing.

We thus define our algorithm as maintaining, for a ratcheting threshold $n(\colorT)$, all items $\colorH(\colorTbar) > n(\colorT)$ for some threshold $n(\colorT)$.
(The threshold $n(\colorT)$ must increase over time to ensure space for new high h.v. data items as we encounter them.)
Formally,
\begin{align*}
\textsf{goal\_steady}
\coloneq \{
\colorTbar \in [0 \twodots \colorT]
: \colorH(\colorTbar) > n(\colorT)
\}.
\end{align*}
In practice, this requires repeatedly discarding all items with lowest hanoi value $\colorH(\colorTbar) = n(\colorT)$ as time elapses elapse.
Lemma \ref{thm:steady-hv-geq-epoch} shows that $n(\colorT) = \colort - 1$ fills available buffer space $\colorS$.%
\footnote{%
More exactly, $\colorS - 1$ sites are required.
So, \textit{nearly} all buffer space is needed to store $\{\colorTbar : \colorH(\colorT) \geq \colort\}$.
In practice, the surplus site can be convenient to store the very first or the very most recent data item.
}

\subsection{Steady Algorithm Mechanism}
\label{sec:steady-mechanism}

\begin{figure*}
  \centering
  \includegraphics[width=\textwidth]{img/hsurf-steady-intuition}
  \caption{TODO}
  \label{fig:hsurf-steady-intuition}
\end{figure*}


Each epoch $\colort$, all items with $\colorH(\colorT) = \colort - 1$ must be overwritten to make space for new items with \hv{} $\colorh \geq \colort$.
Figure \ref{fig:hsurf-steady-intuition} overviews the layout procedure used to orchestrate replacement of data items with \hv{} $\colorh = \colort - 1$ each epoch.
We divide buffer space into $\colors$ ``bunches,'' themselves divided into ``segments.''
Bunch $i=0$ contains one segment of length $\colors$ sites.
Then, for $i > 0$, bunch $i$ contains $2^{i-1}$ segments.
Although segment count increases across bunch, we define segment length as $\colors - i$, decreasing by 1 each bunch.
So, segments in the last bunch contain only one site.
With $\colors$ bunches, available buffer space $\colorS$ is nearly filled by this reservation layout,
\begin{align*}
\colors + \sum_{i=0}^{\colors-1} (\colors - i - 1) \times 2^{i} = 2^{\colors} - 1 = \colorS - 1.
\end{align*}

For each hanoi value $\colorh$, if we store one data item $\colorH(\colorTbar) = \colorh$ per segment, each epoch $\colort$ will fill one segment bunch.
Bunch 0 will contain the first data item with \hv{} $\colorh$, which is encountered in epoch $\colort=\colorh - \colors + 1$.
Bunch 1 contains the data item with that \hv{} $\colorh$ from epoch $\colort=\colorh - \colors + 2$, bunch 2 contains the two data items $\{ \colorTbar \in \lBrace \colort = \colorh - \colors + 2 \rBrace : \colorH(\colorTbar) = \colorh \}$, and so forth.
Because only one not-yet-encountered \hv{} surfaces each epoch, each epoch only takes data item instances from only the top $\colors$ \hv's $\colorh \geq \colort$ into storage.
% Segment size (decreasing by one each bunch) is arranged so that one instance of all $\colors - n$ \hv's $\colorh$ that have ``progressed'' to bunch $n$ can be stored within each segment in that bunch.

\begin{figure*}[htbp!]
  \centering

\begin{minipage}{\textwidth}
  \scriptsize
  \setlength{\tabcolsep}{2.5pt}
  \begin{tabularx}{\textwidth}{
    r
    Y|Y|Y|Y|Y|Y|Y|Y|
    Y|Y|Y|Y|Y Y Y|Y
    |Y|Y|Y|Y|Y|Y|Y
    |Y|Y|Y|Y Y
    }
     { Time $\colorT$} & \textbf{0} & \textbf{1} & \textbf{2} & \textbf{3} & \textbf{4} & \textbf{5} & \textbf{6} & \textbf{7}
    & \textbf{8} & \textbf{9} & \textbf{10} & \textbf{11} & \textbf{12} %& 13 & 14 & 15
    % & 16 & 17 & 18 & 19 & 20 & 21 & 22 & 23
    &  \ldots
    & \textbf{28} & \textbf{29} & \textbf{30} & \textbf{31}
    & \textbf{32} & \textbf{33} & \textbf{34} & \textbf{35}
    & \textbf{36} & \textbf{37} & \textbf{38} & \textbf{39} & \textbf{40}
    & \ldots \\ \hline
     \rowcolor{lightgray!30}
   { Epoch $\colort$} & 0 & 0 & 0 & 0 & 0 & 0 & 0 & 0
    & 0 & 0 & 0 & 0 & 0 %& 13 & 14 & 15
    % & 16 & 17 & 18 & 19 & 20 & 21 & 22 & 23
    &  \ldots
    & 0 & 0 & 0 & 0
    & 1 & 1 & 1 & 1
    & 1 & 1 & 1 & 1 & 1
    & \ldots \\
     % \rowcolor{lightgray!30}
    { \scriptsize$\colorH(\colorT)$} & 0 & 1 & 0 & 2 & 0 & 1 & 0 & 3
    & 0 & 1 & 0 & 2 & 0 %& 13 & 14 & 15
    % & 16 & 17 & 18 & 19 & 20 & 21 & 22 & 23
    &  \ldots
    & 0 & 1 & 0 & 5
    & 0 & 1 & 0 & 2
    & 0 & 1 & 0 & 3 & 0
    & \ldots \\
    \hline
     % \rowcolor{lightgray!30}
     { \scriptsize $\colorK(\colorT)$} & \textbf{0} & \textbf{1} & \textbf{6} & \textbf{2} & \textbf{10} & \textbf{7} & \textbf{13} & \textbf{3}
     & \textbf{16} & \textbf{11} & \textbf{18} & \textbf{8} & \textbf{20} & \ldots
 & \textbf{30} & \textbf{23} & \textbf{31} & \textbf{5} & {\tiny \texttt{\textbf{null\hphantom{}}}}  % \vspace{-3ex}
 & \textbf{24} & {\tiny \texttt{\textbf{null\hphantom{}}}} & \textbf{16}
 & {\tiny \texttt{\textbf{null\hphantom{}}}} & \textbf{25} & {\tiny \texttt{\textbf{null\hphantom{}}}} & \textbf{10} & {\tiny \texttt{\textbf{null\hphantom{}}}}  &\ldots
  \end{tabularx}
\end{minipage}
\vspace{-0.5ex}
\begin{subfigure}{\textwidth}
\caption{\footnotesize Steady policy site selection $\colorK(\colorT)$ with buffer size $\colorS=32$. Ingests marked \nullval{} indicate item discarded without storing.}
\label{fig:hsurf-steady-implementation-site-selection}
\end{subfigure}
\vspace{-0.5ex}

\begin{subfigure}[b]{\linewidth}
\includegraphics[width=\linewidth]{
binder/teeplots/12/num-generations=128+reservation-mode=steady-full+surface-size=32+viz=site-reservation-by-rank-spliced-at-heatmap+ext=.png}
\vspace{-4.5ex}\caption{
  Buffer composition across time, split by epoch with data items color-coded by hanoi value $\colorH(\colorTbar)$.
}
\label{fig:hsurf-steady-implementation-schematic}
\end{subfigure}

\vspace{0.5ex}
\begin{minipage}[]{\textwidth}
 \vspace{-2pt}
  \begin{subfigure}[t]{0.65\linewidth}
    \vspace{0pt}
    \centering
  \includegraphics[width=0.88\linewidth,clip]{binder/teeplots/12/cnorm=log+num-generations=4096+surface-size=256+viz=site-ingest-depth-by-rank-heatmap+ynorm=linear+ext=.png}  % pdf cbar is scrambled
  \end{subfigure}%
  \begin{subfigure}[t]{0.35\linewidth}
  \vspace{-2pt}
  \caption{%
    \footnotesize
    Stored data item age across buffer sites for buffer size $\colorS=256$ from $\colorT=0$ to 4,096.
  }
  \label{fig:hsurf-steady-implementation-heatmap}
\end{subfigure}
\end{minipage}

  \vspace{-0.5ex}
   \begin{minipage}[]{\textwidth}
   \vspace{-2pt}
  \begin{subfigure}[t]{0.65\linewidth}
  \vspace{0pt}
    \centering
    \includegraphics[width=0.88\linewidth,clip]{binder/teeplots/12/num-generations=262144+surface-size=64+viz=stratum-persistence-dripplot+ext=.pdf}
  \end{subfigure}%
  \begin{subfigure}[t]{0.35\linewidth}
  \vspace{-2pt}
  \caption{%
    \footnotesize
    Data item retention time spans by ingestion time point for buffer size $\colorS=64$ from $\colorT=0$ to 3,000.
  }
  \label{fig:hsurf-steady-implementation-dripplot}
  \end{subfigure}
  \end{minipage}

  \vspace{-0.5ex}
 \begin{minipage}[]{\textwidth}
 \vspace{-2pt}
\begin{subfigure}[t]{0.65\linewidth}
\vspace{0pt}
  \centering
  \includegraphics[width=0.88\linewidth,clip]{binder/teeplots/12/hue=kind+surface-size=16+viz=criterion-satisfaction-lineplot+x=rank+y=steady-criterion+yscale=symlog+ext=.pdf}
\end{subfigure}%
\begin{subfigure}[t]{0.35\linewidth}
\vspace{-2pt}
\caption{%
  \footnotesize
  Steady criterion satisfaction across time points for buffer size $\colorS=16$.
}
\label{fig:hsurf-steady-implementation-satisfaction}
\end{subfigure}
\end{minipage}

\vspace{-2ex}\caption{%
  \textbf{Steady algorithm implementation.}
  \footnotesize
  Top panel \ref{fig:hsurf-steady-implementation-site-selection} enumerates initial steady policy site selection on a 32-site buffer.
  Panel \ref{fig:hsurf-steady-implementation-schematic} summarizes how data items are ingested and retained over time within a 32-site buffer, color-coded by data items' hanoi values $\colorH(\colorT)$.
  Between $\colorT=0$ and $\colorT=126$, time is segmented into epochs $\colort=0$, $\colort=1$, and $\colort=2$; strips before  each epoch show hanoi values assigned to each buffer site during that epoch.
  Time increases along the $y$ axis.
  Rectangles with small white ``$\blkhorzoval$'' symbol denote buffer site where the ingested data item from each timestep $\colorT$ is placed.
  Buffer space is split into ``reservation segments.''
  Reservation segments occur in five ``bunches'' --- (1) one 5-site reservation segment, (2) one 4-site reservation segment, (3) two 3-site segments, (4) four 2-site segments, and (5) eight 1-site segments.
  At each epoch, data items are filled into sites newly assigned for their ingestion-order hanoi value from left to right.
  In epoch $\colort=0$, all sites are filled with a first data item.
  During each subsequent epoch $\colort>0$, segments within bunch $i$ each accept one data item with h.v. $\colorh=\colort + \colors - 1 - i$.
  All newly-assigned sites were previously assigned to the overall now-lowest hanoi value $\colorh=\colort - 1$.
  In this way, all instances of the overall lowest hanoi value are overwritten each epoch.
  Note that the rightmost site $\colork=\colorS-1$ is unused.
  Heatmap panel \ref{fig:hsurf-steady-implementation-heatmap} shows the evolution of data item age at each site on a 256-bit field over the course of 4,096 time steps.
  Dripplot panel \ref{fig:hsurf-steady-implementation-dripplot} shows retention spans for 3,000 ingested time points.
  Vertical lines span durations between ingestion and elimination for data items from successive time points.
  Time points previously eliminated are marked in red.
  Lineplot panel \ref{fig:hsurf-steady-implementation-satisfaction} shows steady criterion satisfaction on a 16-bit surface over $2^{16}$ timepoints.
  Lower and upper shaded areas are best- and worst-case bounds, respectively.
  }
\label{fig:hsurf-steady-implementation}

\end{figure*}


The particulars of our layout become useful in managing elimination of data items with \hv{} $\colorh$ during epoch $\colort=\colorh + 1$.
We have that \hv{} $\colorh + \colors$ will store one data item in bunch 0 during epoch $\colort=\colorh + 1$.
This is the same number of data items left by \hv{} $\colorh$ in bunch 0.
The same holds for all bunches, with data items left by \hv{} $\colorh$ equivalent in number to those to be stored in bunch $n$ from \hv{} $\colorh + \colors - n$ during epoch $\colort = \colorh + 1$.

As shown in Figure \ref{fig:hsurf-steady-implementation}, we can take advantage of one-to-one correspondence between incoming data items and data items of \hv{} $\colorh=\colort-1$ to choreograph clean elimination of \hv{} $\colorh=\colort-1$ by overwrites each epoch.
In determining storage site $\colork$ for ingest $\colorTbar$, we map incoming data items with \hv{} $\colorh \geq \colort$ over items $\colorh = \colort - 1$ slated for elimination by placing them at segment positions $\colorh$ modulus segment size.
Calculating the number of \hv{} instances $\colorh = \colorH(\colorTbar)$ already seen $\colorTbar < \colorT$ identifies the segment where data item $\colorTbar$ should be stored.
Lemma \ref{thm:steady-hv-elimination} verifies the behavior of this procedure.

We refer the reader to our supplemental Python-language implementation for an exact step-by-step listing of this procedure, which comprises a handful of fast $\mathcal{O}(1)$ binary operations (e.g., bit mask, bit shift, count leading zeros).
Also implemented fully in accompanying materials, ingestion time calculation for lookup of the data item at $\colork$ at time $\colorT$ boils down to decoding its segment/bunch indices and checking whether (if slated) it has yet been replaced during the current epoch $\colort$.
Batch calculation of ingestion times of all data items stored in a buffer proceeds with time complexity $\mathcal{O}(1)$ per item.

\subsection{Steady Algorithm Criterion Satisfaction}
\label{sec:stready-satisfaction}

In this final subsection, we establish an upper bound on gap size $\colorg$ for a buffer of size $\colorS$ at time $\colorT$ under the proposed steady curation algorithm.
Figure \ref{fig:hsurf-steady-implementation-satisfaction} plots an example of actual worst gap size over time under this algorithm.

\begin{theorem}[Steady algorithm gap size upper bound]
\label{thm:steady-gap-size}
Under the steady curation algorithm,
\begin{align*}
\mathsf{cost\_steady}(\colorT) \leq 2 \frac{\colorS + 1}{\colorS} \hat{\colorg} + 1,
\end{align*}
where $\hat{\colorg}$ is the optimal lower bound on $\mathsf{cost\_steady}(\colorT)$ given in Equation \ref{eqn:steady-optimal-gap-size}.
\end{theorem}
\begin{proof}
Recall that the time between instances of a data item with \hv{} $\colorH(\colorTbar) = \colorh$ is $2^{\colorh + 1}$ data items.
Recall also that the time elapsed between a \hv{} $\colorh$ and a data item with \hv{} greater than $\colorh$ is $2^{\colorh}$ data items.

Under the proposed algorithm, we retain all data items for hanoi values $\colorh \geq \colort$.
So, retained data items occur at most $2^{\colort}$ time steps apart.
This corresponds to gap size at most $2^{\colort} - 1$.
Finally, we test
\begin{align*}
2 \frac{\colorS + 1}{\colorS} \left\lceil \frac{\colorT - \colorS}{\colorS + 1} \right\rceil + 1
&\stackrel{?}{\geq}
2^{\colort} - 1\\
2 \frac{\colorS + 1}{\colorS} \left\lceil \frac{\colorT - \colorS}{\colorS + 1} \right\rceil
&\stackrel{?}{\geq}
2^{\left\lfloor \log_2(\colorT) \right\rfloor - \colors + 1} - 2 \tag{definition $\colort$, Equation \ref{eqn:epoch-defn}}\\
&\stackrel{?}{\geq}
2^{\left\lfloor \log_2(\colorT) - \log_2(\colorS) \right\rfloor + 1} - 2\\
&\stackrel{?}{\geq}
2\left\lfloor \frac{\colorT}{\colorS} \right\rfloor_{\mathrm{bin}} - 2\frac{\colorS}{\colorS}\\
\frac{\colorS + 1}{\colorS} \frac{\colorT - \colorS}{\colorS + 1}
&\stackrel{?}{\geq}
\frac{\colorT - \colorS}{\colorS}\\
\frac{\colorT - \colorS}{\colorS + 1}
&\stackrel{\checkmark}{\geq}
\frac{\colorT - \colorS}{\colorS + 1}.
\end{align*}
\end{proof}

% Suppose surface size $S = 2^s$ at rank $R$, with $r = \left\lceil \log_2 R \right\rceil$.
% Note that $2 \times R \leq 2^r$.
% In order for the steady algorithm to work, we need to have all values with hanoi value greater than or equal to
% \begin{align*}
% \max(r - s, 0)
% \end{align*}


% We will have seen 1 of hv $r - 1$, 2 of hv $r - 2$, 4 of hv $r - 4$, and $2 ^ {s - 1}$ of hv $\max(r - s, 0)$.
% In general, we will have seen $\left\lfloor 2 ^ (r - h) \right\rfloor$ instances of hanoi value $h$ by epoch $r$.
% Recall $\sum_{i = 0}^{q} 2^i = 2 ^ {q + 1} - 1$.
% Do we have enough space?
% \begin{align*}
% \sum_{i = 0}^{(r - 1) - \max(r - s, 0)} 2 ^ i \\
% &= \sum_{i = 0}^{\min(r - 1, s - 1)} 2 ^ i \\
% &\leq \sum_{i = 0}^{s - 1} 2 ^ i \\
% &\leq 2^s - 1\\
% &\stackrel{\checkmark}{\leq} 2^s
% \end{align*}

% Need to prove:
% \begin{itemize}
% \item the number of reservation slots in $j$th reservation equals the number of hanoi value instances observed during any window where $r = k$.
% \item note, because we always fill completely we will therefore drop instances of hanoi value $r - s - 1$.
% \item that hanoi value $r - s - 1$ is in the $h \mod n$th slot
% \end{itemize}


% The $j$th reservation has $\max(1, 2^{j - 1})$ slots.
% The number of new observations of hanoi value during epoch $r$ is $h$ is $\# h(r) -  \# h(r - 1)$.
% This is
% \begin{align*}
% \left\lfloor 2 ^ {r - h} \right\rfloor - \left\lfloor 2 ^ {r - h - 1} \right\rfloor \\
% &= \left\lfloor 2 ^{r - (r - j)} \right\rfloor - \left\lfloor 2 ^ {r - (r - j) - 1} \right\rfloor \\
% &= 2^j - \left\lfloor 2 ^ {j - 1} \right\rfloor \\
% &= \min(2^j - 2^{j - 1}, 1) \\
% &= \min(2 \times 2^{j - 1} - 2^{j - 1}, 1) \\
% &\stackrel{\checkmark}{=} \min(2^{j - 1}, 1).
% \end{align*}

