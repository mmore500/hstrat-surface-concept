\section{Steady Algorithm} \label{sec:steady}

The steady criterion seeks to retain data items from time points evenly spread across observed history.
The criterion can be formulated as minimization of the largest gap size between retained data items.
For a buffer size $\colorS$ and time elapsed $\colorT$, maximum gap size can be bounded at best $\left\lceil \colorT / \colorS \right\rceil$.
This section presents a stream curation algorithm designed to support the steady criterion, achieving maximum gap size no worse than $2\left\lceil \colorT / \colorS \right\rceil$.

Figure \ref{fig:hanoi-intuition-steady} shows the proposed algorithm's core mechanism, which revolves around placing and retaining data items according to the hanoi value of their sequence index $\colorH(\colorTbar)$.
Keeping data items with hanoi value $\colorh > n$ produces gap sizes at most $\colorg \leq 2^n - 1$.
For intuition, imagine discarding items with $\colorH(\colorTbar) = 0$.
This drops every other item, and results in gap size to $\colorg \leq 1$.
Imagine subsequently removing items with $\colorH(\colorTbar) = 1$.
This again drops every other item, and increases gap size to $\colorg \leq 3$.
Pruning according to hanoi value thus provides a well-behaved transition to gracefully increase gap size while keeping even spacing.

Our goal is thus to maintain all items $\colorTbar_y$ with hanoi value $\colorH(\colorTbar_y)$ above some threshold $n(\colorT)$ by repeatedly discarding items $\colorTbar_x$ with lowest hanoi value $\colorH(\colorTbar_x) = n(\colorT)$.
To fill available buffer space $\colorS$, it turns out that, at each epoch $\colort$, we should keep all items with all items $\colorTbar$ with h.v. equal to or greater than the current epoch, $\colorH(\colorT) \geq \colort$.

\begin{lemma}[Data items with h.v. greater than or equal to current epoch.]
At any time $\colorT$ in epoch $\colort$, sufficient buffer space exists to store all data items with h.v. $\colorh \geq \colort$.
That is, $\left| \{\colorTbar \in \{0, \ldots, \colorT \} \ni \colorH(\colorTbar) \geq \colort \} \right| \leq \colorS$.
\end{lemma}

\begin{proof}
Because counts of encountered h.v.'s increase with time, it is sufficient to consider just $\max\{\colorT \in \colort\} = 2^{\colors} - 2$.
Recall that h.v. $\colorh $is encountered for the first time at time $\colorT = 2^{\colorh} - 1$.
So, h.v. $\colorh = \colors + \colort$ is encountered at $\colorT = 2^{\colors + \colort} - 1$.
At that time, there are $\colors$ distinct hanoi values $\colorh$ such that $\colort \leq \colorh \leq \colors + \colort$.
Summing h.v. data item counts and subtracting away $\colorh = \colors + \colort$ (which lies one time point beyond epoch $\colort$) gives
\begin{align*}
\left| \{\colorTbar \in \{0, \ldots, 2^{\colors} - 2 \} \ni \colorH(\colorTbar) \geq \colort \} \right|
&= \left| \{\colorTbar \in \{0, \ldots, 2^{\colors} - 1 \} \ni \colorH(\colorTbar) \geq \colort \} \right| - 1\\
&= 1 + \sum_{i=1}^{\colors} 2^{i - 1} - 1 = 2^{\colors} - 1 = \colorS - 1\\
&\stackrel{\checkmark}{\leq} S.
\end{align*}
\end{proof}

Although we, in fact, have one extra buffer site left over, in practice it is often useful to use this site to permanently retain the very first or the very most recent data item.


% $t = \left\lfloor x \right\rfloor_\mathrm{bin} - s + 1$


% Recall that at epoch $t$, the greatest encountered h.v. is $t + s$
% \begin{align*}
% 1 + \sum_{i=1}^{s} 2^{i - 1} = 2^s = S.
% \end{align*}

So, each epoch, all items with $\colorH(\colorT) = \colort - 1$ must be replaced by new items with higher hanoi value.
Note that, under this scheme, there are never retained items $\colorT$ such that $\colorH(\colorT) < \colort - 1$.

The remainder of the section examines how to arrange data item placement to enact the above hanoi value strategy.
We first detail the conceptual mechanism of the algorithm in managing buffer layout.
We subsequently justify the correctness and viability of aspects of the mechanism.
Then, we prove a lower bound of the performance of the algorithm in upholding the steady criterion.

\subsection{Mechanism}

\begin{figure*}
  \centering
  \includegraphics[width=\textwidth]{img/hsurf-steady-intuition}
  \caption{TODO}
  \label{fig:hsurf-steady-intuition}
\end{figure*}



\begin{figure*}[htbp!]
  \centering

\begin{minipage}{\textwidth}
  \scriptsize
  \setlength{\tabcolsep}{2.5pt}
  \begin{tabularx}{\textwidth}{
    r
    Y|Y|Y|Y|Y|Y|Y|Y|
    Y|Y|Y|Y|Y Y Y|Y
    |Y|Y|Y|Y|Y|Y|Y
    |Y|Y|Y|Y Y
    }
     { Time $\colorT$} & \textbf{0} & \textbf{1} & \textbf{2} & \textbf{3} & \textbf{4} & \textbf{5} & \textbf{6} & \textbf{7}
    & \textbf{8} & \textbf{9} & \textbf{10} & \textbf{11} & \textbf{12} %& 13 & 14 & 15
    % & 16 & 17 & 18 & 19 & 20 & 21 & 22 & 23
    &  \ldots
    & \textbf{28} & \textbf{29} & \textbf{30} & \textbf{31}
    & \textbf{32} & \textbf{33} & \textbf{34} & \textbf{35}
    & \textbf{36} & \textbf{37} & \textbf{38} & \textbf{39} & \textbf{40}
    & \ldots \\ \hline
     \rowcolor{lightgray!30}
   { Epoch $\colort$} & 0 & 0 & 0 & 0 & 0 & 0 & 0 & 0
    & 0 & 0 & 0 & 0 & 0 %& 13 & 14 & 15
    % & 16 & 17 & 18 & 19 & 20 & 21 & 22 & 23
    &  \ldots
    & 0 & 0 & 0 & 0
    & 1 & 1 & 1 & 1
    & 1 & 1 & 1 & 1 & 1
    & \ldots \\
     % \rowcolor{lightgray!30}
    { \scriptsize$\colorH(\colorT)$} & 0 & 1 & 0 & 2 & 0 & 1 & 0 & 3
    & 0 & 1 & 0 & 2 & 0 %& 13 & 14 & 15
    % & 16 & 17 & 18 & 19 & 20 & 21 & 22 & 23
    &  \ldots
    & 0 & 1 & 0 & 5
    & 0 & 1 & 0 & 2
    & 0 & 1 & 0 & 3 & 0
    & \ldots \\
    \hline
     % \rowcolor{lightgray!30}
     { \scriptsize $\colorK(\colorT)$} & \textbf{0} & \textbf{1} & \textbf{6} & \textbf{2} & \textbf{10} & \textbf{7} & \textbf{13} & \textbf{3}
     & \textbf{16} & \textbf{11} & \textbf{18} & \textbf{8} & \textbf{20} & \ldots
 & \textbf{30} & \textbf{23} & \textbf{31} & \textbf{5} & {\tiny \texttt{\textbf{null\hphantom{}}}}  % \vspace{-3ex}
 & \textbf{24} & {\tiny \texttt{\textbf{null\hphantom{}}}} & \textbf{16}
 & {\tiny \texttt{\textbf{null\hphantom{}}}} & \textbf{25} & {\tiny \texttt{\textbf{null\hphantom{}}}} & \textbf{10} & {\tiny \texttt{\textbf{null\hphantom{}}}}  &\ldots
  \end{tabularx}
\end{minipage}
\vspace{-0.5ex}
\begin{subfigure}{\textwidth}
\caption{\footnotesize Steady policy site selection $\colorK(\colorT)$ with buffer size $\colorS=32$. Ingests marked \nullval{} indicate item discarded without storing.}
\label{fig:hsurf-steady-implementation-site-selection}
\end{subfigure}
\vspace{-0.5ex}

\begin{subfigure}[b]{\linewidth}
\includegraphics[width=\linewidth]{
binder/teeplots/12/num-generations=128+reservation-mode=steady-full+surface-size=32+viz=site-reservation-by-rank-spliced-at-heatmap+ext=.png}
\vspace{-4.5ex}\caption{
  Buffer composition across time, split by epoch with data items color-coded by hanoi value $\colorH(\colorTbar)$.
}
\label{fig:hsurf-steady-implementation-schematic}
\end{subfigure}

\vspace{0.5ex}
\begin{minipage}[]{\textwidth}
 \vspace{-2pt}
  \begin{subfigure}[t]{0.65\linewidth}
    \vspace{0pt}
    \centering
  \includegraphics[width=0.88\linewidth,clip]{binder/teeplots/12/cnorm=log+num-generations=4096+surface-size=256+viz=site-ingest-depth-by-rank-heatmap+ynorm=linear+ext=.png}  % pdf cbar is scrambled
  \end{subfigure}%
  \begin{subfigure}[t]{0.35\linewidth}
  \vspace{-2pt}
  \caption{%
    \footnotesize
    Stored data item age across buffer sites for buffer size $\colorS=256$ from $\colorT=0$ to 4,096.
  }
  \label{fig:hsurf-steady-implementation-heatmap}
\end{subfigure}
\end{minipage}

  \vspace{-0.5ex}
   \begin{minipage}[]{\textwidth}
   \vspace{-2pt}
  \begin{subfigure}[t]{0.65\linewidth}
  \vspace{0pt}
    \centering
    \includegraphics[width=0.88\linewidth,clip]{binder/teeplots/12/num-generations=262144+surface-size=64+viz=stratum-persistence-dripplot+ext=.pdf}
  \end{subfigure}%
  \begin{subfigure}[t]{0.35\linewidth}
  \vspace{-2pt}
  \caption{%
    \footnotesize
    Data item retention time spans by ingestion time point for buffer size $\colorS=64$ from $\colorT=0$ to 3,000.
  }
  \label{fig:hsurf-steady-implementation-dripplot}
  \end{subfigure}
  \end{minipage}

  \vspace{-0.5ex}
 \begin{minipage}[]{\textwidth}
 \vspace{-2pt}
\begin{subfigure}[t]{0.65\linewidth}
\vspace{0pt}
  \centering
  \includegraphics[width=0.88\linewidth,clip]{binder/teeplots/12/hue=kind+surface-size=16+viz=criterion-satisfaction-lineplot+x=rank+y=steady-criterion+yscale=symlog+ext=.pdf}
\end{subfigure}%
\begin{subfigure}[t]{0.35\linewidth}
\vspace{-2pt}
\caption{%
  \footnotesize
  Steady criterion satisfaction across time points for buffer size $\colorS=16$.
}
\label{fig:hsurf-steady-implementation-satisfaction}
\end{subfigure}
\end{minipage}

\vspace{-2ex}\caption{%
  \textbf{Steady algorithm implementation.}
  \footnotesize
  Top panel \ref{fig:hsurf-steady-implementation-site-selection} enumerates initial steady policy site selection on a 32-site buffer.
  Panel \ref{fig:hsurf-steady-implementation-schematic} summarizes how data items are ingested and retained over time within a 32-site buffer, color-coded by data items' hanoi values $\colorH(\colorT)$.
  Between $\colorT=0$ and $\colorT=126$, time is segmented into epochs $\colort=0$, $\colort=1$, and $\colort=2$; strips before  each epoch show hanoi values assigned to each buffer site during that epoch.
  Time increases along the $y$ axis.
  Rectangles with small white ``$\blkhorzoval$'' symbol denote buffer site where the ingested data item from each timestep $\colorT$ is placed.
  Buffer space is split into ``reservation segments.''
  Reservation segments occur in five ``bunches'' --- (1) one 5-site reservation segment, (2) one 4-site reservation segment, (3) two 3-site segments, (4) four 2-site segments, and (5) eight 1-site segments.
  At each epoch, data items are filled into sites newly assigned for their ingestion-order hanoi value from left to right.
  In epoch $\colort=0$, all sites are filled with a first data item.
  During each subsequent epoch $\colort>0$, segments within bunch $i$ each accept one data item with h.v. $\colorh=\colort + \colors - 1 - i$.
  All newly-assigned sites were previously assigned to the overall now-lowest hanoi value $\colorh=\colort - 1$.
  In this way, all instances of the overall lowest hanoi value are overwritten each epoch.
  Note that the rightmost site $\colork=\colorS-1$ is unused.
  Heatmap panel \ref{fig:hsurf-steady-implementation-heatmap} shows the evolution of data item age at each site on a 256-bit field over the course of 4,096 time steps.
  Dripplot panel \ref{fig:hsurf-steady-implementation-dripplot} shows retention spans for 3,000 ingested time points.
  Vertical lines span durations between ingestion and elimination for data items from successive time points.
  Time points previously eliminated are marked in red.
  Lineplot panel \ref{fig:hsurf-steady-implementation-satisfaction} shows steady criterion satisfaction on a 16-bit surface over $2^{16}$ timepoints.
  Lower and upper shaded areas are best- and worst-case bounds, respectively.
  }
\label{fig:hsurf-steady-implementation}

\end{figure*}


\subsection{Justification}

Let xj bj be the functions that give position for XXXX at time $T$.
We do not provide formal, closed-form definitions of them here but they can be computed in $\mathcal{O}(1)$ time with availability of binary operators (e.g., bit mask, bit shift, bitwise logical operators).


Suppose surface size $S = 2^s$ at rank $R$, with $r = \left\lceil \log_2 R \right\rceil$.
Note that $2 \times R \leq 2^r$.
In order for the steady algorithm to work, we need to have all values with hanoi value greater than or equal to
\begin{align*}
\max(r - s, 0)
\end{align*}

Recall that the distance between hanoi value instances is $2^{h + 1}$ and the distance between a hanoi value and a greater or equal hanoi value is $2^h$.
If we have $h = \max (r - s, 0)$, then gap size is at most $2^{\max(r - s, 0)}$.
The ideal gap size would be $\max( \left\lceil R / S \right\rceil, 1)$.

Can we bound the our bound on gap size is at most twice the ideal gap size?
\begin{align*}
\frac{
  2^{\max(r - s, 0)}
}{
  \max(\left\lceil R / S \right\rceil, 1)
}
&=
\frac{
  \max(2^r / 2^s, 1)
}{
  \max(\left\lceil R / S \right\rceil, 1)
} \\
&=
\frac{
  \max(2^r / S, 1)
}{
  \max(\left\lceil R / S \right\rceil, 1)
} \\
&\leq
\frac{
  \max(2R / S, 1)
}{
  \max(\left\lceil R / S \right\rceil, 1)
} \\
&\stackrel{\checkmark}{\leq} 2.
\end{align*}

We will have seen 1 of hv $r - 1$, 2 of hv $r - 2$, 4 of hv $r - 4$, and $2 ^ {s - 1}$ of hv $\max(r - s, 0)$.
In general, we will have seen $\left\lfloor 2 ^ (r - h) \right\rfloor$ instances of hanoi value $h$ by epoch $r$.
Recall $\sum_{i = 0}^{q} 2^i = 2 ^ {q + 1} - 1$.
Do we have enough space?
\begin{align*}
\sum_{i = 0}^{(r - 1) - \max(r - s, 0)} 2 ^ i \\
&= \sum_{i = 0}^{\min(r - 1, s - 1)} 2 ^ i \\
&\leq \sum_{i = 0}^{s - 1} 2 ^ i \\
&\leq 2^s - 1\\
&\stackrel{\checkmark}{\leq} 2^s
\end{align*}

Need to prove:
\begin{itemize}
\item the number of reservation slots in $j$th reservation equals the number of hanoi value instances observed during any window where $r = k$.
\item note, because we always fill completely we will therefore drop instances of hanoi value $r - s - 1$.
\item that hanoi value $r - s - 1$ is in the $h \mod n$th slot
\end{itemize}

The hanoi value $h = r - j$ is placed in the $j$th reservation during epoch $r$.
The $j$th reservation has $s - j$ slot size.
So, at epoch $r$ in reservation $j$ the hanoi value $h(j, r)$ will overwrite hanoi value
\begin{align*}
h(j, r) - (s - j) - 1 \\
&= (r - j) - (s - j) - 1 \\
&\stackrel{\checkmark}{=} r - s - 1.
\end{align*}

The $j$th reservation has $\max(1, 2^{j - 1})$ slots.
The number of new observations of hanoi value during epoch $r$ is $h$ is $\# h(r) -  \# h(r - 1)$.
This is
\begin{align*}
\left\lfloor 2 ^ {r - h} \right\rfloor - \left\lfloor 2 ^ {r - h - 1} \right\rfloor \\
&= \left\lfloor 2 ^{r - (r - j)} \right\rfloor - \left\lfloor 2 ^ {r - (r - j) - 1} \right\rfloor \\
&= 2^j - \left\lfloor 2 ^ {j - 1} \right\rfloor \\
&= \min(2^j - 2^{j - 1}, 1) \\
&= \min(2 \times 2^{j - 1} - 2^{j - 1}, 1) \\
&\stackrel{\checkmark}{=} \min(2^{j - 1}, 1).
\end{align*}


\subsection{Criterion Satisfaction}

Take buffer size $\colorS$ at time $\colorT$.
At any one point in time, the best possible coverage is $\left\lceil \colorT / \colorS \right\rceil$.
The proposed curation algorithm achieves no worse than this by a constant factor.

\begin{theorem}[Steady Algorithm Worst-case Gap Size]
\label{thm:steady-gap-size}
Under the steady curation algorithm, no gap size exceeds $\left\lceil 2 \colorT / \colorS \right\rceil$.  %TODO check this
\end{theorem}
\begin{proof}
TODO
\end{proof}
