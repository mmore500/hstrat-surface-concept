\section{Conclusions and Further Directions} \label{sec:conclusion}

\begin{figure*}

\begin{subfigure}{\textwidth}
\includegraphics[width=\textwidth]{binder-2025-07-19-mem-perf-tilted/binder/teeplots/2025-07-19-mem-perf-tilted/col=data-type+font.family=serif+hue=metric+kind=bar+lut=False+orient=h+qosnumitems=10000+row=num-sites+viz=catplot+x=advantage+y=tilted-memory-bytes+ext=.pdf}
\caption{without lut}
\label{fig:mem-perf-tilted:nolut}
\end{subfigure}

\begin{subfigure}{\textwidth}
\includegraphics[
    width=\textwidth, trim={0 0 0 1.2cm}, clip
]{binder-2025-07-19-mem-perf-tilted/binder/teeplots/2025-07-19-mem-perf-tilted/col=data-type+font.family=serif+hue=metric+kind=bar+lut=True+orient=h+qosnumitems=10000+row=num-sites+viz=catplot+x=advantage+y=tilted-memory-bytes+ext=.pdf}
\caption{with lut}
\label{fig:mem-perf-tilted:lut}
\end{subfigure}

\caption{
\textbf{Generalized ring buffer performance characteristics.}
\footnotesize
Comparison is against saturating bucket algorithm.
Annotations report symmetric fold-improvement, calculated as $\max(x, y) / \min(x, y) - 1$.
For legibility, bar heights are symmetric fold-advantage, calculated as $1 - \min(x, y) / \max(x, y)$.
Curation quality comparisons are performed on a same-memory-footprint basis,with \textit{inf} indicating that no saturating bucket data structure fits within the memory footprint of the generalized ring buffer.
Speed comparisons are performed on a same-item-capacity basis.
}
\label{fig:mem-perf-tilted}

\end{figure*}


In this work, we have explored rolling extraction of temporally-representative online subsamples from data streams using memory-efficient methods based on generalizing the ring buffer data structure.
To enable a broader set of use cases for this approach, we extend existing work providing steady (even-spaced) retention \citep{gunther2014algorithm} to introduce an approach supporting tilted (recency-proportional) retention.
We demonstrate this new tilted algorithm to provide $\mathcal{O}(1)$ ingest and prove worst-case lower bounds on curation quality across elapsed stream history.

Our interest in generalized ring buffer approaches arises with respect to applications in resource-constrained computing contexts, such as emerging compute-in-memory hardware accelerator architectures \citep{lie2023cerebras,vasiljevic2021compute}.
To assess performance characteristics among generalized ring buffer algorithms and alternate stream curation approaches, we conducted a suite of benchmark experiments targeting a model resource-constrained platform.
We find that generalized ring buffers provide faster throughput and less memory use compared to conventional stream binning approaches.
Although fastest throughput came from trivial approaches applying intermittent strided purges, these approaches provide substantially lower quality sample composition.

\subsection{Future Work}

One limitation of this work is restriction of the tilted algorithm to $2^{\colorS} - 2$ data item ingests.
Although we expect support for $2^{\colorS} - 2$ ingests to suffice in many use cases, work remains to design behavior past this point.
Structurally, the obstacle preventing naive generalization past this point is occurence of the first instance of Hanoi value $\colorH(\colorT) = \colorS$.
At this point, available buffer space no longer suffices to allocate a reservation for every encountered \hv{}.
As such, downsampling of retained \hv's becomes necessary.
Given that the layout of \hv{} reservations matches initial layout of Gunther's steady algorithm at this point, one possibility is to apply steady curation to allocate hanoi value reservations (i.e., rather than on logical time $\colorT$ itself, as originally formulated).

Our foremost motivation for this work is application-driven, stemming from data management challenges encountered tracking history within forward-time agent-based simulations running on a memory-constrained hardware accelerator architecture \citep{moreno2024trackable}.
As these highly distributed platforms continue to mature, we anticipate efforts to harness their processing power for general-purpose HPC applications will warrant further investigations and algorithm development work around broader themes of sampling/approximation-based strategies for simulation observability.
Indeed, such approaches are foundational in many approaches used to track real-world natural systems.
Memory-efficient methods for stream curation may prove useful in this context, as well, such as for unattended or sporadically uplinked devices in distributed sensor networks, which must record incoming observation streams on an indefinite or indeterminate basis, with limited memory capacity \citep{elnahrawy2003research,jain2022survey,hadiatna2016design}.

\subsection{Algorithm Implementation}

To facilitate practical applications of work presented here, we have organized algorithm implementations for Rust, C++, Python, Zig, and the Cerebras Software Language (CSL) within the ``downstream'' software library \citep{moreno2024downstream}.
The library additionally includes flexible, language-agnostic validation tools to support the development of implementations targeting additional platforms.
% Such as high-throughput CLI tools for decoding lookup operations and extracting bulk hex data to extract time series.

From a software engineering perspective, a significant benefit of the generalized ring buffer approach is flexibility in data handling, as reconfiguration of curation policy can be achieved through a simple function-for-function substitution of control over storage site selection.
Indeed, further flexibility in enacting hybrid curation patterns may be trivially achieved by splitting buffer space and alternating site assignment between two or more site assignment algorithms.
Such flexibility and composability provide substantial benefit, independent of runtime performance characteristics highlighted elsewhere in this work.
