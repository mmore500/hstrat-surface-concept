\section{Introduction} \label{sec:introduction}

Efficient operations over data streams are critical in harnessing ever-increasing volume and velocity of data generation.
Formally, a data stream is considered to comprise a strictly-ordered sequences of read-once inputs.
Such streams' ordering may be dictated by inherently real-time processes (e.g., physical sensor inputs) or by access patterns for physical storage media (e.g., a tape archive) \citep{henzinger1998computing}.
They may also result from non-reversible computations (e.g., forward-time simulation) \citep{abdulla2004simulation,schutzel2014stream}.
Work with data streams assumes input greatly exceeds memory capacity, with streams often simply treated as unbounded \citep{jiang2006research}.
Indeed, real-world computing often requires real-time operations on a continuous, indefinite basis \citep{cordeiro2016online}.
Notable application domains involving data streams include sensor networks \citep{elnahrawy2003research}, big-data analytics \citep{he2010comet}, real-time network traffic analysis \citep{johnson2005streams,muthukrishnan2005data}, systems administration \citep{fischer2012real}, financial analytics \citep{rajeshwari2016real,agarwal2009faster}, environment monitoring \citep{hill2009real}, and astronomy \citep{graham2012data}.

Here, we focus on just one possible operation over data streams: subsampling, and develop $\mathcal{O}(1)$ operations for space-efficient curation of data subsets comprising either
\begin{enumerate}
\item even composition over sequence history (``steady''),
\item early-biased composition (``stretched''), or
\item recency-biased composition (``tilted'').
\end{enumerate}
For each algorithm, we demonstrate worst-case bounds on error in curated collection composition.
We refer to this rolling subset problem as ``data stream curation,'' which we will define next.

\subsection{Stream Curation Problem}

Our work concerns online sampling of discrete data items from a one-dimensional data stream.
In selecting retained data items, we seek to ``curate'' a collection containing samples spanning the first items ingested from the data stream through the most recently ingested items \citep{moreno2024algorithms}.
The objective, ultimately, is to preserve a representative, approximate record of stream history.

We assume curation as an online process, with new data items being ingested on an ongoing basis, with a properly curated archive also needed on a continual basis.
In practice, such fully-online curation can be necessary when either (a) stream records are consulted on a frequent basis or (b) time point(s) for which stream records are needed are not known \textit{a priori} (i.e., query- or trigger-driven events).

\begin{figure}
\begin{minipage}[t]{\linewidth}
    \begin{minipage}[]{\textwidth}
    \noindent\fcolorbox{gray!3}{gray!3}{%
    \begin{minipage}[c][0.4in]{0.04\textwidth}
    \hspace{-0.5ex}%
    \rotatebox{90}{$\colorT = 100$}
    \end{minipage}}%
    \hspace{-1.5ex}%
    {\vrule width 1pt}%
    \noindent\fcolorbox{orange!4}{orange!4}{%
    \begin{minipage}[c][0.4in]{0.45\textwidth}
    \includegraphics[trim={0 1.2cm 0 0},clip,width=\linewidth]{binder/teeplots/coverage-criteria/criterion=steady+stop=100+viz=rugplot+ext=.pdf}%
    \end{minipage}}%
    {\vrule width 1pt}%
    \noindent\fcolorbox{purple!4}{purple!4}{%
    \begin{minipage}[c][0.4in]{0.45\textwidth}
    \includegraphics[trim={0 1.2cm 0 0},clip,width=\linewidth]{binder/teeplots/coverage-criteria/criterion=tilted+stop=100+viz=rugplot+ext=.pdf}%
    \end{minipage}}%
    \end{minipage}\vspace{-1ex}

    \begin{minipage}[]{\textwidth}
    \noindent\fcolorbox{gray!11}{gray!11}{%
    \begin{minipage}[c][0.4in]{0.04\textwidth}
    \hspace{-0.5ex}%
    \rotatebox{90}{$\colorT = 50$}
    \end{minipage}}%
    \hspace{-1.5ex}%
    {\vrule width 1pt}%
    \noindent\fcolorbox{orange!10}{orange!10}{%
    \begin{minipage}[c][0.4in]{0.45\textwidth}
    \includegraphics[trim={0 0 0 0},clip,width=\linewidth]{binder/teeplots/coverage-criteria/criterion=steady+stop=50+viz=rugplot+ext=.pdf}%
    \end{minipage}}%
    {\vrule width 1pt}%
    \noindent\fcolorbox{purple!10}{purple!10}{%
    \begin{minipage}[c][0.4in]{0.45\textwidth}
    \includegraphics[trim={0 0 0 0},clip,width=\linewidth]{binder/teeplots/coverage-criteria/criterion=tilted+stop=51+viz=rugplot+ext=.pdf}%
    \end{minipage}}%
    \end{minipage}\vspace{-0.5ex}

    \begin{minipage}[]{\textwidth}
    \noindent\fcolorbox{white!100}{white!100}{%
    \begin{minipage}[b][0.2cm]{0.04\textwidth}
    \hspace{-0.5ex}%
    \rotatebox{90}{$~$}
    \end{minipage}}%
    \hspace{-1.5ex}%
    {\vrule width 1pt}%
    \noindent\fcolorbox{orange!2}{orange!2}{%
    \begin{minipage}[]{0.45\textwidth}
    \hspace{-1ex}\begin{subfigure}[b]{\linewidth}
    \caption{steady criterion}
    \label{fig:criteria-intuition-steady}
    \end{subfigure}%
    \end{minipage}}%
    {\vrule width 1pt}%
    \noindent\fcolorbox{purple!2}{purple!2}{%
    \begin{minipage}[]{0.45\textwidth}
    \hspace{-1ex}\begin{subfigure}[b]{\linewidth}
    \caption{tilted criterion}
    \label{fig:criteria-intuition-tilted}
    \end{subfigure}%
    \end{minipage}}%
    \end{minipage}
\end{minipage}%
\begin{minipage}[t]{0.04\textwidth}
~
\end{minipage}

\begin{minipage}[t]{\linewidth}
    \vspace{-2ex}%
    \caption{%
      \textbf{Steady and tilted coverage criteria.}
      \footnotesize
      Ideal distributions of ingestion time points for retained data items under the each criterion are shown at $\colorT=50$ (bottom) and $\colorT=100$ (top).
      Vertical bars represent a retained data item.
      In this illustration, collection size is 12 retained items.
      All other ingested data items have been discarded.
      The steady criterion \ref{fig:criteria-intuition-steady} seeks to minimize largest absolute gap size.
      So, ideal retention maintains items spread evenly across data stream history.
      In contrast, under the tilted criterion \ref{fig:criteria-intuition-tilted} recency-proportional gap size is to be minimized, necessitating over-retention of recent data items.
      }
    \label{fig:criteria-intuition}
\end{minipage}
\end{figure}


We consider three possible requirements on sampled data, as described above.
\textit{{Steady}} retention seeks a sample spaced uniformly across elapsed stream history.
\textit{{Stretched}} retention, by contrast, proportionally prioritizes early data items.
\textit{{Tilted}} retention proportionally prioritizes recent data items.
Formal definitons of these three criteria will be introduced in Section \ref{sec:notation-coverage}.
Related objectives appear in a variety of related data stream work, reviewed in Section \ref{sec:prior-work} \citep{aggarwal2003framework,han2005stream}.

%Each contributed policy includes indexing schemes that simultaneously support both efficient update operations and efficient storage of retained stream values in a flat array, requiring only $\mathcal{O}(1)$ storage overhead --- a single counter value.

\subsection{Applications of Stream Curation}

Efficient stream curation operations benefit a variety of use cases requiring synopses of data stream history.
A straightforward application of stream curation is in unattended or sporadically-uplinked sensor devices, which must record incoming observation streams on an indefinite or indeterminate basis, with limited memory capacity \citep{jain2022survey}.
In practice, however, even well-resourced centralized systems require thinning of full fidelity data --- raising the possibility of use cases in long-term telemetry and log management \citep{kent2006guide,miebach2002hubble}.
Checkpoint-rollback state might also be managed through stream curation in scenarios where the possibility of non-halting silent errors require support for arbitrary rollback extents \citep{aupy2013combination}.
Extensions could be imagined to support more general aggregation and approximation operations over stream history besides sampling \citep{schoellhammer2024lightweight}, although we do not directly investigate these possibilities here.

% Existing work has largely applied rolling full retention of most recent data within available buffer space \citep{fincham1995use} or dismissal of incoming data after storage reaches capacity \citep{saunders1989portable,mahzan2017design}.
% Strategies to maintain a cross-sectional time sample appear scant, although there has been some work to extend the record capacity of data loggers through application-specific online compression algorithms \citep{hadiatna2016design}.

Algorithm developments reported here originally stem from supporting work on ``\textit{hereditary stratigraphy},'' a recently-developed technique for distributed tracking of digital ancestry trees --- for instance in analysis of highly-distributed agent-based evolution simulations, content in decentralized social networks, peer-to-peer file sharing, and computer viruses \citep{moreno2022hereditary}.
Although separate from our objectives here, we will briefly motivate this particular use case of stream curation.
Hereditary stratigraphy takes a reconstruction-based approach to tracking, annotating surveiled artifacts with checkpoint data, which is extended by a new ``fingerprint'' with each copy event.
Comparing two artifacts' accreted records reveals the duration of their common ancestry, with the first mismatched fingerprints indicating the end of common descent.

Hereditary stratigraphy relies on stream curation to prevent unbounded growth of generational fingerprint records from bloating memory use.
These records can be considered as a data stream, in that they accrue piece-by-piece, on an indefinite basis.
Downsampling fingerprints saves memory, but introduces uncertainty as to the timing of lineage divergence.
To this end, the manner in which retained checkpoints are spaced across generational history is crucial to inference quality.
Recent work, for instance, finds that recency-biased ``\textit{tilted}'' retention --- as opposed to ``\textit{steady}'' retention --- maximizes phylogeny reconstruction accuracy for common evolutionary scenarios \citep{moreno2024guide}.
Minimizing per-item storage overhead is also critical to hereditary stratigraphy, with \citet{moreno2024guide} finding that single-bit checkpoint values maximize reconstruction quality (i.e., by allowing more fingerprints to be retained).


\subsection{Prior Work}

Given the broad applicability of the data stream paradigm, a rich ecosystem of algorithms exist to support many types of analysis and summarization over sequenced input --- such as rolling summary statistic calculations \citep{lin2004continuously}, on-the-fly data clustering \citep{silva2013data}, live anomaly detection \citep{cai2004maids}, and rolling event frequency estimation \citep{manku2002approximate}.
Stream curation aligns, in particular, at the intersection of two general data stream processing strategies:
\begin{enumerate}
\item sampling, where the data stream corpus is coarsened through extraction of exemplar data items \citep{sibai2016sampling}; and
\item temporal binning/windowing, where data stream content is aggregated (e.g., summarized, compressed, or sampled) with respect to discrete time spans over stream history \citep{gama2007data}.
\end{enumerate}
Stream curation further situates within the purview of logical (rather than physical) time algorithms \citep{sibai2016sampling}, as retention objectives are organized vis-a-vis sequence index rather than real time.

Owing to the fundamental role of dimension-reduction in supporting more advanced data stream operations, downsampling over on logical bins has garnered longstanding algorithmic attention.
Notably, schemes for fixed-capacity steady (``equi-segmented'') and tilted (``vari-segmented'') retention appear in \citep{zhao2005generalized}, with the latter resembling additional ``pyramidal'' and ``tilted'' time window schemes appearing elsewhere \citep{aggarwal2003framework,han2005stream}.
Despite congruities with present work in their objectives and, to an extent, curated collection composition, these references do not prescribe non-iterative layout and update procedures or emphasize minimization of representational overhead (e.g., timestamps, segment length values), as pursued here.
Work on ``amnesic approximation,' a generalized scheme for down-samping satisfying an arbitrary temporal cost function, is also notable in its closely related objectives --- although catering to a substantially more resource-intensive use case \cite{palpanas2004online}.

% To our knowledge, these previous implementations all unfold through stateful iteration, with representational overhead for each stored value ; stateless enumerations of retained set composition are original to our work in this paper.

\citet{moreno2024algorithms}, which also develops stream curation techniques in service of hereditary stratigraphy \citep{moreno2022hstrat}, should be particularly noted in relation to present work.
Whereas this earlier work also focuses on minimizing the representational footprint around stored data, algorithm design caters better to variable-capacity storage, rather than fixed-capacity.
Although configurations oriented to fixed-capacity use cases targeted herein are also explored in \citet{moreno2024algorithms}, they require a more expensive update process keeping data in sorted order and, in practice, can leave unused buffer capacity in order to guarantee bounded adherence to size limits.
Indeed, head-to-head comparison has demonstrated present algorithms as improving execution speed by order of magnitude in speed and also, in the case of tilted retention, enhancing buffer space utilization \citep{moreno2024guide,moreno2024trackable}.
In addition to these benchmark analyses --- beyond the scope of formal algorithmic exposition herein --- presented algorithms have also already seen use implementing hereditary stratigraphy for lineage tracking within very large scale agent-based evolution simulations on the 850,000 core Cerebras Wafer-Scale Engine platform \citep{moreno2024trackable}.


\subsection{Proposed Approach}

% In this work, we have developed new strategies for ``data stream curation'' --- subsampling from a rolling sequence of data items to dynamically maintain a representative cross-sample across observed time points, focusing in particular on fixed-capacity procedures amenable to resource-constrained use cases.

Our proposed approach revolves around a strong simplifying constraint: we assume a fixed number of buffer sites where items ingested from a data stream may be written and, once stored, we do not allow them to be subsequently inspected or moved.
Only one further event may occur after a data item is stored, in that it may be overwriten by a later data item.
Under this regime, the composition of retained data emerges implicitly as a consequence of overwrite order.
Put another way, curation policy is restricted to be exercised solely through ``site selection,'' picking a buffer index to write the $n$th received data item into.
(We also allow ingested data items to be discarded without storage.)

Note that this operational scheme inherently forgoes any explicit data labeling, timestamping, or other structure (e.g., pointers).
Instead, we require site selection to be computable \textit{a priori}.
As a further consequence, efficient attribution of data items' origin time now requires support for straightforward ``inverse'' decoding of a stored data item's provenance based solely on its buffer index and how many items have been ingested from the data stream.
We term this operation ``site lookup.''

\subsection{Major Results}

This paper contributes three site section algorithms, with corresponding site lookup procedures.
These algorithms differ in temporal composition of retained data items, targeting steady, stretched, and tilted distributions, respectively.
All three proposed algorithms support fast $\mathcal{O}(1)$ site selection.
Accompanying site lookip is $\mathcal{O}(\colorS)$ in decoding all $\colorS$ buffer sites' ingest times.
We provide worst-case upper bounds on curation quality across elapsed data item ingests, with the steady algorithm notable in guaranteeing performance matching best case within a factor of two.

