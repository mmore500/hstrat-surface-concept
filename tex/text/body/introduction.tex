\section{Introduction} \label{sec:introduction}

Within the domain of high-performance computing (HPC), efficient algorithms to analyze and summarize data streams are key in contending with the ever-increasing volume and velocity of data generation.
In recent years, explosive growth AI/ML workloads has fueled the emergence of a rich ecosystem of highly-capable, specialized hardware-accelerator platforms.
Although GPGPU computing has long been most prominent in this space, a broader, highly diverse set of hardware architectures are being fielded by startup market players.
Notable emerging AI/ML hardware accelerator platforms include the Cerebras Systems' Wafer-Scale Engine \citep{lie2023cerebras}, Graphcore's Intelligence Processing Unit \citep{gepner2024performance}, Tenstorrent's Tensix processors \citep{vasiljevic2021compute}, and Groq's GropChip \citep{abts2022groq}.
Pursuing an aggressive scale-out paradigm, such platforms bring hundreds, thousands, or --- in the case of the Wafer-Scale Engine --- hundreds of thousands of processing elements to bear within a single silicon die.

Although these emerging AI/ML hardware accelerators harbor immense potential for general-purpose HPC applications, realizing their benefit will require substantial adaptations to satisfy constraints imposed by their unconventional hardware architectures.
One fundamental manner in which these accelerator devices differ from typical CPU-based hardware is with respect to memory locality.
Rather than exposing a high-capacity shared memory space backed by hardware-level fetching and caching, emerging AI/ML hardware accelerator platforms tend to rely on explicit synchronization or message passing steps to relay data from device to host or between processor elements.
As such, available directly-addressable memory capacity can be very small, consisting of just processing elements' attached on-chip memory banks.
At the high end, Tenstorrent's Grayskull device and GraphCore's recent Colossus MK2 provide on-chip memory of just 1MB and 612kb, respectively, per processing element \citep{vasiljevic2021compute,gepner2024performance}.
Cerebras and Groq's hardware is most extreme in this regard, furnishing less than 50kb per processing element \citep{lie2023cerebras,abts2022groq}.

Forward-time simulations exploiting a high degree of locality (e.g., stencil-based methods) have been identified as a prime candidate for these accelerators \citep{jacquelin2022scalable,brown2023exploring,brown2024accelerating,louw2021using}.
In targeting hardware accelerator platforms, these simulation kernels must budget records of simulation state within local memory capacity and available device-to-host bandwidth.
In this context, dynamic coarsening approaches ptovide key means to buffer time series data within available bandwidth capacity.

Feasibility of such an approach, however, requires lightweight \textit{in situ} stream processing techniques capable of dynamically recording an indeterminate time interval within fixed storage capacity.
Appropriate summarization approaches are highly use-case dependent, ranging from rolling summary statistic calculations \citep{lin2004continuously}, on-the-fly data clustering \citep{silva2013data}, live anomaly detection \citep{cai2004maids}, and rolling event frequency estimation \citep{manku2002approximate}.

Here, we focus on a core general-purpose sketching operation: ``stream curation,'' or downsampling via thinning, in which a collection of data items comprising a systematic, deterministic temporal cross section of elapsed stream history is dynamically curated.
For this purpose, we explore the suitability of algorithms generalizing the common ring buffer data structure for sketching stream history in highly resource-constrained contexts \citep{gunther2014compressing}.
Our work applies a novel conceptual framework to extend state-of-the-art by enabling retention patterns prioritizing recent stream history.
We then compare performance characteristics of curation via generalized ring buffer against alternate approaches through on-hardware benchmarks measuring targeting a model resource-constrained hardware platform.

In the following sections, we formalize our definition of stream curation, describe the generalized ring buffer approach, and review other related work.

\subsection{Stream Curation}
\label{sec:stream-curation-problem}

Our work concerns online sampling of discrete data items from a one-dimensional data stream, a task we term ``stream curation.''
We assume a fixed storage capacity; thus, once reached, storing any new data item requires discarding another previously-retained retained item.
To ensure a representative approximate record of stream history, thinning decisions should ``curate'' a collection of stream items spanning the entirety of stream history.
The challenge of this task arises, when a properly curated archive is needed at all times.
In practice, such fully online curation can be necessary when either (a) stream records are consulted frequently or (b) time point(s) for which stream records are needed are not known \textit{a priori} (i.e., query- or trigger-driven events).

We consider a retained collection's coverage over history solely in terms of the timepoints (i.e., sequence positions) of retained data items.
Note that we disregard data items' semantic values under this timepoint-based framing.
We consider two possible requirements on sampled data, as described above.
\textit{{Steady}} retention seeks a sample spaced uniformly across elapsed stream history.
\textit{{Tilted}} retention proportionally prioritizes recent data items.
Related objectives appear in a variety of related data stream work, reviewed in Section \ref{sec:prior-work} \citep{aggarwal2003framework,han2005stream}.
\footnote{%
Note that stream curation should be considered orthogonal to the question of $\ell_p$ sampling (e.g., $\ell_0$ or $\ell_1$ sampling) in that the stream curation objective is to optimize for deterministic temporal balance rather than statistical properties of stochastic composition, which provides advantages in guaranteeing predictable, systematic coverage over temporal history.
With regard to $\ell_p$ sampling, we refer the reader to well-established online techniques to extract $\ell_p$-representative samples over stream distributions such as reservoir sampling, sketching, and hash-based methods \citep{gaber2005mining,muthukrishnan2005data,cormode2019lp}.
}

\begin{figure}
\begin{minipage}[t]{\linewidth}
    \begin{minipage}[]{\textwidth}
    \noindent\fcolorbox{gray!3}{gray!3}{%
    \begin{minipage}[c][0.4in]{0.04\textwidth}
    \hspace{-0.5ex}%
    \rotatebox{90}{$\colorT = 100$}
    \end{minipage}}%
    \hspace{-1.5ex}%
    {\vrule width 1pt}%
    \noindent\fcolorbox{orange!4}{orange!4}{%
    \begin{minipage}[c][0.4in]{0.45\textwidth}
    \includegraphics[trim={0 1.2cm 0 0},clip,width=\linewidth]{binder/teeplots/coverage-criteria/criterion=steady+stop=100+viz=rugplot+ext=.pdf}%
    \end{minipage}}%
    {\vrule width 1pt}%
    \noindent\fcolorbox{purple!4}{purple!4}{%
    \begin{minipage}[c][0.4in]{0.45\textwidth}
    \includegraphics[trim={0 1.2cm 0 0},clip,width=\linewidth]{binder/teeplots/coverage-criteria/criterion=tilted+stop=100+viz=rugplot+ext=.pdf}%
    \end{minipage}}%
    \end{minipage}\vspace{-1ex}

    \begin{minipage}[]{\textwidth}
    \noindent\fcolorbox{gray!11}{gray!11}{%
    \begin{minipage}[c][0.4in]{0.04\textwidth}
    \hspace{-0.5ex}%
    \rotatebox{90}{$\colorT = 50$}
    \end{minipage}}%
    \hspace{-1.5ex}%
    {\vrule width 1pt}%
    \noindent\fcolorbox{orange!10}{orange!10}{%
    \begin{minipage}[c][0.4in]{0.45\textwidth}
    \includegraphics[trim={0 0 0 0},clip,width=\linewidth]{binder/teeplots/coverage-criteria/criterion=steady+stop=50+viz=rugplot+ext=.pdf}%
    \end{minipage}}%
    {\vrule width 1pt}%
    \noindent\fcolorbox{purple!10}{purple!10}{%
    \begin{minipage}[c][0.4in]{0.45\textwidth}
    \includegraphics[trim={0 0 0 0},clip,width=\linewidth]{binder/teeplots/coverage-criteria/criterion=tilted+stop=51+viz=rugplot+ext=.pdf}%
    \end{minipage}}%
    \end{minipage}\vspace{-0.5ex}

    \begin{minipage}[]{\textwidth}
    \noindent\fcolorbox{white!100}{white!100}{%
    \begin{minipage}[b][0.2cm]{0.04\textwidth}
    \hspace{-0.5ex}%
    \rotatebox{90}{$~$}
    \end{minipage}}%
    \hspace{-1.5ex}%
    {\vrule width 1pt}%
    \noindent\fcolorbox{orange!2}{orange!2}{%
    \begin{minipage}[]{0.45\textwidth}
    \hspace{-1ex}\begin{subfigure}[b]{\linewidth}
    \caption{steady criterion}
    \label{fig:criteria-intuition-steady}
    \end{subfigure}%
    \end{minipage}}%
    {\vrule width 1pt}%
    \noindent\fcolorbox{purple!2}{purple!2}{%
    \begin{minipage}[]{0.45\textwidth}
    \hspace{-1ex}\begin{subfigure}[b]{\linewidth}
    \caption{tilted criterion}
    \label{fig:criteria-intuition-tilted}
    \end{subfigure}%
    \end{minipage}}%
    \end{minipage}
\end{minipage}%
\begin{minipage}[t]{0.04\textwidth}
~
\end{minipage}

\begin{minipage}[t]{\linewidth}
    \vspace{-2ex}%
    \caption{%
      \textbf{Steady and tilted coverage criteria.}
      \footnotesize
      Ideal distributions of ingestion time points for retained data items under the each criterion are shown at $\colorT=50$ (bottom) and $\colorT=100$ (top).
      Vertical bars represent a retained data item.
      In this illustration, collection size is 12 retained items.
      All other ingested data items have been discarded.
      The steady criterion \ref{fig:criteria-intuition-steady} seeks to minimize largest absolute gap size.
      So, ideal retention maintains items spread evenly across data stream history.
      In contrast, under the tilted criterion \ref{fig:criteria-intuition-tilted} recency-proportional gap size is to be minimized, necessitating over-retention of recent data items.
      }
    \label{fig:criteria-intuition}
\end{minipage}
\end{figure}


Figure \ref{fig:criteria-intuition} contrasts optimal sample compositions under \textit{steady} and \textit{tilted} retention.
Under steady retention, retained items should be spaced evenly across time;
under tilted retention, spacing between items should be proportional to how far back in time.
Deciding between these two approaches is application-specific.
Such an approach is relevant when recent data is more important \citep{phithakkitnukoon2010recent}.
Existing work on generalized ring buffer, reviewed below, provides steady curation.
Extensions developed on this work target tilted curation.

Formally, we frame stream curation objectives in terms of cost functions on the timepoints of discarded data items, formulated in terms of ``gap size'' $\colorG_{\colorT}(\colorTbar)$ in the curated record, the number of discarded data items around timepoint $\colorTbar$ at logical time $\colorT$.
Section \ref{sec:notation} provides a full introduction of this notation, with formal definitions.
We define the following cost function on the timepoints of discarded data items:
\begin{empheq}[left={\hspace{1.5in}\displaystyle \mathsf{cost}(\colorT) \coloneq %\empheqlbrace
}]{align}
% \begin{align}
% \mathsf{cost\_steady}(\colorT) \coloneq
\max_{\colorTbar \in [0\twodots\colorT)} \colorG_{\colorT}(\colorTbar) &&\text{for \textit{``steady''} curation,}  &&& ~ &&& ~
  \label{eqn:steady-cost}
 \\
% &\max_{\colorTbar \in [1\twodots\colorT)} \frac{\colorG_{\colorT}(\colorTbar)}{\colorTbar} &&\text{for \textit{``stretched''} curation, and}  &&& ~ &&& ~ \label{eqn:stretched-cost} \\
&\max_{\colorTbar \in [0\twodots\colorT - 1)} \frac{\colorG_{\colorT}(\colorTbar)}{\colorT - 1 - \colorTbar} &&\text{for \textit{``tilted''} curation,}  &&& ~ &&& ~ \label{eqn:tilted-cost}
\end{empheq}
% \end{align}
Under this framing, curation quality may be assessed by maintaining cost function $\mathsf{cost}(\colorT)$ below an upper bound $\mathsf{bound}(\colorT)$ across logical time $\colorT$.

\subsection{Generalized Ring Buffer}

\begin{figure}

\begin{subfigure}{0.3\linewidth}
\centering
\caption{simple ring buffer}
\label{fig:curation-ingest-lookup:ring}
\end{subfigure}
\begin{minipage}{0.6\linewidth}
\centering
\includegraphics[width=\linewidth]{img/ring-buffer-simple}
\end{minipage}

\vspace{1ex}

\noindent\rule{\linewidth}{1pt}

\begingroup\setlength{\fboxsep}{0pt}%
\noindent\fcolorbox{blue!0}{blue!10}{%
\begin{minipage}{0.02\linewidth}~\end{minipage}%
\begin{subfigure}{0.3\linewidth}
\centering
\caption{generalized ring buffer: steady curation}
\label{fig:curation-ingest-lookup:steady}
\end{subfigure}%
\begin{minipage}{0.02\linewidth}~\end{minipage}%
\noindent\fcolorbox{blue!4}{blue!4}{%
\begin{minipage}{0.02\linewidth}~\end{minipage}%
\begin{minipage}{0.6\linewidth}
\centering
\vspace{1ex}

\includegraphics[width=\linewidth]{img/ring-buffer-steady}
\vspace{1ex}
\end{minipage}%
}%
}
\endgroup

\noindent\rule{\linewidth}{1pt}

\begingroup\setlength{\fboxsep}{0pt}%
\noindent\fcolorbox{purple!0}{purple!10}{%
\begin{minipage}{0.02\linewidth}~\end{minipage}%
\begin{subfigure}{0.3\linewidth}
\centering
\caption{generalized ring buffer: tilted curation}
\label{fig:curation-ingest-lookup:tilted}
\end{subfigure}%
\begin{minipage}{0.02\linewidth}~\end{minipage}%
\noindent\fcolorbox{purple!4}{purple!4}{%
\begin{minipage}{0.02\linewidth}~\end{minipage}%
\begin{minipage}{0.6\linewidth}
\centering
\vspace{1ex}

\includegraphics[width=\linewidth]{img/ring-buffer-tilted}
\vspace{1ex}
\end{minipage}%
}%
}
\endgroup

\caption{
  {\textbf Comparison of simple ring buffer to generalized ring buffers.}
  \footnotesize
  TODO
  Both approaches accumulate items ingested from a data stream into a fixed-capacity memory buffer.
  However, whereas the ring buffer retains only most-recent items, the generalized algorithm maintains a sample spanning the entirety of elapsed history.
  We notate ingest site assignment and stored item attribution, respectively, as $\colorK$ and $\colorL$.
  For the proposed DStream tilted algorithm these operations are described in Listings and \ref{alg:tilted-site-selection} and \ref{alg:tilted-time-lookup}.
}
\label{fig:curation-ingest-lookup}
\end{figure}
% https://docs.google.com/presentation/d/1-7eLWTSH16s1MpGyT0iTJGAVLEYdKM7hK3ao2y_bW9A


Ring buffers are a ubiquitous and well-known data structure that provides a fast, space-efficient, and simple mechanism to keep last-$n$ samples on a rolling basis \citep{dusseau2018operating}.
Where $\colorK(\colorT) = \colorT \bmod \colorS$, where $\colorT$ counts stored stream items and $\colorS$ indicates buffer size.
Figure \ref{fig:curation-ingest-lookup:ring} shows this pattern.
A key aspect of the simplicity of this algorithm is the coupling of storage and deletion operations once buffer space fills.

However, alternate sample compositions can be
\citet{gunther2014compressing} proposes generalizing the ring buffer by using some other $\colorK(\colorT)$ (i.e., $\neq \colorT \bmod \colorS$)
This approach retains some of the attractive qualities of ring buffers.
- fixed-capacity, contiguous memory
- strict constant-time update operations,
- full use of memory (no unused or overhead),
- simple implementation, determined by a stateless function $\colorK(\colorT)$

Although lesser-known within the applied stream processing community, the generalized ring buffer approach offers several key advantages to resource-constrained contexts:
One limitation of that work is in.
Here, we develop an alternate conceptual framework around

This approach retains the advantageous characteristics of the ring buffer approach noted above.
These characteristics are valuable in the resource-constrained environments and streamline the amount of data that needs to be shuttled.
 dynamic on-device processing of

In this vein,  .

We assess how
We judge
We assess
We test
 data structure proposed by  constitutes a promising possibility for resource-constrained application domains.
However, rather than ,
 contrasts this approach to a classic ring buffer with the generalized approach.


well-suited to resource-constrained
One promising lightweight technique is  .


However, this .
Compared to more widely known approaches

However, application-oriented explorations of this technique are lacking.

In this work, we introduce an extension of this approach to
Here, we focus on the.
- memory use should be a fixed amount up front
- the entire representation should be contiguous in memory, allowing for trivial serialization or RDMA access
- data should make full use of available memory, avoiding unused memory or data structure overhead
- update time should be O(1).

Although we do not explore it here, we note the opportunity for generalizations of the generalized ring buffer approach to summarization strategies beyond thinning: notably, summation \citep{gunther2014compressing}.
Extensions could be imagined to support more general aggregation and approximation operations over stream history besides sampling \citep{schoellhammer2024lightweight}, although we do not directly investigate these possibilities here.
\citet{gunther2014compressing}:
(1) retaining the sum over each aggregation interval,
(2) retaining the average over each aggregation interval,
(3) retaining the minimum over each aggregation interval,
(4) retaining the maximum over each aggregation interval,
(5) retaining a randomly selected value from each aggregation interval.

\subsection{Related Work}

Efficient operations over data streams hinge on accumulation and summarization .
distributed big-data processing \citep{he2010comet}, real-time network traffic analysis \citep{johnson2005streams,muthukrishnan2005data}, systems log management \citep{fischer2012real}, fraud monitoring \citep{rajeshwari2016real}, trading in financial markets \citep{agarwal2009faster}, environmental monitoring \citep{hill2009real}, and astronomy \citep{graham2012data}.
algorithms exist for analysis and summarization over sequenced input --- such as rolling summary statistic calculations \citep{lin2004continuously}, on-the-fly data clustering \citep{silva2013data}, live anomaly detection \citep{cai2004maids}, and rolling event frequency estimation \citep{manku2002approximate}.
Stream curation relates mostdirectly to the themes of
\begin{enumerate}
\item \textit{sampling}, where the data stream corpus is coarsened through extraction of exemplar data items \citep{sibai2016sampling}; and
\item \textit{binning/windowing}, where data stream content is aggregated (e.g., summarized, compressed, or sampled) with respect to discrete time spans over stream history \citep{gama2007data}.
\end{enumerate}

Given the broad applicability of the data stream paradigm, many algorithms exist for analysis and summarization over sequenced input --- such as rolling summary statistic calculations \citep{lin2004continuously}, on-the-fly data clustering \citep{silva2013data}, live anomaly detection \citep{cai2004maids}, and rolling event frequency estimation \citep{manku2002approximate}.
Here, we focus on a very basic core method downsampling via thinning.
\begin{enumerate}
\item \textit{sampling}, where the data stream corpus is coarsened through extraction of exemplar data items \citep{sibai2016sampling}; and
\item \textit{binning/windowing}, where data stream content is aggregated (e.g., summarized, compressed, or sampled) with respect to discrete time spans over stream history \citep{gama2007data}.
\end{enumerate}

Efficient stream curation operations benefit a variety of use cases requiring synopses of data stream history.
A straightforward application of stream curation is in unattended or sporadically uplinked sensor devices, which must record incoming observation streams on an indefinite or indeterminate basis, with limited memory capacity \citep{jain2022survey}.
In practice, however, even well-resourced centralized systems require thinning of full fidelity data --- raising the possibility of use cases in long-term telemetry and log management \citep{kent2006guide,miebach2002hubble}.
%Checkpoint-rollback state might also be managed through stream curation in scenarios where the possibility of non-halting silent errors requires support for arbitrary rollback extents \citep{aupy2013combination}.
A need for lightweight data stream processing techniques exists beyond these accelerator domains, as well, given the increasing prominence of RDMA \citep{wang2024survey} and broader growth of compute capacity relative to memory capacity.
A major advantage in having contiguous, fixed-size representation that can be dynamically exported off the chip or between processor elements.
per core expected to drop by 30\% every two years \citep{qureshi2014memory}.
Lightweight data stream processing methods are also applicable to embedded applications, such as distributed sensor networks and \citep{elnahrawy2003research},

% Existing work has largely applied rolling full retention of most recent data within available buffer space \citep{fincham1995use} or dismissal of incoming data after storage reaches capacity \citep{saunders1989portable,mahzan2017design}.
% Strategies to maintain a cross-sectional time sample appear scant, although there has been some work to extend the record capacity of data loggers through application-specific online compression algorithms \citep{hadiatna2016design}.

Algorithms reported here stem from work on ``\textit{hereditary stratigraphy},'' a recently-developed technique for tracking of digital ancestry trees in highly-distributed systems --- for instance, in analysis of many-processor agent-based evolution simulations, content in decentralized social networks, peer-to-peer file sharing, or computer viruses \citep{moreno2022hereditary}.
Although beyond the scope of objectives here, we will briefly motivate this particular use case of stream curation.
Hereditary stratigraphy annotates surveilled artifacts with checkpoint data, which is extended by a new ``fingerprint'' value with each copy event.
Comparing two artifacts' accreted records reconstructs the duration of their common ancestry, with the first mismatched fingerprints signifying divergence from common descent.

This use case relies on stream curation to prevent unbounded growth of generational fingerprint records.
These records can be considered a data stream in that they accrue indefinitely, piece by piece.
Downsampling fingerprints saves memory, but introduces uncertainty in estimating the timing of lineage divergence.
For this reason, spacing of retained checkpoints across generational history is crucial to inference quality.
Minimizing per-item storage overhead is also critical to hereditary stratigraphy, with \citet{moreno2024guide} finding that single-bit checkpoint values maximize reconstruction quality (i.e., by allowing more fingerprints to be retained).
TODO: memory use is important due to memory constraints on accelerator device and due to dispersal of agents through message-passing; allowing memory and messages to be strictly fixed-length, while guaranteeing full use of assigned space, is crucual for both performance and implementation simplificiation.
Both of these concerns are prioritized in present work.

Although not covered here, in other preliminary work we have also investigated ``DStream'' algorithms that \textit{skew older} over elapsed history (``\textit{stretched}'' algorithm), or \textit{skew newer} over elapsed history (``\textit{tilted}'' algorithm) \citep{moreno2024structured}.
Together, these algorithms support a variety of use cases differing in what data is prioritized.
Also mention library.

% resource-constrained

% Formally, a data stream is considered to be composed of a strictly-ordered sequence of read-once inputs.
% Such streams' ordering may be dictated by inherently real-time processes (e.g., physical sensor inputs) or by access patterns for physical storage media (e.g., a tape archive) \citep{henzinger1998computing}.
% They may also result from non-reversible computations (e.g., forward-time simulation) \citep{abdulla2004simulation,schutzel2014stream}.
% Work with data streams assumes input greatly exceeds memory capacity, with streams often treated as unbounded \citep{jiang2006research}.
% Indeed, real-world computing often requires real-time operations on a continuous, indefinite basis \citep{cordeiro2016online}.

% Here, we focus specifically on subsampling over data streams and introduce an $\mathcal{O}(1)$ algorithms for space-efficient curation of data items \textit{evenly covering} elapsed history.
% We call this algorithm the \textit{steady} ``DStream'' algorithm.
% Figure \ref{fig:criteria-intuition} shows an example of ideal ideal steady retention.
% Among other results, we demonstrate worst-case bounds on error in curated collection composition under the steady algorithm.
% We refer to this rolling subset problem as ``data stream curation,'' which we will define next.

%\subsection{Stream Curation Problem}

Our work concerns online sampling of discrete data items from a one-dimensional data stream.
In selecting retained data items, we seek to ``curate'' a collection containing samples spanning the first items ingested from the data stream through the most recently ingested items \citep{moreno2024algorithms}.
The objective, ultimately, is to preserve a representative, approximate record of stream history.

We assume curation as an online process, with new data items being ingested on an ongoing basis, with a properly curated archive also needed on a continual basis.
In practice, such fully-online curation can be necessary when either (a) stream records are consulted on a frequent basis or (b) time point(s) for which stream records are needed are not known \textit{a priori} (i.e., query- or trigger-driven events).

\begin{figure}
\begin{minipage}[t]{\linewidth}
    \begin{minipage}[]{\textwidth}
    \noindent\fcolorbox{gray!3}{gray!3}{%
    \begin{minipage}[c][0.4in]{0.04\textwidth}
    \hspace{-0.5ex}%
    \rotatebox{90}{$\colorT = 100$}
    \end{minipage}}%
    \hspace{-1.5ex}%
    {\vrule width 1pt}%
    \noindent\fcolorbox{orange!4}{orange!4}{%
    \begin{minipage}[c][0.4in]{0.45\textwidth}
    \includegraphics[trim={0 1.2cm 0 0},clip,width=\linewidth]{binder/teeplots/coverage-criteria/criterion=steady+stop=100+viz=rugplot+ext=.pdf}%
    \end{minipage}}%
    {\vrule width 1pt}%
    \noindent\fcolorbox{purple!4}{purple!4}{%
    \begin{minipage}[c][0.4in]{0.45\textwidth}
    \includegraphics[trim={0 1.2cm 0 0},clip,width=\linewidth]{binder/teeplots/coverage-criteria/criterion=tilted+stop=100+viz=rugplot+ext=.pdf}%
    \end{minipage}}%
    \end{minipage}\vspace{-1ex}

    \begin{minipage}[]{\textwidth}
    \noindent\fcolorbox{gray!11}{gray!11}{%
    \begin{minipage}[c][0.4in]{0.04\textwidth}
    \hspace{-0.5ex}%
    \rotatebox{90}{$\colorT = 50$}
    \end{minipage}}%
    \hspace{-1.5ex}%
    {\vrule width 1pt}%
    \noindent\fcolorbox{orange!10}{orange!10}{%
    \begin{minipage}[c][0.4in]{0.45\textwidth}
    \includegraphics[trim={0 0 0 0},clip,width=\linewidth]{binder/teeplots/coverage-criteria/criterion=steady+stop=50+viz=rugplot+ext=.pdf}%
    \end{minipage}}%
    {\vrule width 1pt}%
    \noindent\fcolorbox{purple!10}{purple!10}{%
    \begin{minipage}[c][0.4in]{0.45\textwidth}
    \includegraphics[trim={0 0 0 0},clip,width=\linewidth]{binder/teeplots/coverage-criteria/criterion=tilted+stop=51+viz=rugplot+ext=.pdf}%
    \end{minipage}}%
    \end{minipage}\vspace{-0.5ex}

    \begin{minipage}[]{\textwidth}
    \noindent\fcolorbox{white!100}{white!100}{%
    \begin{minipage}[b][0.2cm]{0.04\textwidth}
    \hspace{-0.5ex}%
    \rotatebox{90}{$~$}
    \end{minipage}}%
    \hspace{-1.5ex}%
    {\vrule width 1pt}%
    \noindent\fcolorbox{orange!2}{orange!2}{%
    \begin{minipage}[]{0.45\textwidth}
    \hspace{-1ex}\begin{subfigure}[b]{\linewidth}
    \caption{steady criterion}
    \label{fig:criteria-intuition-steady}
    \end{subfigure}%
    \end{minipage}}%
    {\vrule width 1pt}%
    \noindent\fcolorbox{purple!2}{purple!2}{%
    \begin{minipage}[]{0.45\textwidth}
    \hspace{-1ex}\begin{subfigure}[b]{\linewidth}
    \caption{tilted criterion}
    \label{fig:criteria-intuition-tilted}
    \end{subfigure}%
    \end{minipage}}%
    \end{minipage}
\end{minipage}%
\begin{minipage}[t]{0.04\textwidth}
~
\end{minipage}

\begin{minipage}[t]{\linewidth}
    \vspace{-2ex}%
    \caption{%
      \textbf{Steady and tilted coverage criteria.}
      \footnotesize
      Ideal distributions of ingestion time points for retained data items under the each criterion are shown at $\colorT=50$ (bottom) and $\colorT=100$ (top).
      Vertical bars represent a retained data item.
      In this illustration, collection size is 12 retained items.
      All other ingested data items have been discarded.
      The steady criterion \ref{fig:criteria-intuition-steady} seeks to minimize largest absolute gap size.
      So, ideal retention maintains items spread evenly across data stream history.
      In contrast, under the tilted criterion \ref{fig:criteria-intuition-tilted} recency-proportional gap size is to be minimized, necessitating over-retention of recent data items.
      }
    \label{fig:criteria-intuition}
\end{minipage}
\end{figure}


We consider three possible requirements on sampled data, as described above.
\textit{{Steady}} retention seeks a sample spaced uniformly across elapsed stream history.
\textit{{Stretched}} retention, by contrast, proportionally prioritizes early data items.
\textit{{Tilted}} retention proportionally prioritizes recent data items.
Formal definitons of these three criteria will be introduced in Section \ref{sec:notation-coverage}.
Related objectives appear in a variety of related data stream work, reviewed in Section \ref{sec:prior-work} \citep{aggarwal2003framework,han2005stream}.

%Each contributed policy includes indexing schemes that simultaneously support both efficient update operations and efficient storage of retained stream values in a flat array, requiring only $\mathcal{O}(1)$ storage overhead --- a single counter value.

\subsection{Applications of Stream Curation}

Efficient stream curation operations benefit a variety of use cases requiring synopses of data stream history.
A straightforward application of stream curation is in unattended or sporadically-uplinked sensor devices, which must record incoming observation streams on an indefinite or indeterminate basis, with limited memory capacity \citep{jain2022survey}.
In practice, however, even well-resourced centralized systems require thinning of full fidelity data --- raising the possibility of use cases in long-term telemetry and log management \citep{kent2006guide,miebach2002hubble}.
Checkpoint-rollback state might also be managed through stream curation in scenarios where the possibility of non-halting silent errors require support for arbitrary rollback extents \citep{aupy2013combination}.
Extensions could be imagined to support more general aggregation and approximation operations over stream history besides sampling \citep{schoellhammer2024lightweight}, although we do not directly investigate these possibilities here.

% Existing work has largely applied rolling full retention of most recent data within available buffer space \citep{fincham1995use} or dismissal of incoming data after storage reaches capacity \citep{saunders1989portable,mahzan2017design}.
% Strategies to maintain a cross-sectional time sample appear scant, although there has been some work to extend the record capacity of data loggers through application-specific online compression algorithms \citep{hadiatna2016design}.

Algorithm developments reported here originally stem from supporting work on ``\textit{hereditary stratigraphy},'' a recently-developed technique for distributed tracking of digital ancestry trees --- for instance in analysis of highly-distributed agent-based evolution simulations, content in decentralized social networks, peer-to-peer file sharing, and computer viruses \citep{moreno2022hereditary}.
Although separate from our objectives here, we will briefly motivate this particular use case of stream curation.
Hereditary stratigraphy takes a reconstruction-based approach to tracking, annotating surveiled artifacts with checkpoint data, which is extended by a new ``fingerprint'' with each copy event.
Comparing two artifacts' accreted records reveals the duration of their common ancestry, with the first mismatched fingerprints indicating the end of common descent.

Hereditary stratigraphy relies on stream curation to prevent unbounded growth of generational fingerprint records from bloating memory use.
These records can be considered as a data stream, in that they accrue piece-by-piece, on an indefinite basis.
Downsampling fingerprints saves memory, but introduces uncertainty as to the timing of lineage divergence.
To this end, the manner in which retained checkpoints are spaced across generational history is crucial to inference quality.
Recent work, for instance, finds that recency-biased ``\textit{tilted}'' retention --- as opposed to ``\textit{steady}'' retention --- maximizes phylogeny reconstruction accuracy for common evolutionary scenarios \citep{moreno2024guide}.
Minimizing per-item storage overhead is also critical to hereditary stratigraphy, with \citet{moreno2024guide} finding that single-bit checkpoint values maximize reconstruction quality (i.e., by allowing more fingerprints to be retained).


%\subsection{Prior Work}

Given the broad applicability of the data stream paradigm, a rich ecosystem of algorithms exist to support many types of analysis and summarization over sequenced input --- such as rolling summary statistic calculations \citep{lin2004continuously}, on-the-fly data clustering \citep{silva2013data}, live anomaly detection \citep{cai2004maids}, and rolling event frequency estimation \citep{manku2002approximate}.
Stream curation aligns, in particular, at the intersection of two general data stream processing strategies:
\begin{enumerate}
\item sampling, where the data stream corpus is coarsened through extraction of exemplar data items \citep{sibai2016sampling}; and
\item temporal binning/windowing, where data stream content is aggregated (e.g., summarized, compressed, or sampled) with respect to discrete time spans over stream history \citep{gama2007data}.
\end{enumerate}
Stream curation further situates within the purview of logical (rather than physical) time algorithms \citep{sibai2016sampling}, as retention objectives are organized vis-a-vis sequence index rather than real time.

Owing to the fundamental role of dimension-reduction in supporting more advanced data stream operations, downsampling over on logical bins has garnered longstanding algorithmic attention.
Notably, schemes for fixed-capacity steady (``equi-segmented'') and tilted (``vari-segmented'') retention appear in \citep{zhao2005generalized}, with the latter resembling additional ``pyramidal'' and ``tilted'' time window schemes appearing elsewhere \citep{aggarwal2003framework,han2005stream}.
Despite congruities with present work in their objectives and, to an extent, curated collection composition, these references do not prescribe non-iterative layout and update procedures or emphasize minimization of representational overhead (e.g., timestamps, segment length values), as pursued here.
Work on ``amnesic approximation,' a generalized scheme for down-samping satisfying an arbitrary temporal cost function, is also notable in its closely related objectives --- although catering to a substantially more resource-intensive use case \cite{palpanas2004online}.

% To our knowledge, these previous implementations all unfold through stateful iteration, with representational overhead for each stored value ; stateless enumerations of retained set composition are original to our work in this paper.

\citet{moreno2024algorithms}, which also develops stream curation techniques in service of hereditary stratigraphy \citep{moreno2022hstrat}, should be particularly noted in relation to present work.
Whereas this earlier work also focuses on minimizing the representational footprint around stored data, algorithm design caters better to variable-capacity storage, rather than fixed-capacity.
Although configurations oriented to fixed-capacity use cases targeted herein are also explored in \citet{moreno2024algorithms}, they require a more expensive update process keeping data in sorted order and, in practice, can leave unused buffer capacity in order to guarantee bounded adherence to size limits.
Indeed, head-to-head comparison has demonstrated present algorithms as improving execution speed by order of magnitude in speed and also, in the case of tilted retention, enhancing buffer space utilization \citep{moreno2024guide,moreno2024trackable}.
In addition to these benchmark analyses --- beyond the scope of formal algorithmic exposition herein --- presented algorithms have also already seen use implementing hereditary stratigraphy for lineage tracking within very large scale agent-based evolution simulations on the 850,000 core Cerebras Wafer-Scale Engine platform \citep{moreno2024trackable}.


%\subsection{Proposed Approach}

% In this work, we have developed new strategies for ``data stream curation'' --- subsampling from a rolling sequence of data items to dynamically maintain a representative cross-sample across observed time points, focusing in particular on fixed-capacity procedures amenable to resource-constrained use cases.

Our proposed approach revolves around a strong simplifying constraint: we assume a fixed number of buffer sites where items ingested from a data stream may be written and, once stored, we do not allow them to be subsequently inspected or moved.
Only one further event may occur after a data item is stored, in that it may be overwriten by a later data item.
Under this regime, the composition of retained data emerges implicitly as a consequence of overwrite order.
Put another way, curation policy is restricted to be exercised solely through ``site selection,'' picking a buffer index to write the $n$th received data item into.
(We also allow ingested data items to be discarded without storage.)

Note that this operational scheme inherently forgoes any explicit data labeling, timestamping, or other structure (e.g., pointers).
Instead, we require site selection to be computable \textit{a priori}.
As a further consequence, efficient attribution of data items' origin time now requires support for straightforward ``inverse'' decoding of a stored data item's provenance based solely on its buffer index and how many items have been ingested from the data stream.
We term this operation ``site lookup.''

\subsection{Major Results}

This paper contributes three site section algorithms, with corresponding site lookup procedures.
These algorithms differ in temporal composition of retained data items, targeting steady, stretched, and tilted distributions, respectively.
All three proposed algorithms support fast $\mathcal{O}(1)$ site selection.
Accompanying site lookip is $\mathcal{O}(\colorS)$ in decoding all $\colorS$ buffer sites' ingest times.
We provide worst-case upper bounds on curation quality across elapsed data item ingests, with the steady algorithm notable in guaranteeing performance matching best case within a factor of two.

