\section{Preliminaries, Notations, and Terminology} \label{sec:notation}

\begin{figure*}
  \centering
  \begin{subfigure}{0.55\textwidth}
  \includegraphics[width=\textwidth]{img/surface-site-ingest}
  \caption{ingest site selection for storage}
  \label{fig:surface-site-ingest}
  \end{subfigure}%
  \hfill
  \begin{subfigure}{0.4\textwidth}
  \centering
  \includegraphics[width=\textwidth]{img/ingest-rank-calculation}
  \caption{ingested time calculation for site lookup}
  \label{fig:ingest-rank-calculation}
  \end{subfigure}
  \caption{%
  \textbf{Core stream curation algorithm operations.}
  \footnotesize
  The ingestion site selection operation (\ref{fig:surface-site-ingest}) takes the current time $\colorT$ and determines the buffer site $\colork$ to store the ingested data item.
  This operation is performed when storing data into a curated buffer, once for each data item received from the data stream.
  The ingested time calculation operation (\ref{fig:ingest-rank-calculation}) provides the previous time $\colorTbar$ when the data item present at buffer site $\colork$ was ingested, given the current time $\colorT$.
  This operation is performed when reading data from a curated buffer, in order to identify the provenance of stored data.
  Note that the ingestion time of stored data at a buffer site $\colork$ is determined by the sequence of sites that were selected by the ingestion site selection operation --- in a loose sense this calculation operation can be considered as ``decoding'' or ``inverse'' to the selection operation.
  }
  \label{fig:ingest-and-lookup}
\end{figure*}


The core function of proposed algorithms is to dynamically filter out a bounded-size subset of incoming data that, according to a desired \textbf{coverage criterion}, minimizes \textbf{time gaps} between retained data items.
Incoming data is assumed to arrive on a rolling basis, as a \textbf{data stream} comprised of sequential \textbf{data items} $v_i$.
We assume the data stream to be ephemeral (i.e., ``read once''), and refer to the act of reading an item from the data stream as \textbf{ingesting} it.
We call this scenario the \textbf{data stream curation problem}.
Pertinent examples might include rolling downsample operations on sensor readings occuring at a regular frequency, managing checkpoint stores in a rollback-enabled algorithm, or storing telemetry on a production server.

As defined, the objective of stream curation considers data items according only to their \textit{temporal context} (i.e., sequence position), and not actual contained values.
We assume data item values to be fixed size, so as to fit interchangeably in memory buffer slots, but do not consider or discuss semantic values of data items further.
For the purposes of present discussion, we assume data items to be fixed-size and do not consider the actual values of the data items further.

\subsection{Buffer Storage $\colorS$}
\label{sec:notation-buffer}

We assume a fixed \textit{number of available buffer sites}, sufficient to store $\colorS$ data items.%
\footnote{%
In associated materials, the fixed-size buffer used to store curated data items is referred to as a ``surface.''
Space-efficient solutions for the stream curation problem under extensible memory capacity have been considered in other work \citep{moreno2024algorithms}.%
}
Proposed algorithms require minimal capacity $\colorS\geq4$ and for buffer size to be an even power of two, $\colorS = 2^{\colors}$ for some integer $\colors$.
On occasion it will become necessary to refer to a specific buffer position $\colork$.
We will take a zero-indexing convention, so $0 \leq \colork < \colorS$.

We consider only one update operation on the buffer: storage of an ingested data item at a buffer site $\colork$.
As a \textit{simplifying limitation}, algorithms do not move, swap, or read already-stored data.
As such, the only way to discard a stored data item is by overwriting it.
Under this scheme, control of what data is retained and for how long occurs solely as a consequence of \textit{ingestion site selection} --- picking where (and if) to store incoming data items.
The site selection operation, schematized in Figure \ref{fig:surface-site-ingest}), takes primary focus herein.

As a space-saving optimization, we store only the data items themselves in buffer space --- no metadata (e.g., ingestion time) or data structure components (e.g., indices or pointers) are stored.
This optimization is critical, in particular, when data items are small --- such as single bits or single bytes \citep{moreno2022hereditary}.
Note, though, that, without metadata, identifying stored data items requires capability deduce ingestion time solely from buffer position.
Figure \ref{fig:ingest-rank-calculation} depicts an example \textit{ingested-time calculation} operation for data retrieval.

\subsection{Time $\colorTbar$}
\label{sec:notation-time}

We will refer to each data item's stream sequence index as its \textbf{ingestion time} $\colorTbar$ and the most recent ingestion time as the \textbf{current time} $\colorT$.
In referring to time, we will also take a zero-indexing convention, so that the first element of the data stream has ingestion time $\colorTbar=0$.
We assume $\colorT$ to be known for each data ingestion, which is simple to accomplish in practice by simply incrementing a counter each time a data item is ingested.
Because we are only concerned with the sequence order of data items, and not the values of data items themselves, we will sometimes shorthand $\colorTbar$ as referring to $v_{\colorTbar}$ (the data item ingested at that time).

In principle, data stream curation would support indefinite ingestions, $\colorT \in 0, 1, \ldots \,$.
Our proposed \textit{steady curation} algorithm, introduced below, operates in this fashion.
However, our proposed \textit{stretched} and \textit{tilted curation} algorithms accept only $2^{\colorS}$ ingestions, $\colorT \in 0, 1, \ldots, 2^{\colorS} - 1$.
We expect this capacity to suffice for many applications using even moderately-sized buffers.
For instance, a 64-site buffer suffices to ingest items continuously at 5GHz for well over 100 years.
As such, we leave behavior for stretched and tilted curation past $2^{\colorS}$ ingests to future work.

\subsection{Gap Size $\colorg$}
\label{sec:notation-gapsize}

Be reminded that \textit{coverage criteria} for retained data items considered here operate solely in terms of items' time indices $\colorTbar$ --- not their data values.
We define coverage criteria in terms of \textbf{gap sizes} in the retained record.
Formally, we define gap size as a count of consecutive data items discarded (i.e., overwritten).
Let $\colorB_{\colorT}$ denote data items retained in buffer at time $\colorT$ (including $v_{\colorT}$) and $\colorBnot_{\colorT}$ refer to data items discarded (i.e., overwritten) up to that point.
So established, gap size at record index $\colorTbar$. $0 \leq \colorTbar \leq \colorT$ follows as
\begin{align*}
\colorG_{\colorT}(\colorTbar) = \left| \{\colorTbar - m, \,\ldots, \colorTbar, \,\ldots, \colorTbar+n\} \in \colorBnot_{\colorT} \right|_\mathrm{max}.
\end{align*}
Note that if $\colorTbar \in \colorB_{\colorT}$, then $\colorG_{\colorT}(\colorTbar) = 0$.

\subsection{Coverage Criteria}
\label{sec:notation-coverage}

Algorithms cover three main coverage criteria:
\begin{enumerate}
\item \textbf{steady criterion}: seeks to maintain data items \textit{evenly covering} elapsed history (Section \ref{sec:steady}),
\item \textbf{stretched criterion}: favors \textit{older} data items proportionally to their time index $\colorTbar$ (Section \ref{sec:stretched}), and
\item \textbf{tilted criteiron}: favors \textit{newer} items proportionally to their time index recency $\colorT - \colorTbar$ (Section \ref{sec:tilted}).
\end{enumerate}
Figure \ref{fig:criteria-intuition} commpares example ideal retention distributions under each criterion.
Formal definitions are provided in its criterion's corresponding section.

\subsection{Time Hanoi Value $\colorh$}
\label{sec:notation-hanoi}

\definecolor{C0}{HTML}{1F77B4}  % tab10[0]
\definecolor{C1}{HTML}{FF7F0E}  % tab10[1]
\definecolor{C2}{HTML}{2CA02C}  % tab10[2]
\definecolor{C3}{HTML}{D62728}  % tab10[3]
\definecolor{C4}{HTML}{9467BD}  % tab10[4]
\definecolor{C5}{HTML}{8C564B}  % tab10[5]
\definecolor{C6}{HTML}{E377C2}  % tab10[6]
\definecolor{C7}{HTML}{7F7F7F}  % tab10[7]
\definecolor{C8}{HTML}{BCBD22}  % tab10[8]
\definecolor{C9}{HTML}{17BECF}  % tab10[9]
\definecolor{C10}{HTML}{A1C9F4} % pastel[0]
\definecolor{C11}{HTML}{FFB482} % pastel[1]
\definecolor{C12}{HTML}{8DE5A1} % pastel[2]
\definecolor{C13}{HTML}{FF9F9B} % pastel[3]
\definecolor{C14}{HTML}{D0BBFF} % pastel[4]
\definecolor{C15}{HTML}{DEBB9B} % pastel[5]
\newcommand{\boxnum}[2]{%
  \begingroup
  \setlength{\fboxsep}{2pt}%  % padding inside the box
  \colorbox{#1}{\textcolor{white}{\large\bfseries #2}}%
  \endgroup
}

\begin{figure*}
  \centering

\begin{subfigure}{\linewidth}
  %------------- first (τ = 1) row ----------------------------
  \noindent
  % gray label column – twice as wide & twice as tall
  \fcolorbox{gray!3}{gray!3}{%
    \begin{minipage}[c][1.68in][c]{0.03\linewidth}
      \centering
      \rotatebox{90}{\shortstack{Meta-epoch $\colortau = 1$ \footnotesize(\@ Time $\colorT = 32$)\\~}}
    \end{minipage}}%
  \hspace{-0.8ex}{\vrule width 1pt}\hspace{-0.8ex}%
  % purple panel – picture + automatic purple fill on the right
  \fcolorbox{purple!5}{purple!5}{%
    \begin{minipage}[c][1.68in][c]{0.44\linewidth}
      \hspace{-1.5ex}
      \vspace{-2ex}
      \begin{tikzpicture}[>=Stealth,decoration={brace}]
        \node[inner sep=0pt] (imgA) {\includegraphics[trim={0 0.69cm 0 0},clip,height=1.30in]{binder-hanoi-strategy-small-epoch/binder/teeplots/hanoi-strategy-small-epoch/font.family=serif+hue=hanoi-value+viz=lolliplot+what=last-n-after+x=index+y=hanoi-value+ext=.pdf}};
        \draw[->] ($(imgA.east)+(0.15,1.20)$) -- ($(imgA.east)+(-0.34,1.20)$);
        \node[inner sep=2pt,right] at ($(imgA.east)+(0.15,1.20)$) {\small \; $1\times \boxnum{C5}{5}$};
        \draw[->] ($(imgA.east)+(0.15,0.80)$) -- ($(imgA.east)+(-0.34,0.80)$);
        \node[inner sep=2pt,right] at ($(imgA.east)+(0.15,0.80)$) {\small \; $1\times \boxnum{C4}{4}$};
        \draw[->] ($(imgA.east)+(0.15,0.40)$) -- ($(imgA.east)+(-0.34,0.40)$);
        \node[inner sep=2pt,right] at ($(imgA.east)+(0.15,0.40)$) {\small \; $2\times \boxnum{C3}{3}$};
        \draw[->] ($(imgA.east)+(0.15,0)$) -- ($(imgA.east)+(-0.34,0)$);
        \node[inner sep=2pt,right] at ($(imgA.east)+(0.15,0)$) {\small \; $4\times \boxnum{C2}{2}$};
        \draw[->] ($(imgA.east)+(0.15,-0.40)$) -- ($(imgA.east)+(-0.34,-0.40)$);
        \node[inner sep=2pt,right] at ($(imgA.east)+(0.15,-0.40)$) {\small \; $4\times \boxnum{C1}{1}$};
        \draw[->] ($(imgA.east)+(0.15,-0.80)$) -- ($(imgA.east)+(-0.34,-0.80)$);
        \node[inner sep=2pt,right] at ($(imgA.east)+(0.15,-0.80)$) {\small \; $4\times \boxnum{C0}{0}$};
        \draw[line width=1.4pt,decorate,decoration={brace,amplitude=4pt}] ($(imgA.east)+(0.3,1.50)$) -- ($(imgA.east)+(1.15,1.50)$)
          node[midway,anchor=south,yshift=0.19cm,align=center,xshift=-2.8cm] {\Large Keep up to $4$ most recent items of each \hv{}\\(If $<4$ items of \hv{} seen, keep all \hv{} items.)};
      \end{tikzpicture}
      \hspace*{\fill}%
    \end{minipage}}\\[-1ex]

  %------------- second (τ = 0) row ---------------------------
  \noindent
  \fcolorbox{gray!11}{gray!11}{%
    \begin{minipage}[c][1.68in][c]{0.03\linewidth}
      \centering
      \rotatebox{90}{\shortstack{Meta-epoch $\colortau = 0$ \footnotesize(\@ Time $\colorT = 16$)\\~}}
    \end{minipage}}%
  \hspace{-0.8ex}{\vrule width 1pt}\hspace{-0.8ex}%
  \fcolorbox{purple!11}{purple!11}{%
    \begin{minipage}[c][1.68in][c]{0.44\linewidth}
      \vspace{-1.3cm}
      \hspace{-1.5ex}
      \vspace{-4ex}
      \begin{tikzpicture}[>=Stealth,decoration={brace}]
        \node[inner sep=0pt] (imgB) {\includegraphics[trim={0 0.75cm 0 0},height=1.30in]{binder-hanoi-strategy-small-epoch/binder/teeplots/hanoi-strategy-small-epoch/font.family=serif+hue=hanoi-value+viz=lolliplot+what=last-n-before+x=index+xlim=32+y=hanoi-value+ext=.pdf}};
        \draw[->] ($(imgB.east)+(0.15,0.80)$) -- ($(imgB.east)+(-2.77,0.80)$);
        \node[inner sep=2pt,right] at ($(imgB.east)+(0.15,0.80)$) {\small \; $1\times \boxnum{C4}{4}$};
        \draw[->] ($(imgB.east)+(0.15,0.40)$) -- ($(imgB.east)+(-2.77,0.40)$);
        \node[inner sep=2pt,right] at ($(imgB.east)+(0.15,0.40)$) {\small \; $1\times \boxnum{C3}{3}$};
        \draw[->] ($(imgB.east)+(0.15,0)$) -- ($(imgB.east)+(-2.77,0)$);
        \node[inner sep=2pt,right] at ($(imgB.east)+(0.15,0)$) {\small \; $2\times \boxnum{C2}{2}$};
        \draw[->] ($(imgB.east)+(0.15,-0.40)$) -- ($(imgB.east)+(-2.77,-0.40)$);
        \node[inner sep=2pt,right] at ($(imgB.east)+(0.15,-0.40)$) {\small \; $4\times \boxnum{C1}{1}$};
        \draw[->] ($(imgB.east)+(0.15,-0.80)$) -- ($(imgB.east)+(-2.77,-0.80)$);
        \node[inner sep=2pt,right] at ($(imgB.east)+(0.15,-0.80)$) {\small \; $8\times \boxnum{C0}{0}$};
        \draw[line width=1.4pt,decorate,decoration={brace,amplitude=4pt}] ($(imgB.east)+(0.3,1.10)$) -- ($(imgB.east)+(1.15,1.10)$)
          node[midway,anchor=south,yshift=0.19cm,align=center,xshift=-2.8cm] {\Large Keep up to $8$ most recent items of each \hv{}\\(If $<8$ items of \hv{} seen, keep all \hv{} items.)};
      \end{tikzpicture}
      \vspace{-0.7cm}
      \hspace*{\fill}%
    \end{minipage}}

  \caption{\footnotesize last-$n$ \hv{} retention}
  \label{fig:hanoi-intuition:hv-retention}
\end{subfigure}

\vspace{1.5ex}

\begin{subfigure}{\linewidth}

  \begin{tikzpicture}

    % ---------- first (left) plot ----------
    \node[anchor=north west, inner sep=0] (X) at (0,0){%
      \includegraphics[height=3.283in,
                       trim={0 0.55in 8.06cm 0},clip]%
                      {binder-20a-tilted-eulerian-surface/binder/%
                       teeplots/20a/color-epoch-t=black+color-metaepoch-tau=black+%
                       color-site-k=black+font.family=serif+plotter=clear_ax+surface-size=16+%
                       swap-yaxes=True+symbol-color=black+%
                       viz=site-reservation-at-ranks-heatmap+ext=.pdf}};

      \fill[gray,fill opacity=0.12] ($(X.south west) - (-0.5cm, 0.2cm)$)
                       rectangle ($(X.south west) - (-1cm, -0.52cm)$);
      \fill[gray,fill opacity=0.055] ($(X.south west) - (-0.5cm, -1.22cm)$)
                       rectangle ($(X.south west) - (-1cm, -0.52cm)$);


                       % ---------- overlays for the first plot ----------
    \begin{scope}[x={(X.south east)},y={(X.north west)}]

      \fill[purple,fill opacity=0.11] ($(X.south west) - (-1cm, 0.2cm)$)
                       rectangle ($(X.south east) - (0.7cm, 0) + (0, 0.5cm)$);
      \fill[purple, fill opacity=0.052] ($(X.south west) - (-1cm,-0.5cm)$)
                       rectangle ($(X.south east) - (0.7cm, 0) + (0, 1.2cm)$);

      % annotation rows
      \node[text width=2in] at ($(X.south west) - (0.6cm, 0) + (4.52cm, 7.8cm)$) {%
      $1 \times \boxnum{C15}{15}\; \boxnum{C14}{14}\; \boxnum{C13}{13}\;  \boxnum{C12}{12}\; \boxnum{C11}{11}\; \boxnum{C10}{10}\;\;\;\;\;\rightarrow$

      $\;\;\;\;\;\;\boxnum{C9}{9}\; \boxnum{C8}{8}\; \boxnum{C7}{7}\; \boxnum{C6}{6}\; \boxnum{C5}{5}\; \boxnum{C4}{4}\; \boxnum{C3}{3}\; \boxnum{C2}{2}\; \boxnum{C1}{1}\; \boxnum{C0}{0}$};

      \node[text width=1.5in] at ($(X.south west) - (0.6cm, 0) + (4.2cm, 6.73cm)$) {%
      \Large Keep last $1$ seen item of each \hv{}
      };

      \node[text width=2in] at ($(X.south west) - (0.55cm, 0) + (4.18cm, 3.05cm)$) {%
        $1 \times \boxnum{C8}{8}\; \boxnum{C7}{7}\;$%
        \;
        $2 \times \boxnum{C6}{6}\; \boxnum{C5}{5}\; \boxnum{C4}{4}\;%
                 \boxnum{C3}{3}\; \boxnum{C2}{2}\; \boxnum{C1}{1}\;%
                 \boxnum{C0}{0}\;\rightarrow$};

      \node[text width=1.5in] at ($(X.south west) - (0.6cm, 0) + (4.2cm, 2.1cm)$) {%
      \Large Keep up to $2$ most recent items of each \hv{} \small (Keep all \hv{} items, if fewer.)
      };

      \node[text width=2in] at ($(X.south west) - (0.6cm, 0) + (4.65cm, 0.9cm)$) {%
        $1 \times \boxnum{C5}{5}\; \boxnum{C4}{4}\;$%
        \;
        $2 \times \boxnum{C3}{3}\;$%
        \;
        $4 \times \boxnum{C2}{2}\; \boxnum{C1}{1}\; \boxnum{C0}{0}\;\rightarrow$};

      \node[text width=2in] at ($(X.south west) - (0.6cm, 0) + (4.21cm, 0.19cm)$) {%
        \;
      $1 \times \boxnum{C4}{4}\; \boxnum{C3}{3}\;$%
        \;
        $2 \times \boxnum{C2}{2}\;$%
        \;
        $4 \times \boxnum{C1}{1}\;$%
        \;
        $8 \times \boxnum{C0}{0}\;\rightarrow$};

    \end{scope}

    % ---------- second (right) plot ----------
    \node[anchor=north west, inner sep=0] (Y)
         at ($(X.north east) - (0.7cm, 0) + (0,0)$){%
      \includegraphics[height=3.7in,
                       trim={1.4cm 0 0 0},clip]%
                      {binder-20a-tilted-eulerian-surface/binder/%
                       teeplots/20a/color-epoch-t=black+color-metaepoch-tau=black+%
                       color-site-k=black+font.family=serif+surface-size=16+swap-yaxes=True+%
                       symbol-color=black+viz=site-reservation-at-ranks-heatmap+%
                       ext=.pdf}};
    % (optional) you can add more TikZ annotations that refer to Y here
  \end{tikzpicture}

  \vspace{-2.5ex}

\caption{\footnotesize time course of buffer layout; time from bottom to top}
\label{fig:hanoi-intuition:buffer-layout}

\end{subfigure}
% \begin{minipage}[]{1.42\linewidth}
% \noindent\fcolorbox{white!100}{white!100}{%
% \begin{minipage}[b][0.2cm]{0.02\linewidth}
% \hspace{-0.5ex}%
% \rotatebox{90}{$~$}
% \end{minipage}}%
% \hspace{-1.5ex}%
% {\vrule width 1pt}%
% \noindent\fcolorbox{orange!2}{orange!2}{%
% \begin{minipage}[]{0.34\linewidth}
% \centering
% \hspace{-1.2ex}\begin{subfigure}[b]{\linewidth}
% \captionsetup{justification=centering}
% \vspace{-2ex}
% \caption{steady strategy\\ \footnotesize all data items of top $n$ hanoi values}
% \label{fig:hanoi-intuition-steady}
% \end{subfigure}%
% \end{minipage}}%
% \hspace{-1ex}%
% {\vrule width 1pt}%
% \noindent\fcolorbox{pink!3}{pink!3}{%
% \begin{minipage}[]{0.31\linewidth}
% \centering
% \hspace{-1.2ex}\begin{subfigure}[b]{\linewidth}
% \captionsetup{justification=centering}
% \vspace{-2ex}
% \caption{stretched strategy\\ \footnotesize first $n$ data items of all hanoi values}
% \label{fig:hanoi-intuition-stretched}
% \end{subfigure}%
% \end{minipage}}%
% \hspace{-1ex}%
% {\vrule width 1pt}%
% \noindent\fcolorbox{purple!3}{purple!3}{%
% \begin{minipage}[]{0.31\linewidth}
% \centering
% \hspace{-1.2ex}\begin{subfigure}[b]{\linewidth}
% \captionsetup{justification=centering}
% \vspace{-2ex}
% \caption{tilted strategy\\ \footnotesize last $n$ data items of all hanoi values}
% \label{fig:hanoi-intuition-tilted}
% \end{subfigure}%
% \end{minipage}\hspace{-1ex}}%
% \end{minipage}
% \vspace{-0.1in}
\caption{%
  \textbf{Tilted-generalized ring buffer strategy.}
  \footnotesize
  Data retention is prioritized based on ``hanoi value'' (\hv) of arrival time $\colorT$ (``lollipop'' bars in upper panel).
  To achieve tilted coverage, our proposed strategy maintains the set of $n$ most-recent items for each \hv{} (green boxes in upper panel).
  As time ``epochs'' elapse, corresponding to arrival of progressively higher \hv 's, per-\hv{} $n$ shrinks.
  Lower panel shows epoch-by-epoch storage layout of sites ``reserved'' to store items of each \hv{} set, for buffer size $\colorS=16$.
  (For within-epoch detail of individual data items, see Figure \ref{fig:hsurf-tilted-implementation}.)
  In each successive ``meta-epoch'' (dashed lines) $n$ is halved across low \hv 's in a rolling fashion to store newly-arrived high-\hv items.
  Note nested composition of consecutive-\hv ``segments'' (solid lines); design is described in Figure \ref{fig:hsurf-stretched-intuition}.
  }
\label{fig:hanoi-intuition}
\end{figure*}


As shown in Figure \ref{fig:hanoi-intuition} --- and discussed in greater detail later on --- it happens that the structure of proposed algorithms heavily rely on abstractions based around data items' ingestion time $\colorTbar$.
One foundational abstraction is the corresponding value in the ``ruler'' integer sequence $\colorH(\colorTbar) = \max \{ n \in \mathbb{Z}_{\geq0} : \colorTbar \bmod 2^n = 0 \}$.%
\footnote{%
Note that, in implementation, $\colorH(\colorTbar)$ can be calculated fast $\mathcal{O}(1)$ using bit-level operations --- e.g., in Python \texttt{(~T \& T-1).bit\_length()} \citep{oeis}.
}
The first few terms of this sequence are $0,\allowbreak 1,\allowbreak 0,\allowbreak 2,\allowbreak 0,\allowbreak 1,\allowbreak 0,\allowbreak 3,\allowbreak 0,\allowbreak 1,\allowbreak 0,\allowbreak 2,\allowbreak 0,\allowbreak 1,\allowbreak 0,\allowbreak 4,\allowbreak 0,\allowbreak \,\ldots$ \quad .
This sequence appears as A007814 in the On-Line Encyclopedia of Integer Sequences \citep{oeis}.
We maintain our zero-indexing convention, so $\colorH(0) = 0$, $\colorH(1) = 1$, $\colorH(2) = 0$, etc.

We refer to $\colorH(\colorTbar) = \colorh$ as the ``\textbf{hanoi value}'' of $\colorTbar$ in reference to the famous ``tower of hanoi'' puzzle \citep{lucas1889jeux}.
On occasion, we abbreviate this term as ``h.v.''

Some intuition for the structure of the Hanoi sequence will benefit the reader.
As depicted in Figure \ref{fig:hanoi-intuition}, the hanoi sequence exhibits recursively-nested fractal structure.
Element 0 appears every 2nd entry, element 1 appears every 4th entry, and in the general case element $\colorh$ appears every $2^{\colorh+1}$th entry.
So, a hanoi value $\colorh$ appears twice as often as value $\colorh + 1$.
Remark, though, that when hanoi value $\colorh$ appears for the first time, the value $\colorh - 1$ has appeared exactly once.
So, we have seen exactly one instance of $\colorh$ and also exactly one instance of $\colorh - 1$.
At this point, the value $\colorh - 2$ has appeared exactly twice and, in general, the value $\colorh - n$ has appeared $2^{n - 1}$ tim.

\subsection{Time Epoch $\colort$}
\label{sec:notation-epoch}

Owing to h.v.-based structure, the binary order of magnitude of the elapsed time $\colorT$ plays a useful conceptual role.
We call this value $\colort \sim \left\lfloor \log_2(\colorT) \right\rfloor$ the \textbf{epoch} of $\colorT$, correcting to begin epoch $\colort=1$ at $\colorT = \colorS$ and taking $\colorT < \colorS$ (before the buffer fills) as epoch $\colort=0$.

We associate time $\colorT=0$ to epoch $\colort=0$ and treat $\colorT < \colorS$ --- before the buffer fills --- as epoch 0.

For the \textit{steady algorithm}, epoch transition occur exactly at each binary orders of magnitude (e.g., $\colorT = 16$ or $\colorT = 32$) for $\colorT \geq \colorS$.
For the \textit{stretched} and \textit{tilted algorithms}, epoch transition instead occurs immediately \textit{after} a new binary order of magnitude (e.g.,  $\colorT = 17$).
Formally, for $\colorT > \colorS$,
\begin{align*}
\colort &= \left\lfloor \log_2(\colorT) \right\rfloor - \colors + 1 \tag{steady criterion algorithm}\\
&\text{or}\\
\colort &= \left\lfloor \log_2(\colorT - 1) \right\rfloor - \colors + 1 \tag{stretched/tilted criterion algorithms}.
\end{align*}

As noted above, stretched and tilted algorithms limit consideration to $\colorT < 2^{\colorS}$.
Thus, for these algorithms,
\begin{align*}
\colort &\leq \left\lfloor\log_2(2^{\colorS} - 2)\right\rfloor - \colors + 1 \\
&\leq \colorS - \colors,
\end{align*}
assuming $\colorS \geq 2$.

\subsection{Time Meta-epoch $\colortau$}
\label{sec:notation-metaepoch}

In the case of the \textit{stretched} and \textit{tilted} algorithms, it becomes useful to organize a further abstraction on top of time epoch $\colort$: \textbf{meta-epoch} $\colortau$.
We define $\colortau=1$ as beginning at $\colort = 1$, with $\colortau=0$ for $\colort=0$.
As later motivated in Lemma \ref{thm:stretched-meta-epoch}, we define meta-epochs $\colortau\geq1$ as lasting $2^{\colortau} - 1$ epochs.
Under this definition, we have $\colortau\geq1$ as beginning at epoch
\begin{align*}
\min\{\colort \in \colortau\}
&= 1 + \sum_{i=1}^{\colortau - 1} (2^{i} - 1) \\
&= 2^{\colortau} - \colortau.
\end{align*}
For epoch $\colort$, we can therefore calculate the current meta-epoch $\colortau$ as
\begin{align*}
\colortau
=
\begin{cases}
\left\lfloor \log_2(\colort) \right\rfloor & \text{if } \colort = 2^{\left\lfloor \log_2(\colort) \right\rfloor} - \left\lfloor \log_2(\colort) \right\rfloor \\
\left\lfloor \log_2(\colort) \right\rfloor - 1 & \text{otherwise.}
\end{cases}
\end{align*}

Recalling that stretched and tilted algorithms limit $\colort \leq \colorS - \colors$, Supplementary Lemma \ref{thm:meta-epoch-bound} establishes the following upper bounds on $\colortau$,
\begin{align*}
\colortau
&\leq
\min\Big(
  \log_2(\colort + \colors),
  \log_2(\colort) + 1
\Big)
\text{ for } \colort \in 1, 2, \ldots, \colorS - \colors.
\end{align*}
Taking $\colort = \colorS - \colors$, we can also bound $\colortau$ over the stretched and tilted algorithms' domains by
\begin{align*}
\colortau \leq \colors.
\end{align*}

% \subsection{Miscellania}

% As a final piece of minutiae, let the binary floor of a value $x$ be denoted $\left\lfloor x \right\rfloor_\mathrm{bin} = 2^{\left\lfloor \log_2 x \right\rfloor}$.
