\section{Stretched Algorithm} \label{sec:stretched}

The stretched criterion favors early data items, targeting a record with gap sizes proportional to data item ingest time $\colorTbar$.
As given in Equation \ref{eqn:stretched-cost} in Section \ref{sec:stream-curation-problem}, the stretched criterion's cost function is the largest ratio of gap size to ingest time,
\begin{align*}
\max\Big\{\frac{\colorG_{\colorT}(\colorTbar)}{\colorTbar} : \colorTbar \in [1 \twodots \colorT)\Big\}.
\end{align*}
For buffer size $\colorS$ and time elapsed $\colorT$, ideal retention would space retained items so that gap size grows proportionally to $\colorTbar$.
Under such a layout, spacing between data items would scale exponentially, and --- counting from zero --- the $n$th retained data item would have ingestion time $\colorT^{n/(\colorS - 1)}$.
Deriving an approximate bound without accounting for discretization effects, gap size ratio would be minimized at best,
\begin{align}
\label{eqn:approx-gap-bound}
\frac{\colorG_{\colorT}(\colorTbar)}{\colorTbar}
&\stackrel{\sim}{\geq}
\colorT^{1/\colorS} - 1
\text{ for }
\colorTbar > 0.
\end{align}

Lemma \ref{thm:stretched-ideal-strict} works in discretization to prove a strict lower bound on gap size ratio,
\begin{align}
\frac{\colorG_{\colorT}(\colorTbar)}{\colorTbar}
\geq
\frac{
  1
}{
  1 + \colorS
  - \left\lfloor \colorS \log_{\colorT}\Big(
    (\colorT - \colorS)(\colorT^{1/\colorS} - 1) + 1
  \Big)\right\rfloor
}
\geq
\frac{
  1
}{
  1 + \colorS
}
\text{ for }
\colorTbar > 0.
\label{eqn:stretched-best}
\end{align}

This section proposes a stream curation algorithm tailored to the stretched criterion, achieving gap size ratios no worse than
\begin{align}
\frac{\colorG_{\colorT}(\colorTbar)}{\colorTbar}
&\leq
\min\Big(
  \frac{2^{\colortau + 1}}{\colorS},\;\;
  \frac{2(\colort + \colors)}{\colorS},\;\;
  \frac{4\colort}{\colorS}
\Big)
\text{ for }
\colorTbar > 0
\label{eqn:stretchednoworse}
\end{align}
over supported epochs $\colort \in [0\twodots\colorS - \colors)$.
This bound ensures gap size ratio ${\colorG_{\colorT}(\colorTbar)}/{\colorTbar} \leq 1$.
More generally, guarantees gap size ratio can be shown guaranteed within a factor of $(1 + 1/\colorS)\times\min(2\colort + 2\colors, \;\; 4\colort, \;\; 2^{\colortau + 1})$ times the optimal bound established in Equation \ref{eqn:stretched-best}.

% (X/S) / (1 / (1 + S))
% X(1 + 1/S)

\subsection{Stretched Algorithm Strategy}
\label{sec:stretched-strategy}

As with the steady algorithm, processing data items $\colorTbar$ based on their hanoi value $\colorH(\colorTbar)$ provides the backbone of our approach to stretched curation.
However, instead of keeping just the $m$ highest \hv{}'s encountered, we approximate a stretched distribution by keeping the first $n$ instances of all encountered \hv{}'s.
Figure \ref{fig:hanoi-intuition-stretched} shows how keeping the first $n$ instances of each \hv{} approximates stretched distribution.

To respect fixed buffer capacity, per-\hv{} capacity $n$ must degrade as we encounter new \hv{}'s.
We thus set out to maintain -- for a declining threshold $n(\colorT)$ --- the set of data items,
\begin{align*}
\textsf{goal\_stretched}
&\coloneq
\bigcup_{\colorh \geq 0}
\{ \colorTbar = i2^{\colorh + 1} + 2^{\colorh} - 1 \text{ for } i \in [0 \twodots n(\colorT) - 1] : \colorTbar < \colorT \}.
% \quad
% \forall \colorh \in \{ \colorH(\colorTbar) \forall \colorTbar \leq \colorT \}
\end{align*}
The set $\textsf{goal\_stretched}$ is constructed as a union of the smallest $n(\colorT)$ instances of each \hv{}, excluding those not yet encountered at current time $\colorTbar$.
By construction, $\textsf{goal\_stretched} \subset [0 \twodots \colorT)$.
Lemma \ref{thm:stretched-first-n-space} shows setting $n(\colorT) \coloneq 2^{\colors - 1 - \colortau}$ suffices to respect available buffer capacity $\colorS$.

% Note that, by the nature of the hanoi sequence, the distinct \hv{}'s encountered by time $\colorTbar$, $\{ \colorH(\colorTbar) \forall \colorTbar \leq \colorT \}$, , equals $[0 \twodots \left\lfloor \log_2\colorT + 1 \right\rfloor]\}$.

\FloatBarrier  % ensure no peeking ahead by readers!
\begin{figure*}[htbp!]
  \centering
\begin{subfigure}{0.48\textwidth}
\includegraphics[width=\textwidth, clip, trim={0 0 1.25cm 0}]{binder/teeplots/20/surface-size=16+viz=site-reservation-at-ranks-heatmap+ext=}
\caption{\footnotesize Site \hv{} reservations $\colorHcal(\colork)$ for epochs $\colort=0$ to $\colort=11$.}
\label{fig:hsurf-stretched-intuition-reservations}
\end{subfigure}%
~
~
~
\vline
~
~
~
\begin{subfigure}{0.48\textwidth}
\includegraphics[width=\textwidth, clip, trim={1.25cm 0 0 0}]{binder/teeplots/20/plotter=size+surface-size=16+viz=site-reservation-at-ranks-heatmap+ext=}
\caption{\footnotesize Initialized $r$ and mature $R$ reservation segment sizes.}
\label{fig:hsurf-stretched-intuition-reservations-size}
\end{subfigure}
  \caption{
    \textbf{Stretched algorithm strategy.}
    \footnotesize
    Left panel \ref{fig:hsurf-stretched-intuition-reservations} shows progression of \hv{} reservations $\colorHcal(\colork)$ on a buffer with size $\colorS=16$ across supported epochs $\colort \in [0\twodots\colorS - \colors)$.
    Horizontal rows track epoch $\colort$, indicated on the leftmost axis.
    The rightmost axis, in the right panel, indicates meta-epoch $\colortau$.
    Observe, for instance, that four sites, colored blue, are reserved for \hv{} $\colorh=0$ during epoch $\colort=0$.
    As shown in right panel \ref{fig:hsurf-stretched-intuition-reservations-size}, reservation segment bunches are nested recursively, with inner bunches having shorter segments.
    Reservation segments are separated by black lines in both diagrams; bunches are indicated by color code in the right diagram, with segments having same initial size $r$ belonging to the same bunch.
    As epochs elapse, segments grow from initial size $r$ to mature size $R$ and are then invaded to elimination by their larger left neighbor.
    Note how recursive nesting ensures that the shortest segments are eliminated first.
    To ensure it lasts longest, the first item with \hv{} $\colorH(\colorT) = 0$ is placed in the leftmost (and largest) segment $r=5$.
    Subsequent \hv{} instances are accomodated in segment $r=3$, the two $r=2$ segments, and then the four $r=1$ segments.
    Once available segment reservations are filled, subsequent \hv{} instances are discarded without storage.
    Because the segment sizes $r$ mirror the hanoi sequence, invasion of one site per epoch $\colort$ ensures buffer space for instances of high \hv{} as they are encountered at later $\colorT$.
  In this manner, layout approximates the first-$n$ \hv{} strategy depicted in Figure \ref{fig:hanoi-intuition-stretched}, with $n$ progressively decreasing as segments are invaded and lost.
  }
  \label{fig:hsurf-stretched-intuition}
\end{figure*}


\subsection{Stretched Algorithm Mechanism}
\label{sec:stretched-mechanism}

\begin{figure*}[htbp!]
  \centering

\begin{minipage}{\textwidth}
  \scriptsize
  \setlength{\tabcolsep}{2.5pt}
  \begin{tabularx}{\textwidth}{
    r
    Y|Y|Y|Y|Y|Y|Y|Y|
    Y|Y|Y|Y|Y Y Y|Y
    |Y|Y|Y|Y|Y|Y|Y
    |Y|Y|Y|Y Y
    }
     { Time $\colorT$} & \textbf{0} & \textbf{1} & \textbf{2} & \textbf{3} & \textbf{4} & \textbf{5} & \textbf{6} & \textbf{7}
    & \textbf{8} & \textbf{9} & \textbf{10} & \textbf{11} & \textbf{12} %& 13 & 14 & 15
    % & 16 & 17 & 18 & 19 & 20 & 21 & 22 & 23
    &  \ldots
    & \textbf{28} & \textbf{29} & \textbf{30} & \textbf{31}
    & \textbf{32} & \textbf{33} & \textbf{34} & \textbf{35}
    & \textbf{36} & \textbf{37} & \textbf{38} & \textbf{39} & \textbf{40}
    & \ldots \\ \hline
   { Epoch $\colort$} & 0 & 0 & 0 & 0 & 0 & 0 & 0 & 0
    & 0 & 0 & 0 & 0 & 0 %& 13 & 14 & 15
    % & 16 & 17 & 18 & 19 & 20 & 21 & 22 & 23
    &  \ldots
    & 0 & 0 & 0 & 0
    & 1 & 1 & 1 & 1
    & 1 & 1 & 1 & 1 & 1
    & \ldots \\
     \rowcolor{lightgray!30}
   { Meta-epoch $\colortau$} & 0 & 0 & 0 & 0 & 0 & 0 & 0 & 0
    & 0 & 0 & 0 & 0 & 0 %& 13 & 14 & 15
    % & 16 & 17 & 18 & 19 & 20 & 21 & 22 & 23
    &  \ldots
    & 0 & 0 & 0 & 0
    & 1 & 1 & 1 & 1
    & 1 & 1 & 1 & 1 & 1
    & \ldots \\
    { \scriptsize$\colorH(\colorT)$} & 0 & 1 & 0 & 2 & 0 & 1 & 0 & 3
    & 0 & 1 & 0 & 2 & 0 %& 13 & 14 & 15
    % & 16 & 17 & 18 & 19 & 20 & 21 & 22 & 23
    &  \ldots
    & 0 & 1 & 0 & 5
    & 0 & 1 & 0 & 2
    & 0 & 1 & 0 & 3 & 0
    & \ldots \\ \hline
     { \scriptsize $\colorK(\colorT)$} & \textbf{0} & \textbf{1} & \textbf{17} & \textbf{2} & \textbf{10} & \textbf{18} & \textbf{25} & \textbf{3}
     & \textbf{7} & \textbf{11} & \textbf{14} & \textbf{19} & \textbf{22} & \ldots
 & \textbf{28} & \textbf{30} & \textbf{31} & \textbf{5} & {\tiny \texttt{\textbf{null\hphantom{}}}}  % \vspace{-3ex}
 & {\tiny \texttt{\textbf{null\hphantom{}}}} & {\tiny \texttt{\textbf{null\hphantom{}}}}  & \textbf{9} & {\tiny \texttt{\textbf{null\hphantom{}}}}
 & {\tiny \texttt{\textbf{null\hphantom{}}}} & {\tiny \texttt{\textbf{null\hphantom{}}}} & \textbf{13} & {\tiny \texttt{\textbf{null\hphantom{}}}}  &\ldots
  \end{tabularx}
\end{minipage}
\begin{subfigure}{\textwidth}
\vspace{-1ex}
\caption{\footnotesize Stretched policy site selection $\colorK(\colorT)$ with buffer size $\colorS=32$. Ingests marked \nullval{} indicate item discarded without storing.}
\label{fig:hsurf-stretched-implementation-site-selection}
\end{subfigure}

\begin{subfigure}[b]{\linewidth}
\includegraphics[width=\linewidth]{
binder/teeplots/41/num-generations=128+surface-size=32+viz=site-reservation-by-rank-spliced-at-heatmap+ext=}
\vspace{-4.5ex}\caption{
  Buffer composition across time, split by epoch with data items color-coded by hanoi value of ingestion time $\colorH(\colorTbar)$.
}
\label{fig:hsurf-stretched-implementation-schematic}
\end{subfigure}

\vspace{0.5ex}
\begin{minipage}[]{\textwidth}
 \vspace{-2pt}
  \begin{subfigure}[t]{0.65\linewidth}
    \vspace{0pt}
    \centering
  \includegraphics[width=0.88\linewidth,clip]{binder/teeplots/41/cnorm=log+num-generations=4096+surface-size=256+viz=site-ingest-depth-by-rank-heatmap+ynorm=linear+ext=.png}  % pdf cbar is scrambled
  \end{subfigure}%
  \begin{subfigure}[t]{0.35\linewidth}
  \vspace{-2pt}
  \caption{%
    \footnotesize
    Stored data item age across buffer sites for buffer size $\colorS=256$ from $\colorT=0$ to 4,096.
  }
  \label{fig:hsurf-stretched-implementation-heatmap}
\end{subfigure}
\end{minipage}

  \vspace{-0.5ex}
   \begin{minipage}[]{\textwidth}
   \vspace{-2pt}
  \begin{subfigure}[t]{0.65\linewidth}
  \vspace{0pt}
    \centering
    \includegraphics[width=0.88\linewidth,clip]{binder/teeplots/41/num-generations=4096+surface-size=64+viz=stratum-persistence-dripplot+ext=}
  \end{subfigure}%
  \begin{subfigure}[t]{0.35\linewidth}
  \vspace{-2pt}
  \caption{%
    \footnotesize
    Data item retention time spans by ingestion time point for buffer size $\colorS=64$ from $\colorT=0$ to 3,000.
  }
  \label{fig:hsurf-stretched-implementation-dripplot}
  \end{subfigure}
  \end{minipage}

  \vspace{-0.5ex}
 \begin{minipage}[]{\textwidth}
 \vspace{-2pt}
\begin{subfigure}[t]{0.65\linewidth}
\vspace{0pt}
  \centering
  \includegraphics[width=0.88\linewidth,clip]{binder/teeplots/41/hue=kind+surface-size=16+viz=criterion-satisfaction-lineplot+x=rank+y=stretched-criterion+ext=}
\end{subfigure}%
\begin{subfigure}[t]{0.35\linewidth}
\vspace{-2pt}
\caption{%
  \footnotesize
  Stretched criterion satisfaction across time points for buffer size $\colorS=16$.
}
\label{fig:hsurf-stretched-implementation-satisfaction}
\end{subfigure}
\end{minipage}

\vspace{-2ex}\caption{%
  \textbf{Stretched algorithm implementation.}
  \footnotesize
  Top panel \ref{fig:hsurf-stretched-implementation-site-selection} enumerates initial stretched policy site selection on a 32-site buffer.
  Panel \ref{fig:hsurf-stretched-implementation-schematic} summarizes how data items are ingested and retained over time within a 32-site buffer, color-coded by data items' hanoi values $\colorH(\colorTbar)$.
  Between $\colorT=0$ and $\colorT=127$, time is segmented into epochs $\colort=0$, $\colort=1$, and $\colort=2$.
  Spliced-in strips show hanoi values assigned to each buffer site for the upcoming epoch, separated into ``reservation segments'' by vertical black lines.
  Reservation segments occur in five recursively nested ``bunches'' --- (1) one 6-site reservation segment, (2) one 4-site reservation segment, (3) two 3-site segments, (4) four 2-site segments, and (5) eight 1-site segments.
  At each epoch, data items are filled into sites newly assigned for their ingestion-order hanoi value from left to right.
  In epoch 0, all sites are filled with a first data item.
  At subsequent epochs, the first site of all innermost-nested segments is ``invaded'' by new high \hv{} sites added to other segments.
  When data items are placed, they remain retained until invaded by a higher-\hv{} data item.
  This process continues until only one segment remains, as shown in Figure \ref{fig:hsurf-stretched-intuition-reservations}.
  Heatmap panel \ref{fig:hsurf-stretched-implementation-heatmap} shows the evolution of data item age at each site on a 256-bit field over the course of 4,096 time steps.
  Dripplot panel \ref{fig:hsurf-stretched-implementation-dripplot} shows retention spans for 3,000 ingested time points.
  Vertical lines span durations between ingestion and elimination for data items from successive time points.
  Time points previously eliminated are marked in red, although in this case they are largely obscured by crowding in small $\colorTbar$.
  Lineplot panel \ref{fig:hsurf-stretched-implementation-satisfaction} shows stretched criterion satisfaction on a 16-bit surface over $2^{16}$ timepoints.
  Lower and upper shaded areas are best- and worst-case bounds, respectively.
  }
\label{fig:hsurf-stretched-implementation}

\end{figure*}


Be reminded that our stretched retention plan is to guarantee space for the first $n(\colorT) =  2^{\colors - 1 - \colortau}$ instances of each hanoi value.
A naive layout might reserve a full $n(\colorT)$ sites for all $2^{\colors + \colort}$ \hv{}'s $\colorh$ that have been encountered by time $\colorT$.
However, such a naive approach would exceed available buffer capacity.
For example, at $\colortau=\colort=0$,
\begin{align*}
2^{\colors - 1 - \colortau} \times 2^{\colors + \colort}
&\geq
2^{2\colors - 1}\\
&\geq
\frac{\colorS^{2}}{2}\\
&> \colorS \text{ for } \colorS > 1.
\end{align*}
A more sophisticated approach will be needed, which we develop next.

\subsubsection{Stretched Algorithm Layout at $\colort,\colortau=0$}

In motivating a more apt stretched layout strategy, begin by restricting focus to epoch $\colort=\colortau=0$, where $\colorT < \colorS$.
Assume that we assign one site to each data item $0\leq \colorTbar < \colorS$ and arrange site assignments according to \hv{} $\colorh = \colorH(\colorTbar)$.
Suppose organization of reserved sites into contiguous segments, with no two items in the same segment allowed to share the same hanoi value $\colorh$.

Under this scheme, we will have at least $\colorS/2$ segments --- one per \hv{} $\colorh=0$ instance encountered.
In constructing segments, half of these $\colorh=0$ segments can be augmented with a site to house one of the $\colorS/4$ \hv{} $\colorh=1$ data items.
We can continue, and further augment $\colorS/8$ segments with \hv{} $\colorh=2$, etc.
Continuing this pattern to place all encountered \hv{} $\colorh\leq\colors$ yields segment sizes that turn out to recapitulate the hanoi sequence.
Special-casing the largest segment, constructed segment sizes can be enumerated as
\begin{align}
\colors + 1,\;\; \colorH(0) + 1,\;\; \colorH(1) + 1,\;\; \ldots,\;\; \colorH(\colorS/2 - 2) + 1.
\label{eqn:stretched-segment-sizes}
\end{align}
These segment sizes can be shown to exactly fill available buffer space $\colorS$,
\begin{align*}
\colors + 1
+  \sum_{\colorh = 0}^{\colors - 2}
2^{\colors - 2 - \colorh} \times (\colorh + 1)
&=
\colors + 1 +
2^{\colors} - \colors - 1
\stackrel{\checkmark}{=}
\colorS.
\end{align*}

Thus far, we have only considered segment sizes --- and not discussed the arrangement of segment order within buffer space $\colorS$.
One naive approach would simply order segments by length, as previously in Section \ref{sec:steady}.
However, as we will see shortly, it turns out that adopting the hanoi sequence's natural ordering (as done in Formula \ref{eqn:stretched-segment-sizes}) better serves our objectives.
The bottom row (``epoch 0'') of Figure \ref{fig:hsurf-stretched-intuition-reservations} shows application of this layout strategy to a 32-site buffer, with segments sized and arranged directly as enumerated in Formula \ref{eqn:stretched-segment-sizes}.

\subsubsection{Stretched Algorithm Layout at $\colort,\colortau\geq1$}

What about $\colorT \geq \colorS$ (i.e., $\colort \geq 1$)?
At epoch $\colort=\colortau=0$, we have successfully guaranteed $n(\colorT) = 2^{\colors - 1 - \colortau} = \colorS / 2$ reserved sites per hanoi value.
To satisfy $\textsf{goal\_stretched}$ at $\colort=\colortau=1$, we only need to guarantee $n(\colorT) = \colorS/4$ reserved sites --- half as many as at $\colort=\colortau=0$.
So, half of our $S/2$ sites reserved to \hv{} $\colorh=0$ may be freed up.
One way to do this is by releasing all singleton segments containing \textit{only} \hv{} $\colorh=0$.

Because singleton segments intersperse all other segments, their elimination makes space for all remaining segments to ``invade'' by growing one site.
Sticking with our convention of at most one site with each \hv{} $\colorh$ per reservation segment, invading segments accrue space to host an additional high hanoi value data item.
For instance, the largest segment will grow a site reserved to \hv{} $\colorh=\colors +1$.
Two reservation sites will be added for \hv{} $\colorh=\colors - 1$, four for \hv{} $\colorh=\colors - 2$, etc. --- crucially, mirroring the incidence counts for these \hv{}'s during epoch $\colort=1$.

In subsequent epochs $\colort>1$, we can continue dissolving the smallest, innermost-nested reservation segments to grow capacity for new high-\hv{} data items.
Figure \ref{fig:hsurf-stretched-intuition-reservations} shows several steps through this ``invasion'' process on a 32-site buffer.
At final epoch $\colort=\colorS-\colors - 1$ (i.e., $\colorT \approx 2^{\colorS - 1}$), the proposed process of progressive, nested segment subsumption culminates to a single reservation segment containing one site for each \hv{} $0 \leq \colorh < \colorS$.

We will next show that meta-epochs $\colortau$, as defined earlier in Section \ref{sec:notation-metaepoch}, correspond precisely to the timing with which successive inner segments are subsumed.

\begin{lemma}[Meta-epoch Semantics]
\label{thm:stretched-meta-epoch}

The duration of meta-epochs $\colortau$, defined in Section \ref{sec:notation-metaepoch} as lasting $2^{\colortau} - 1$ epochs for $\colortau\geq1$, corresponds to the succesive epoch durations of ``invasion'' reservation-removal events.
\end{lemma}

\begin{proof}

Recall that under the stretched algorithm's proposed layout strategy, buffer space is filled without any overwrites during epoch 0.
Then, during subsequent epochs, reservations are repeatedly eliminated.
Half of reservations, designated ``invading'' reservations, grow by addition of new high-h.v. sites.
The other half of reservations are subsumed, successively losing low-h.v. sites to their invading neighbors.
Note that ``invaded'' reservations do not add high-h.v. sites --- during the invasion process, they are frozen while being eliminated.
By construction, ``invaded'' reservations are always those of smallest remaining size.
Note that, owing to the recursively nested structure of reservation layout, smallest-remaining-size reservations are always interspersed every-other position and always constitutes half of active reservations.

Because invading reservations grow by exactly one site per epoch, the number of epochs comprising a ``meta-epoch'' during which a reservation set is invaded to eliminations corresponds exactly to their reservation size at invasion outset.
So, our proof objective can be recast as determining the final, frozen-for-elimination size $R$ of reservations that took initial size $r$ at epoch $\colort=0$.
Supposing $\colortau = r$, we would have $R = |\{\colort \in \colortau\}|$.
Our goal is therefore to show $R = 2^{r} - 1$.

As already mentioned, initially-singleton $r=1$ reservations are always invaded first, at epoch $\colort=1$.
Trivially, these reservations also have $R = 1 \stackrel{\checkmark}{=} 2^1 - 1$ on account of never having the opportunity to act as invader.
Initially $r=2$ reservations are invaded next.
These reservations acted as invader during epoch $\colort=1$ so grew to $R = 3 \stackrel{\checkmark}{=} 2^2 - 1$.
Subsequent reservations $r>2$, however, grow exponentially --- owing to having invaded reservations that themselves already grew by invasion.
For instance, reservations $r=3$ begin by invading their singleton neighbors $r=1$ during .
Thus, for $r=3$,
\begin{align*}
R
&= 3 + 1 + 2 + 1\\
&\stackrel{\checkmark}{=} 2^3 - 1.
\end{align*}

This pattern generalizes as
\begin{align*}
r + \sum_{i=1}^{r-1} (r - i - 1) \times 2^{i}
&\stackrel{\checkmark}{=} 2^{r} - 1. TODO cleanup
\end{align*}

\end{proof}

\begin{corollary}[Number h.v. 0 Reservations]
\label{thm:num-hv0-reservations}
The number of buffer sites $n$ reserved to h.v. 0 is $2^{\colors-1-\colortau}$.
\end{corollary}

\begin{proof}
By construction, the number of buffer sites reserved to h.v. 0 at epoch 0 is $2^{\colors - 1}$.
Lemma \ref{thm:stretched-meta-epoch} demonstrates that meta-epoch $\colortau$ corresponds to the total number of active/completed invasion cycles.
By construction, the number of available buffer sites for a hanoi value halve when that hanoi value is invaded.
With h.v. 0 always first to be invaded, occuring at the opening epoch of each meta-epoch,
\begin{align*}
n
&= 2^{\colors - 1} \times \frac{1}{2}^{\colortau}\\
&\stackrel{\checkmark}{=} 2^{\colors - \colortau - 1}.
\end{align*}
\end{proof}

\begin{corollary}[Discarded item h.v. incidence count]
\label{thm:discarded-incidence-count}
No data item is discarded unless its h.v. incidence is greater than number of buffer sites reserved to h.v. 0.
\end{corollary}

\begin{proof}
By construction, this proposition is trivially true for hanoi values with reservation site counts greater than or equal to that for h.v. 0.
Corollary \ref{thm:num-hv0-reservations} gives this as $2^{\colors-1-\colortau}$.
However, we must consider h.v.'s fewer than $2^{\colors-1-\colortau}$ sites reserved TODO and show that the number of incidences of that h.v. encountered is less than or equal to that h.v.'s number of sites reserved.

For our next step, begin by noting that at the outset of each epoch $\colortau$, eall reservations provide a site for all h.v. $\colorh < R$, where $R$ is number of sites in the to-be-invaded reservation when frozen for elimination.
From Lemma \ref{thm:stretched-meta-epoch}, we have $R = 2^{\colortau} - 1$.
So, we can restrict consideration to $\colorh > 2^{\colortau} - 1$ or, equivalently, $\colorh \geq 2^{\colortau}$.

Recall that as a propoerty of the hanoi sequence during epoch $\colort$, we have encountered one of the highest-value reserved h.v., one of the second highest-value reserved h.v., two of the third-highest reserved h.v., etc.
Also recall that highest-value encountered h.v. increases by one per epoch.

Initial reservation sizes are laid out with sizes corresponding to hanoi values.
By construction, retained reservations grow exactly one site per epoch.
Because reservations are eliminated in increasing order of initial size $r$, we will always have the largest (position 0) reservation with a site for the single encountered instance of both the highest-value hanoi values.
We can store the two instances of the the next-smallest h.v. in the largest and second-largest reservations.
Proceeding into deeper uninvaded reservation nesting layers, reservation count doubles in sync with the number of encountered h.v. instances.
So, we can safely store all encountered h.v. instances for the top $\colors - \colortau$ encountered hanoi values.

During epoch $\colort$, the highest-encountered h.v. is $\colors + \colort$.
So, we can safely store all encountered h.v. instances for
\begin{align*}
\colorh
&\geq
\colors + \colort - (\colors - \colortau)\\
&\geq
\colort - \colortau.
\end{align*}

Combining the above, our question boils down to
\begin{align*}
2^{\colortau}
\stackrel{?}{\leq} \colort - \colortau.
\end{align*}

With $\colort \geq 2^{\colortau} - \colortau$, TODO THIS INEQUALITY IS WRONG
\begin{align*}
2^{\colortau}
\stackrel{?}{\leq} \colort - \colortau \\
\stackrel{?}{\leq} 2^{\colortau} - \colortau - \colortau \\
\stackrel{?}{\leq} 2^{\colortau} - 2\colortau.
\end{align*}

\end{proof}


With relationship between segment subsumption and meta-epoch $\colortau$ thus established, Lemma \ref{thm:stretched-discarded-incidence-count} shows that our scheme maintains reservation layout sufficient to accommodate at least $n(\colorT) = 2^{\colors - 1 - \colortau}$ items of each hanoi value.

\subsubsection{Stretched Algorithm Implementation}
\label{sec:stretched-implementation}

Having determined reservation segment layout strategy, the remaining details of site selection can be addressed succinctly.

As we encounter data items with $\colorH(\colorTbar) = \colorh$, we fill reserved sites for that item's \hv{} in descending order of initialized segment size $r$.
Among same-size segments, we simply fill from left to right.
As invasion eliminates the smallest initialized segments first, this approach guarantees retention of the oldest data items with $\colorH(\colorTbar) = \colorh$.
We may thus reinterpret Lemma \ref{thm:stretched-discarded-incidence-count} as providing guarantees on the first $n$ instances of each \hv{} retained.
Once sites reserved to \hv{} $\colorh$ fill, it is necessary to discard further instances $\colorH(\colorTbar) = \colorh$ without storage.
Figure \ref{fig:hsurf-stretched-implementation-schematic} illustrates the resulting site selection process $\colorK(\colorT)$ over epochs $\colort \in \{0,1,2\}$ on an example buffer, size $\colorS=32$.
Algorithm \ref{alg:stretched-site-selection} provides a step-by-step listing of the stretched site selection procedure $\colorK(\colorT)$, which is $\mathcal{O}(1)$.

\begin{algorithm}[H]
\caption{Stretched algorithm site selection $\colorK(\colorT)$.}
\label{alg:stretched-site-selection}
\begin{minipage}{0.53\textwidth}
    \label{alg:example}
    \hspace*{\algorithmicindent} \textbf{Input:} $\colorS \in \{2^{\mathbb{N}}\},\;\; \colorT \in \mathbb{N}$ \Comment{Buffer size and current logical time}\\
    \hspace*{\algorithmicindent} \textbf{Output:} $\in [0 \twodots \colorS - 1) \cup \{\nullval\}$ \Comment{Selected site, if any}
    \begin{algorithmic}[1]
        \State $\texttt{uint\_t} ~ ~ \colors \gets \Call{BitLength}{\colorS} - 1$
        \State $\texttt{uint\_t} ~ ~ \colort \gets \max(0,\;\; \Call{BitLength}{\colorT+1} - \colors)$ \Comment{Current epoch}
        \State $\texttt{uint\_t} ~ ~ \colorh \gets \Call{CountTrailingZeros}{\colorT + 1}$ \Comment{Current \hv{}}
        \Statex
        \State $\texttt{uint\_t} ~ ~ i \gets \Call{RightShift}{\colorT, \colorh + 1}$ \Comment{Hanoi value incidence (i.e., num seen)}
        \State $\texttt{bool\_t} ~ ~ \epsilon_{\colortau} \gets \Call{BitFloorSafe}{2\colort} \;\; > \;\; \colort + \Call{BitLength}{\colortau}$ \Comment{Correction factor}
        \State $\texttt{uint\_t} ~ ~ \colortau \gets  \Call{BitLength}{\colortau} - \Call{Bool2Int}{\epsilon_{\colortau}}$ \Comment{Current meta-epoch}
        \State $\texttt{uint\_t} ~ ~ b \gets \min(1,\;\; \Call{RightShift}{\colorS, \colortau + 1})$ \Comment{Num bunches available to \hv}
        \If{$i \geq b$} \Comment{If seen more than sites reserved to \hv{}\;\ldots}
            \State \Return \nullval \Comment{\ldots discard without storing}
        \EndIf
        \Statex
        \State $\texttt{uint\_t} ~ ~ b_l \gets i$ \Comment{Logical bunch index, in order filled \ldots}
        \Statex \Comment{\ldots i.e., increasing nestedness/decreasing init size $r$}
        \Statex
        \Statex \Comment{Need to calculate physical bunch index\ldots}
        \Statex \Comment{\ldots i.e., among bunches left-to-right in buffer space}
        \Statex
        \State $\texttt{uint\_t} ~ ~ v \gets \Call{BitLength}{b_l}$ \Comment{Nestedness depth level for physical bunch}
        \State $\texttt{uint\_t} ~ ~ w \gets \Call{RightShift}{\colorS, v} \;\; \times \;\;\Call{Bool2Int}{v > 0}$ \Comment{Num bunches spaced between bunches in same nest level}
        \State $\texttt{uint\_t} ~ ~ o \gets 2w$  \Comment{Offset of nestedness level in physical bunch order}
        \State $\texttt{uint\_t} ~ ~ p \gets b_l - \Call{BitFloorSafe}{b_l}$ \Comment{Bunch position within nestedness level}
        \State $\texttt{uint\_t} ~ ~ b_p \gets o + wp$ \Comment{Physical bunch index\ldots}
        \Statex \Comment{\ldots i.e., in left-to-right buffer space ordering}
        \Statex
        \Statex \Comment{Need to calculate buffer position of $b_p$\textsuperscript{th} bunch}
        \Statex
        \State $\texttt{uint\_t} ~ ~ \epsilon_{\colork} = \Call{Bool2Int}{b_l > 0}$  \Comment{Correction factor, 0\textsuperscript{th} bunch (i.e., bunch $r=\colors$ at site $\colork=0$)}
        \State $\texttt{uint\_t} ~ ~ \colork \gets \Call{BitCount}{2b_p +(2S - b_p)} - 1 - \epsilon_{\colork}$  \Comment{Site index of bunch}
        \Statex
        \State \Return $\colork + \colorh$ \Comment{Calculate placement site, \hv{} $\colorh$ is offset within bunch}
    \end{algorithmic}
\end{minipage}
\end{algorithm}


Stretched site lookup $\colorL(\colorT)$ is provided in supplementary material, as Algorithm \ref{alg:stretched-time-lookup}.
Reference Python implementations appear in Supplementary Listings \ref{lst:stretched_site_selection.py} and \ref{lst:stretched_time_lookup.py}, as well as accompanying tests.
The data item $\colorTbar$ present at buffer site $\colork$ at time $\colorT$ can be determined by decoding that site's segment index and checking whether (if slated) it has yet been replaced during the current epoch $\colort$.
Both site selection and ingest time calculation can be accomplished through fast $\mathcal{O}(1)$ binary operations (e.g., bit mask, bit shift, count leading zeros, popcount).

\subsection{Stretched Algorithm Criterion Satisfaction}
\label{sec:stretched-satisfaction}

In this final subsection, we establish an upper bound on gap size ratio $\colorg / \colorTbar$ for a buffer of size $\colorS$ at time $\colorT$ under the proposed stretched curation algorithm.

\begin{theorem}[Stretched Algorithm Worst-case Gap Size Ratio]
\label{thm:stretched-gap-size}
Under the stretched curation algorithm, gap size ratio is bounded
\begin{align*}
\frac{\colorG_{\colorT}(\colorTbar)}{\colorTbar}
&\leq
\min\Big(
  2^{\colortau+1-\colors}
  ,
  \frac{2(\colort + \colors)}{\colorS},
  \frac{4\colort}{\colorS}
\Big).
\end{align*}
\end{theorem}
\begin{proof}

Suppose an ingest time $\colorTbar \leq \colorT$.
What is the smallest hanoi value $\colorh$ with a guaranteed non-discarded instance $\colorTbar'$ with $\colorH(\colorTbar') = \colorh$ that captures $\colorTbar$, $\colorTbar \leq \colorTbar'$?
We can answer this question by determining the smallest $m \geq 0$ such that $m \times n \geq \colorTbar$ and $m \in 2^{\mathbb{N}}$,
\begin{align*}
m \times n
&\geq \colorTbar\\
m
&\geq \colorTbar / n\\
m
&= \left\lceil \colorTbar / n \right\rceil_{\mathrm{bin}}.
% m \leq 2\left\lfloor \colorTbar / n \right\rfloor_{\mathrm{bin}}
\end{align*}

Owing to the nested nature of the hanoi sequence, data items are retained at twice the cadence of the smallest retained hanoi value.
(The smallest h.v. items are interspersed every-other with higher h.v. items.)
So, $\colorG_{\colorT}(\colorTbar) = \left\lceil \colorTbar / n \right\rceil_{\mathrm{bin}}/2 - 1$ and gap size ratio can be bounded
\begin{align*}
\frac{\colorG_{\colorT}(\colorTbar)}{\colorTbar}
&\leq
\frac{
\frac{\left\lceil \colorTbar / n \right\rceil_{\mathrm{bin}}}{2} - 1
}{
\colorTbar
}\\
&\leq
\frac{
\frac{2\colorTbar / n}{2} - 1
}{
\colorTbar
}\\
&\leq
1 / n - 1 / \colorTbar\\
&\leq
1 / n.
\end{align*}

Substituting the expressions for per-h.v. reservation count $n$ derived in Lemma \ref{thm:stretched-discarded-incidence-count} and Theorem \ref{thm:stretched-reservation-count} gives,
\begin{align*}
  \frac{\colorG_{\colorT}(\colorTbar)}{\colorTbar}
  &\leq
  \Big[
    \max\Big(
      % \left\lceil 2^{\colors - \colortau - 1} \right\rceil,
      2^{\colors - \colortau - 1},
      \frac{\colorS}{2(\colort + \colors)},
      \frac{\colorS}{4\colort},
    \Big)
  \Big]^{-1}.
\end{align*}

Simplifying terms gives the result.

\end{proof}


During early epoch $\colort = 1$, $\colorG_{\colorT}(\colorTbar)/\colorTbar \leq 4/\colorS$.
Likewise, at the opposite extremum, $\colorG_{\colorT}(\colorTbar)/\colorTbar \leq 1$ during the last supported meta-epoch $\colortau = \colors - 1$.
Figure \ref{fig:hsurf-stretched-implementation-satisfaction} shows algorithm performance on the stretched criterion for buffer size $\colorS=16$, $\colorT \in [0\twodots 2^{\colorS} - 1)$.
