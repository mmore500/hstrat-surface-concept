\begin{abstract}
In working with data streams, it is often necessary to record elapsed history on an ongoing basis.
In memory-constrained contexts, this task requires efficient, online mechanisms to aggregate or summarize history.
To these ends, work by Gunther (2014) is notable in introducing an algorithmic generalization of the ring buffer data structure that provides a fixed-capacity, rolling subsample spread evenly across elapsed data stream history.
This approach maintains attractive properties of the naive ring buffer, including a fixed contiguous memory footprint, minimal bookkeeping memory overhead, and strict constant-time update operations.
In this work, we introduce a new ``tilted'' generalized ring buffer algorithm to enable use cases that require sampling density proportional to data recency.
As the basis for our approach, we develop a simple conceptual framework around 2-adic valuation of sequence positions.
Based on this framework, we provide upper bounds on deviations in sample composition under our approach.
Finally, our work turns to asssess the practical performance characteristics of generalized ring buffer approaches.
In on-hardware benchmark tests targeting a model resource-constrained platform, we find generalized ring buffers to outperform conventional stream binning approaches, with regard to both execution speed and memory use.
By contrast, trivial approaches applying intermittent strided purges support the highest throughput per unit time, but at the cost of lower memory efficiency and less representative sample composition.
These advantages make generalized ring buffer approaches promising for stream processing applications resource-constrained computing platforms, such as emerging many-processor compute-in-memory chips and embedded devices within distributed sensor networks.
To further practical applications of introduced algorithms, we provide accompanying open-source implementations across a variety of programming languages.
\end{abstract}
