\section{Experimental Evaluation}
\label{sec:performance}

% \begin{figure*}

\begin{subfigure}{\textwidth}
\centering
\includegraphics[width=0.92\textwidth]{binder/downstream-benchmark/binder/teeplots/2025-01-18-cpp-bench-speed/exclude=dstream.stretched_algo,dstream.tilted_algo,zhao_tilted_algo+num_items=1000000+palette=tab10-r+row=is-naive+viz=outsetgrid+x=num-sites+y=duration-per-item-ns+ext=.pdf}
\caption{per-item walltime over 1 million data ingests, including no-store simple ring buffer controls}
\label{fig:perf:speed}
\end{subfigure}

\begin{subfigure}{0.38\textwidth}
\includegraphics[width=0.9\textwidth,angle=90]{binder/downstream-benchmark/binder/teeplots/2025-01-18-cpp-bench-memory/hue=strategy+kind=bar+palette=set2+viz=catplot+x=num-sites+y=memory-bytes+ext=.pdf}
\centering
\begin{minipage}{0.9\textwidth}
\caption{net memory use for retained data, timestamps, and data structure components}
\label{fig:perf:memory}
\end{minipage}
\end{subfigure}%
\begin{subfigure}{0.62\textwidth}
\includegraphics[width=0.9\textwidth]{binder/downstream-benchmark/binder/teeplots/2025-01-18-qos-dstream-vs-naive-steady/exclude=naive steady greedy+hue=algorithm+kind=line+palette=set2+style=algorithm+viz=relplot+x=num-items-ingested+y=max-gap-size+ext=.pdf}
\centering
\begin{minipage}{0.9\textwidth}
\caption{worst-case time gap size between retained items across 10k data ingests into 64-item buffer storage}
\label{fig:perf:qos}
\end{minipage}
\end{subfigure}

\caption{
\textbf{DownStream steady curation improves speed and memory footprint, while maintaining equivalent or smaller time gaps between retained data items.}
On-hardware performance downsampling from single-bit data stream for proposed ``dstream'' algorithm is compared against existing ``naive'' approach \citep{zhao2005generalized}, which requires storage of a timestamp delta with each retained data item.
Across surveyed conditions, the dstream approach provided between $2.5\times$ and $72\times$ speedup over the naive approach, shown in subpanel \ref{fig:perf:speed}.
In fact, dstream performance closely resembles that of our minimal no-store control.
Over shorter data streams, where a smaller fraction of data items are discarded, dstream performance more closely resembles that of the ring buffer control (Figure \ref{fig:perf-num-ingests}).
Shaded bands represent bootstrapped 95\% CI over 40 replicate timings.
In the case of single-bit data items, dstream also reduces memory use --- shown in subpanel \ref{fig:perf:memory} --- more than 30-fold, owing to omission of 32-bit timestamp deltas.
Finally, subpanel \ref{fig:perf:qos} compares growth rates for maximimal time gap among retained items.
Here, dstream behavior closely matches, or slightly improves, the naive approach.
}
\label{fig:perf}
\end{figure*}


% \begin{figure*}

\begin{subfigure}{\textwidth}
\centering
\includegraphics[width=0.92\textwidth]{binder/downstream-benchmark/binder/teeplots/2025-01-18-cpp-bench-speed/exclude=dstream.stretched_algo,dstream.tilted_algo,zhao_tilted_algo+num_items=1000000+palette=tab10-r+row=is-naive+viz=outsetgrid+x=num-sites+y=duration-per-item-ns+ext=.pdf}
\caption{per-item walltime over 1 million data ingests, including no-store simple ring buffer controls}
\label{fig:perf:speed}
\end{subfigure}

\begin{subfigure}{0.38\textwidth}
\includegraphics[width=0.9\textwidth,angle=90]{binder/downstream-benchmark/binder/teeplots/2025-01-18-cpp-bench-memory/hue=strategy+kind=bar+palette=set2+viz=catplot+x=num-sites+y=memory-bytes+ext=.pdf}
\centering
\begin{minipage}{0.9\textwidth}
\caption{net memory use for retained data, timestamps, and data structure components}
\label{fig:perf:memory}
\end{minipage}
\end{subfigure}%
\begin{subfigure}{0.62\textwidth}
\includegraphics[width=0.9\textwidth]{binder/downstream-benchmark/binder/teeplots/2025-01-18-qos-dstream-vs-naive-steady/exclude=naive steady greedy+hue=algorithm+kind=line+palette=set2+style=algorithm+viz=relplot+x=num-items-ingested+y=max-gap-size+ext=.pdf}
\centering
\begin{minipage}{0.9\textwidth}
\caption{worst-case time gap size between retained items across 10k data ingests into 64-item buffer storage}
\label{fig:perf:qos}
\end{minipage}
\end{subfigure}

\caption{
\textbf{DownStream steady curation improves speed and memory footprint, while maintaining equivalent or smaller time gaps between retained data items.}
On-hardware performance downsampling from single-bit data stream for proposed ``dstream'' algorithm is compared against existing ``naive'' approach \citep{zhao2005generalized}, which requires storage of a timestamp delta with each retained data item.
Across surveyed conditions, the dstream approach provided between $2.5\times$ and $72\times$ speedup over the naive approach, shown in subpanel \ref{fig:perf:speed}.
In fact, dstream performance closely resembles that of our minimal no-store control.
Over shorter data streams, where a smaller fraction of data items are discarded, dstream performance more closely resembles that of the ring buffer control (Figure \ref{fig:perf-num-ingests}).
Shaded bands represent bootstrapped 95\% CI over 40 replicate timings.
In the case of single-bit data items, dstream also reduces memory use --- shown in subpanel \ref{fig:perf:memory} --- more than 30-fold, owing to omission of 32-bit timestamp deltas.
Finally, subpanel \ref{fig:perf:qos} compares growth rates for maximimal time gap among retained items.
Here, dstream behavior closely matches, or slightly improves, the naive approach.
}
\label{fig:perf}
\end{figure*}

% % data from https://github.com/mmore500/downstream-benchmark/blob/73f0bdd8ef5908198882dbc2c0d926d716dc8391/binder/2025-01-25-cpp-bench-speed-pico.ipynb
\begin{table}[ht]
\centering
\caption{
Xorshift benchmark timings for embedded experiments.
Table \ref{tab:perf-control-embedded} provides timings for ringbuf and discard controls.
Table \ref{tab:perf-embedded-word} provides single-word and double-word timings.
}
\label{tab:perf-embedded}
\small
\begin{tabular}{l c c c c c}
\toprule
\multirow{2}{*}{\makecell{\textit{\textbf{Stream}}\\\textit{\textbf{Contents}}}}
 & \multirow{2}{*}{\makecell{\textit{\textbf{Buffer}}\\\textit{\textbf{Capacity}}}}
 & \multicolumn{2}{c}{\textbf{Ingest Time} ($\pm$ std.\ dev.)}
 & \multirow{2}{*}{\makecell{\textbf{dstream}\\\textbf{Speedup}}} \\
\cmidrule(lr){3-4}
 & & \textit{dstream} & \textit{naive} & \\
\midrule

\multirow{4}{*}{\makecell{\textit{10{,}000}\\\textit{single-bit}\\\textit{items}}}
& \textit{64 sites}
  & \(397 \pm 2.0\)\,ns
  & \(1{,}478 \pm 3.5\)\,ns
  & \(3.7\times\) \\
& \textit{256 sites}
  & \(486 \pm 2.3\)\,ns
  & \(9{,}484 \pm 3.2\)\,ns
  & \(20\times\) \\
& \textit{1024 sites}
  & \(726 \pm 2.7\)\,ns
  & \(94{,}996 \pm 3.0\)\,ns
  & \(131\times\) \\
& \textit{4096 sites}
  & \(1{,}242 \pm 3.1\)\,ns
  & \(729{,}994 \pm 3.6\)\,ns
  & \(588\times\) \\
\cmidrule(lr){2-5}

\multirow{4}{*}{\makecell{\textit{1{,}000{,}000}\\\textit{single-bit}\\\textit{items}}}
& \textit{64 sites}
  & \(338 \pm 0.0\)\,ns
  & \(704 \pm 0.0\)\,ns
  & \(2.1\times\) \\
& \textit{256 sites}
  & \(343 \pm 0.0\)\,ns
  & \(858 \pm 0.0\)\,ns
  & \(2.5\times\) \\
& \textit{1024 sites}
  & \(361 \pm 0.0\)\,ns
  & \(3{,}080 \pm 0.0\)\,ns
  & \(8.5\times\) \\
& \textit{4096 sites}
  & \(381 \pm 0.0\)\,ns
  & \(31{,}923 \pm 0.0\)\,ns
  & \(84\times\) \\
\midrule

\multirow{4}{*}{\makecell{\textit{10{,}000}\\\textit{single-byte}\\\textit{items}}}
& \textit{64 sites}
  & \(386 \pm 3.3\)\,ns
  & \(1{,}232 \pm 3.1\)\,ns
  & \(3.2\times\) \\
& \textit{256 sites}
  & \(468 \pm 3.4\)\,ns
  & \(6{,}911 \pm 3.2\)\,ns
  & \(15\times\) \\
& \textit{1024 sites}
  & \(685 \pm 3.1\)\,ns
  & \(67{,}887 \pm 4.1\)\,ns
  & \(99\times\) \\
& \textit{4096 sites}
  & \(1{,}149 \pm 3.1\)\,ns
  & \(525{,}578 \pm 2.8\)\,ns
  & \(458\times\) \\
\cmidrule(lr){2-5}

\multirow{4}{*}{\makecell{\textit{1{,}000{,}000}\\\textit{single-byte}\\\textit{items}}}
& \textit{64 sites}
  & \(330 \pm 0.0\)\,ns
  & \(682 \pm 0.0\)\,ns
  & \(2.1\times\) \\
& \textit{256 sites}
  & \(335 \pm 0.0\)\,ns
  & \(802 \pm 0.0\)\,ns
  & \(2.4\times\) \\
& \textit{1024 sites}
  & \(352 \pm 0.0\)\,ns
  & \(2{,}371 \pm 0.0\)\,ns
  & \(6.7\times\) \\
& \textit{4096 sites}
  & \(371 \pm 0.0\)\,ns
  & \(22{,}705 \pm 0.0\)\,ns
  & \(61\times\) \\
\bottomrule
\end{tabular}
\end{table}


Given the application-oriented posture of this work, we extended our analysis to assess the practical impact of DStream curation on memory efficiency and stream processing throughput.
In this section, we report on-hardware benchmark experiments to compare the runtime behavior of DStream curation against current practice --- namely, the equi-segmented algorithm within the generalized time series dimension-reduction framework proposed by \citet{zhao2005generalized}.

A notable aspect of the work is in creating a representative picture including resource-constrained computing contexts, where order-of-magnitude memory savings enabled to are particularly impactful, we supplemented trials in a traditional CPU-oriented High-Performance Computing (HPC) environment with an additional set of trials targeting embedded computing microprocessor hardware.
As described earlier, the motivating application in hereditary stratigraphy in developing this work is a synthesis of the two: HPC simulation code harnessing embedded-like peripheral processing engines, ranging from established GPU hardware to emerging AI/ML accelerator platforms such as the Cerebras Wafer-Scale Engine.

A core objective of our algorithm is to generalize across data stream applications.
For this reason, we used an artificial stand-in data stream to ensure our experiments were straightforward and reproducible, but nontrivial.
For these purposes, we used a deterministic sequence generated by 32-bit xorshift PRNG, where each element is related to the prior by three shift/xor operations \citep{marsaglia2003xorshift}.

\subsection{Benchmark Methodology}

Benchmark code was implemented in C++.
We implemented the naive algorithm baseline using the memory-compact \texttt{std::vector<bool>} specialization.
We also used compile-time means to record exact measurents of memory use.

Memory usage was calculated at compile-time using the \texttt{sizeof} builtin.

Experiments were performed on a Raspberry Pi Pico RP2040 Microcontroller Board, which features a Dual-Core ARM Cortex-M0+ chip clocked at 133 MHz.
The RP2040 provides 264KB of SRAM and 2MB of on-board flash memory.
Embedded experiments were compiled with the Raspberry Pi Pico SDK version 2.1.0, which bundles ARM GNU Toolchain 13.3.1 \citep{raspberrypipico2024}.
Ten replicates were taken for each embedded timing.

Across all trials, executables were compiled with optimization level \texttt{O3} with native micro-architecture and, for HPC experiments, ISA extensions enabled.
Appropriate steps were taken to prevent the operations interest from being inadvertently optimized away.
To accurately reflect scenarios where stream length is unknown \textit{a priori}, we took steps to ensure total loop count was unknown as compile time.
Benchmarking code and results are publicly available, as described in Section \ref{sec:materials}.

\subsection{Software and Data Availability}
\label{sec:materials}

Supporting software and executable notebooks for this work are available via Zenodo at \url{https://doi.org/10.5281/zenodo.10779240} \citep{moreno2024hsurf}.
DStream algorithm implementations are also published on PyPI in the \texttt{downstream} Python package and at \url{https://github.com/mmore500/downstream}, where we plan to conduct longer-term, end-user-facing development and maintenance \citep{moreno2024downstream}.
Supplemental materials are available via the Open Science Framework at \url{https://osf.io/na2wp} and \url{https://osf.io/kjpqu/} \citep{foster2017open}.
All accompanying software and materials are provided open-source under the MIT License.

This project benefited significantly from open-source scientific software \citep{2020SciPy-NMeth,harris2020array,reback2020pandas,mckinney-proc-scipy-2010,waskom2021seaborn,hunter2007matplotlib,moreno2023teeplot}.


\subsection{Time Benchmark Results}

\begin{figure}
\resizebox{\linewidth}{!}{%
\input{binder-downstream-benchmark/binder/teeplots/2025-02-19-cpp-bench-speed-pico/col=data-type+font.family=serif+hue=algorithm+kind=line+palette=muted+style=algorithm+viz=relplot+x=buffer-size-s+y=per-item-walltime-ns+ext=.pgf}%
}
\caption{pico performance}
\label{fig:pico-performance}
\end{figure}


\subsection{Memory Benchmark Results}

\begin{figure}
\resizebox{\linewidth}{!}{%
\includegraphics[width=\linewidth,angle=90,origin=c]{binder-downstream-benchmark/binder/teeplots/2025-02-19-cpp-bench-memory-pico-steady/col=data-type+hue=algorithm+kind=bar+palette=set2+viz=catplot+x=num-sites+y=bits-per-item+ext=.pdf}%
}
\caption{steady mem}
\label{fig:steady-mem}
\end{figure}

\begin{figure}
\resizebox{\linewidth}{!}{%
\includegraphics[width=\linewidth,angle=90,origin=c]{binder-downstream-benchmark/binder/teeplots/2025-02-19-cpp-bench-memory-pico-tilted/col=data-type+hue=algorithm+kind=bar+palette=set2+viz=catplot+x=num-sites+y=bits-per-item+ext=.pdf}%
}
\vspace{-20ex}
\caption{%
Per-item memory use for tilted curation strategies.
\footnotesize
Lower is better.
}
\label{fig:tilted-mem}
\end{figure}


\subsection{Solution Quality Results}

\begin{figure}
\includegraphics[width=\linewidth]{binder-downstream-benchmark/binder/teeplots/2025-01-18-qos-dstream-vs-naive-steady/buffer_size=64+font.family=serif+hue=algorithm+kind=line+palette=set2+style=algorithm+viz=relplot+x=num-items-ingested+y=max-gap-size+ext=.pdf}
\caption{%
Progression of record gap size under steady curation strategies, for buffer size 64.
Lower is better.
Supplementary Figure \ref{fig:steady-qos-supp} compares gap size progressions at other buffer sizes, which behave similarly.
}
\label{fig:steady-qos}
\end{figure}

\begin{figure}
\resizebox{\linewidth}{!}{%
\input{binder-downstream-benchmark/binder/teeplots/2025-01-18-qos-dstream-vs-naive-tilted/font.family=serif+hue=algorithm+kind=line+palette=set2+style=algorithm+viz=relplot+x=num-items-ingested+y=gap-size-cost+ext=.pgf}%
}
\caption{tilted qos}
\label{fig:tilted-qos}
\end{figure}


Figure \ref{fig:tilted-qos} shows solution quality, as measured by the growth in largest gap size between retained elements ratio TODO.
% Recall that an optimal solution will keep minimize this value by evenly spacing elements across history.
