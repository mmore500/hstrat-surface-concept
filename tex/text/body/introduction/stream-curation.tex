\subsection{Stream Curation Problem}

Our work concerns online sampling of discrete data items from a one-dimensional data stream.
In selecting retained data items, we seek to ``curate'' a collection containing samples spanning the first items ingested from the data stream through the most recently ingested items \citep{moreno2024algorithms}.
The objective, ultimately, is to preserve a representative, approximate record of stream history.

We assume curation as an online process, with new data items being ingested on an ongoing basis, with a properly curated archive also needed on a continual basis.
In practice, such fully-online curation can be necessary when either (a) stream records are consulted on a frequent basis or (b) time point(s) for which stream records are needed are not known \textit{a priori} (i.e., query- or trigger-driven events).

\begin{figure}
\begin{minipage}[t]{\linewidth}
    \begin{minipage}[]{\textwidth}
    \noindent\fcolorbox{gray!3}{gray!3}{%
    \begin{minipage}[c][0.4in]{0.04\textwidth}
    \hspace{-0.5ex}%
    \rotatebox{90}{$\colorT = 100$}
    \end{minipage}}%
    \hspace{-1.5ex}%
    {\vrule width 1pt}%
    \noindent\fcolorbox{orange!4}{orange!4}{%
    \begin{minipage}[c][0.4in]{0.45\textwidth}
    \includegraphics[trim={0 1.2cm 0 0},clip,width=\linewidth]{binder/teeplots/coverage-criteria/criterion=steady+stop=100+viz=rugplot+ext=.pdf}%
    \end{minipage}}%
    {\vrule width 1pt}%
    \noindent\fcolorbox{purple!4}{purple!4}{%
    \begin{minipage}[c][0.4in]{0.45\textwidth}
    \includegraphics[trim={0 1.2cm 0 0},clip,width=\linewidth]{binder/teeplots/coverage-criteria/criterion=tilted+stop=100+viz=rugplot+ext=.pdf}%
    \end{minipage}}%
    \end{minipage}\vspace{-1ex}

    \begin{minipage}[]{\textwidth}
    \noindent\fcolorbox{gray!11}{gray!11}{%
    \begin{minipage}[c][0.4in]{0.04\textwidth}
    \hspace{-0.5ex}%
    \rotatebox{90}{$\colorT = 50$}
    \end{minipage}}%
    \hspace{-1.5ex}%
    {\vrule width 1pt}%
    \noindent\fcolorbox{orange!10}{orange!10}{%
    \begin{minipage}[c][0.4in]{0.45\textwidth}
    \includegraphics[trim={0 0 0 0},clip,width=\linewidth]{binder/teeplots/coverage-criteria/criterion=steady+stop=50+viz=rugplot+ext=.pdf}%
    \end{minipage}}%
    {\vrule width 1pt}%
    \noindent\fcolorbox{purple!10}{purple!10}{%
    \begin{minipage}[c][0.4in]{0.45\textwidth}
    \includegraphics[trim={0 0 0 0},clip,width=\linewidth]{binder/teeplots/coverage-criteria/criterion=tilted+stop=51+viz=rugplot+ext=.pdf}%
    \end{minipage}}%
    \end{minipage}\vspace{-0.5ex}

    \begin{minipage}[]{\textwidth}
    \noindent\fcolorbox{white!100}{white!100}{%
    \begin{minipage}[b][0.2cm]{0.04\textwidth}
    \hspace{-0.5ex}%
    \rotatebox{90}{$~$}
    \end{minipage}}%
    \hspace{-1.5ex}%
    {\vrule width 1pt}%
    \noindent\fcolorbox{orange!2}{orange!2}{%
    \begin{minipage}[]{0.45\textwidth}
    \hspace{-1ex}\begin{subfigure}[b]{\linewidth}
    \caption{steady criterion}
    \label{fig:criteria-intuition-steady}
    \end{subfigure}%
    \end{minipage}}%
    {\vrule width 1pt}%
    \noindent\fcolorbox{purple!2}{purple!2}{%
    \begin{minipage}[]{0.45\textwidth}
    \hspace{-1ex}\begin{subfigure}[b]{\linewidth}
    \caption{tilted criterion}
    \label{fig:criteria-intuition-tilted}
    \end{subfigure}%
    \end{minipage}}%
    \end{minipage}
\end{minipage}%
\begin{minipage}[t]{0.04\textwidth}
~
\end{minipage}

\begin{minipage}[t]{\linewidth}
    \vspace{-2ex}%
    \caption{%
      \textbf{Steady and tilted coverage criteria.}
      \footnotesize
      Ideal distributions of ingestion time points for retained data items under the each criterion are shown at $\colorT=50$ (bottom) and $\colorT=100$ (top).
      Vertical bars represent a retained data item.
      In this illustration, collection size is 12 retained items.
      All other ingested data items have been discarded.
      The steady criterion \ref{fig:criteria-intuition-steady} seeks to minimize largest absolute gap size.
      So, ideal retention maintains items spread evenly across data stream history.
      In contrast, under the tilted criterion \ref{fig:criteria-intuition-tilted} recency-proportional gap size is to be minimized, necessitating over-retention of recent data items.
      }
    \label{fig:criteria-intuition}
\end{minipage}
\end{figure}


We consider three possible requirements on sampled data, as described above.
\textit{{Steady}} retention seeks a sample spaced uniformly across elapsed stream history.
\textit{{Stretched}} retention, by contrast, proportionally prioritizes early data items.
\textit{{Tilted}} retention proportionally prioritizes recent data items.
Formal definitons of these three criteria will be introduced in Section \ref{sec:notation-coverage}.
Related objectives appear in a variety of related data stream work, reviewed in Section \ref{sec:prior-work} \citep{aggarwal2003framework,han2005stream}.

%Each contributed policy includes indexing schemes that simultaneously support both efficient update operations and efficient storage of retained stream values in a flat array, requiring only $\mathcal{O}(1)$ storage overhead --- a single counter value.

\subsection{Applications of Stream Curation}

Efficient stream curation operations benefit a variety of use cases requiring synopses of data stream history.
A straightforward application of stream curation is in unattended or sporadically-uplinked sensor devices, which must record incoming observation streams on an indefinite or indeterminate basis, with limited memory capacity \citep{jain2022survey}.
In practice, however, even well-resourced centralized systems require thinning of full fidelity data --- raising the possibility of use cases in long-term telemetry and log management \citep{kent2006guide,miebach2002hubble}.
Checkpoint-rollback state might also be managed through stream curation in scenarios where the possibility of non-halting silent errors require support for arbitrary rollback extents \citep{aupy2013combination}.
Extensions could be imagined to support more general aggregation and approximation operations over stream history besides sampling \citep{schoellhammer2024lightweight}, although we do not directly investigate these possibilities here.

% Existing work has largely applied rolling full retention of most recent data within available buffer space \citep{fincham1995use} or dismissal of incoming data after storage reaches capacity \citep{saunders1989portable,mahzan2017design}.
% Strategies to maintain a cross-sectional time sample appear scant, although there has been some work to extend the record capacity of data loggers through application-specific online compression algorithms \citep{hadiatna2016design}.

Algorithm developments reported here originally stem from supporting work on ``\textit{hereditary stratigraphy},'' a recently-developed technique for distributed tracking of digital ancestry trees --- for instance in analysis of highly-distributed agent-based evolution simulations, content in decentralized social networks, peer-to-peer file sharing, and computer viruses \citep{moreno2022hereditary}.
Although separate from our objectives here, we will briefly motivate this particular use case of stream curation.
Hereditary stratigraphy takes a reconstruction-based approach to tracking, annotating surveiled artifacts with checkpoint data, which is extended by a new ``fingerprint'' with each copy event.
Comparing two artifacts' accreted records reveals the duration of their common ancestry, with the first mismatched fingerprints indicating the end of common descent.

Hereditary stratigraphy relies on stream curation to prevent unbounded growth of generational fingerprint records from bloating memory use.
These records can be considered as a data stream, in that they accrue piece-by-piece, on an indefinite basis.
Downsampling fingerprints saves memory, but introduces uncertainty as to the timing of lineage divergence.
To this end, the manner in which retained checkpoints are spaced across generational history is crucial to inference quality.
Recent work, for instance, finds that recency-biased ``\textit{tilted}'' retention --- as opposed to ``\textit{steady}'' retention --- maximizes phylogeny reconstruction accuracy for common evolutionary scenarios \citep{moreno2024guide}.
Minimizing per-item storage overhead is also critical to hereditary stratigraphy, with \citet{moreno2024guide} finding that single-bit checkpoint values maximize reconstruction quality (i.e., by allowing more fingerprints to be retained).
