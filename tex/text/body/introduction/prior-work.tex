\subsection{Prior Work}

Given the broad applicability of the data stream paradigm, a rich ecosystem of algorithms exist to support many types of analysis and summarization over sequenced input --- such as rolling summary statistic calculations \citep{lin2004continuously}, on-the-fly data clustering \citep{silva2013data}, live anomaly detection \citep{cai2004maids}, and rolling event frequency estimation \citep{manku2002approximate}.
Stream curation aligns, in particular, at the intersection of two general data stream processing strategies:
\begin{enumerate}
\item sampling, where the data stream corpus is coarsened through extraction of exemplar data items \citep{sibai2016sampling}; and
\item temporal binning/windowing, where data stream content is aggregated (e.g., summarized, compressed, or sampled) with respect to discrete time spans over stream history \citep{gama2007data}.
\end{enumerate}
Stream curation further situates within the purview of logical (rather than physical) time algorithms \citep{sibai2016sampling}, as retention objectives are organized vis-a-vis sequence index rather than real time.

Owing to the fundamental role of dimension-reduction in supporting more advanced data stream operations, downsampling over on logical bins has garnered longstanding algorithmic attention.
Notably, schemes for fixed-capacity steady (``equi-segmented'') and tilted (``vari-segmented'') retention appear in \citep{zhao2005generalized}, with the latter resembling additional ``pyramidal'' and ``tilted'' time window schemes appearing elsewhere \citep{aggarwal2003framework,han2005stream}.
Despite congruities with present work in their objectives and, to an extent, curated collection composition, these references do not prescribe non-iterative layout and update procedures or emphasize minimization of representational overhead (e.g., timestamps, segment length values), as pursued here.
Work on ``amnesic approximation,' a generalized scheme for down-samping satisfying an arbitrary temporal cost function, is also notable in its closely related objectives --- although catering to a substantially more resource-intensive use case \cite{palpanas2004online}.

% To our knowledge, these previous implementations all unfold through stateful iteration, with representational overhead for each stored value ; stateless enumerations of retained set composition are original to our work in this paper.

\citet{moreno2024algorithms}, which also develops stream curation techniques in service of hereditary stratigraphy \citep{moreno2022hstrat}, should be particularly noted in relation to present work.
Whereas this earlier work also focuses on minimizing the representational footprint around stored data, algorithm design caters better to variable-capacity storage, rather than fixed-capacity.
Although configurations oriented to fixed-capacity use cases targeted herein are also explored in \citet{moreno2024algorithms}, they require a more expensive update process keeping data in sorted order and, in practice, can leave unused buffer capacity in order to guarantee bounded adherence to size limits.
Indeed, head-to-head comparison has demonstrated present algorithms as improving execution speed by order of magnitude in speed and also, in the case of tilted retention, enhancing buffer space utilization \citep{moreno2024guide,moreno2024trackable}.
In addition to these benchmark analyses --- beyond the scope of formal algorithmic exposition herein --- presented algorithms have also already seen use implementing hereditary stratigraphy for lineage tracking within very large scale agent-based evolution simulations on the 850,000 core Cerebras Wafer-Scale Engine platform \citep{moreno2024trackable}.
