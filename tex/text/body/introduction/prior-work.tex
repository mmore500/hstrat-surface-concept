\subsection{Prior Work}
\label{sec:prior-work}

Given the broad applicability of the data stream paradigm, many algorithms exist for analysis and summarization over sequenced input --- such as rolling summary statistic calculations \citep{lin2004continuously}, on-the-fly data clustering \citep{silva2013data}, live anomaly detection \citep{cai2004maids}, and rolling event frequency estimation \citep{manku2002approximate}.
Stream curation touches in particular on two broad paradigms data stream processing:
\begin{enumerate}
\item \textit{sampling}, where the data stream corpus is coarsened through extraction of exemplar data items \citep{sibai2016sampling}; and
\item \textit{binning/windowing}, where data stream content is aggregated (e.g., summarized, compressed, or sampled) with respect to discrete time spans over stream history \citep{gama2007data}.
\end{enumerate}

Although curated data items are, indeed, a sample of a data stream, work here is orthogonal to the question of $\ell_p$ sampling (e.g., $\ell_0$ or $\ell_1$ sampling) in that our objective is to optimize for temporal balance rather than stochastic composition.
Indeed, well-established techniques exist to extract rolling $\ell_p$-representative samples over the distribution of data values from a stream, such as reservoir sampling, sketching, and hash-based methods \citep{gaber2005mining,muthukrishnan2005data,cormode2019lp}.
Note also that stream curation pertains to logical time rather than real time \citep{sibai2016sampling}, as retention objectives are organized vis-a-vis sequence index rather than clock time.

Owing to dimension reduction's fundamental role in supporting more advanced data stream operations, substantial work exists addressing the question of downsampling via temporal binning.
Notably, schemes for fixed-capacity steady (``equi-segmented'') and tilted (``vari-segmented'') retention appear in \citep{zhao2005generalized}, with the latter resembling additional ``pyramidal,'' ``logarithmic,'' and ``tilted'' time window schemes appearing elsewhere \citep{aggarwal2003framework,han2005stream,giannella2003mining,phithakkitnukoon2010recent}.
Although congruities exist in objectives and aspects of algorithm structure, no existing work prescribes non-iterative layout and update procedures that emphasize minimization of representational overhead (e.g., avoiding storage of timestamps, segment length values, etc.) --- as pursued here.
Work on ``amnesic approximation,'' a generalized scheme for downsampling satisfying an arbitrary temporal cost function, has related objectives but caters to a substantially more resource-intensive use case \citep{palpanas2004online}.

% To our knowledge, these previous implementations all unfold through stateful iteration, with representational overhead for each stored value ; stateless enumerations of retained set composition are original to our work in this paper.

\citet{moreno2024algorithms} presented earlier stream curation techniques in service of hereditary stratigraphy.
Whereas that earlier work also focuses on minimizing the representational footprint around stored data, it caters better to variable-capacity storage, rather than fixed-capacity.
Although configurations oriented to fixed-capacity use cases targeted here are also explored in \citet{moreno2024algorithms}, they require a more expensive update process that keeps data in sorted order and can leave buffer capacity unused.
Indeed, head-to-head benchmark trials demonstrate improved execution speed (by an order of magnitude) and enhanced buffer space utilization under tilted retention \citep{moreno2024guide,moreno2024trackable}.
