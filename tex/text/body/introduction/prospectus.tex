\subsection{Proposed Approach}

% In this work, we have developed new strategies for ``data stream curation'' --- subsampling from a rolling sequence of data items to dynamically maintain a representative cross-sample across observed time points, focusing in particular on fixed-capacity procedures amenable to resource-constrained use cases.

Our proposed approach revolves around a strong simplifying constraint: we assume a fixed number of buffer sites where items ingested from a data stream may be written and, once stored, we do not allow them to be subsequently inspected or moved.
Only one further event may occur after a data item is stored, in that it may be overwritten by a later data item.
We also allow ingested data items to be discarded without storage.
Under this regime, the composition of retained data emerges implicitly as a consequence of items targeted for overwrite.
Put another way, curation policy is exercised solely through ``\textit{site selection}'' when picking a buffer index for the $n$th received data item.

Note that this operational scheme inherently forgoes any explicit data labeling, timestamping, or other structure (e.g., pointers).
Instead, we require site selection to be computable \textit{a priori}.
As a further consequence, efficient attribution of data items' origin time hence requires support for efficient ``inverse'' decoding of a stored data item's origin time based solely on its buffer index and how many items have been ingested from the data stream.
We term this operation ``\textit{site lookup}.''
Figure \ref{fig:ingest-and-lookup} schematizes our ``site selection'' and ``site lookup'' operations.

\begin{figure*}
  \centering
  \begin{subfigure}{0.55\textwidth}
  \includegraphics[width=\textwidth]{img/surface-site-ingest}
  \caption{ingest site selection for storage}
  \label{fig:surface-site-ingest}
  \end{subfigure}%
  \hfill
  \begin{subfigure}{0.4\textwidth}
  \centering
  \includegraphics[width=\textwidth]{img/ingest-rank-calculation}
  \caption{ingested time calculation for site lookup}
  \label{fig:ingest-rank-calculation}
  \end{subfigure}
  \caption{%
  \textbf{Core stream curation algorithm operations.}
  \footnotesize
  The ingestion site selection operation (\ref{fig:surface-site-ingest}) takes the current time $\colorT$ and determines the buffer site $\colork$ to store the ingested data item.
  This operation is performed when storing data into a curated buffer, once for each data item received from the data stream.
  The ingested time calculation operation (\ref{fig:ingest-rank-calculation}) provides the previous time $\colorTbar$ when the data item present at buffer site $\colork$ was ingested, given the current time $\colorT$.
  This operation is performed when reading data from a curated buffer, in order to identify the provenance of stored data.
  Note that the ingestion time of stored data at a buffer site $\colork$ is determined by the sequence of sites that were selected by the ingestion site selection operation --- in a loose sense this calculation operation can be considered as ``decoding'' or ``inverse'' to the selection operation.
  }
  \label{fig:ingest-and-lookup}
\end{figure*}


\subsection{Major Results}

This paper contributes three site selection algorithms for stream curation, with corresponding site lookup procedures.
These algorithms differ in temporal composition of retained data items, targeting steady, stretched, and tilted distributions, respectively.
All three proposed algorithms support $\mathcal{O}(1)$ site selection.
Accompanying site lookup is $\mathcal{O}(\colorS)$ to decode all $\colorS$ buffer sites' ingest times.
We provide worst-case upper bounds on curation quality, with the steady algorithm notable in guaranteeing performance matching best case within a factor of two.
