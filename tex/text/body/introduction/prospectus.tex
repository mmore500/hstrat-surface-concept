\subsection{Proposed Approach}

% In this work, we have developed new strategies for ``data stream curation'' --- subsampling from a rolling sequence of data items to dynamically maintain a representative cross-sample across observed time points, focusing in particular on fixed-capacity procedures amenable to resource-constrained use cases.

Our proposed DStream approach adopts a strong simplifying constraint: Once stored, we do not allow data items to be subsequently inspected or moved.
We assume a fixed number of buffer sites where items ingested from a data stream may be written.
The only further event that may occur after a data item is stored is being overwritten by a later data item.
We also allow ingested data items to be discarded without storage.
Under this regime, the composition of retained data emerges implicitly as a consequence of items targeted for overwrite.
Put another way, curation policy is exercised solely through ``\textit{site selection}'' when picking a buffer index for the $n$th received data item.

Note that this operational scheme supports particularly efficient storage of fine-grained data items, as it inherently forgoes overhead from explicit data labeling, timestamping, or other structure (e.g., pointers).
Instead, we require site selection to be computable \textit{a priori}.
As a further consequence, efficient attribution of data items' origin time hence requires support for efficient ``inverse'' decoding of a stored data item's origin time based solely on its buffer index and how many items have been ingested from the data stream.
We term this operation ``\textit{site lookup}.''
Figure \ref{fig:ingest-and-lookup} schematizes our ``site selection'' and ``site lookup'' operations.
Figure \ref{fig:curation-ingest-lookup} TODO.

\begin{figure}

\begin{subfigure}{0.3\linewidth}
\centering
\caption{simple ring buffer}
\label{fig:curation-ingest-lookup:ring}
\end{subfigure}
\begin{minipage}{0.6\linewidth}
\centering
\includegraphics[width=\linewidth]{img/ring-buffer-simple}
\end{minipage}

\vspace{1ex}

\noindent\rule{\linewidth}{1pt}

\begingroup\setlength{\fboxsep}{0pt}%
\noindent\fcolorbox{blue!0}{blue!10}{%
\begin{minipage}{0.02\linewidth}~\end{minipage}%
\begin{subfigure}{0.3\linewidth}
\centering
\caption{generalized ring buffer: steady curation}
\label{fig:curation-ingest-lookup:steady}
\end{subfigure}%
\begin{minipage}{0.02\linewidth}~\end{minipage}%
\noindent\fcolorbox{blue!4}{blue!4}{%
\begin{minipage}{0.02\linewidth}~\end{minipage}%
\begin{minipage}{0.6\linewidth}
\centering
\vspace{1ex}

\includegraphics[width=\linewidth]{img/ring-buffer-steady}
\vspace{1ex}
\end{minipage}%
}%
}
\endgroup

\noindent\rule{\linewidth}{1pt}

\begingroup\setlength{\fboxsep}{0pt}%
\noindent\fcolorbox{purple!0}{purple!10}{%
\begin{minipage}{0.02\linewidth}~\end{minipage}%
\begin{subfigure}{0.3\linewidth}
\centering
\caption{generalized ring buffer: tilted curation}
\label{fig:curation-ingest-lookup:tilted}
\end{subfigure}%
\begin{minipage}{0.02\linewidth}~\end{minipage}%
\noindent\fcolorbox{purple!4}{purple!4}{%
\begin{minipage}{0.02\linewidth}~\end{minipage}%
\begin{minipage}{0.6\linewidth}
\centering
\vspace{1ex}

\includegraphics[width=\linewidth]{img/ring-buffer-tilted}
\vspace{1ex}
\end{minipage}%
}%
}
\endgroup

\caption{
  {\textbf Comparison of simple ring buffer to generalized ring buffers.}
  \footnotesize
  TODO
  Both approaches accumulate items ingested from a data stream into a fixed-capacity memory buffer.
  However, whereas the ring buffer retains only most-recent items, the generalized algorithm maintains a sample spanning the entirety of elapsed history.
  We notate ingest site assignment and stored item attribution, respectively, as $\colorK$ and $\colorL$.
  For the proposed DStream tilted algorithm these operations are described in Listings and \ref{alg:tilted-site-selection} and \ref{alg:tilted-time-lookup}.
}
\label{fig:curation-ingest-lookup}
\end{figure}
% https://docs.google.com/presentation/d/1-7eLWTSH16s1MpGyT0iTJGAVLEYdKM7hK3ao2y_bW9A

% \begin{figure*}
  \centering
  \begin{subfigure}{0.55\textwidth}
  \includegraphics[width=\textwidth]{img/surface-site-ingest}
  \caption{ingest site selection for storage}
  \label{fig:surface-site-ingest}
  \end{subfigure}%
  \hfill
  \begin{subfigure}{0.4\textwidth}
  \centering
  \includegraphics[width=\textwidth]{img/ingest-rank-calculation}
  \caption{ingested time calculation for site lookup}
  \label{fig:ingest-rank-calculation}
  \end{subfigure}
  \caption{%
  \textbf{Core stream curation algorithm operations.}
  \footnotesize
  The ingestion site selection operation (\ref{fig:surface-site-ingest}) takes the current time $\colorT$ and determines the buffer site $\colork$ to store the ingested data item.
  This operation is performed when storing data into a curated buffer, once for each data item received from the data stream.
  The ingested time calculation operation (\ref{fig:ingest-rank-calculation}) provides the previous time $\colorTbar$ when the data item present at buffer site $\colork$ was ingested, given the current time $\colorT$.
  This operation is performed when reading data from a curated buffer, in order to identify the provenance of stored data.
  Note that the ingestion time of stored data at a buffer site $\colork$ is determined by the sequence of sites that were selected by the ingestion site selection operation --- in a loose sense this calculation operation can be considered as ``decoding'' or ``inverse'' to the selection operation.
  }
  \label{fig:ingest-and-lookup}
\end{figure*}


\subsection{Major Results}

This paper contributes a novel site selection algorithm for ``steady'' stream curation, with a corresponding site lookup procedures
% These algorithms differ in the temporal composition of retained data items, targeting steady, stretched, and tilted distributions, respectively.
The proposed algorithm supports $\mathcal{O}(1)$ site selection, and accompanying site lookup is $\mathcal{O}(\colorS)$ to decode all $\colorS$ buffer sites' ingest times.
We provide a worst-case lower bound on curation quality, guaranteeing performance within a constant factor of best case quality.
