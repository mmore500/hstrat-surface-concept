\section{Tilted Algorithm} \label{sec:tilted}

The tilted criterion favors recent data items, enjoining a record spaced proportionally to time elapsed since data item's ingest time, $\colorT - \colorTbar$.
This is opposite the stretched criterion, which favors early data items.

The tilted criterion can be formulated as minimization of the largest ratio of gap size to ingest time,
\begin{align*}
\frac{\colorG_{\colorT}(\colorTbar)}{\colorT - \colorTbar} \text{ for } \colorTbar < \colorT.
\end{align*}

The same approximate lower bound on best-case gap size ratio derived for the stretched curation can be applied also to tilted curation,
\begin{align*}
\frac{\colorG_{\colorT}(\colorTbar)}{\colorTbar}
&\stackrel{\sim}{\geq}
\colorT^{1/\colorS} - 1.
\tag{Equation \ref{eqn:approx-gap-bound}}
\end{align*}
As can the strict bound on stretched algorithm gap size ratio acounting for discretiztion effects,
\begin{align*}
\frac{\colorG_{\colorT}(\colorTbar)}{\colorT - \colorTbar}
&\geq
\frac{
  1
}{
  1 + \colorS
  - \left\lfloor \colorS \log_{\colorT}\Big(
    (\colorT - \colorS)(\colorT^{1/\colorS} - 1) + 1
  \Big)\right\rfloor
}.
\tag{Theorem \ref{thm:stretched-ideal-strict}}
\end{align*}

In this section, we present a stream curation algorithm tailored to the tilted criterion, achieving maximum gap size ratio no worse than,
\begin{align*}
  \frac{\colorG_{\colorT}(\colorTbar)}{\colorT - \colorTbar}
  &\leq
  \frac{
    1
  }{
    \max\Big(
      \frac{\colorS}{2(\colort + \colors)},
      \frac{\colorS}{4\colort},
      \frac{\colorS}{\left\lceil 2^{\colors - \colortau - 1} \right\rceil}
    \Big)
    - 1/2
  }.
\end{align*}

\subsection{Tilted Algorithm Strategy}
\label{sec:tilted-strategy}

\begin{figure*}
  \centering
\begin{subfigure}{0.43\textwidth}
\includegraphics[width=\textwidth]{img/hsurf-tilted-intuition}
\caption{TODO}
\end{subfigure}%
\begin{subfigure}{0.57\textwidth}
\includegraphics[width=\textwidth]{binder/teeplots/20/surface-size=16+viz=site-reservation-at-ranks-heatmap+ext=}
\caption{TODO}
\end{subfigure}
  \caption{
    \textbf{Tilted algorithm strategy.}
    \footnotesize
    TODO
    % Overview of placement strategy for data item instances of hanoi value $\colorh$.
    % Data items are shown in yellow, spanning across the epoch timeline from when they are stored to when they are overwritten.
    % Available memory buffer sites, broken into conceptual segments, are shown across the top of the schematic.
    % The first data item with $\colorH(\colorT) = \colorh$ is placed in segment 0.
    % The next data item with h.v. $\colorh$ is encountered in the following epoch, and it is placed in segment 1.
    % In epoch 3, two data items with h.v. $\colorh$ are encountered, and placed in segment 2.
    % In the epoch 4, four h.v. $\colorh$ data items (twice as many) are encountered.
    % We place them in segment 3.
    % If working with a larger memory buffer, this pattern would continue in subsequent epochs.
    % During the epoch after the last segment is written to, all h.v. $\colorh$ data items are overwritten.
    % In segment 0 they are overwritten by h.v. $\colorh - 4$, in segment 1, h.v. $\colorh - 3$, and so forth.
    % In this manner, data items with highest hanoi value are retained on a rolling basis.
    % This approach ensures equal gap sizes across history, with gap sizes gracefully doubling in size as intermediate values are overwritten.
  }
  \label{fig:hsurf-tilted-intuition}
\end{figure*}


The retention strategy for the tilted algorithm strongly resembles that of the stretched algorithm.
Recall that under the stretched algorithm the first $n(\colorT)$ data items of each hanoi value $\colorH(\colorTbar)$ are retained, with $n(\colorT)$ degrading with time in shifting from many copies of few encountered hanoi values to few copies of many encountered hanoi values.
Under the tilted algorithm, we instead keep the \textit{last} $n(\colorT)$ data items of each hanoi value.
Figure \ref{fig:hanoi-intuition-tilted} shows how keeping the last $n$ instances of each h.v. approximates tilted distribution.

We thus define our algorithmic goal as maintaining --- for a declining threshold $n(\colorT)$ --- the set of data items,
\begin{align*}
\textsf{goal\_tilted}
&=
\bigcup_{\colorh \geq 0}
\{ \colorTbar \in \colorT, \colorT - 1, \, \ldots, \max(0, 2^{\colorh + 1}n(\colorT) - 1) : \colorH(\colorTbar) = \colorh \}.
\end{align*}
Note that by construction, $\textsf{goal\_tilted} \subset \{ \colorTbar \in 0, 1, \, \ldots, \colorT \}$.
As shown previously in Theorem \ref{thm:stretched-first-n-space}, setting $n(\colorT) = 2^{\colors - 1 - \colortau}$ suffices to respect available buffer capacity $\colorS$.

\subsection{Tilted Algorithm Mechanism}
\label{sec:tilted-mechanism}

Because the tilted algorithm, like the stretched algorithm, also follows an equal-$n$-per-h.v. scheme, hanoi value reservation layout is maintained identically to that of the stretched algorithm.
Refer to Section \ref{sec:stretched-mechanism} for a detailed description and justification of hanoi value reservation layout, and how it is progressed across epochs $\colort$ and meta-epochs $\colortau$.

A pertinent result of the stretched layout, though, is that the at least $2^{\colors - 1 - \colortau}$ data item instances of each h.v. are retained (Lemma \ref{thm:stretched-reservation-count}).
However, unlike the stretched algorithm, for the tilted algorithm we wish to keep the \textit{last} $n$ rather than the \textit{first} $n$ instances of each hanoi value.
We can do that by continuing to write data items for each h.v. into reservation slots for that h.v. after they initially fill --- overwriting older instances of the h.v. to keep a ``ring buffer'' of fresh h.v. instances.

In the following, we prove several results related to this cyclical-overwrite mechanism --- notably, including that it aligns evenly to epoch and meta-epoch transitions.
Ultimately, these results lead up to Lemma \ref{thm:tilted-most-recent-retained}, which confirms that our strategy always preserves the last $2^{\colors - 1 - \colortau}$ instances of each h.v., including through transitions where reservation site count for a h.v. is shrunk by half due to invasion.

\begin{lemma}[Last Instance of a Hanoi Value within Epoch $\colort$]
\label{thm:tilted-last-touched}
The final instance of each h.v. encountered during an epoch is placed in the rightmost site reserved for that hanoi value.
That is, during any epoch $\colort$,
\begin{align*}
\colorK\Big(
  \max\{\colorT \in \colort : \colorH(\colorT) = \colorh\}
\Big)
=
\max\{\colork \in \colorS : \colorHcal(\colork) = \colorh \}
\end{align*}
for all $\colorh \in \{\colorHcal(\colork) : \colork \in \colorS\}$.
\end{lemma}

\begin{proof}
Because we are only concerned with pertinent h.v. $\colorh \in \{\colorHcal(\colork) : \colork \in \colorS\}$, we may assume $\colorh \leq \colors + \colort$.
In traversing a hanoi value's ``ring buffer'' of reserved sites, the rightmost reservation segment is visited last, as it is positioned rightmost within the set of smallest size segments.
Our proof objective can thus be fulfilled by showing that the number of sites $\colork$ reserved to h.v. $\colorh$ evenly divides the number of h.v. $\colorh$ instances encountered during epoch $\colort$.
That is,
\begin{align*}
|\{\colorT \in \colort : \colorH(\colort) = \colorh\}| \bmod |\{\colork \in \colorS : \colorHcal(\colork) = \colorh\} = 0.
\end{align*}

How many instances of a hanoi value $\colorh$ are encountered during epoch $\colort$?
This is
\begin{align*}
|\{\colorT \in \colort : \colorH(\colorT) = \colorh\}|
&=
2^{\colort + \colors - \colorh} - \left\lfloor2^{\colort + \colors - \colorh - 1} \right\rfloor.
\end{align*}
Adopting $\mathfrak{C}_{\colorH(\colorTbar)}$ as shorthand for this cardinality, observe that $\mathfrak{C}_{\colorH(\colorTbar)} \in 2^{\mathbb{N}}$ for $\colorh \leq \colors + \colort$.

How many sites are reserved to a hanoi value $\colorh$ during epoch $\colort$?
Recall from Lemma \ref{thm:stretched-discarded-incidence-count} that this is
\begin{align*}
|\{\colork \in \colorS : \colorHcal(\colork) = \colorh\}|
&=
\min\Big(
\left\lceil 2^{\colors - \colortau - 1} \right\rceil,
2^{\colors + \colort - \colorh} - \left\lfloor 2^{\colors + \colort - \colorh - 1} \right\rfloor
\Big).
\end{align*}
Denoting this cardinality $\mathfrak{C}_{\colorHcal(\colork)}$, observe also that
$\mathfrak{C}_{\colorHcal(\colork)} \in 2^{\mathbb{N}}$ for $\colorh \leq \colors + \colort$.

Because both $\mathfrak{C}_{\colorH(\colorTbar)} \in 2^{\mathbb{N}}$ and $\mathfrak{C}_{\colorHcal(\colork)} \in 2^{\mathbb{N}}$, all that remains to be shown is $\mathfrak{C}_{\colorH(\colorTbar)} \geq \mathfrak{C}_{\colorHcal(\colork)}$.
Consider,
\begin{align*}
2^{\colort + \colors - \colorh} - \left\lfloor2^{\colort + \colors - \colorh - 1} \right\rfloor
&\stackrel{?}{\geq}
\min\Big(
\left\lceil 2^{\colors - \colortau - 1} \right\rceil,
2^{\colors + \colort - \colorh} - \left\lfloor 2^{\colors - \colorh - 1} \right\rfloor
\Big)\\
&\stackrel{\checkmark}{\geq}
2^{\colors + \colort - \colorh} - \left\lfloor 2^{\colors - \colorh - 1} \right\rfloor.
\end{align*}
\end{proof}


\begin{lemma}[Leftmost Invaded Site is Overwritten Last]
\label{thm:tilted-last-overwritten}
Among the invaded data items $\colork$ overwritten at epoch $\colort > 0$, the leftmost data item is overwritten last.
That is,
\begin{align*}
\min\{ \colorK(\colorT) \text{ for } \colorT \in \colort : \colorHcal_{\colort-1}(\colorK(\colorT)) \neq \colorH(\colorT) \}
&=
\colorK\Big(\max \{ \colorT \in \colort \}\Big).
\end{align*}
\end{lemma}
\begin{proof}
By construction, the leftmost invaded site is $\colork = \colort + \colors$, invaded by h.v. $\colorh = \colort + \colors$ for $\colort > 0$.
H.v. $\colorh = \colort + \colors$ occurs first at ingest time $\colorT = 2^{\colort + \colors} - 1$.
Epoch $\colort + 1$ begins at time $\colorT = 2^{\colort + \colors}$, so epoch $\colort$ (which begins at $\colorT = 2^{\colort + \colors - 1}$) ends at ingest time $2^{\colort + \colors - 1}$, giving the result.
\end{proof}


\begin{lemma}[Monotonicity of Hanoi Value Reservation $\colorHcal(\colork)$ for Buffer Site $\colork$]
\label{thm:tilted-invader-minus-invaded}
A site's assigned hanoi value reservation never decreases.
Where it increases, it does so by at least 2.
Formally, where $\colorHcal_{\colort + 1}(\colork) \neq \colorHcal_{\colort}(\colork)$,
\begin{align*}
\colorHcal_{\colort + 1}(\colork) - \colorHcal_{\colort}(\colork) \geq 2.
\end{align*}
\end{lemma}

\begin{proof}
By design, invasion of any segment begins at the segment's leftmost site $\colork$, always assigned $\colorHcal(\colork) = 0$.
Because singleton $r=0$ reservation segments never invade, the invader of this leftmost $\colorh = 0$ site will always stem from a segment $r \geq 1$ and have h.v. $\colorh > 1$.
The delta $\colorHcal_{\colort + 1}(\colork) - \colorHcal_{\colort}(\colork) \geq 2$ remains constant over subsequent invasion steps because invader and invaded h.v.'s both increment by exactly 1 each epoch (until complete elimination of the invaded segment).
\end{proof}


\begin{lemma}[Invasion Overwrite Order within Epoch $\colort$]
\label{thm:tilted-invading-overwrite-order}
Except for the leftmost invaded site in segment $r=\colors$, invaded sites are overwritten left-to-right. For $\colort > 0$, pick
\begin{align*}
\\
\colork',\colork''
\in
\eqnmarkbox[WildStrawberry]{invadedsites}{
  \mathsf{invaded\_sites}_{\colort}
}
: \colors < \colork' < \colork'' < \colorS.
\end{align*}
\annotate[yshift=1em]{above,left}{invadedsites}{$\{
  \colork \in \colorS
  : \colorHcal_{\colort - 1}(\colork) \neq \colorHcal_{\colort}(\colork)
\}$}
Then,
\begin{align*}
\min\{
  \colorT \in \colort
  : \colorK(\colorT) = \colork'
\}
<
\min\{
  \colorT \in \colort
  : \colorK(\colorT) = \colork''
\}.
\end{align*}

\end{lemma}
\begin{proof}
Recall from Lemma \ref{thm:stretched-meta-epoch} that $\colortau$ tells the number of reservation segment subsumption cycles that have elapsed.
Recall also that $\min\{\colort \in \colortau\} = 2^{\colortau} - \colortau$.

There are $\left\lceil 2^{\colors - 1 - \colortau} \right\rceil$ uninvaded reservation segments at meta-epoch $\colortau$.
Invader h.v.'s at $\colort \geq 1$ are
\begin{align*}
\colort + \colors, \colorH(0) + \colort + \colortau, \colorH(1) + \colort + \colortau, \,\ldots, \colorH(\left\lceil 2^{\colors - 1 - \colortau} \right\rceil - 2) + \colort + \colortau.
\end{align*}

Rewritten based on properties of the hanoi sequence and excluding the leftmost invader (covered in Lemma \ref{thm:tilted-last-overwritten}),
\begin{align*}
\colorH(2^{\colort + \colors - 1} + 1 \times 2^{\colort + \colortau} - 1) , \colorH(2^{\colort + \colors - 1} + 2 \times 2^{\colort + \colortau} - 1), \,\ldots,
\colorH(2^{\colort + \colors - 1} + (\left\lceil2^{\colors - 1 - \colortau}\right\rceil - 1)  \times 2^{\colort + \colortau} - 1).
\end{align*}

We will confirm that,
\begin{align*}
\{
\colorT \in
2^{\colort + \colors - 1} + 1 \times 2^{\colort + \colortau} - 1,
2^{\colort + \colors - 1} + 2 \times 2^{\colort + \colortau} - 1,
\,\ldots,
2^{\colort + \colors - 1} + (\left\lceil2^{\colors - 1 - \colortau} \right\rceil - 1) \times 2^{\colort + \colortau} - 1
\} \subseteq \colort.
\end{align*}
Consider the range of ingest times comprising epoch $\colort$,
\begin{align*}
\{
\colorTbar \in \colort
\} = \{
2^{\colort + \colors - 1},
2^{\colort + \colors - 1} + 1,
\,\ldots,
2^{\colort + \colors} - 1
\}.
\end{align*}
Showing that our sequence of h.v.'s occurs during $\colort$ therefore requires,
\begin{align*}
2^{\colort + \colors} - 1
&\stackrel{?}{\geq}
(\left\lceil 2^{\colors - 1 - \colortau} \right\rceil  - 1)
2^{\colort + \colortau} + 2^{\colort + \colors - 1}\\
2^{\colort + \colors - 1} - 1
&\stackrel{?}{\geq}
\left\lceil 2^{\colors - 1 - \colortau} \right\rceil \times 2^{\colort + \colortau} - 2^{\colort + \colortau}.
\end{align*}
Given $\colort + \colortau \geq 2$ for all invasions because both $\colort > 0$ and $\colortau > 0$,
\begin{align*}
2^{\colort + \colors - 1} + 3
&\stackrel{\checkmark}{\geq}
2^{\colort + \colors - 1}.
\end{align*}

Remark that for all non-invading $\colorT \in \colort$, $\colorH(\colorT) < \colort + \colortau$.
Because same-h.v. reservations are filled from left-to-right and invading hanoi values accrue sequentially across $\colorT$, invading segments will overwrite their invaded data items left-to-right.
\end{proof}


\begin{lemma}[Minimum Recent Items Retained per Hanoi Value]
\label{thm:tilted-most-recent-retained}
At least the most recent $\left\lceil2^{\colors - 1 - \colortau}\right\rceil$ encountered instances of every h.v. $\colorh$ are retained.
\end{lemma}
\begin{proof}
From Lemma \ref{thm:stretched-discarded-incidence-count} we have that reservations are available to store at least the first $\left\lceil 2^{\colors - 1 - \colortau} \right\rceil$ instances of each hanoi value.
So, we can restrict our consideration to instances where h.v. instance count exceeds $\left\lceil 2^{\colors - 1 - \colortau} \right\rceil$.

In the absence of invasion, data item placement (by design) cycles around the sites reserved to a hanoi value.
With at least $\left\lceil 2^{\colors - 1 - \colortau} \right\rceil$ sites reserved, at least the last $\left\lceil 2^{\colors - 1 - \colortau} \right\rceil$ instances of each h.v. are retained.

What about in the case of invasion?
In this case, again from Lemma \ref{thm:stretched-discarded-incidence-count}, we have the number of reserved sites as dropping from $\left\lceil 2^{\colors - \colortau} \right\rceil$ to $\left\lceil 2^{\colors - 1 - \colortau} \right\rceil$.
Recall from Lemma \ref{thm:tilted-last-touched} each epoch, the final encounttered instance of each h.v. is placed into the rightmost reservation segment.
By design, we therefore know that the final encountered $\left\lceil 2^{\colors - 1 - \colortau} \right\rceil$ instances (ending at the rightmost reservation segment) are laid out left-to-right across the smallest remaining segments, $r = \colortau$.

So, at the outset of epoch $\colort$, to-be-invaded sites $\{\colork \in \colorS : \colorHcal_{\colort - 1}(\colork) \neq \colorHcal_{\colort}(\colork)\}$ always contain the most-recent $\left\lceil 2^{\colors - 1 - \colortau} \right\rceil$ ingested h.v. $\colorh = \colorHcal_{\colort - 1}(\colork)$ data items, sequenced left-to-right.
If data items in these sites were lost instantaneously at time $\min\{\colorT \in \colort\}$, we would not meet our proof objectives, having at that point none of the most-recent $\left\lceil 2^{\colors - 1 - \colortau} \right\rceil$ h.v. $\colorh$ data items retained.
However, this is not the case --- data items in these sites $\colork$ linger until they are \textit{actually} overwritten by incoming data items $\colorT \in \colort$ with $\colorK(\colorT) = \colork$.

From Lemma \ref{thm:tilted-invading-overwrite-order}, we have that --- over the course of an epoch --- data items are overwritten in the left-to-right order they were deposited --- except the leftmost reservation, which is overwritten last.
To ensure that the most recent $\left\lceil 2^{\colors - 1 - \colortau} \right\rceil$ data items are stored, we therefore must show two conditions:
\begin{enumerate}
\item that at least two depositions of our overwritten h.v. $\colorh$ occur before the first invasion overwrite, and
\item that the cadence of overwrites proceeds slower than the cadence of new instances of the hanoi value $\colorh$ being overwritten.
\end{enumerate}

Imagine the sequence of $\left\lceil 2^{\colors - 1 - \colortau} \right\rceil$ most recent data items $\colorT$ for h.v. $\colorH(\colorT) = \colorh$ like the classic video game ``snake'' \citep{de2016complexity}.
In that game, the eponymous snake grows at its head and shrinks at its tail.
Analogously, our sequence of most recent data items adds new items at the front and has tail items overwritten.
When an invasion occurs and half of ring buffer reservations are lost, the snake's head has just arrived to the last site of the \textit{lost half of the ring buffer} and its leading $\left\lceil 2^{\colors - 1 - \colortau} \right\rceil$ sites are \textit{stretched across the lost half of the ring buffer}.
From that point, our $\left\lceil 2^{\colors - 1 - \colortau} \right\rceil$ item snake will be chased into the remaining half of the ring buffer by overwrites at its rear.
The two conditions described above ensure that writes of fresh instances of h.v. $\colorh$ (1) pull far enough ahead and (2) stay far enough ahead of invading overwrites to keep at least the snake's leading $\left\lceil 2^{\colors - 1 - \colortau} \right\rceil$ body segments intact.
Mixing metaphors, the snake drags its head and then its tail to safety in the remaining $\left\lceil 2^{\colors - 1 - \colortau} \right\rceil$ reserved ring buffer sites as the rickety bridge of re-reserved but not-yet-overwritten sites it had been occupying collapses behind it.
After escaping, the snake happily crawls in circles around its $\left\lceil 2^{\colors - 1 - \colortau} \right\rceil$ reserved sites --- at least, until invaded again.

\begin{proofpart}[Before First Overwrite]
Let $\colorT' = \min\{\colorT \in \colort + 1\}$.
Because $\colorT' \in 2^{\mathbb{N}} \forall \colort$, we have $\colorH(\colorT') = 0$ and, by the fractal property of the hanoi sequence,
\begin{align*}
\colorH(\colorT'), \colorH(\colorT'+1), \,\ldots, \colorH(2\colorT' - 1) = \colorH(0), \colorH(1), \,\ldots, \colorH(\colorT' - 1).
\end{align*}
Note that $2\colorT' - 1 = \max\{\colorT \in \colort + 1\}$.

By Lemma \ref{thm:tilted-invader-minus-invaded} for h.v. $\colorHcal_{\colort}(\colork)$ invaded by h.v. $\colorHcal_{\colort + 1}(\colork)$ ($\colorHcal_{\colort}(\colork) \neq \colorHcal_{\colort + 1}(\colork)$), we have $\colorHcal_{\colort + 1}(\colork) \geq \colorHcal_{\colort}(\colork) + 2$.
Recalling that the first instance of h.v. $\colorh$ occurs at ingest time $\colorT = 2^{\colorh} - 1$, note that $\forall\colorh$,
\begin{align*}
|\{\colorT \in 0,\,\ldots, 3 \times 2^{\colorh} : \colorH(\colorT) = \colorh\}| = 2.
\end{align*}
\end{proofpart}
Because $\colorT = 3 \times 2^{\colorh} < 2^{\colorh + 2} - 1$, we have our result.

\begin{proofpart}[Overwrite Cadence]
The cadence of h.v. $\colorh$, after its first incidence at ingest time $\colorT=2^{\colorh} - 1$ is to occur every $2^{\colorh + 1}$ data items, where $\colorT \bmod 2^{\colorh + 1} = 2^{\colorh} - 1$.
Observe also that the cadence at which a h.v. $\geq\colorh$ occurs is every $2^{\colorh}$ items, where $\colorT \bmod 2^{\colorh} = 2^{\colorh} - 1$.

Again, by Lemma \ref{thm:tilted-invader-minus-invaded} for h.v. $\colorHcal_{\colort}(\colork)$ invaded by h.v. $\colorHcal_{\colort + 1}(\colork)$ ($\colorHcal_{\colort}(\colork) \neq \colorHcal_{\colort + 1}(\colork)$), we have $\colorHcal_{\colort + 1}(\colork) \geq \colorHcal_{\colort}(\colork) + 2$.
New incidences of invaded h.v. $\colorHcal_{\colort}(\colork)$ accrue faster than the final $\left\lceil 2^{\colors - 1 - \colortau} \right\rceil$ instances of h.v. $\colorh$ from epoch $\colort$ are overwritten because
\begin{align*}
2^{\colorh + 1} \stackrel{\checkmark}{<} 2^{\colorh + 2}.
\end{align*}
\end{proofpart}

\end{proof}


\subsection{Tilted Algorithm Implementation}
\label{sec:tilted-implementation}

\begin{figure}[htbp!]
  \centering

\begin{minipage}{\linewidth}
  \footnotesize
  \setlength{\tabcolsep}{2.5pt}
  % key-column + │ + 11 value columns = 12 columns total
  \begin{tabularx}{\linewidth}{@{}r|*{11}{Y}@{}}

  % ----------  T = 0–9 ----------
  { Time $\colorT$} &
    {0} & {1} & {2} & {3} & {4}
    & {5} & {6} & {7} & {8} & {9} & \ldots \\ \hline
  { \textit{Epoch} $\colort$} &
    \textit{0} & \textit{0} & \textit{0} & \textit{0} & \textit{0} & \textit{0} & \textit{0} & \textit{0} & \textit{0} & \textit{0} & \ldots \\
  { \textit{Meta-epoch} $\colortau$} &
    \textit{0} & \textit{0} & \textit{0} & \textit{0} & \textit{0} & \textit{0} & \textit{0} & \textit{0} & \textit{0} & \textit{0} & \ldots \\
  { \footnotesize \textit{Hanoi value} $\colorH(\colorT)$} &
    \textit{0} & \textit{1} & \textit{0} & \textit{2} & \textit{0} & \textit{1} & \textit{0} & \textit{3} & \textit{0} & \textit{1} & \ldots \\
  \rowcolor{lightgray!30}
  { \footnotesize \textbf{Buffer site} $\colorK(\colorT)$} &
    \textbf{0} & \textbf{1} & \textbf{9} & \textbf{2} & \textbf{6} & \textbf{10}
    & \textbf{13} & \textbf{3} & \textbf{5} & \textbf{7} & \ldots \\
  \multicolumn{12}{c}{} \\[1ex]

  % ----------  T = 10–19 ----------
  { $\colorT$} &
    {10} & {11} & {12} & {13} & {14}
    & {15} & {16} & {17} & {18} & {19} & \ldots \\ \hline
  { $\colort$} &
    \textit{0} & \textit{0} & \textit{0} & \textit{0} & \textit{0} & \textit{0} & \textit{1} & \textit{1} & \textit{1} & \textit{1} & \ldots \\
  { $\colortau$} &
    \textit{0} & \textit{0} & \textit{0} & \textit{0} & \textit{0} & \textit{0} & \textit{1} & \textit{1} & \textit{1} & \textit{1} & \ldots \\
  { $\colorH(\colorT)$} &
    \textit{0} & \textit{2} & \textit{0} & \textit{1} & \textit{0} & \textit{4} & \textit{0} & \textit{1} & \textit{0} & \textit{2} & \ldots \\
  \rowcolor{lightgray!30}
  { $\colorK(\colorT)$} &
    \textbf{8} & \textbf{11} & \textbf{12} & \textbf{14} & \textbf{15}
    & \textbf{4} & \textbf{0} & \textbf{1} & \textbf{9} & \textbf{8} & \ldots \\
  \multicolumn{12}{c}{} \\[1ex]

  % ----------  T = 20–29 ----------
  { $\colorT$} &
    {20} & {21} & {22} & {23} & {24}
    & {25} & {26} & {27} & {28} & {29} & \ldots \\ \hline
  { $\colort$} &
    \textit{1} & \textit{1} & \textit{1} & \textit{1} & \textit{1} & \textit{1} & \textit{1} & \textit{1} & \textit{1} & \textit{1} & \ldots \\
  { $\colortau$} &
    \textit{1} & \textit{1} & \textit{1} & \textit{1} & \textit{1} & \textit{1} & \textit{1} & \textit{1} & \textit{1} & \textit{1} & \ldots \\
  { $\colorH(\colorT)$} &
    \textit{0} & \textit{1} & \textit{0} & \textit{3} & \textit{0} & \textit{1} & \textit{0} & \textit{2} & \textit{0} & \textit{1} & \ldots \\
  \rowcolor{lightgray!30}
  { $\colorK(\colorT)$} &
    \textbf{6} & \textbf{10} & \textbf{13} & \textbf{12} & \textbf{0}
    & \textbf{7} & \textbf{9} & \textbf{15} & \textbf{6} & \textbf{14} & \ldots \\
  \multicolumn{12}{c}{} \\[1ex]

  % ----------  T = 30–39 ----------
  { $\colorT$} &
    {30} & {31} & {32} & {33} & {34}
    & {35} & {36} & {37} & {38} & {39} & \ldots \\ \hline
  { $\colort$} &
    \textit{1} & \textit{1} & \textit{2} & \textit{2} & \textit{2} & \textit{2} & \textit{2} & \textit{2} & \textit{2} & \textit{2} & \ldots \\
  { $\colortau$} &
    \textit{1} & \textit{1} & \textit{2} & \textit{2} & \textit{2} & \textit{2} & \textit{2} & \textit{2} & \textit{2} & \textit{2} & \ldots \\
  { $\colorH(\colorT)$} &
    \textit{0} & \textit{5} & \textit{0} & \textit{1} & \textit{0} & \textit{2} & \textit{0} & \textit{1} & \textit{0} & \textit{3} & \ldots \\
  \rowcolor{lightgray!30}
  { $\colorK(\colorT)$} &
    \textbf{13} & \textbf{5} & \textbf{0} & \textbf{1} & \textbf{9}
    & \textbf{2} & \textbf{0} & \textbf{10} & \textbf{9} & \textbf{3} & \ldots \\
  \end{tabularx}
  \vspace{-2ex}
\end{minipage}


\begin{subfigure}{\linewidth}
\caption{\footnotesize Site selection $\colorK(\colorT)$ for buffer size $\colorS=16$.}
\label{fig:hsurf-tilted-implementation-site-selection}
\end{subfigure}
\vspace{-3ex}

\begin{subfigure}[b]{\linewidth}
\includegraphics[width=\linewidth,trim={0 0 0 16.5cm},clip]{%
binder-20a-tilted-eulerian-surface/binder/teeplots/20a/font.family=serif+num-generations=256+strip-plotter=site_reservation_at_rank_stripped_heatmap+surface-size=16+viz=site-reservation-by-rank-spliced-at-heatmap+ext=.pdf}
\vspace{-4.5ex}\caption{\footnotesize
  Buffer composition over time from bottom to top, split by epoch with data items color-coded by \hv{} $\colorH(\colorTbar)$.
  Animated version at \url{https://hopth.ru/ef}.
}
\label{fig:hsurf-tilted-implementation-schematic}
\end{subfigure}

% \vspace{0.5ex}
% \begin{minipage}[]{\textwidth}
%  \vspace{-2pt}
%   \begin{subfigure}[t]{0.65\linewidth}
%     \vspace{0pt}
%     \centering
%   \includegraphics[width=0.88\linewidth,clip]{binder/teeplots/20/cnorm=log+num-generations=4096+surface-size=256+viz=site-ingest-depth-by-rank-heatmap+ynorm=linear+ext=.png}  % pdf cbar is scrambled
%   \end{subfigure}%
%   \begin{subfigure}[t]{0.35\linewidth}
%   \vspace{-2pt}
%   \caption{%
%     \footnotesize
%     Stored data item age across buffer sites for buffer size $\colorS=256$ from $\colorT=0$ to 4,096.
%   }
%   \label{fig:hsurf-tilted-implementation-heatmap}
% \end{subfigure}
% \end{minipage}

%   \vspace{-0.5ex}
%    \begin{minipage}[]{\textwidth}
%    \vspace{-2pt}
%   \begin{subfigure}[t]{0.65\linewidth}
%   \vspace{0pt}
%     \centering
%     \includegraphics[width=0.88\linewidth,clip]{binder/teeplots/20/num-generations=262144+surface-size=64+viz=stratum-persistence-dripplot+ext=.pdf}
%   \end{subfigure}%
%   \begin{subfigure}[t]{0.35\linewidth}
%   \vspace{-2pt}
%   \caption{%
%     \footnotesize
%     Data item retention time spans by ingestion time point for buffer size $\colorS=64$ from $\colorT=0$ to 3,000.
%   }
%   \label{fig:hsurf-tilted-implementation-dripplot}
%   \end{subfigure}
%   \end{minipage}

%   \vspace{-0.5ex}
%  \begin{minipage}[]{\textwidth}
%  \vspace{-2pt}
% \begin{subfigure}[t]{0.65\linewidth}
% \vspace{0pt}
%   \centering
%   \includegraphics[width=0.88\linewidth,clip]{binder/teeplots/20/hue=kind+surface-size=16+viz=criterion-satisfaction-lineplot+x=rank+y=tilted-criterion+ext=.pdf}
% \end{subfigure}%
% \begin{subfigure}[t]{0.35\linewidth}
% \vspace{-2pt}
% \caption{%
%   \footnotesize
%   Tilted criterion satisfaction across time points for buffer size $\colorS=16$.
% }
% \label{fig:hsurf-tilted-implementation-satisfaction}
% \end{subfigure}
% \end{minipage}

\vspace{-2ex}\caption{%
  \textbf{Example tilted-generalized ring buffer behavior.}
  \footnotesize
  Top panel \ref{fig:hsurf-tilted-implementation-site-selection} enumerates storage site selection on a 16-site buffer for $\colorT=0 \ldots 40$.
  Bottom panel \ref{fig:hsurf-tilted-implementation-schematic} depicts time course of stored data items, color-coded by data items' hanoi values $\colorH(\colorTbar)$.
  Time progresses from bottom to top.
  Between $\colorT=0$ and $\colorT=63$, time is segmented into epochs $\colort=0$, $\colort=1$, $\colort=2$, and $\colort=3$; strips before each epoch show hanoi values assigned to each buffer site during that epoch.
  Rectangles with small white ``$\blkhorzoval$'' symbol denote storage of ingested data item.
  In epoch 0, all sites are filled with a first data item.
  Then, in subsequent epochs, low \hv{} data items for which a newly-allocated reservation site is not available ``cycle'' within sites reserved for that \hv{} (ensuring most recent data items corresponding to that \hv{} are retained).
  % Heatmap panel \ref{fig:hsurf-tilted-implementation-heatmap} shows evolution of data item age at buffer sites.
  % Dripplot panel \ref{fig:hsurf-tilted-implementation-dripplot} shows retention spans for 3,000 ingested time points.
  % Vertical lines span durations between ingestion and elimination for data items from successive time points.
  % Time points previously eliminated are marked in red.
  % Lineplot panel \ref{fig:hsurf-tilted-implementation-satisfaction} shows tilted criterion satisfaction on a 16-bit surface over $2^{16}$ timepoints.
  % Lower and upper shaded areas are best- and worst-case bounds, respectively.
}
\label{fig:hsurf-tilted-implementation}

\end{figure}


Site selection for data ingest proceeds similarly to that of the stretched algorithm, described in Section \ref{sec:stretched} -- with the exception that, instead of discarding data items once available sites reserved to that h.v. have filled, we simply cycle back and overwrite data items within that h.v.'s reservations in the order that they were added.
In practice, the target index among available sites reserved to a h.v. can be calculated as the number of previous times a h.v. has been encountered before time $\colorT$, modulus the number of sites reserved to that hanoi value.
Figure \ref{fig:hsurf-stretched-implementation-schematic} illustrates site selection over epochs $\colort=0,1,2$ on buffer size $\colorS=32$.

Ingestion time calculation for lookup of the data item at buffer site $\colork$ at time $\colorT$ can be accomplished by decoding that site's segment index and checking whether (if slated) it has yet been replaced during the current epoch $\colort$.
As before, both site selection and ingest time calculation can be accomplished through fast $\mathcal{O}(1)$ binary operations (e.g., bit mask, bit shift, count leading zeros).
We refer the reader to our supplemental Python-language implementation for an exact step-by-step listing of both operations for the stretched algorithm.

\subsection{Tilted Algorithm Criterion Satisfaction}
\label{sec:tilted-satisfaction}

In this final subsection, we establish an upper bound on gap size ratio $\colorg/(\colorT - \colorTbar)$ for a buffer of size $\colorS$ at time $\colorT$ under the proposed tilted curation algorithm.

\begin{theorem}[Tilted algorithm gap size ratio upper bound]
\label{thm:tilted-gap-size}
Under the tilted curation algorithm, gap size ratio is bounded per Equation \ref{eqn:tilted-gap-size-bound}.
\end{theorem}
\begin{proof}

From Lemma \ref{thm:gap-size-ratio-tilted}, we have that if the first $n$ instances of each \hv{} $\colorh$ are retained then gap size ratio is bounded below by $1/(n - 1/2)$.
Substituting expressions for the number of sites reserved per \hv{} $n$ from Lemma \ref{thm:stretched-discarded-incidence-count} and Corollary \ref{thm:stretched-reservation-count} gives the result.
\end{proof}


Consideration of the gap size bound result found in Theorem \ref{thm:stretched-gap-size} at early and late epochs $\colort$ is instructive.
During early epoch $\colort = 1$, $\colorG_{\colorT}(\colorTbar)/(\colorT - \colorTbar) \leq 4/\colorS$.
Likewise, during the last supported epoch, where $\colortau = \colors$, $\colorG_{\colorT}(\colorTbar)/(\colorT - \colorTbar) \leq 2$.
Figure \ref{fig:hsurf-tilted-implementation-satisfaction} shows algorithm performance on the stretched criterion for buffer size $\colorS=16$, $0 \leq \colorT < 2^{\colorS}$.
