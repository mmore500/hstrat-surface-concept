\section{Tilted Algorithm} \label{sec:tilted}

The tilted criterion favors recent data items, mandating a record spaced proportionally to time elapsed since ingest, $\colorT - \colorTbar$.
This is opposite to the stretched criterion, which favors early data items.
As given in Equation \ref{eqn:tilted-cost} in Section \ref{sec:stream-curation-problem}, the tilted criterion's cost function is the largest ratio of gap size to ingest time,
\begin{align*}
\max_{\colorTbar} \frac{\colorG_{\colorT}(\colorTbar)}{\colorT - \colorTbar} \text{ for } \colorTbar < \colorT.
\end{align*}

The approximate lower bound on best-case gap size ratio provided in Equation \ref{eqn:approx-gap-bound} for the stretched curation can also be applied to tilted curation, as can the strict bound on best-case gap size ratio accounting for discretization effects established in Theorem \ref{thm:stretched-ideal-strict}.
In this section, we present a stream curation algorithm tailored to the tilted criterion, achieving maximum gap size ratio no worse than
\begin{align}
  \frac{\colorG_{\colorT}(\colorTbar)}{\colorT - \colorTbar}
  &\leq
  \frac{
    1
  }{
    \max\Big(
      \frac{\colorS}{2(\colort + \colors)},\;\;
      \frac{\colorS}{4\colort},\;\;
      \frac{\colorS}{2^{\colortau + 1}}
    \Big)
    - 1/2
  }
  \label{eqn:tilted-gap-size-bound}
\end{align}
over supported epochs $\colort \in [0\twodots\colorS - \colors)$.
Because $\min(2\colort + 2\colors,\;\; 4\colort,\;\; 2^{\colors - \colortau - 1}) \leq \colorS$, tilted gap size ratio is no greater than a factor of $2(1 + 1/\colorS)\times\min(2\colort + 2\colors, \;\; 4\colort, \;\; 2^{\colortau + 1})$ times the optimal bound established in Equation \ref{eqn:stretched-best}.
Additionally, gap size ratio is bounded $\frac{\colorG_{\colorT}(\colorTbar)}{\colorT - \colorTbar} \leq 2$.

% (1/((S/X) - 1/2)) / (1 / (1 + S))
% (2/((S/X))) / (1 / (1 + S))
% 2(1 + 1/S) * X / S

\subsection{Tilted Algorithm Strategy}
\label{sec:tilted-strategy}

\begin{figure*}
  \centering
\begin{subfigure}{0.43\textwidth}
\includegraphics[width=\textwidth]{img/hsurf-tilted-intuition}
\caption{TODO}
\end{subfigure}%
\begin{subfigure}{0.57\textwidth}
\includegraphics[width=\textwidth]{binder/teeplots/20/surface-size=16+viz=site-reservation-at-ranks-heatmap+ext=}
\caption{TODO}
\end{subfigure}
  \caption{
    \textbf{Tilted algorithm strategy.}
    \footnotesize
    TODO
    % Overview of placement strategy for data item instances of hanoi value $\colorh$.
    % Data items are shown in yellow, spanning across the epoch timeline from when they are stored to when they are overwritten.
    % Available memory buffer sites, broken into conceptual segments, are shown across the top of the schematic.
    % The first data item with $\colorH(\colorT) = \colorh$ is placed in segment 0.
    % The next data item with h.v. $\colorh$ is encountered in the following epoch, and it is placed in segment 1.
    % In epoch 3, two data items with h.v. $\colorh$ are encountered, and placed in segment 2.
    % In the epoch 4, four h.v. $\colorh$ data items (twice as many) are encountered.
    % We place them in segment 3.
    % If working with a larger memory buffer, this pattern would continue in subsequent epochs.
    % During the epoch after the last segment is written to, all h.v. $\colorh$ data items are overwritten.
    % In segment 0 they are overwritten by h.v. $\colorh - 4$, in segment 1, h.v. $\colorh - 3$, and so forth.
    % In this manner, data items with highest hanoi value are retained on a rolling basis.
    % This approach ensures equal gap sizes across history, with gap sizes gracefully doubling in size as intermediate values are overwritten.
  }
  \label{fig:hsurf-tilted-intuition}
\end{figure*}


The retention strategy for the tilted algorithm strongly resembles that of the stretched algorithm.
Recall that under the stretched algorithm the first $n(\colorT)$ data items of each \hv{} $\colorH(\colorTbar)$ are retained, with $n(\colorT)$ decreasing so as to shift from many copies of few encountered hanoi values to few copies of many encountered hanoi values.
Under the tilted algorithm, we instead keep the \textit{last} $n(\colorT)$ data items of each hanoi value.
Figure \ref{fig:hanoi-intuition-tilted} shows how keeping the last $n$ instances of each \hv{} approximates tilted distribution.

We thus define our algorithmic goal as maintaining --- for a declining threshold $n(\colorT)$ --- the set of data items,
\begin{align*}
\\
\textsf{goal\_tilted}
&\coloneq
\bigcup_{\colorh \in [0 \twodots \left\lfloor \log_2(\colorT + 1) \right\rfloor]}
\{ \colorTbar =
\eqnmarkbox[gray]{maxhanoi}{
  \left\lfloor
  \frac{\colorT - 2^{\colorh} + 1}{2^{\colorh + 1}}
  \right\rfloor
  2^{\colorh + 1}
  + 2^{\colorh}
  - 1
}
- i2^{\colorh + 1} \text{ for } i \in [0 \twodots n(\colorT) - 1] : \colorTbar \geq 0 \}.
\annotate[yshift=1em]{above,left}{maxhanoi}{$\max\{
  \colorTbar' \in [0 \twodots \colorT] : \colorH(\colorTbar') = \colorh
\}$}
\end{align*}

Note also that by construction, $\textsf{goal\_tilted} \subset \{ \colorTbar \in [0 \twodots \colorT] \}$.
It can be shown analogously to the stretched algorithm's Lemma \ref{thm:stretched-first-n-space} that setting $n(\colorT) \coloneq 2^{\colors - 1 - \colortau}$ suffices to respect available buffer capacity $\colorS$ under the tilted algorithm.

\subsection{Tilted Algorithm Mechanism}
\label{sec:tilted-mechanism}

Because the tilted algorithm, like the stretched algorithm, also approximates an equal-$n$-per-\hv{} scheme, hanoi value reservation layout is maintained identically to the stretched algorithm's segment-based scheme.
Refer to Section \ref{sec:stretched-mechanism} for a detailed description of this \hv{} reservation layout, and how it unfolds across epochs $0 \leq \colort \leq \colorS - \colors$.

A pertinent result of the stretched layout is that at least $2^{\colors - 1 - \colortau}$ data item instances of each \hv{} are retained (Lemma \ref{thm:stretched-discarded-incidence-count}).
However, unlike the stretched algorithm, for the tilted algorithm we wish to keep the \textit{last} $n$ rather than the \textit{first} $n$ instances of each hanoi value.
We can do that by continuing to write data items for each \hv{} into buffer sites reserved for that \hv{} after they initially fill --- overwriting older instances of the \hv{} to keep a ``ring buffer'' of fresh \hv{} instances.

Supplemental materials prove several results related to the tilted algorithm's ring buffer mechanism, including that fill cycles align evenly to epoch and meta-epoch transitions (Lemma \ref{thm:tilted-last-touched}).
These results build to Lemma \ref{thm:tilted-most-recent-retained}, which confirms that our strategy always preserves the last $2^{\colors - 1 - \colortau}$ instances of each hanoi value.
We take particular care in considering transitions where the ``ring buffer'' of sites reserved to a \hv{} is halved by growth of invading segments.

\subsection{Tilted Algorithm Implementation}
\label{sec:tilted-implementation}

\begin{figure}[htbp!]
  \centering

\begin{minipage}{\linewidth}
  \footnotesize
  \setlength{\tabcolsep}{2.5pt}
  % key-column + │ + 11 value columns = 12 columns total
  \begin{tabularx}{\linewidth}{@{}r|*{11}{Y}@{}}

  % ----------  T = 0–9 ----------
  { Time $\colorT$} &
    {0} & {1} & {2} & {3} & {4}
    & {5} & {6} & {7} & {8} & {9} & \ldots \\ \hline
  { \textit{Epoch} $\colort$} &
    \textit{0} & \textit{0} & \textit{0} & \textit{0} & \textit{0} & \textit{0} & \textit{0} & \textit{0} & \textit{0} & \textit{0} & \ldots \\
  { \textit{Meta-epoch} $\colortau$} &
    \textit{0} & \textit{0} & \textit{0} & \textit{0} & \textit{0} & \textit{0} & \textit{0} & \textit{0} & \textit{0} & \textit{0} & \ldots \\
  { \footnotesize \textit{Hanoi value} $\colorH(\colorT)$} &
    \textit{0} & \textit{1} & \textit{0} & \textit{2} & \textit{0} & \textit{1} & \textit{0} & \textit{3} & \textit{0} & \textit{1} & \ldots \\
  \rowcolor{lightgray!30}
  { \footnotesize \textbf{Buffer site} $\colorK(\colorT)$} &
    \textbf{0} & \textbf{1} & \textbf{9} & \textbf{2} & \textbf{6} & \textbf{10}
    & \textbf{13} & \textbf{3} & \textbf{5} & \textbf{7} & \ldots \\
  \multicolumn{12}{c}{} \\[1ex]

  % ----------  T = 10–19 ----------
  { $\colorT$} &
    {10} & {11} & {12} & {13} & {14}
    & {15} & {16} & {17} & {18} & {19} & \ldots \\ \hline
  { $\colort$} &
    \textit{0} & \textit{0} & \textit{0} & \textit{0} & \textit{0} & \textit{0} & \textit{1} & \textit{1} & \textit{1} & \textit{1} & \ldots \\
  { $\colortau$} &
    \textit{0} & \textit{0} & \textit{0} & \textit{0} & \textit{0} & \textit{0} & \textit{1} & \textit{1} & \textit{1} & \textit{1} & \ldots \\
  { $\colorH(\colorT)$} &
    \textit{0} & \textit{2} & \textit{0} & \textit{1} & \textit{0} & \textit{4} & \textit{0} & \textit{1} & \textit{0} & \textit{2} & \ldots \\
  \rowcolor{lightgray!30}
  { $\colorK(\colorT)$} &
    \textbf{8} & \textbf{11} & \textbf{12} & \textbf{14} & \textbf{15}
    & \textbf{4} & \textbf{0} & \textbf{1} & \textbf{9} & \textbf{8} & \ldots \\
  \multicolumn{12}{c}{} \\[1ex]

  % ----------  T = 20–29 ----------
  { $\colorT$} &
    {20} & {21} & {22} & {23} & {24}
    & {25} & {26} & {27} & {28} & {29} & \ldots \\ \hline
  { $\colort$} &
    \textit{1} & \textit{1} & \textit{1} & \textit{1} & \textit{1} & \textit{1} & \textit{1} & \textit{1} & \textit{1} & \textit{1} & \ldots \\
  { $\colortau$} &
    \textit{1} & \textit{1} & \textit{1} & \textit{1} & \textit{1} & \textit{1} & \textit{1} & \textit{1} & \textit{1} & \textit{1} & \ldots \\
  { $\colorH(\colorT)$} &
    \textit{0} & \textit{1} & \textit{0} & \textit{3} & \textit{0} & \textit{1} & \textit{0} & \textit{2} & \textit{0} & \textit{1} & \ldots \\
  \rowcolor{lightgray!30}
  { $\colorK(\colorT)$} &
    \textbf{6} & \textbf{10} & \textbf{13} & \textbf{12} & \textbf{0}
    & \textbf{7} & \textbf{9} & \textbf{15} & \textbf{6} & \textbf{14} & \ldots \\
  \multicolumn{12}{c}{} \\[1ex]

  % ----------  T = 30–39 ----------
  { $\colorT$} &
    {30} & {31} & {32} & {33} & {34}
    & {35} & {36} & {37} & {38} & {39} & \ldots \\ \hline
  { $\colort$} &
    \textit{1} & \textit{1} & \textit{2} & \textit{2} & \textit{2} & \textit{2} & \textit{2} & \textit{2} & \textit{2} & \textit{2} & \ldots \\
  { $\colortau$} &
    \textit{1} & \textit{1} & \textit{2} & \textit{2} & \textit{2} & \textit{2} & \textit{2} & \textit{2} & \textit{2} & \textit{2} & \ldots \\
  { $\colorH(\colorT)$} &
    \textit{0} & \textit{5} & \textit{0} & \textit{1} & \textit{0} & \textit{2} & \textit{0} & \textit{1} & \textit{0} & \textit{3} & \ldots \\
  \rowcolor{lightgray!30}
  { $\colorK(\colorT)$} &
    \textbf{13} & \textbf{5} & \textbf{0} & \textbf{1} & \textbf{9}
    & \textbf{2} & \textbf{0} & \textbf{10} & \textbf{9} & \textbf{3} & \ldots \\
  \end{tabularx}
  \vspace{-2ex}
\end{minipage}


\begin{subfigure}{\linewidth}
\caption{\footnotesize Site selection $\colorK(\colorT)$ for buffer size $\colorS=16$.}
\label{fig:hsurf-tilted-implementation-site-selection}
\end{subfigure}
\vspace{-3ex}

\begin{subfigure}[b]{\linewidth}
\includegraphics[width=\linewidth,trim={0 0 0 16.5cm},clip]{%
binder-20a-tilted-eulerian-surface/binder/teeplots/20a/font.family=serif+num-generations=256+strip-plotter=site_reservation_at_rank_stripped_heatmap+surface-size=16+viz=site-reservation-by-rank-spliced-at-heatmap+ext=.pdf}
\vspace{-4.5ex}\caption{\footnotesize
  Buffer composition over time from bottom to top, split by epoch with data items color-coded by \hv{} $\colorH(\colorTbar)$.
  Animated version at \url{https://hopth.ru/ef}.
}
\label{fig:hsurf-tilted-implementation-schematic}
\end{subfigure}

% \vspace{0.5ex}
% \begin{minipage}[]{\textwidth}
%  \vspace{-2pt}
%   \begin{subfigure}[t]{0.65\linewidth}
%     \vspace{0pt}
%     \centering
%   \includegraphics[width=0.88\linewidth,clip]{binder/teeplots/20/cnorm=log+num-generations=4096+surface-size=256+viz=site-ingest-depth-by-rank-heatmap+ynorm=linear+ext=.png}  % pdf cbar is scrambled
%   \end{subfigure}%
%   \begin{subfigure}[t]{0.35\linewidth}
%   \vspace{-2pt}
%   \caption{%
%     \footnotesize
%     Stored data item age across buffer sites for buffer size $\colorS=256$ from $\colorT=0$ to 4,096.
%   }
%   \label{fig:hsurf-tilted-implementation-heatmap}
% \end{subfigure}
% \end{minipage}

%   \vspace{-0.5ex}
%    \begin{minipage}[]{\textwidth}
%    \vspace{-2pt}
%   \begin{subfigure}[t]{0.65\linewidth}
%   \vspace{0pt}
%     \centering
%     \includegraphics[width=0.88\linewidth,clip]{binder/teeplots/20/num-generations=262144+surface-size=64+viz=stratum-persistence-dripplot+ext=.pdf}
%   \end{subfigure}%
%   \begin{subfigure}[t]{0.35\linewidth}
%   \vspace{-2pt}
%   \caption{%
%     \footnotesize
%     Data item retention time spans by ingestion time point for buffer size $\colorS=64$ from $\colorT=0$ to 3,000.
%   }
%   \label{fig:hsurf-tilted-implementation-dripplot}
%   \end{subfigure}
%   \end{minipage}

%   \vspace{-0.5ex}
%  \begin{minipage}[]{\textwidth}
%  \vspace{-2pt}
% \begin{subfigure}[t]{0.65\linewidth}
% \vspace{0pt}
%   \centering
%   \includegraphics[width=0.88\linewidth,clip]{binder/teeplots/20/hue=kind+surface-size=16+viz=criterion-satisfaction-lineplot+x=rank+y=tilted-criterion+ext=.pdf}
% \end{subfigure}%
% \begin{subfigure}[t]{0.35\linewidth}
% \vspace{-2pt}
% \caption{%
%   \footnotesize
%   Tilted criterion satisfaction across time points for buffer size $\colorS=16$.
% }
% \label{fig:hsurf-tilted-implementation-satisfaction}
% \end{subfigure}
% \end{minipage}

\vspace{-2ex}\caption{%
  \textbf{Example tilted-generalized ring buffer behavior.}
  \footnotesize
  Top panel \ref{fig:hsurf-tilted-implementation-site-selection} enumerates storage site selection on a 16-site buffer for $\colorT=0 \ldots 40$.
  Bottom panel \ref{fig:hsurf-tilted-implementation-schematic} depicts time course of stored data items, color-coded by data items' hanoi values $\colorH(\colorTbar)$.
  Time progresses from bottom to top.
  Between $\colorT=0$ and $\colorT=63$, time is segmented into epochs $\colort=0$, $\colort=1$, $\colort=2$, and $\colort=3$; strips before each epoch show hanoi values assigned to each buffer site during that epoch.
  Rectangles with small white ``$\blkhorzoval$'' symbol denote storage of ingested data item.
  In epoch 0, all sites are filled with a first data item.
  Then, in subsequent epochs, low \hv{} data items for which a newly-allocated reservation site is not available ``cycle'' within sites reserved for that \hv{} (ensuring most recent data items corresponding to that \hv{} are retained).
  % Heatmap panel \ref{fig:hsurf-tilted-implementation-heatmap} shows evolution of data item age at buffer sites.
  % Dripplot panel \ref{fig:hsurf-tilted-implementation-dripplot} shows retention spans for 3,000 ingested time points.
  % Vertical lines span durations between ingestion and elimination for data items from successive time points.
  % Time points previously eliminated are marked in red.
  % Lineplot panel \ref{fig:hsurf-tilted-implementation-satisfaction} shows tilted criterion satisfaction on a 16-bit surface over $2^{16}$ timepoints.
  % Lower and upper shaded areas are best- and worst-case bounds, respectively.
}
\label{fig:hsurf-tilted-implementation}

\end{figure}


Site selection for data ingest proceeds similarly to the stretched algorithm, described in Section \ref{sec:stretched}.
However, instead of discarding data items after available sites reserved to that \hv{} fill, we simply cycle back and overwrite the first data items within that \hv{}'s reservations.
In practice, the target index among available sites reserved to a \hv{} can be calculated as the number of previous times a \hv{} has been encountered before time $\colorT$, modulus the number of sites reserved to that hanoi value.
Figure \ref{fig:hsurf-tilted-implementation-schematic} illustrates site selection over epochs $\colort \in \{0,1,2\}$ on buffer size $\colorS=32$.
Algorithm \ref{alg:tilted-site-selection} provides a step-by-step listing of the tilted site selection procedure, which is $\mathcal{O}(1)$.

\begin{algorithm}[H]
\caption{Tilted algorithm site selection $\colorK(\colorT)$.}
\label{alg:tilted-site-selection}
\begin{minipage}{0.53\textwidth}
    \hspace*{\algorithmicindent} \textbf{Input:} $\colorS \in \{2^{\mathbb{N}}\},\;\; \colorT \in \mathbb{N}$ \Comment{Buffer size and current logical time}\\
    \hspace*{\algorithmicindent} \textbf{Output:} $\colork \in [0 \twodots \colorS - 1) \cup \{\nullval\}$ \Comment{Selected site, if any}
    \begin{algorithmic}[1]
        \State $\texttt{uint\_t} ~ ~ \colors \gets \Call{BitLength}{\colorS} - 1$
        \State $\texttt{uint\_t} ~ ~ \colort \gets \max(0,\;\; \Call{BitLength}{\colorT+1} - \colors)$ \Comment{Current epoch}
        \State $\texttt{uint\_t} ~ ~ \colorh \gets \Call{CountTrailingZeros}{\colorT + 1}$ \Comment{Current \hv{}}
        \Statex
        \State $\texttt{uint\_t} ~ ~ i \gets \Call{RightShift}{\colorT, \;\; \colorh + 1}$ \Comment{Hanoi value incidence (i.e., num seen)}
        \State $\texttt{bool\_t} ~ ~ \epsilon_{\colortau} \gets \Call{BitFloorSafe}{2\colort} \;\; > \;\; \colort + \Call{BitLength}{\colort}$ \Comment{Correction factor}
        \State $\texttt{uint\_t} ~ ~ \colortau \gets  \Call{BitLength}{\colort} - \Call{Bool2Int}{\epsilon_{\colortau}}$ \Comment{Current meta-epoch}
        \State $\texttt{uint\_t} ~ ~ \colort_0 \gets 2^{\colortau} - \colortau$ \Comment{First epoch of meta-epoch}
        \State $\texttt{uint\_t} ~ ~ \colort_1 \gets 2^{\colortau + 1} - (\colortau + 1)$ \Comment{First epoch of next meta-epoch}
        \State $\texttt{uint\_t} ~ ~ \epsilon_b \gets \Call{Bool2Int}{\colort \;\; < \;\; \colorh + \colort_0 \;\; < \;\; \colort_1}$ \Comment{Uninvaded correction factor}
        \State $\texttt{uint\_t} ~ ~ b \gets \max(1,\;\; \Call{RightShift}{\colorS, \;\; \colortau + 1 - \epsilon_b})$ \Comment{Num bunches available to \hv}
        \Statex
        \State $\texttt{uint\_t} ~ ~ b_l \gets \Call{ModPow2}{i, \;\; b}$ \Comment{Logical bunch index, in order filled \ldots}
        \Statex \Comment{\ldots i.e., increasing nestedness/decreasing init size $r$}
        \Statex
        \Statex \Comment{Need to calculate physical bunch index\ldots}
        \Statex \Comment{\ldots i.e., among bunches left-to-right in buffer space}
        \Statex
        \State $\texttt{uint\_t} ~ ~ v \gets \Call{BitLength}{b_l}$ \Comment{Nestedness depth level for physical bunch}
        \State $\texttt{uint\_t} ~ ~ w \gets \Call{RightShift}{\colorS, \;\; v} \;\; \times \;\;\Call{Bool2Int}{v > 0}$ \Comment{Num bunches spaced between bunches in same nest level}
        \State $\texttt{uint\_t} ~ ~ o \gets 2w$  \Comment{Offset of nestedness level in physical bunch order}
        \State $\texttt{uint\_t} ~ ~ p \gets b_l - \Call{BitFloorSafe}{b_l}$ \Comment{Bunch position within nestedness level}
        \State $\texttt{uint\_t} ~ ~ b_p \gets o + wp$ \Comment{Physical bunch index\ldots}
        \Statex \Comment{\ldots i.e., in left-to-right buffer space ordering}
        \Statex
        \Statex \Comment{Need to calculate buffer position of $b_p$\textsuperscript{th} bunch}
        \Statex
        \State $\texttt{uint\_t} ~ ~ \epsilon_{\colork} = \Call{Bool2Int}{b_l > 0}$  \Comment{Correction factor, 0\textsuperscript{th} bunch (i.e., bunch $r=\colors$ at site $\colork=0$)}
        \State $\texttt{uint\_t} ~ ~ \colork \gets \Call{BitCount}{2b_p +(2\colorS - b_p)} - 1 - \epsilon_{\colork}$  \Comment{Site index of bunch}
        \Statex
        \State \Return $\colork + \colorh$ \Comment{Calculate placement site, \hv{} $\colorh$ is offset within bunch}
    \end{algorithmic}
\end{minipage}
\end{algorithm}


The data item $\colorTbar$ present at buffer site $\colork$ at time $\colorT$ can be determined by decoding that site's segment index and checking whether (if slated) it has yet been replaced during the current epoch $\colort$.
Both site selection $\colorK$ and lookup $\colorL$ can be accomplished through fast $\mathcal{O}(1)$ binary operations (e.g., bit mask, bit shift, count leading zeros).
Tilted site lookup is provided in supplementary material, as Algorithm \ref{alg:tilted-time-lookup}.
Reference Python implementations appear in Supplementary Listings \ref{lst:tilted_site_selection.py} and \ref{lst:tilted_time_lookup.py}, as well as accompanying tests.

\subsection{Tilted Algorithm Criterion Satisfaction}
\label{sec:tilted-satisfaction}

In this final subsection, we establish an upper bound on gap size ratio $\colorg/(\colorT - \colorTbar)$ for a buffer of size $\colorS$ at time $\colorT$ under the proposed tilted curation algorithm.

\begin{theorem}[Tilted algorithm gap size ratio upper bound]
\label{thm:tilted-gap-size}
Under the tilted curation algorithm, gap size ratio is bounded per Equation \ref{eqn:tilted-gap-size-bound}.
\end{theorem}
\begin{proof}

From Lemma \ref{thm:gap-size-ratio-tilted}, we have that if the first $n$ instances of each \hv{} $\colorh$ are retained then gap size ratio is bounded below by $1/(n - 1/2)$.
Substituting expressions for the number of sites reserved per \hv{} $n$ from Lemma \ref{thm:stretched-discarded-incidence-count} and Corollary \ref{thm:stretched-reservation-count} gives the result.
\end{proof}


During early epoch $\colort = 1$, $\colorG_{\colorT}(\colorTbar)/(\colorT - \colorTbar) \leq 4/\colorS$.
Likewise, during the last supported meta-epoch $\colortau = \colors - 1$, $\colorG_{\colorT}(\colorTbar)/(\colorT - \colorTbar) \leq 2$.
Figure \ref{fig:hsurf-tilted-implementation-satisfaction} shows algorithm performance on the tilted criterion for buffer size $\colorS=16$, $\colorT \in [0\twodots 2^{\colorS} - 1)$.
