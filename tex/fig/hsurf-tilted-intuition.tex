\begin{figure*}[htbp!]
  \centering

  \hfill
\begin{subfigure}{0.43\textwidth}
  \includegraphics[width=\textwidth]{img/hsurf-stretched-intuition}
  \caption{\footnotesize site selection for \hv{} $\colorh=0$ under \textit{\textbf{stretched}} algorithm}
  \label{fig:hsurf-tilted-intuition-site-selection}
\end{subfigure}
\hfill
\begin{subfigure}{0.43\textwidth}
  \includegraphics[width=\textwidth]{img/hsurf-tilted-intuition}
  \caption{\footnotesize site selection for \hv{} $\colorh=0$ under \textit{\textbf{tilted}} algorithm}
  \label{fig:hsurf-tilted-intuition-tilted}
\end{subfigure}
\hfill

  \caption{
    \textbf{Tilted algorithm strategy.}
    \footnotesize
    Tilted algorithm strategy relates closely to stretched algorithm strategy.
    In particular, the tilted algorithm uses \hv{} reservation layout $\colorHcal_{\colort}(\colork)$ exactly identical to the stretched algorithm ( shown in Figure \ref{fig:hsurf-stretched-intuition}).
    As contrasted between left and right panels, the tilted and stretched algorithms differ in how they handle \hv{} instances after available reservation segments have been filled.
    Schematics show site selection strategy for items with \hv{} $\colorH(\colorTbar) = 0$ on a buffer of size $\colorS=8$.
    Whereas the stretched algorithm discards these items, the tilted algorithm treats reserved segments as a ring buffer by ``wrapping around'' and beginning again from the largest (and leftmost) segment $r=\colors$.
    In this way, the most recent $n$ (as opposed to the first $n$) data items corresponding to each hanoi value are kept, satisfying the tilted retention objective depicted in Figure \ref{fig:hanoi-intuition-tilted}.
  % Because the distribution of segment lengths mirrors the hanoi sequence, buffer space becomes available for the first-seen instances of high hanoi values as they are encountered.
    So, placements for a particular \hv{} cycle around available reservation sites, and then continue cycling around remaining sites after the \hv{} is invaded and half of reservation sites for that \hv{} are ceded.
    % Invasion is sequenced to correspond cleanly with the cycle returning to segment 0 after filling to-be-eliminated sites.
    % Subsequent overwrites of data items at ceded reservation sites occur slowly enough to ensure that, with $n$ remaining reservation sites, the $n$ most recent \hv{} data items are retained as remaining sites are refilled.
  % In this manner, this layout strategy approximates the last-$n$ \hv{} strategy depicted in , with $n$ decreasing to respect available buffer space as segments are invaded and lost (just as for the stretched algorithm, Figure \ref{fig:hsurf-stretched-intuition-reservations}).
  }
  \label{fig:hsurf-tilted-intuition}
\end{figure*}
