\begin{figure*}
  \centering
\begin{subfigure}{0.43\textwidth}
\includegraphics[width=\textwidth]{img/hsurf-tilted-intuition}
\caption{TODO}
\end{subfigure}%
\begin{subfigure}{0.57\textwidth}
\includegraphics[width=\textwidth]{binder/teeplots/20/surface-size=16+viz=site-reservation-at-ranks-heatmap+ext=}
\caption{TODO}
\end{subfigure}
  \caption{
    \textbf{Tilted algorithm strategy.}
    \footnotesize
    TODO
    % Overview of placement strategy for data item instances of hanoi value $\colorh$.
    % Data items are shown in yellow, spanning across the epoch timeline from when they are stored to when they are overwritten.
    % Available memory buffer sites, broken into conceptual segments, are shown across the top of the schematic.
    % The first data item with $\colorH(\colorT) = \colorh$ is placed in segment 0.
    % The next data item with h.v. $\colorh$ is encountered in the following epoch, and it is placed in segment 1.
    % In epoch 3, two data items with h.v. $\colorh$ are encountered, and placed in segment 2.
    % In the epoch 4, four h.v. $\colorh$ data items (twice as many) are encountered.
    % We place them in segment 3.
    % If working with a larger memory buffer, this pattern would continue in subsequent epochs.
    % During the epoch after the last segment is written to, all h.v. $\colorh$ data items are overwritten.
    % In segment 0 they are overwritten by h.v. $\colorh - 4$, in segment 1, h.v. $\colorh - 3$, and so forth.
    % In this manner, data items with highest hanoi value are retained on a rolling basis.
    % This approach ensures equal gap sizes across history, with gap sizes gracefully doubling in size as intermediate values are overwritten.
  }
  \label{fig:hsurf-tilted-intuition}
\end{figure*}
