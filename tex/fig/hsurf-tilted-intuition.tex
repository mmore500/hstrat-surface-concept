\begin{figure*}[htbp!]
  \centering
\begin{minipage}{0.43\textwidth}
\includegraphics[width=\textwidth]{img/hsurf-tilted-intuition}
\end{minipage}%
\begin{minipage}{0.57\textwidth}
\centering
\begin{minipage}{0.95\linewidth}
  \caption{
    \textbf{Tilted algorithm strategy.}
    \footnotesize
    Schematic shows placement strategy for data items with hanoi value $\colorH(\colorTbar) = 0$ on a buffer of size $\colorS=8$.
    During epoch 0, an initial data item is placed at each buffer site juast as under the stretched algorithm strategy (Figure \ref{fig:hsurf-stretched-intuition}).
    The tilted strategy differs from stretched in that during epoch 1, subsequent h.v. 0 data items continue to be placed in available h.v. 0 reservation sites.
    In this way, the last --- as opposed to first --- data items corresponding to each hanoi value are kept.
  % Because the distribution of segment lengths mirrors the hanoi sequence, buffer space becomes available for the first-seen instances of high hanoi values as they are encountered.
    Each epoch, all invaded sites share the same hanoi value, causing the available reservation sites for that h.v. to instantaneously halve.
    So, placements for a particular h.v. cycle around available reservation sites, and then continue cycling around remaining sites after the h.v. is invaded and half of reservation sites for that h.v. are ceded.
    Invasion is sequenced to correspond cleanly with the cycle returning to segment 0 after filling to-be-eliminated sites.
    Subsequent overwrites of data items at ceded reservation sites occur slowly enough to ensure that, with $n$ remaining reservation sites, the $n$ most recent h.v. data items are retained as remaining sites are refilled.
  In this manner, this layout strategy approximates the last-$n$ h.v. strategy depicted in Figure \ref{fig:hanoi-intuition-tilted}, with $n$ decreasing to respect available buffer space as segments are invaded and lost (just as for the stretched algorithm, Figure \ref{fig:hsurf-stretched-intuition-reservations}).
  }
  \label{fig:hsurf-tilted-intuition}
\end{minipage}
\end{minipage}
\end{figure*}
