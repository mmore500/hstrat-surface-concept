\begin{figure*}[htbp!]
  \centering
  \includegraphics[width=\textwidth]{img/hsurf-steady-intuition}
  \caption{
    \textbf{Steady algorithm strategy.}
    \footnotesize
    Overview of placement strategy for data item instances of hanoi value $\colorh$.
    Data items are shown in yellow, spanning across the epoch timeline from when they are stored to when they are overwritten.
    Available memory buffer sites, broken into conceptual segments, are shown across the top of the schematic.
    The first data item with $\colorH(\colorTbar) = \colorh$ is placed in segment 0.
    The next data item with \hv{} $\colorh$ is encountered in the following epoch, and it is placed in segment 1.
    In epoch 3, two data items with \hv{} $\colorh$ are encountered and placed in segment 2.
    In the epoch 4, four \hv{} $\colorh$ data items (twice as many) are encountered.
    We place them in segment 3.
    If working with a larger memory buffer, this pattern would continue in subsequent epochs.
    During the epoch after the last segment is written to, all \hv{} $\colorh$ data items are overwritten.
    In segment 0 they are overwritten by \hv{} $\colorh - 4$, in segment 1, \hv{} $\colorh - 3$, and so forth.
    In this manner, data items with highest hanoi value are retained on a rolling basis.
    This approach ensures equal gap sizes across history, with gap sizes gracefully doubling in size as intermediate values are overwritten.
  }
  \label{fig:hsurf-steady-intuition}
\end{figure*}
