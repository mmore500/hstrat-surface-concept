\begin{figure*}[htbp!]
  \centering
  \begin{subfigure}[b]{\linewidth}
    \includegraphics[width=\textwidth, trim={0cm 0cm 0.3cm 1cm}, clip]{img/hsurf-steady-intuition}
    \vspace{-4.5ex}
    \caption{\footnotesize Site selection $\colorK(\colorT)$ for data items of one \hv{}, $\colorH(\colorT) = \colorh$.}
    \label{fig:hsurf-steady-intuition-diagram}
  \end{subfigure}
  \begin{subfigure}[b]{\linewidth}
    \includegraphics[width=0.011\textwidth, trim={0.2cm 2.8cm 31.8cm 2.8cm}, clip]{binder/teeplots/11/reservation-mode=steady+surface-size=32+viz=site-reservation-at-ranks-heatmap+ext=}%
    \begin{tikzpicture}

    % Include the image in a node
    \node [
      above right,
      inner sep=0] (image) at (0,0) {\includegraphics[width=0.989\textwidth, trim={0.7cm 0cm 1.4cm 5.6cm}, clip]{binder/teeplots/11/reservation-mode=steady+surface-size=32+viz=site-reservation-at-ranks-heatmap+ext=}};

    % Create scope with normalized axes
    \begin{scope}[
      x={($0.1*(image.south east)$)},
      y={($0.1*(image.north west)$)}]

    % Grid
      % \draw[lightgray,step=0.5] (image.south west) grid (image.north east);
    % Axes' labels
      % \foreach \x in {0,1,...,10} { \node [below] at (\x,0) {\x}; }
      % \foreach \y in {0,1,...,10} { \node [left] at (0,\y) {\y};}

      \draw[stealth-, ultra thick,seabornBrown] (0.37,2.72) -- ++(0, -0.5)node{};
      \draw[stealth-, ultra thick,seabornBrown] (2.21,3.79) -- ++(0, -0.5)node{};
      \draw[stealth-, ultra thick,seabornBrown] (3.74,4.86) -- ++(0, -0.5)node{};
      \draw[stealth-, ultra thick,seabornBrown] (4.66,4.86) -- ++(0, -0.5)node{};
      \draw[stealth-, ultra thick,seabornBrown] (5.27,5.96) -- ++(0, -0.5)node{};
      \draw[stealth-, ultra thick,seabornBrown] (5.88,5.96) -- ++(0, -0.5)node{};
      \draw[stealth-, ultra thick,seabornBrown] (6.5,5.96) -- ++(0, -0.5)node{};
      \draw[stealth-, ultra thick,seabornBrown] (7.12,5.96) -- ++(0, -0.5)node{};
      \draw[stealth-, ultra thick,seabornBrown] (7.41,7.06) -- ++(0, -0.5)node{};
      \draw[stealth-, ultra thick,seabornBrown] (7.72,7.06) -- ++(0, -0.5)node{};
      \draw[stealth-, ultra thick,seabornBrown] (8.02,7.06) -- ++(0, -0.5)node{};
      \draw[stealth-, ultra thick,seabornBrown] (8.32,7.06) -- ++(0, -0.5)node{};
      \draw[stealth-, ultra thick,seabornBrown] (8.61,7.06) -- ++(0, -0.5)node{};
      \draw[stealth-, ultra thick,seabornBrown] (8.94,7.06) -- ++(0, -0.5)node{};
      \draw[stealth-, ultra thick,seabornBrown] (9.25,7.06) -- ++(0, -0.5)node{};
      \draw[stealth-, ultra thick,seabornBrown] (9.55,7.06) -- ++(0, -0.5)node{};

      \draw[Bar-, very thick,seabornBrown] (0.37,7.9) -- ++(0, -0.3)node{};
      \draw[Bar-, very thick,seabornBrown] (2.21,7.9) -- ++(0, -0.3)node{};
      \draw[Bar-, very thick,seabornBrown] (3.74,7.9) -- ++(0, -0.3)node{};
      \draw[Bar-, very thick,seabornBrown] (4.66,7.9) -- ++(0, -0.3)node{};
      \draw[Bar-, very thick,seabornBrown] (5.27,7.9) -- ++(0, -0.3)node{};
      \draw[Bar-, very thick,seabornBrown] (5.88,7.9) -- ++(0, -0.3)node{};
      \draw[Bar-, very thick,seabornBrown] (6.5,7.9) -- ++(0, -0.3)node{};
      \draw[Bar-, very thick,seabornBrown] (7.12,7.9) -- ++(0, -0.3)node{};
      \draw[Bar-, very thick,seabornBrown] (7.41,7.9) -- ++(0, -0.3)node{};
      \draw[Bar-, very thick,seabornBrown] (7.72,7.9) -- ++(0, -0.3)node{};
      \draw[Bar-, very thick,seabornBrown] (8.02,7.9) -- ++(0, -0.3)node{};
      \draw[Bar-, very thick,seabornBrown] (8.32,7.9) -- ++(0, -0.3)node{};
      \draw[Bar-, very thick,seabornBrown] (8.61,7.9) -- ++(0, -0.3)node{};
      \draw[Bar-, very thick,seabornBrown] (8.94,7.9) -- ++(0, -0.3)node{};
      \draw[Bar-, very thick,seabornBrown] (9.25,7.9) -- ++(0, -0.3)node{};
      \draw[Bar-, very thick,seabornBrown] (9.55,7.9) -- ++(0, -0.3)node{};

    \end{scope}

    \end{tikzpicture}
    \vspace{-4.5ex}
    \caption{\footnotesize Sites reserved for \hv{} $\colorHcal_{\colort}(\colork)$ over epochs $\colort=0$ to $\colort=5$.}
    \label{fig:hsurf-steady-intuition-heatmap}
  \end{subfigure}%
  \caption{
    \textbf{Steady algorithm strategy.}
    \footnotesize
    Top panel \ref{fig:hsurf-steady-intuition-diagram} shows sites selected for itmes with \hv{} $\colorh$ from their first occurance during epoch $\colort=\colorh-4$ to epoch $\colort=\colorh+1$, when stored instances of that \hv{} are overwritten.
    Available memory buffer sites are shown across the bottom of the schematic.
    Data items' vertical span provides a timeline from when they are stored to when they are overwritten.
    The first data item with hanoi value $\colorH(\colorT) = \colorh$ is placed in segment $\colort=\colorh-4$.
    The next data item with \hv{} $\colorh$ is encountered in the following epoch, and it is placed in bunch 1.
    In epoch $\colort=\colorh-1$, two data items with \hv{} $\colorh$ are encountered and placed into segments within bunch 2.
    Epoch $\colort=\colorh-1$, encounters 4 data items with \hv{} $\colorh-1$ places them in bunch 3 segments.
    In epoch $\colort=\colorh$, eight \hv{} $\colorh$ data items (twice as many) are encountered.
    We place them in bunch 4's one-site segments.
    Finally, during epoch $\colort=\colorh+1$, incoming data items with \hv{} $\colorh$ discarded and all stored \hv{} $\colorh$ items are overwritten.
    In this manner, data items with highest \hv{} are retained on a rolling basis to provide uniformly-spaced gaps --- as laid out in Figure \ref{fig:hanoi-intuition-steady}.
    Bottom panel \ref{fig:hsurf-steady-intuition-heatmap} shows \hv{} site reservations $\colorHcal_{\colort}(\colork)$ from epoch $\colort=0$ through $\colort=5$ with buffer size $\colorS=16$.
    Numbering/color coding corresponds to \hv{} a site is reserved for.
    Segments are separated by black dividers; bunches are not explicitly indicated.
    Annotations highlight ingest/overwrite lifecycle of \hv{} $\colorh=5$
    Note that rightmost site $\colork=\colorS-1$ is unused.
  }
  \label{fig:hsurf-steady-intuition}
\end{figure*}
