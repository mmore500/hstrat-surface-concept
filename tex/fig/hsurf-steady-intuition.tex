\begin{figure*}[htbp!]
  \flushleft
  \begin{subfigure}[b]{1.008\linewidth}
    \hspace{-1.7ex}
    \includegraphics[width=\textwidth, trim={0cm 0cm 0cm 1cm}, clip]{img/hsurf-steady-intuition}
    \vspace{-2ex}
    \caption{\footnotesize Site selection $\colorK(\colorT)$ for data items of one \hv{}, $\colorh = 6$.}
    \label{fig:hsurf-steady-intuition-diagram}
  \end{subfigure}
\vspace{-5ex}
  \flushright
  \begin{subfigure}[b]{0.98\linewidth}
    \includegraphics[width=0.011\textwidth, trim={0.2cm 2.8cm 31.8cm 2.8cm}, clip]{binder/teeplots/12/reservation-mode=steady-full+surface-size=32+viz=site-reservation-at-ranks-heatmap+ext=.pdf}%
    \begin{tikzpicture}

    % Include the image in a node
    \node [
      above right,
      inner sep=0] (image) at (0,0) {\includegraphics[width=0.989\textwidth, trim={0.7cm 0cm 1.4cm 5.6cm}, clip]{binder/teeplots/12/reservation-mode=steady-full+surface-size=32+viz=site-reservation-at-ranks-heatmap+ext=.pdf}};

    % Create scope with normalized axes
    \begin{scope}[
      x={($0.1*(image.south east)$)},
      y={($0.1*(image.north west)$)}]

    % Grid
      % \draw[lightgray,step=0.5] (image.south west) grid (image.north east);
    % Axes' labels
      % \foreach \x in {0,1,...,10} { \node [below] at (\x,0) {\x}; }
      % \foreach \y in {0,1,...,10} { \node [left] at (0,\y) {\y};}

\draw[stealth-, ultra thick, gray](0.37,2.68) -- ++(0, -0.5) node[font=\footnotesize] at (1.62, 2.38) {incidence $i=0$ of $\colorH(\colorT)=6$ at $\colorT=63$};
\draw[stealth-, ultra thick, gray](2.21+0.6,1.1+3.84) -- ++(0, -0.5) node[font=\footnotesize] at (2.45+0.6, 3.5+1.1) {$i=1$~~~of $\colorH(\colorT)=6$};
\draw[stealth-, ultra thick, gray](3.74-0.3,1.1+4.9) -- ++(0, -0.5) node[font=\footnotesize] at (3.53-0.3, 4.61+1.1) {$i=2$};
\draw[stealth-, ultra thick, gray](4.66-0.3,1.1+4.9) -- ++(0, -0.5) node[font=\footnotesize] at (4.45-0.3, 4.61+1.1) {$i=3$};
\draw[stealth-, ultra thick, gray](5.27,1.1+5.96) -- ++(0, -0.5) node[font=\footnotesize] at (5.06, 5.67+1.1) {$i=4$};
\draw[stealth-, ultra thick, gray](5.88,1.1+5.96) -- ++(0, -0.5) node[font=\footnotesize] at (5.67, 5.71+1.1) {$i=5$};
\draw[stealth-, ultra thick, gray](6.5,1.1+5.96) -- ++(0, -0.5) node[font=\footnotesize] at (6.29, 5.71+1.1) {$i=6$};
\draw[stealth-, ultra thick, gray](7.12,1.1+5.96) -- ++(0, -0.5) node[font=\footnotesize] at (6.91, 5.68+1.1) {$i=7$};
\draw[stealth-, ultra thick, gray](7.41+0.3,1.1+7.06) -- ++(0, -0.5) node[font=\footnotesize] at (7.30+0.3, 6.65+1.1) {$8$};
\draw[stealth-, ultra thick, gray](7.72+0.3,1.1+7.06) -- ++(0, -0.5) node[font=\footnotesize] at (7.61+0.3, 6.65+1.1) {$9$};
\draw[stealth-, ultra thick, gray](8.02+0.3,1.1+7.06) -- ++(0, -0.5) node[font=\footnotesize] at (7.91+0.3, 6.65+1.1) {$10$};
\draw[stealth-, ultra thick, gray](8.32+0.3,1.1+7.06) -- ++(0, -0.5) node[font=\footnotesize] at (8.21+0.3, 6.65+1.1) {$11$};
\draw[stealth-, ultra thick, gray](8.61+0.3,1.1+7.06) -- ++(0, -0.5) node[font=\footnotesize] at (8.5+0.3, 6.65+1.1) {$12$};
\draw[stealth-, ultra thick, gray](8.94+0.3,1.1+7.06) -- ++(0, -0.5) node[font=\footnotesize] at (8.83+0.3, 6.65+1.1) {$13$};
\draw[stealth-, ultra thick, gray](9.25+0.3,1.1+7.06) -- ++(0, -0.5) node[font=\footnotesize] at (9.14+0.3, 6.65+1.1) {$14$};
\draw[stealth-, ultra thick, gray](9.55+0.3,1.1+7.06) -- ++(0, -0.5) node[font=\footnotesize] at (9.44+0.3, 6.65+1.1) {$15$};
    \end{scope}

    \end{tikzpicture}
    \vspace{-5ex}
    \caption{\footnotesize Sites reserved for \hv{} $\colorHcal_{\colort}(\colork)$ over epochs $\colort=0$ to $\colort=7$. H.v.{} $\colorh=6$ annotated as an example.}
    \label{fig:hsurf-steady-intuition-heatmap}
  \end{subfigure}%
  \vspace{-2ex}
  \caption{
    \textbf{Steady algorithm strategy.}
    \footnotesize
    Top panel \ref{fig:hsurf-steady-intuition-diagram} shows sites selected for items with \hv{} $\colorh=6$ from their first occurrence during epoch $\colort=2$ to epoch $\colort=7$, when stored instances of that \hv{} are overwritten.
    Memory buffer sites are shown across the bottom of the schematic.
    Data items' vertical span stretches across time from the epoch when they are stored to the epoch when they are overwritten.
    The first data item with hanoi value $\colorH(\colorT) = \colorh$ is placed in bunch 0 during epoch $\colort=\colorh-4$.
    The next data item with \hv{} $\colorh$ is encountered in the following epoch, and it is placed in bunch 1.
    In epoch $\colort=\colorh-2$, two data items with \hv{} $\colorh$ are encountered and placed into segments within bunch 2.
    Epoch $\colort=\colorh-1$, encounters 4 data items with \hv{} $\colorh-1$ places them in bunch 3's segments.
    In epoch $\colort=\colorh$, eight \hv{} $\colorh$ data items (twice as many) are encountered.
    We place them in bunch 4's one-site segments.
    Finally, during epoch $\colort=\colorh+1$, all further ingested data items with \hv{} $\colorh$ are discarded and all existing stored \hv{} $\colorh$ items are overwritten.
    In this manner, data items with highest \hv{} are retained on a rolling basis to provide uniformly-spaced gaps --- as laid out in Figure \ref{fig:hanoi-intuition-steady}.
    Bottom panel \ref{fig:hsurf-steady-intuition-heatmap} shows \hv{} site reservations $\colorHcal_{\colort}(\colork)$ from epoch $\colort=0$ through $\colort=5$ with buffer size $\colorS=16$.
    Numbering/color coding corresponds to which \hv{} a site is reserved for.
    Black dividers separate bunches; white space divides segments within bunches.
    Annotations highlight the lifecycle of data items with \hv{} $\colorh=6$.
  }
  \label{fig:hsurf-steady-intuition}
\end{figure*}
