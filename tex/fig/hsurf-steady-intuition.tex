\begin{figure*}[htbp!]
  \flushleft
  \begin{subfigure}[b]{0.972\linewidth}
    \includegraphics[width=\textwidth, trim={0cm 0cm 0.3cm 1cm}, clip]{img/hsurf-steady-intuition}
    \vspace{-4ex}
    \caption{\footnotesize Site selection $\colorK(\colorT)$ for data items of one \hv{}, $\colorH(\colorT) = \colorh$.}
    \label{fig:hsurf-steady-intuition-diagram}
  \end{subfigure}
\vspace{-2ex}
  \flushright
  \begin{subfigure}[b]{0.98\linewidth}
    \includegraphics[width=0.011\textwidth, trim={0.2cm 2.8cm 31.8cm 2.8cm}, clip]{binder/teeplots/11/reservation-mode=steady+surface-size=32+viz=site-reservation-at-ranks-heatmap+ext=}%
    \begin{tikzpicture}

    % Include the image in a node
    \node [
      above right,
      inner sep=0] (image) at (0,0) {\includegraphics[width=0.989\textwidth, trim={0.7cm 0cm 1.4cm 5.6cm}, clip]{binder/teeplots/11/reservation-mode=steady+surface-size=32+viz=site-reservation-at-ranks-heatmap+ext=}};

    % Create scope with normalized axes
    \begin{scope}[
      x={($0.1*(image.south east)$)},
      y={($0.1*(image.north west)$)}]

    % Grid
      % \draw[lightgray,step=0.5] (image.south west) grid (image.north east);
    % Axes' labels
      % \foreach \x in {0,1,...,10} { \node [below] at (\x,0) {\x}; }
      % \foreach \y in {0,1,...,10} { \node [left] at (0,\y) {\y};}

\draw[stealth-, ultra thick, seabornBrown](0.37,2.68) -- ++(0, -0.5) node[font=\footnotesize] at (1.62, 2.38) {incidence $i=0$ of $\colorH(\colorT)=5$ at $\colorT=31$};
\draw[stealth-, ultra thick, seabornBrown](2.21,3.84) -- ++(0, -0.5) node[font=\footnotesize] at (2.45, 3.5) {$i=1$~~~of $\colorH(\colorT)=5$};
\draw[stealth-, ultra thick, seabornBrown](3.74,4.9) -- ++(0, -0.5) node[font=\footnotesize] at (3.53, 4.61) {$i=2$};
\draw[stealth-, ultra thick, seabornBrown](4.66,4.9) -- ++(0, -0.5) node[font=\footnotesize] at (4.45, 4.61) {$i=3$};
\draw[stealth-, ultra thick, seabornBrown](5.27,5.96) -- ++(0, -0.5) node[font=\footnotesize] at (5.06, 5.67) {$i=4$};
\draw[stealth-, ultra thick, seabornBrown](5.88,5.96) -- ++(0, -0.5) node[font=\footnotesize] at (5.67, 5.71) {$i=5$};
\draw[stealth-, ultra thick, seabornBrown](6.5,5.96) -- ++(0, -0.5) node[font=\footnotesize] at (6.29, 5.71) {$i=6$};
\draw[stealth-, ultra thick, seabornBrown](7.12,5.96) -- ++(0, -0.5) node[font=\footnotesize] at (6.91, 5.68) {$i=7$};
\draw[stealth-, ultra thick, seabornBrown](7.41,7.06) -- ++(0, -0.5) node[font=\footnotesize] at (7.30, 6.65) {$8$};
\draw[stealth-, ultra thick, seabornBrown](7.72,7.06) -- ++(0, -0.5) node[font=\footnotesize] at (7.61, 6.65) {$9$};
\draw[stealth-, ultra thick, seabornBrown](8.02,7.06) -- ++(0, -0.5) node[font=\footnotesize] at (7.91, 6.65) {$10$};
\draw[stealth-, ultra thick, seabornBrown](8.32,7.06) -- ++(0, -0.5) node[font=\footnotesize] at (8.21, 6.65) {$11$};
\draw[stealth-, ultra thick, seabornBrown](8.61,7.06) -- ++(0, -0.5) node[font=\footnotesize] at (8.5, 6.65) {$12$};
\draw[stealth-, ultra thick, seabornBrown](8.94,7.06) -- ++(0, -0.5) node[font=\footnotesize] at (8.83, 6.65) {$13$};
\draw[stealth-, ultra thick, seabornBrown](9.25,7.06) -- ++(0, -0.5) node[font=\footnotesize] at (9.14, 6.65) {$14$};
\draw[stealth-, ultra thick, seabornBrown](9.55,7.06) -- ++(0, -0.5) node[font=\footnotesize] at (9.44, 6.65) {$15$};
    \end{scope}

    \end{tikzpicture}
    \vspace{-5ex}
    \caption{\footnotesize Sites reserved for \hv{} $\colorHcal_{\colort}(\colork)$ over epochs $\colort=0$ to $\colort=5$. H.v.{} $\colorh=5$ annotated as an example.}
    \label{fig:hsurf-steady-intuition-heatmap}
  \end{subfigure}%
  \vspace{-2ex}
  \caption{
    \textbf{Steady algorithm strategy.}
    \footnotesize
    Top panel \ref{fig:hsurf-steady-intuition-diagram} shows sites selected for items with \hv{} $\colorh$ from their first occurrence during epoch $\colort=\colorh-4$ to epoch $\colort=\colorh+1$, when stored instances of that \hv{} are overwritten.
    Available memory buffer sites are shown across the bottom of the schematic.
    Data items' vertical span provides a timeline from when they are stored to when they are overwritten.
    The first data item with hanoi value $\colorH(\colorT) = \colorh$ is placed in bunch 0 during epoch $\colort=\colorh-4$.
    The next data item with \hv{} $\colorh$ is encountered in the following epoch, and it is placed in bunch 1.
    In epoch $\colort=\colorh-2$, two data items with \hv{} $\colorh$ are encountered and placed into segments within bunch 2.
    Epoch $\colort=\colorh-1$, encounters 4 data items with \hv{} $\colorh-1$ places them in bunch 3's segments.
    In epoch $\colort=\colorh$, eight \hv{} $\colorh$ data items (twice as many) are encountered.
    We place them in bunch 4's one-site segments.
    Finally, during epoch $\colort=\colorh+1$, all further ingested data items with \hv{} $\colorh$ are discarded and all existing stored \hv{} $\colorh$ items are overwritten.
    In this manner, data items with highest \hv{} are retained on a rolling basis to provide uniformly-spaced gaps --- as laid out in Figure \ref{fig:hanoi-intuition-steady}.
    Bottom panel \ref{fig:hsurf-steady-intuition-heatmap} shows \hv{} site reservations $\colorHcal_{\colort}(\colork)$ from epoch $\colort=0$ through $\colort=5$ with buffer size $\colorS=16$.
    Numbering/color coding corresponds to which \hv{} a site is reserved for.
    Black dividers separate segments; bunches are not explicitly indicated.
    Annotations highlight the lifecycle of data items with \hv{} $\colorh=5$.
    Note that the rightmost site $\colork=\colorS-1$ is unused.
  }
  \label{fig:hsurf-steady-intuition}
\end{figure*}
