\begin{figure*}

\begin{subfigure}{\textwidth}
\includegraphics[width=\textwidth]{binder-2025-07-19-mem-perf-tilted/binder/teeplots/2025-07-19-mem-perf-tilted/col=data-type+font.family=serif+hue=metric+kind=bar+lut=False+orient=h+qosnumitems=10000+row=num-sites+viz=catplot+x=advantage+y=tilted-memory-bytes+ext=.pdf}
\caption{without lut}
\label{fig:mem-perf-tilted:nolut}
\end{subfigure}

\begin{subfigure}{\textwidth}
\includegraphics[
    width=\textwidth, trim={0 0 0 1.2cm}, clip
]{binder-2025-07-19-mem-perf-tilted/binder/teeplots/2025-07-19-mem-perf-tilted/col=data-type+font.family=serif+hue=metric+kind=bar+lut=True+orient=h+qosnumitems=10000+row=num-sites+viz=catplot+x=advantage+y=tilted-memory-bytes+ext=.pdf}
\caption{with lut}
\label{fig:mem-perf-tilted:lut}
\end{subfigure}

\caption{
\textbf{Generalized ring buffer performance characteristics.}
\footnotesize
Comparison is against saturating bucket algorithm.
Annotations report symmetric fold-improvement, calculated as $\max(x, y) / \min(x, y) - 1$.
For legibility, bar heights are symmetric fold-advantage, calculated as $1 - \min(x, y) / \max(x, y)$.
Curation quality comparisons are performed on a same-memory-footprint basis,with \textit{inf} indicating that no saturating bucket data structure fits within the memory footprint of the generalized ring buffer.
Speed comparisons are performed on a same-item-capacity basis.
}
\label{fig:mem-perf-tilted}

\end{figure*}
