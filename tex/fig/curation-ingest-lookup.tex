\begin{figure*}

\begin{subfigure}{0.5\linewidth}
\centering
\includegraphics[width=0.83\linewidth]{img/curation-ingest-lookup-ring}
\caption{simple ring buffer}
\label{fig:curation-ingest-lookup:ring}
\end{subfigure}%
\begin{subfigure}{0.5\linewidth}
\centering
\includegraphics[width=0.83\linewidth]{img/curation-ingest-lookup-steady}
\caption{generalized ring buffer}
\label{fig:curation-ingest-lookup:steady}
\end{subfigure}%

\caption{
  {\textbf Comparison of simple ring buffer to DStream steady curation.}
  \footnotesize
  Both approaches accumulate items ingested from a data stream into a fixed-capacity memory buffer.
  However, whereas the ring buffer retains only most-recent items, the DStream algorithm maintains a sample spanning the entirety of elapsed history.
  In the case of DStream steady curation, ingest site assignment and stored item attribution, respectively, are governed by functions $\colorK$ and $\colorL$, described later on.
}
\label{fig:curation-ingest-lookup}
\end{figure*}
% https://docs.google.com/presentation/d/1-7eLWTSH16s1MpGyT0iTJGAVLEYdKM7hK3ao2y_bW9A
