\begin{figure*}

\begin{subfigure}{0.5\linewidth}
\centering
\includegraphics[width=0.83\linewidth]{img/curation-ingest-lookup-ring}
\caption{simple ring buffer}
\label{fig:curation-ingest-lookup:ring}
\end{subfigure}%
\begin{subfigure}{0.5\linewidth}
\centering
\includegraphics[width=0.83\linewidth]{img/curation-ingest-lookup-steady}
\caption{generalized ring buffer}
\label{fig:curation-ingest-lookup:steady}
\end{subfigure}%

\caption{
  {\textbf Comparison of simple ring buffer to generalized ring buffer curation.}
  \footnotesize
  Both approaches accumulate items ingested from a data stream into a fixed-capacity memory buffer.
  However, whereas the ring buffer retains only most-recent items, the generalized algorithm maintains a sample spanning the entirety of elapsed history.
  We notate ingest site assignment and stored item attribution, respectively, as $\colorK$ and $\colorL$.
  For the proposed DStream tilted algorithm these operations are described in Listings and \ref{alg:tilted-site-selection} and \ref{alg:tilted-time-lookup}.
}
\label{fig:curation-ingest-lookup}
\end{figure*}
% https://docs.google.com/presentation/d/1-7eLWTSH16s1MpGyT0iTJGAVLEYdKM7hK3ao2y_bW9A
