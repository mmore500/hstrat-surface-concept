\begin{figure}


\begin{subfigure}{\linewidth}
\includegraphics[width=\linewidth, clip, trim={0 7.2cm 0 0}]{binder-2025-06-30-cpp-bench-speed-tilted/binder/teeplots/2025-06-30-cpp-bench-speed-tilted/col=data-type+errorbar=se+font.family=serif+hue=algorithm+kind=line+row=platform+style=algorithm+viz=relplot+x=buffer-size-s+y=per-item-walltime-ns+ext=.pdf}
\caption{\small x86 workstation}
\label{fig:speed:thinkpad}
\end{subfigure}

\begin{subfigure}{\linewidth}
\includegraphics[width=\linewidth, clip, trim={0 0 0 9cm}]{binder-2025-06-30-cpp-bench-speed-tilted/binder/teeplots/2025-06-30-cpp-bench-speed-tilted/col=data-type+errorbar=se+font.family=serif+hue=algorithm+kind=line+row=platform+style=algorithm+viz=relplot+x=buffer-size-s+y=per-item-walltime-ns+ext=.pdf}
\caption{\small ARM embedded device}
\label{fig:speed:pi}
\end{subfigure}

\caption{%
\textbf{Performance comparison for tilted curation approaches.}
\small
Shaded bands indicate standard error over $n=10$ trials.
Results for byte and word dtypes were similar to double word dtype; shown in Supplemental Figure \ref{fig:speed-supp}.
Likewise, results for ARM workstation were similar to x86 workstation; also shown in Supplemental Figure \ref{fig:speed-supp}.
Table \ref{tab:algorithms} overviews compared treatments.
}
\label{fig:speed}
\end{figure}
