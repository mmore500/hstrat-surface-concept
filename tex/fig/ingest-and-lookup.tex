\begin{figure*}[htbp!]
  \centering
  \begin{minipage}[]{\textwidth}
    \centering
    \begin{minipage}{0.36\textwidth}
    \includegraphics[width=\textwidth]{img/ingest-and-lookup}

    % dummy subcaptions
    % adapted from https://tex.stackexchange.com/a/40182/316176
    \begin{subfigure}{0in}
      \captionsetup{labelformat=empty}
      \caption{}\label{fig:surface-site-ingest}
    \end{subfigure}
    \begin{subfigure}{0in}
      \captionsetup{labelformat=empty}
      \caption{}\label{fig:ingest-rank-calculation}
    \end{subfigure}

  \end{minipage}%
    \hfill
    \begin{minipage}{0.63\textwidth}
      \caption{%
      \textbf{Core stream curation algorithm operations.}
      \footnotesize
      The ingest site selection operation (item ``a'') takes the current time $\colorT$ and determines the buffer site $\colork$ to store the ingested data item.
      This operation is performed when storing data into a curated buffer, once for each data item received from the data stream.
      Data is not moved after it is stored.
      The ingested time calculation operation (item ``b'') provides the previous time $\colorTbar$ when the data item present at buffer site $\colork$ was ingested, given the current time $\colorT$.
      This operation is performed when reading data from a curated buffer, in order to identify the provenance of stored data.
      Note that the ingest time of stored data at a buffer site $\colork$ is determined by the sequence of sites that were selected by the ingest site selection operation --- in a loose sense, this calculation operation can be considered as ``decoding'' or ``inverse'' to the selection operation.
      Panels with diamond markers on the right show curated collection composition at $\colorT=4$ and $\colorT=8$.
      Target curated collection compositions considered in this work are shown in Figure \ref{fig:criteria-intuition}.
      }
      \label{fig:ingest-and-lookup}
      \end{minipage}
  \end{minipage}
\end{figure*}
