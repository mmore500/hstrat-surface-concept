\begin{figure*}
  \centering
  \begin{subfigure}{0.55\textwidth}
  \includegraphics[width=\textwidth]{img/surface-site-ingest}
  \caption{ingest site selection for storage}
  \label{fig:surface-site-ingest}
  \end{subfigure}%
  \hfill
  \begin{subfigure}{0.4\textwidth}
  \centering
  \includegraphics[width=\textwidth]{img/ingest-rank-calculation}
  \caption{ingested time calculation for site lookup}
  \label{fig:ingest-rank-calculation}
  \end{subfigure}
  \caption{%
  \textbf{Core stream curation algorithm operations.}
  \footnotesize
  The ingestion site selection operation (\ref{fig:surface-site-ingest}) takes the current time $\colorT$ and determines the buffer site $\colork$ to store the ingested data item.
  This operation is performed when storing data into a curated buffer, once for each data item received from the data stream.
  The ingested time calculation operation (\ref{fig:ingest-rank-calculation}) provides the previous time $\colorTbar$ when the data item present at buffer site $\colork$ was ingested, given the current time $\colorT$.
  This operation is performed when reading data from a curated buffer, in order to identify the provenance of stored data.
  Note that the ingestion time of stored data at a buffer site $\colork$ is determined by the sequence of sites that were selected by the ingestion site selection operation --- in a loose sense this calculation operation can be considered as ``decoding'' or ``inverse'' to the selection operation.
  }
  \label{fig:ingest-and-lookup}
\end{figure*}
