\begin{figure}
\begin{subfigure}{0.6\linewidth}
\includegraphics[width=\linewidth]{binder-downstream-benchmark/binder/teeplots/2025-01-18-qos-dstream-vs-naive-tilted/buffer_size=256+hue=algorithm+kind=line+palette=set2+style=algorithm+viz=relplot+x=num-items-ingested+y=gap-size-cost+ext=.pdf}
\caption{buffer size 256}
\end{subfigure}
\begin{subfigure}{0.6\linewidth}
\includegraphics[width=\linewidth]{binder-downstream-benchmark/binder/teeplots/2025-01-18-qos-dstream-vs-naive-tilted/buffer_size=1024+hue=algorithm+kind=line+palette=set2+style=algorithm+viz=relplot+x=num-items-ingested+y=gap-size-cost+ext=.pdf}
\caption{buffer size 1024}
\end{subfigure}
\begin{subfigure}{0.6\linewidth}
\includegraphics[width=\linewidth]{binder-downstream-benchmark/binder/teeplots/2025-01-18-qos-dstream-vs-naive-tilted/buffer_size=4096+hue=algorithm+kind=line+palette=set2+style=algorithm+viz=relplot+x=num-items-ingested+y=gap-size-cost+ext=.pdf}
\caption{buffer size 4096}
\end{subfigure}
\caption{%
Progression of recency-proportional record gap size under tilted curation strategies, for buffer sizes 256, 1024, and 4096.
\footnotesize
Lower is better.
Figure \ref{fig:tilted-qos} compares recency-proportional gap size progressions at buffer size 64.
}
\label{fig:tilted-qos-supp}
\end{figure}
