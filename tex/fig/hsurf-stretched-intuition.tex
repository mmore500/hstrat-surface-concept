\begin{figure*}[htbp!]
  \centering
\begin{subfigure}{0.49\textwidth}
\includegraphics[width=\textwidth, clip, trim={0 0 1.25cm 0}]{binder/teeplots/20/surface-size=16+viz=site-reservation-at-ranks-heatmap+ext=}
\caption{hanoi value reservations across epochs}
\label{fig:hsurf-stretched-intuition-reservations}
\end{subfigure}%
~
\vline
~
\begin{subfigure}{0.49\textwidth}
\includegraphics[width=\textwidth, clip, trim={1.25cm 0 0 0}]{binder/teeplots/20/plotter=size+surface-size=16+viz=site-reservation-at-ranks-heatmap+ext=}
\caption{TODO}
\label{fig:hsurf-stretched-intuition-reservations-size}
\end{subfigure}
  \caption{
    \textbf{Stretched algorithm strategy.}
    \footnotesize
    First panel \ref{fig:hsurf-stretched-intuition-site-selection} shows placement strategy for data items with hanoi value $\colorH(\colorTbar) = 0$ on a buffer of size $\colorS=8$.
    Four sites, shown in dark blue, are reserved for hanoi value 0 during epoch 0.
    Reservation segment bunches are nested recursively, with successive bunches having shorter segments.
    The first data item instance with hanoi value $\colorH(\colorTbar) = 0$ is placed in segment bunch 0, which only contains one segment: segment 0.
    The second \hv{} 0 data item is placed in segment bunch 1, which comprises just segment 1.
    The third and fourth \hv{} 0 items are placed in segment bunch 2, which comprises segments 2 and 3 --- both containing only one buffer site.
    During epoch 1, \hv{} 0 sites in segment bunch 2 are ceded, leaving two sites reserved for \hv{} 0.
    Note how these first-overwritten sites contain the most recent, as opposed to most ancient, \hv{} 0 data items.
    Subsequently, during epoch 2, reserved space for \hv{} 0 contracts to just segment 0 and the second \hv{} 0 data item is overwritten.
    Second panel \ref{fig:hsurf-stretched-intuition-reservations} shows \hv{} reservation layout across epochs for a 32-site buffer.
    Reservation segments are separated by black lines.
    As epochs elapse, inner-nested segments are invaded as outer-nested segments extend by one hanoi value.
    Recursive nesting ensures that the shortest --- and last-filled --- segments are lost first.
    % Each epoch, all invaded sites share the same hanoi value, causing the available reservation sites for that \hv{} to instantaneously halve.
    Because the distribution of segment lengths mirrors the hanoi sequence, buffer space becomes available for the first-seen instances of high hanoi values as they are encountered.
  In this manner, this layout strategy approximates the first-$n$ \hv{} strategy depicted in Figure \ref{fig:hanoi-intuition-stretched}, with $n$ decreasing to respect available buffer space as segments are invaded and lost.
  }
  \label{fig:hsurf-stretched-intuition}
\end{figure*}
