\begin{figure*}[htbp!]
  \centering
\begin{subfigure}{0.48\textwidth}
\includegraphics[width=\textwidth, clip, trim={0 0 1.25cm 0}]{binder/teeplots/20/surface-size=16+viz=site-reservation-at-ranks-heatmap+ext=}
\caption{\footnotesize Site \hv{} reservations $\colorHcal(\colork)$ for epochs $\colort=0$ to $\colort=11$.}
\label{fig:hsurf-stretched-intuition-reservations}
\end{subfigure}%
~
~
~
\vline
~
~
~
\begin{subfigure}{0.48\textwidth}
\includegraphics[width=\textwidth, clip, trim={1.25cm 0 0 0}]{binder/teeplots/20/plotter=size+surface-size=16+viz=site-reservation-at-ranks-heatmap+ext=}
\caption{\footnotesize Initialized $r$ and mature $R$ reservation segment sizes.}
\label{fig:hsurf-stretched-intuition-reservations-size}
\end{subfigure}
  \caption{
    \textbf{Stretched algorithm strategy.}
    \footnotesize
    Left panel \ref{fig:hsurf-stretched-intuition-reservations} shows progression of \hv{} reservations $\colorHcal(\colork)$ on a buffer with size $\colorS=16$ across supported epochs $\colort \in [0\twodots\colorS - \colors)$.
    Horizontal rows track epoch $\colort$, indicated on the leftmost axis.
    The rightmost axis, in the right panel, indicates meta-epoch $\colortau$.
    Observe, for instance, that four sites, colored blue, are reserved for \hv{} $\colorh=0$ during epoch $\colort=0$.
    As shown in right panel \ref{fig:hsurf-stretched-intuition-reservations-size}, reservation segment bunches are nested recursively, with inner bunches having shorter segments.
    Reservation segments are separated by black lines in both diagrams; bunches are indicated by color code in the right diagram, with segments having same initial size $r$ belonging to the same bunch.
    As epochs elapse, segments grow from initial size $r$ to mature size $R$ and are then invaded to elimination by their larger left neighbor.
    Note how recursive nesting ensures that the shortest segments are eliminated first.
    To ensure it lasts longest, the first item with \hv{} $\colorH(\colorT) = 0$ is placed in the leftmost (and largest) segment $r=5$.
    Subsequent \hv{} instances are accomodated in segment $r=3$, the two $r=2$ segments, and then the four $r=1$ segments.
    Once available segment reservations are filled, subsequent \hv{} instances are discarded without storage.
    Because the segment sizes $r$ mirror the hanoi sequence, invasion of one site per epoch $\colort$ ensures buffer space for instances of high \hv{} as they are encountered at later $\colorT$.
  In this manner, layout approximates the first-$n$ \hv{} strategy depicted in Figure \ref{fig:hanoi-intuition-stretched}, with $n$ progressively decreasing as segments are invaded and lost.
  }
  \label{fig:hsurf-stretched-intuition}
\end{figure*}
